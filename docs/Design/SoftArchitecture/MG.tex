\documentclass[12pt, titlepage]{article}

\usepackage{fullpage}
\usepackage[round]{natbib}
\usepackage{multirow}
\usepackage{booktabs}
\usepackage{tabularx}
\usepackage{graphicx} 
\usepackage{float} 
\usepackage{hyperref}
\usepackage{enumitem}
\hypersetup{
  colorlinks,
  citecolor=blue,
  filecolor=black,
  linkcolor=red,
  urlcolor=blue
}

\input{../../Comments}
%% Common Parts

\newcommand{\progname}{ProgName} % PUT YOUR PROGRAM NAME HERE
\newcommand{\authname}{Team 22, READING4ALL
\\ Fiza Sehar
\\ Nawaal Fatima
\\ Dhruv Sardana
\\ Moly Mikhail
\\ Casey Francine Bulaclac } % AUTHOR NAMES                  

\usepackage{hyperref}
    \hypersetup{colorlinks=true, linkcolor=blue, citecolor=blue, filecolor=blue,
                urlcolor=blue, unicode=false}
    \urlstyle{same}
                                


\newcounter{acnum}
\newcommand{\actheacnum}{AC\theacnum}
\newcommand{\acref}[1]{AC\ref{#1}}

\newcounter{ucnum}
\newcommand{\uctheucnum}{UC\theucnum}
\newcommand{\uref}[1]{UC\ref{#1}}

\newcounter{mnum}
\newcommand{\mthemnum}{M\themnum}
\newcommand{\mref}[1]{M\ref{#1}}

\begin{document}

\title{Module Guide for \progname{}}
\author{\authname}
\date{\today}

\maketitle

\pagenumbering{roman}

\section{Revision History}

\begin{tabularx}{\textwidth}{p{4cm}p{2cm}X}
  \toprule {\bf Date} & {\bf Version} & {\bf Notes}\\
  \midrule
  November 13 2025 & 1.0 & Rev-1 Design Document\\
  Date 2 & 1.1 & Notes\\
  \bottomrule
\end{tabularx}

\newpage

\section{Reference Material}

This section records information for easy reference.

\subsection{Abbreviations and Acronyms}

\renewcommand{\arraystretch}{1.2}
\begin{tabular}{l l}
  \toprule
  \textbf{symbol} & \textbf{description}\\
  \midrule
  AC & Anticipated Change\\
  DAG & Directed Acyclic Graph \\
  M & Module \\
  MG & Module Guide \\
  OS & Operating System \\
  R & Requirement\\
  SC & Scientific Computing \\
  SRS & Software Requirements Specification\\
  \progname & Explanation of program name\\
  UC & Unlikely Change \\
  WCAG 2.1 &  Web Content Accessibility Guidelines\\
  AODA & Accessibility for Ontarians with Disabilities Act \\
  UI & User Interface \\
  DOM & Document Object Model \\
  AI & Artificial Intelligence \\
  ML & Machine Learning \\
  ARIA & Accessible Rich Internet Applications \\
  \bottomrule
\end{tabular}\\

\newpage

\tableofcontents

\listoftables

\listoffigures

\newpage

\pagenumbering{arabic}

\section{Introduction}

Decomposing a system into modules is a commonly accepted approach to developing
software.  A module is a work assignment for a programmer or programming
team~\citep{ParnasEtAl1984}.  We advocate a decomposition
based on the principle of information hiding~\citep{Parnas1972a}.  This
principle supports design for change, because the ``secrets'' that each module
hides represent likely future changes.  Design for change is valuable in SC,
where modifications are frequent, especially during initial development as the
solution space is explored.

Our design follows the rules layed out by \citet{ParnasEtAl1984}, as follows:
\begin{itemize}
  \item System details that are likely to change independently should be the
    secrets of separate modules.
  \item Each data structure is implemented in only one module.
  \item Any other program that requires information stored in a module's data
    structures must obtain it by calling access programs belonging to
    that module.
\end{itemize}

After completing the first stage of the design, the Software Requirements
Specification (SRS), the Module Guide (MG) is
developed~\citep{ParnasEtAl1984}. The MG
specifies the modular structure of the system and is intended to allow both
designers and maintainers to easily identify the parts of the software.  The
potential readers of this document are as follows:

\begin{itemize}
  \item New project members: This document can be a guide for a new
    project member
    to easily understand the overall structure and quickly find the
    relevant modules they are searching for.
  \item Maintainers: The hierarchical structure of the module guide improves the
    maintainers' understanding when they need to make changes to the
    system. It is
    important for a maintainer to update the relevant sections of the document
    after changes have been made.
  \item Designers: Once the module guide has been written, it can be used to
    check for consistency, feasibility, and flexibility. Designers
    can verify the
    system in various ways, such as consistency among modules,
    feasibility of the
    decomposition, and flexibility of the design.
\end{itemize}

The rest of the document is organized as follows. Section
\ref{SecChange} lists the anticipated and unlikely changes of the software
requirements. Section \ref{SecMH} summarizes the module decomposition that
was constructed according to the likely changes. Section \ref{SecConnection}
specifies the connections between the software requirements and the
modules. Section \ref{SecMD} gives a detailed description of the
modules. Section \ref{SecTM} includes two traceability matrices. One checks
the completeness of the design against the requirements provided in the SRS. The
other shows the relation between anticipated changes and the modules. Section
\ref{SecUse} describes the use relation between modules.

\section{Anticipated and Unlikely Changes} \label{SecChange}

This section lists possible changes to the system. According to the likeliness
of the change, the possible changes are classified into two
categories. Anticipated changes are listed in Section \ref{SecAchange}, and
unlikely changes are listed in Section \ref{SecUchange}.

\subsection{Anticipated Changes} \label{SecAchange}

Anticipated changes are the source of the information that is to be hidden
inside the modules. Ideally, changing one of the anticipated changes will only
require changing the one module that hides the associated decision. The approach
adapted here is called design for
change.

\begin{description}

  \item[\refstepcounter{acnum} \actheacnum \label{acInfra}:] The specific
    infrastructure on which the software is running.
  \item[\refstepcounter{acnum} \actheacnum \label{acInputFormat}:]
    The format of the
    initial input data accepted by the system; in the future the
    system might extend to support PDFs and other formats.
  \item [\refstepcounter{acnum} \actheacnum \label{acMLModel}:] The
    machine learning model used for alt-text generation.
  \item [\refstepcounter{acnum} \actheacnum \label{acLang}:] The
    language used for alt-text generation, allowing the system to
    generate alt-text in multiple languages beyond English.
  \item [\refstepcounter{acnum} \actheacnum \label{acSession}:] The
    duration of time the user remains authenticated before being
    prompted to log in again.
  \item [\refstepcounter{acnum} \actheacnum \label{acInputTopic}:]
    The range of images supported for alt-text generation can be
    extended beyond academic STEM images.
  \item [\refstepcounter{acnum} \actheacnum \label{acPlatform}:] The
    platform that the system operates on can be extended beyond a web
    application to also include desktop or mobile applications.

\end{description}


\subsection{Unlikely Changes} \label{SecUchange}

The module design should be as general as possible. However, a general system is
more complex. Sometimes this complexity is not necessary. Fixing some design
decisions at the system architecture stage can simplify the software design. If
these decision should later need to be changed, then many parts of the design
will potentially need to be modified. Hence, it is not intended that these
decisions will be changed.\\

The following are changes that the system is unlikely to incur: 
\begin{description}

  \item[\refstepcounter{ucnum} \uctheucnum \label{ucIO}:] The
    project's evaluation metrics as shown in Table 5 of the
    Appendix in our \citet{SRS} document will not change as these
    metrics are concrete and
    will be used to evaluate the effectiveness of the generated
    alternative text.
  \item [\refstepcounter{ucnum} \uctheucnum \label{ucIO}:] The
    primary output device for this system will remain as a screen,
    even if the system is later expanded to additional platforms
    (i.e., mobile) and not just the current web-based tool.
  \item [\refstepcounter{ucnum} \uctheucnum \label{ucIO}:] The system
    will always support keyboard input as its primary
    interaction method, with mouse input also supported for selecting
    user interface elements. This is to ensure that
    the functionality of keyboard navigation is always enabled for
    accessibility.

\end{description}

\section{Module Hierarchy} \label{SecMH}

This section provides an overview of the module design. Modules are summarized
in a hierarchy decomposed by secrets in Table \ref{TblMH}. The modules listed
below, which are leaves in the hierarchy tree, are the modules that will
actually be implemented.

\begin{table}[H]
  \centering
  \begin{tabular}{|p{4cm}|p{4cm}|p{4cm}| p{4cm}|}
  \hline
  \textbf{AI Model Modules} & \textbf{Backend Modules} & \textbf{Frontend Modules} & \textbf{System Information Modules}\\ \hline
  
  DataPreprocess-Module & AuthenticationService Module & UserInterface-Interactions Module & Logging Module\\ \hline
  WCAGCompliance Module & SessionManagement Module & MainScreen Module & ~\\ \hline
  ModelOutput Module & ImageHandling-Validation Module & ShowHistoryScreen Module & ~\\ \hline
  AIModelTraining Module & BackendController Module & LogInScreen Module & ~\\ \hline
  CaptionGeneration Module & ~ & UserInterface-Accessibility Module & ~\\ \hline
  InternalMetric-Compliance Module & ~ & ~ & ~\\ \hline
  FeedbackMetrics Module & ~ & ~ & ~\\ \hline
  FeedbackIntegration Module & ~ & ~ & ~\\ \hline
  Feedback Module & ~ & ~ & ~\\ \hline
  
  \end{tabular}
  \caption{Module Hierarchy by Subsystem}
\end{table}

  
  

\section{Connection Between Requirements and Design} \label{SecConnection}

The design of the system is intended to satisfy the requirements developed in
the SRS. In this stage, the system is decomposed into modules. The connection
between requirements and modules is listed in Table~\ref{TblMH}.

\wss{The intention of this section is to document decisions that are made
  ``between'' the requirements and the design.  To satisfy some requirements,
  design decisions need to be made.  Rather than make these decisions implicit,
  they are explicitly recorded here.  For instance, if a program has security
  requirements, a specific design decision may be made to satisfy those
requirements with a password.}

\begin{table}[h!]
  \centering
  \begin{tabular}{p{0.3\textwidth} p{0.6\textwidth}}
    \toprule
    \textbf{Level 1} & \textbf{Level 2}\\
    \midrule

    {Hardware-Hiding Module [Section (\ref{hhmmain})]} & N/A \\
    \midrule

    \multirow{13}{0.3\textwidth}{Behaviour-Hiding Module [Section
    (\ref{bhmmain})]} &
    {DataPreprocess Module (\refstepcounter{mnum}\mthemnum \label{mod:preprocessing_table}) (see \ref{mod:preprocessing})}\\ & 
    {WCAGCompliance Module (\refstepcounter{mnum}\mthemnum \label{mod:wcagcompliance_table}) (see \ref{mod:wcagcompliance})}\\ &
    {ModelOutput Module (\refstepcounter{mnum}\mthemnum \label{mod:modeloutput_table}) (see \ref{mod:modeloutput})}\\ &
    {AuthenticationService Module (\refstepcounter{mnum}\mthemnum \label{mod:auth_table} )(see \ref{mod:auth})}\\ &
    {SessionManagement Module (\refstepcounter{mnum}\mthemnum \label{mod:mess}) (see \ref{mod:mess})}\\ &
    {BackendController Module (\refstepcounter{mnum}\mthemnum \label{mod:bcm_table}) (see \ref{mod:bcm})}\\ &
    {Logging Module (\refstepcounter{mnum}\mthemnum) (see \ref{mod:logmod})}\\ &
    {ImageValidation Module (\refstepcounter{mnum}\mthemnum) (see \ref{mod:imgvalidmod})}\\ & 
    {User Interface Interactions Module (\refstepcounter{mnum}\mthemnum) \label{mod:uii_table} (see \ref{mod: uii})} \\ & 
    {Main Screen Module (\refstepcounter{mnum}\mthemnum) \label{mod:msm_table} (see \ref{mod: msm})} \\ &
    {Show History Screen Module (\refstepcounter{mnum}\mthemnum) \label{mod:showhm_table} (see \ref{mod: showhm})} \\ &
    {Log In Screen Module (\refstepcounter{mnum}\mthemnum) \label{mod:lis_table} (see \ref{mod: lis})} \\ &
    {User Interface Accessibility Module (\refstepcounter{mnum}\mthemnum) \label{mod:uia_table} (see \ref{mod: uia})} \\ 
    \midrule

    \multirow{6}{0.3\textwidth}{Software Decision Module [Section (\ref{mod:sdmmain})]} &
    {AIModelTraining Module (\refstepcounter{mnum}\mthemnum \label{mod:modeltraining}) (see \ref{mod:modeltraining})}\\ &
    {CaptionGeneration Module (\refstepcounter{mnum}\mthemnum \label{mod:captiongeneration_table}) (see \ref{mod:captiongeneration})}\\ &
    {InternalMetricCompliance (\refstepcounter{mnum}\mthemnum \label{mod:internalmetriccompliance_table}) (see \ref{mod:internalmetriccompliance})}\\ &
    {FeedbackMetrics Module (\refstepcounter{mnum}\mthemnum \label{mod:feedbackmetrics_table} )(see \ref{mod:feedbackmetrics})}\\ &
    {FeedbackLoop Module (\refstepcounter{mnum}\mthemnum \label{mod:feedbackloop_table}) (see \ref{mod:feedbackloop})}\\ &
    {Feedback Module (\refstepcounter{mnum}\mthemnum \label{mod:feedback_table}) (see \ref{mod:feedback})}\\
    \bottomrule
  \end{tabular}
  \caption{Module Hierarchy}
  \label{TblMH}
\end{table}

\section{Module Decomposition} \label{SecMD}

Modules are decomposed according to the principle of ``information hiding''
proposed by \citet{ParnasEtAl1984}. The \emph{Secrets} field in a module
decomposition is a brief statement of the design decision hidden by the
module. The \emph{Services} field specifies \emph{what} the module will do
without documenting \emph{how} to do it. For each module, a suggestion for the
implementing software is given under the \emph{Implemented By} title. If the
entry is \emph{OS}, this means that the module is provided by the operating
system or by standard programming language libraries.
\emph{\progname{}} means the
module will be implemented by the \progname{} software.

Only the leaf modules in the hierarchy have to be implemented. If a dash
(\emph{--}) is shown, this means that the module is not a leaf and will not have
to be implemented.

\subsection{Hardware Hiding Modules \label{hhmmain}}
This subsection is N/A as the Reading4All system has no hardware components.

\subsection{Behaviour-Hiding Modules \label{bhmmain}}
This subsection includes modules that pertain to the required
behavior of the Reading4All system.
The modules are defined as their own subsection and follow the following format:

\begin{description}
  \item[Secrets:]This part describes the essential behavioral
    components that define a module's intended functionality.
  \item[Services:]This component defines the module's programs that
    implement the
    externally visible behaviors of the system, as outlined in the
    Software Requirements Specification (SRS). Any modifications to the
    SRS will likely require corresponding updates to the programs
    within this module.
  \item[Implemented By:] This part mentions if the Reading4All system
    or external systems will
    implement the specific module.
  \item[Type of Module:] This section defines the moduel type into a
    Record, Library, Abstract
    Object, or Abstract Data Type. This section also includes
    information for leaf modules in the decomposition by secrets tree.
\end{description}

\subsubsection{DataPreprocess Module (\mref{mod:preprocessing_table})}
\label{mod:preprocessing}
\begin{description}
  \item[Secrets:] Includes the steps used to prepare input images for
    further processing.
  \item[Services:] Normalizes, sorts, and standardizes the image
    input so it can be correctly used by the training and caption
    generation modules.
  \item[Implemented By:] Reading4All
  \item[Type of Module:] Library
\end{description}

\subsubsection{WCAGCompliance Module (\mref{mod:wcagcompliance_table})}
\label{mod:wcagcompliance}

\begin{description}
  \item[Secrets:] The rules used to decide whether a caption meets
    AODA (\cite{AODA}) and WCAG 2.1 AA accessibility standards (\citet{WCAG}).
  \item[Services:] Checks the generated text against WCAG 2.1 AA (\citet{WCAG})
    criteria and flags or adjusts any generated alt text that does not meet these standards.
  \item[Implemented By:] Reading4All
  \item[Type of Module:] Library
\end{description}

\subsubsection{ModelOutput Module (\mref{mod:modeloutput_table})}
\label{mod:modeloutput}

\begin{description}
  \item[Secrets:] The structure used to store and present the final
    text to the user via the BackendController as mentioned in
    Section \ref{mod:bcm}.
  \item[Services:] Produces and formats the final version of the
    accessibility-compliant text for output or download.
  \item[Implemented By:] Reading4All
  \item[Type of Module:] Record
\end{description}

\subsubsection{AuthenticationService Module (\mref{mod:auth_table})}
\label{mod:auth}
\begin{description}
  \item[Secrets:] The internal logic for credential verification,
    token structure and signing, token refresh rules, lockout
    thresholds, and trust boundaries to external identity stores.
    Includes error classification for authentication failures and
    audit-safe handling of sensitive data.
  \item[Services:] Verifies user credentials, issues and refreshes
    authentication tokens, validates tokens on incoming requests,
    resolves current user identity (userID/role) for access control,
    and coordinates with Session Management to create/terminate sessions.
  \item[Implemented By:] Reading4All
  \item[Type of Module:] Library
\end{description}

\subsubsection{UserInterfaceInterfactions Module (\mref{mod:uii_table})}
\label{mod: uii}
\begin{description}
  \item[Secrets:] The internal logic and mapping of user-triggered actions such as clicks and keyboard presses 
  on the user interface into the backend and display screens modules. 
  \item[Services:] Captures user actions from the UI and connects UI controls to application 
  behavior by attaching event handlers that calls on other modules’ access programs with 
  validated parameters.
  \item[Implemented By:] Reading4All Team 
  \item[Type of Module:] Library
\end{description}

\subsubsection{MainScreen Module (\mref{mod:msm_table})}
\label{mod: msm}
\begin{description}
  \item[Secrets:] The internal layout composition of the main screen.
  This includes how the upload panel and supporting UI elements are arranged and rendered into the \texttt{DOM}. 
  \item[Services:] Renders the “Main” screen of the application. This screen displays the image-upload entry point for users.
  It calls on the UserInterfaceAccessibility module 
  to assign the main landmark, set initial focus, and announce the screen to screen readers. 
  \item[Implemented By:] Reading4All Team 
  \item[Type of Module:] Library
\end{description}

\subsubsection{ShowHistoryScreen Module (\mref{mod:showhm_table})}
\label{mod: showhm}
\begin{description}
  \item[Secrets:] The internal layout composition of the show history screen when an authenticated user wants to see their 
  previously generated alt-text history. This includes how the history items are formatted and displayed. 
  \item[Services:] Renders the user’s current session history to the screen. It relies on the UserInterfaceAccessibility module
   to set the main landmark, focus the first history item (if any), and announce the screen.
  \item[Implemented By:] Reading4All Team 
  \item[Type of Module:] Library
\end{description}

\subsubsection{LogInScreen Module (\mref{mod:lis_table})}
\label{mod: lis}
\begin{description}
  \item[Secrets:] The internal layout composition of the log in screen including username/email and password fields, and 
  a log in button.
  \item[Services:] Renders the system’s Log In screen. It calls on the UserInterfaceAccessibility module 
  to set focus on the fields on the log in form and announce the screen. 
  \item[Implemented By:] Reading4All Team 
  \item[Type of Module:] Library
\end{description}

\subsubsection{UserInterfaceAccessibility Module (\mref{mod:uia_table})}
\label{mod: uia}
\begin{description}
  \item[Secrets:] The internal mechanism for providing screen reader and keyboard navigation compatibility in the browser.
  This includes how ARIA (\citet{ARIA}) live regions are implemented, how focus is manipulated, and how keyboard navigation is enabled 
  internally.
  \item[Services:] Provides accessibility capabilities to other user interface modules by: assigning and maintaining 
  the correct main landmark on each screen; managing which element receives initial keyboard focus; sending 
  announcements to screen readers using ARIA (\citet{ARIA}) live regions; and enabling keyboard navigation patterns. 
  \item[Implemented By:] Reading4All Team 
  \item[Type of Module:] Library
\end{description}
\subsubsection{SessionManagement Module}
\label{mod: mess}
\begin{description}
  \item[Secrets:] The representation and storage of active sessions,
    expiry/refresh policies, session-history data structures
    (interaction entries and Reading4All data references), and
    eviction/cleanup strategies. Includes indexing strategies for
    fast token lookups and privacy rules for session-scoped logs.
  \item[Services:] Creates, validates, refreshes, and deletes
    sessions; appends interaction entries; links session activity to
    Reading4All data records; retrieves filtered session history and
    recent data references for the current session.
  \item[Implemented By:] Reading4All
  \item[Type of Module:] Abstract Data Type
\end{description}

\subsubsection{BackendController Module (\mref{mod:bcm_table})}
\label{mod:bcm}
\begin{description}
  \item[Secrets:] The internal application flow of data and user
    inputs, including how backend requests are processed and routed
    between services.
  \item[Services:]Coordinates backend responses by receiving requests
    from the frontend and directing it to the appropriate service
    such as the User Authentication or  AI Model, and returning the results.
  \item[Implemented By:] Reading4All
  \item[Type of Module:] Abstract Objects
\end{description}

\subsubsection{Logging Module}
\label{mod:logmod}
\begin{description}
  \item[Secrets:] The internal details of which events and errors are logged and the specific format used for log entries.
  \item[Services:] Records system events and errors received from the BackendController Module. 
  \item[Implemented By:] Reading4All
  \item[Type of Module:] Library
\end{description}

\subsubsection{ImageValidation Module}
\label{mod:imgvalidmod}
\begin{description}
  \item[Secrets:] The internal details of how uploaded images are verified, including how file type, size are checked and whether a file is reachable. 
  \item[Services:] Validates uploaded images to ensure that they meet system requirements before the alt text generation process. 
  \item[Implemented By:] Reading4All
  \item[Type of Module:] Library
\end{description}



\subsection{Software Decision Modules}
\label{mod:sdmmain}
This subsection includes the modules' internal operations,
connections with other modules and decision making components.
Similar to Section \ref{bhmmain} the modules are defined as their own
subsection and follow the following format:

\begin{description}
  \item[Secrets:] Internal logic, data flow, and processing steps
    that define how the system produces and improves alternative text.
  \item[Services:] Implements the internal operations of the system,
    such as model training, caption generation, quality validation,
    feedback collection, and output production.
\end{description}

\subsubsection{AIModelTraining Module}
\label{mod:modeltraining}
\begin{description}
  \item[Secrets:] The internal design of how the model is built and
    improved. This includes how image and text data are linked, how
    training is divided into sets, and what hidden patterns the model
    learns. The specific model structure and training process are not
    exposed to other modules.
  \item[Services:] Trains the system to understand relationships
    between images and text so it can later describe new images
    accurately. Produces a trained model and supporting data (such as
    validation results) that are used by the CaptionGeneration module.
  \item[Implemented By:] Reading4All
  \item[Type of Module:] Abstract Object
\end{description}

\subsubsection{CaptionGeneration Module (\mref{mod:captiongeneration_table})}
\label{mod:captiongeneration}
\begin{description}
  \item[Secrets:] The process the system uses to generate descriptive
    text from an image. This includes how image features are
    interpreted and how words are selected and arranged into sentences.
  \item[Services:] Uses the trained model from the AIModelTraining module to produce alternative text
    for a given image. Passes the generated caption to the WCAG (\citet{WCAG}) and
    InternalMetricCompliance modules to ensure it meets accessibility
    and clarity standards.
  \item[Implemented By:] Reading4All
  \item[Type of Module:] Library
\end{description}

\subsubsection{InternalMetricCompliance Module (\mref{mod:internalmetriccompliance_table})}
\label{mod:internalmetriccompliance}
\begin{description}
  \item[Secrets:] The internal measures chosen by Team 22 to judge
    how clear, concise or relevant the generated text is.
  \item[Services:] Reviews each caption using internal quality checks
    that complement the AODA (\citet{AODA}) and WCAG 2.1 AA (\citet{WCAG}) validation standards.
  \item[Implemented By:] Reading4All
  \item[Type of Module:] Library
\end{description}

\subsubsection{FeedbackMetrics Module (\mref{mod:feedbackmetrics_table})}
\label{mod:feedbackmetrics}
\begin{description}
  \item[Secrets:] How user feedback is collected, interpreted, and
    organized for later use in model improvement. The exact method
    for analyzing or weighting different types of feedback is hidden.
  \item[Services:] Receives direct input from users in the form of
    comments, ratings, or corrections about the generated alternative
    text. Summarizes this information into a structured format that
    can be used to guide model retraining and future system updates.
  \item[Implemented By:] Reading4All
  \item[Type of Module:] Library
\end{description}

\subsubsection{FeedbackLoop Module (\mref{mod:feedbackloop_table})}
\label{mod:feedbackloop}
\begin{description}
  \item[Secrets:] The rules and internal logic that determine how
    user feedback is used to update and improve the system’s model.
    The specific approach to retraining or fine-tuning is hidden.
  \item[Services:] Takes processed feedback data from the
    FeedbackMetrics module and uses it to retrain or adjust the model
    over time. This allows the system to gradually improve its
    accuracy and produce captions that better reflect user expectations.
  \item[Implemented By:] Reading4All
  \item[Type of Module:] Abstract Data Type
\end{description}

\subsubsection{Feedback Module (\mref{mod:feedback_table})}
\label{mod:feedback}
\begin{description}
  \item[Secrets:] How feedback information is stored and organized.
  \item[Services:] Stores the user’s feedback entries and makes them
    available to the FeedbackMetrics and FeedbackLoop modules for
    review and improvement purposes.
  \item[Implemented By:] Reading4All
  \item[Type of Module:] Record
\end{description}

\section{Traceability Matrix} \label{SecTM}

This section shows two traceability matrices: between the modules and the
requirements and between the modules and the anticipated changes.

% the table should use mref, the requirements should be named, use something
% like fref
\begin{table}[H]
  \centering
  \begin{tabular}{p{0.2\textwidth} p{0.6\textwidth}}
    \toprule
    \textbf{Req.} & \textbf{Modules}\\
    \midrule
    FR1 & \mref{mod:preprocessing_table}\\
    FR2 & \mref{mod:captiongeneration_table}\\
    FR3 & \mref{}\\
    FR4 & \mref{}\\
    FR5 & \mref{}\\
    FR6 & \mref{}\\
    \bottomrule
  \end{tabular}
  \caption{Trace Between Functional Requirements and Modules}
  \label{TblFRT}
\end{table}

\begin{table}[H]
  \centering
  \begin{tabular}{p{0.2\textwidth} p{0.6\textwidth}}
    \toprule
    \textbf{Req.} & \textbf{Modules}\\
    \midrule
    LFR-AR1 & \mref{mod:uii_table}, \mref{mod:msm_table}, \mref{mod:showhm_table}, \mref{mod:lis_table}, \mref{mod:uia_table}\\
    LFR-AR2 & \mref{mod:msm_table}, \mref{mod:uii_table}, \mref{mod:showhm_table}, \mref{mod:lis_table}, \mref{mod:uia_table}\\
    LFR-AR3 & \mref{mod:msm_table}, \mref{mod:uii_table}, \mref{mod:showhm_table}, \mref{mod:lis_table}, \mref{mod:uia_table}\\
    LFR-AR4 & \mref{mod:internalmetriccompliance_table}, \mref{mod:msm_table}, \mref{mod:uii_table}, \mref{mod:showhm_table}, \mref{mod:lis_table}. \mref{mod:uia_table}\\
    \bottomrule
  \end{tabular}
  \caption{Trace Between Look and Feel Requirements and Modules}
  \label{TblFRT}
\end{table}

\begin{table}[H]
  \centering
  \begin{tabular}{p{0.2\textwidth} p{0.6\textwidth}}
    \toprule
    \textbf{Req.} & \textbf{Modules}\\
    \midrule
    LFR-SR1 & \mref{mod:uii_table}, \mref{mod:msm_table}, \mref{mod:showhm_table}, \mref{mod:lis_table}, \mref{mod:uia_table}\\
    LFR-SR2 & \mref{mod:uii_table}, \mref{mod:msm_table}, \mref{mod:showhm_table}, \mref{mod:lis_table}, \mref{mod:uia_table}\\
    LFR-SR3 & \mref{mod:uii_table}, \mref{mod:msm_table}, \mref{mod:showhm_table}, \mref{mod:lis_table}, \mref{mod:uia_table}\\
    \bottomrule
  \end{tabular}
  \caption{Trace Between Style Requirements and Modules}
  \label{TblSRT}
\end{table}

\begin{table}[H]
  \centering
  \begin{tabular}{p{0.2\textwidth} p{0.6\textwidth}}
    \toprule
    \textbf{Req.} & \textbf{Modules}\\
    \midrule
    UHR-EUR1 & \mref{mod:wcagcompliance_table}, \mref{mod:modeloutput_table}, \mref{mod:uii_table}, \mref{mod:msm_table}, \mref{mod:showhm_table}, \mref{mod:lis_table}. \mref{mod:uia_table}\\
    UHR-EUR2 & \mref{mod:uii_table}, \mref{mod:msm_table}, \mref{mod:showhm_table}, \mref{mod:lis_table}, \mref{mod:uia_table}\\
    UHR-EUR3 & \mref{mod:uii_table}, \mref{mod:msm_table}, \mref{mod:showhm_table}, \mref{mod:lis_table}, \mref{mod:uia_table}\\
    UHR-EUR4 & \mref{mod:uii_table}, \mref{mod:msm_table}, \mref{mod:showhm_table}, \mref{mod:lis_table}, \mref{mod:uia_table}\\
    \bottomrule
  \end{tabular}
  \caption{Trace Between Usability and Humanity Requirements and Modules}
  \label{TblURT}
\end{table}

\begin{table}[H]
  \centering
  \begin{tabular}{p{0.2\textwidth} p{0.6\textwidth}}
    \toprule
    \textbf{Req.} & \textbf{Modules}\\
    \midrule
    UHR-PIR1 & \mref{mod:uii_table}, \mref{mod:msm_table}, \mref{mod:uia_table}\\
    \bottomrule
  \end{tabular}
  \caption{Trace Between Personalization and Internationalization
  Requirements and Modules}
  \label{TblPIRT}
\end{table}

\begin{table}[H]
  \centering
  \begin{tabular}{p{0.2\textwidth} p{0.6\textwidth}}
    \toprule
    \textbf{Req.} & \textbf{Modules}\\
    \midrule
    UHR-LR1 & \mref{mod:uii_table}, \mref{mod:msm_table}, \mref{mod:showhm_table}, \mref{mod:lis_table}, \mref{mod:uia_table}\\
    \bottomrule
  \end{tabular}
  \caption{Trace Between Learning Requirements and Modules}
  \label{TblPIRT}
\end{table}

\begin{table}[H]
  \centering
  \begin{tabular}{p{0.2\textwidth} p{0.6\textwidth}}
    \toprule
    \textbf{Req.} & \textbf{Modules}\\
    \midrule
    UHR-UPR1 & \mref{mod:uii_table}, \mref{mod:msm_table}, \mref{mod:showhm_table}, \mref{mod:lis_table}, \mref{mod:uia_table}\\
    \bottomrule
  \end{tabular}
  \caption{Trace Between Understandability and Politeness
  Requirements and Modules}
  \label{TblUPRT}
\end{table}

\begin{table}[H]
  \centering
  \begin{tabular}{p{0.2\textwidth} p{0.6\textwidth}}
    \toprule
    \textbf{Req.} & \textbf{Modules}\\
    \midrule
    UHR-AR1 & \mref{mod:uii_table}, \mref{mod:msm_table}, \mref{mod:showhm_table}, \mref{mod:lis_table}, \mref{mod:uia_table}\\
    UHR-AR2 & \mref{mod:uii_table}, \mref{mod:msm_table}, \mref{mod:showhm_table}, \mref{mod:lis_table}, \mref{mod:uia_table}\\
    \bottomrule
  \end{tabular}
  \caption{Trace Between Accessibility Requirements and Modules}
  \label{TblART}
\end{table}

\begin{table}[H]
  \centering
  \begin{tabular}{p{0.2\textwidth} p{0.6\textwidth}}
    \toprule
    \textbf{Req.} & \textbf{Modules}\\
    \midrule
    PR-SL1 & \mref{mod:captiongeneration_table}\\
    PR-SL2 &  \mref{mod:uia_table} \\
    PR-SCR1 & \mref{}\\
    PR-SCR2 & \mref{mod:internalmetriccompliance_table}\\
    PR-SCR3 & \mref{mod:uii_table}, \mref{mod:msm_table}, \mref{mod:showhm_table}, \mref{mod:lis_table}, \mref{mod:uia_table}\\
    PR-SR-HA1 & \mref{}\\
    PR-SR-HA2 & \mref{}\\
    PR-SR-HA3 & \mref{}\\
    PR-PAR1 & \mref{mod:captiongeneration_table}\\
    PR-PAR2 & \mref{mod:internalmetriccompliance_table}\\
    PR-PAR3 & \mref{mod:feedbackmetrics_table}\\
    PR-RFT1 & \mref{}\\
    PR-RFT2 & \mref{}\\
    PR-CR1 & \mref{}\\
    PR-CR2 & \mref{}\\
    PR-SER1 & \mref{mod:modeltraining_table}\\
    PR-LR1 & \mref{mod:modeltraining_table}, \mref{mod:wcagcompliance_table}\\
    PR-LR2 & \mref{}\\
    \bottomrule
  \end{tabular}
  \caption{Trace Between Performance Requirements and Modules}
  \label{TblFRT}
\end{table}

\begin{table}[H]
  \centering
  \begin{tabular}{p{0.2\textwidth} p{0.6\textwidth}}
    \toprule
    \textbf{Req.} & \textbf{Modules}\\
    \midrule
    OER-EP1 & \mref{}\\
    OER-EP2 & \mref{}\\
    OER-WE1 & \mref{mod:wcagcompliance_table}\\
    OER-WE2 & \mref{}\\
    OER-IAS1 & \mref{mod:preprocessing_table}\\
    OER-IAS2 & \mref{}\\
    OER-IAS3 & \mref{mod:internalmetriccompliance_table}\\
    OER-PR1 & \mref{}\\
    OER-RL1 & \mref{mod:preprocessing_table}, \mref{mod:modeltraining_table}, \mref{mod:captiongeneration_table}, \mref{mod:wcagcompliance_table}, \mref{mod:internalmetriccompliance_table}\\
    OER-RL2 & \mref{}\\
    \bottomrule
  \end{tabular}
  \caption{Trace Between Operational and Environmental Requirements and Modules}
  \label{TblOERT}
\end{table}

\begin{table}[H]
  \centering
  \begin{tabular}{p{0.2\textwidth} p{0.6\textwidth}}
    \toprule
    \textbf{Req.} & \textbf{Modules}\\
    \midrule
    MS-MNT1 & \mref{mod:preprocessing_table}, \mref{mod:captiongeneration_table}, \mref{mod:wcagcompliance_table}, \mref{mod:uii_table}, \mref{mod:msm_table}, \mref{mod:showhm_table}, \mref{mod:lis_table}, \mref{mod:uia_table}\\\\
    MS-MNT2 & \mref{}\\
    MS-MNT3 & \mref{}\\
    MS-SUP1 & \mref{}\\
    MS-AD1 & \mref{mod:modeltraining_table}\\
    MS-AD2 & \mref{}\\
    MS-AD3 & \mref{}\\
    \bottomrule
  \end{tabular}
  \caption{Trace Between Maintainability and Support Requirements and Modules}
  \label{TblMSRT}
\end{table}

\begin{table}[H]
  \centering
  \begin{tabular}{p{0.2\textwidth} p{0.6\textwidth}}
    \toprule
    \textbf{Req.} & \textbf{Modules}\\
    \midrule
    SR-AR1 & \mref{}\\
    SR-AR2 & \mref{}\\
    SR-IR1 & \mref{}\\
    SR-IR2 & \mref{mod:preprocessing_table}\\
    SR-PR1 & \mref{}\\
    SR-PR2 & \mref{mod:internalmetriccompliance_table}\\
    SR-AU1 & \mref{}\\
    SR-AU2 & \mref{}\\
    SR-IM1 & \mref{}\\
    SR-AIM2 & \mref{}\\
    \bottomrule
  \end{tabular}
  \caption{Trace Between all Security Requirements and Modules}
  \label{TblSR}
\end{table}


\begin{table}[H]
  \centering
  \begin{tabular}{p{0.2\textwidth} p{0.6\textwidth}}
    \toprule
    \textbf{Req.} & \textbf{Modules}\\
    \midrule
    CR1 & \mref{mod:captiongeneration_table}\\
    CR2 & \mref{mod:internalmetriccompliance_table}\\
    CR3 & \mref{mod:internalmetriccompliance_table}\\
    \bottomrule
  \end{tabular}
  \caption{Trace Between all Cultural Requirements and Modules}
  \label{TblCR}
\end{table}

\begin{table}[H]
  \centering
  \begin{tabular}{p{0.2\textwidth} p{0.6\textwidth}}
    \toprule
    \textbf{Req.} & \textbf{Modules}\\
    \midrule
    CR-LR1 & \mref{mod:wcagcompliance_table}\\
    CR-SCR1 & \mref{}\\
    CR-SCR2 & \mref{}\\
    \bottomrule
  \end{tabular}
  \caption{Trace Between all Compliance Requirements and Modules}
  \label{TbltCompR}
\end{table}


\begin{table}[H]
  \centering
  \begin{tabular}{p{0.2\textwidth} p{0.6\textwidth}}
    \toprule
    \textbf{AC} & \textbf{Modules}\\
    \midrule
    \acref{acHardware} & \mref{mHH}\\
    \acref{acInput} & \mref{mInput}\\
    \acref{acParams} & \mref{mParams}\\
    \acref{acVerify} & \mref{mVerify}\\
    \acref{acOutput} & \mref{mOutput}\\
    \acref{acVerifyOut} & \mref{mVerifyOut}\\
    \acref{acODEs} & \mref{mODEs}\\
    \acref{acEnergy} & \mref{mEnergy}\\
    \acref{acControl} & \mref{mControl}\\
    \acref{acSeqDS} & \mref{mSeqDS}\\
    \acref{acSolver} & \mref{mSolver}\\
    \acref{acPlot} & \mref{mPlot}\\
    \bottomrule
  \end{tabular}
  \caption{Trace Between Anticipated Changes and Modules}
  \label{TblACT}
\end{table}

\section{Use Hierarchy Between Modules} \label{SecUse}

In this section, the uses hierarchy between modules is
provided. \citet{Parnas1978} said of two programs A and B that A {\em uses} B if
correct execution of B may be necessary for A to complete the task described in
its specification. That is, A {\em uses} B if there exist situations in which
the correct functioning of A depends upon the availability of a correct
implementation of B.  Figure \ref{FigUH} illustrates the use relation between
the modules. It can be seen that the graph is a directed acyclic graph
(DAG). Each level of the hierarchy offers a testable and usable subset of the
system, and modules in the higher level of the hierarchy are essentially simpler
because they use modules from the lower levels.

\wss{The uses relation is not a data flow diagram.  In the code there will often
  be an import statement in module A when it directly uses module B.  Module B
  provides the services that module A needs.  The code for module A needs to be
  able to see these services (hence the import statement).  Since the uses
  relation is transitive, there is a use relation without an import, but the
arrows in the diagram typically correspond to the presence of import statement.}

\wss{If module A uses module B, the arrow is directed from A to B.}

\begin{figure}[H]
  \centering
  %\includegraphics[width=0.7\textwidth]{UsesHierarchy.png}
  \caption{Use hierarchy among modules}
  \label{FigUH}
\end{figure}

%\section*{References}

\section{User Interfaces}

\wss{Design of user interface for software and hardware.  Attach an appendix if
needed. Drawings, Sketches, Figma}

\section{Design of Communication Protocols}

\wss{If appropriate}

\section{Timeline}

\wss{Schedule of tasks and who is responsible}

\wss{You can point to GitHub if this information is included there}

\bibliographystyle {plainnat}
\bibliography{../../../refs/References}

\newpage{}

\end{document}
