\documentclass[12pt, titlepage]{article}

\usepackage{fullpage}
\usepackage[round]{natbib}
\usepackage{multirow}
\usepackage{booktabs}
\usepackage{tabularx}
\usepackage{graphicx}
\usepackage{float}
\usepackage{hyperref}
\hypersetup{
    colorlinks,
    citecolor=blue,
    filecolor=black,
    linkcolor=red,
    urlcolor=blue
}

\input{../../Comments}
%% Common Parts

\newcommand{\progname}{ProgName} % PUT YOUR PROGRAM NAME HERE
\newcommand{\authname}{Team 22, READING4ALL
\\ Fiza Sehar
\\ Nawaal Fatima
\\ Dhruv Sardana
\\ Moly Mikhail
\\ Casey Francine Bulaclac } % AUTHOR NAMES                  

\usepackage{hyperref}
    \hypersetup{colorlinks=true, linkcolor=blue, citecolor=blue, filecolor=blue,
                urlcolor=blue, unicode=false}
    \urlstyle{same}
                                
 

\newcounter{acnum}
\newcommand{\actheacnum}{AC\theacnum}
\newcommand{\acref}[1]{AC\ref{#1}}

\newcounter{ucnum}
\newcommand{\uctheucnum}{UC\theucnum}
\newcommand{\uref}[1]{UC\ref{#1}}

\newcounter{mnum}
\newcommand{\mthemnum}{M\themnum}
\newcommand{\mref}[1]{M\ref{#1}}

\begin{document}

\title{Module Guide for \progname{}} 
\author{\authname}
\date{\today}

\maketitle

\pagenumbering{roman}

\section{Revision History}

\begin{tabularx}{\textwidth}{p{3cm}p{2cm}X}
\toprule {\bf Date} & {\bf Version} & {\bf Notes}\\
\midrule
Date 1 & 1.0 & Notes\\
Date 2 & 1.1 & Notes\\
\bottomrule
\end{tabularx}

\newpage

\section{Reference Material}

This section records information for easy reference.

\subsection{Abbreviations and Acronyms}

\renewcommand{\arraystretch}{1.2}
\begin{tabular}{l l} 
  \toprule		
  \textbf{symbol} & \textbf{description}\\
  \midrule 
  AC & Anticipated Change\\
  DAG & Directed Acyclic Graph \\
  M & Module \\
  MG & Module Guide \\
  OS & Operating System \\
  R & Requirement\\
  SC & Scientific Computing \\
  SRS & Software Requirements Specification\\
  \progname & Explanation of program name\\
  UC & Unlikely Change \\
  \wss{etc.} & \wss{...}\\
  \bottomrule
\end{tabular}\\

\newpage

\tableofcontents

\listoftables

\listoffigures

\newpage

\pagenumbering{arabic}

\section{Introduction}

Decomposing a system into modules is a commonly accepted approach to developing
software.  A module is a work assignment for a programmer or programming
team~\citep{ParnasEtAl1984}.  We advocate a decomposition
based on the principle of information hiding~\citep{Parnas1972a}.  This
principle supports design for change, because the ``secrets'' that each module
hides represent likely future changes.  Design for change is valuable in SC,
where modifications are frequent, especially during initial development as the
solution space is explored.  

Our design follows the rules layed out by \citet{ParnasEtAl1984}, as follows:
\begin{itemize}
\item System details that are likely to change independently should be the
  secrets of separate modules.
\item Each data structure is implemented in only one module.
\item Any other program that requires information stored in a module's data
  structures must obtain it by calling access programs belonging to that module.
\end{itemize}

After completing the first stage of the design, the Software Requirements
Specification (SRS), the Module Guide (MG) is developed~\citep{ParnasEtAl1984}. The MG
specifies the modular structure of the system and is intended to allow both
designers and maintainers to easily identify the parts of the software.  The
potential readers of this document are as follows:

\begin{itemize}
\item New project members: This document can be a guide for a new project member
  to easily understand the overall structure and quickly find the
  relevant modules they are searching for.
\item Maintainers: The hierarchical structure of the module guide improves the
  maintainers' understanding when they need to make changes to the system. It is
  important for a maintainer to update the relevant sections of the document
  after changes have been made.
\item Designers: Once the module guide has been written, it can be used to
  check for consistency, feasibility, and flexibility. Designers can verify the
  system in various ways, such as consistency among modules, feasibility of the
  decomposition, and flexibility of the design.
\end{itemize}

The rest of the document is organized as follows. Section
\ref{SecChange} lists the anticipated and unlikely changes of the software
requirements. Section \ref{SecMH} summarizes the module decomposition that
was constructed according to the likely changes. Section \ref{SecConnection}
specifies the connections between the software requirements and the
modules. Section \ref{SecMD} gives a detailed description of the
modules. Section \ref{SecTM} includes two traceability matrices. One checks
the completeness of the design against the requirements provided in the SRS. The
other shows the relation between anticipated changes and the modules. Section
\ref{SecUse} describes the use relation between modules.

\section{Anticipated and Unlikely Changes} \label{SecChange}

This section lists possible changes to the system. According to the likeliness
of the change, the possible changes are classified into two
categories. Anticipated changes are listed in Section \ref{SecAchange}, and
unlikely changes are listed in Section \ref{SecUchange}.

\subsection{Anticipated Changes} \label{SecAchange}

Anticipated changes are the source of the information that is to be hidden
inside the modules. Ideally, changing one of the anticipated changes will only
require changing the one module that hides the associated decision. The approach
adapted here is called design for
change.

\begin{description}
\item[\refstepcounter{acnum} \actheacnum \label{acHardware}:] The specific
  hardware on which the software is running.
\item[\refstepcounter{acnum} \actheacnum \label{acInputFormat}:] The format of the
  initial input data accepted by the system; in the future the system might extend to support PDFs and other formats. 
\item [\refstepcounter{acnum} \actheacnum \label{acMLModel}:] The machine learning model used for alt-text generation. 
\item [\refstepcounter{acnum} \actheacnum \label{acLang}:] The language used for alt-text generation, allowing the system to generate alt-text in multiple languages beyond English. 
\item [\refstepcounter{acnum} \actheacnum \label{acSession}:] The duration of time the user remains authenticated before being prompted to log in again. 
\item [\refstepcounter{acnum} \actheacnum \label{acInputTopic}:] The range of images supported for alt-text generation can be extended beyond academic STEM images. 
\item [\refstepcounter{acnum} \actheacnum \label{acPlatform}:] The platform that the system operates on can be extended beyond a web application to also include desktop or mobile applications. 
\end{description}

\wss{Anticipated changes relate to changes that would be made in requirements,
design or implementation choices.  They are not related to changes that are made
at run-time, like the values of parameters.}

\subsection{Unlikely Changes} \label{SecUchange}

The module design should be as general as possible. However, a general system is
more complex. Sometimes this complexity is not necessary. Fixing some design
decisions at the system architecture stage can simplify the software design. If
these decision should later need to be changed, then many parts of the design
will potentially need to be modified. Hence, it is not intended that these
decisions will be changed.

\begin{description}
\item[\refstepcounter{ucnum} \uctheucnum \label{ucIO}:] The project's evaluation metrics as shown in Table 5 of the 
Appendix in our \citet{SRS} document will not change as these metrics are concrete and
will be used to evaluate the effectiveness of the generated alternative text.
\item [\refstepcounter{ucnum} \uctheucnum \label{ucIO}:] The primary output device for this system will remain as a screen, 
even if the system is later expanded to additional platforms (i.e., mobile) and not just the current web-based tool.
\item [\refstepcounter{ucnum} \uctheucnum \label{ucIO}:] The system will always support keyboard input as its primary 
interaction method, with mouse input also supported for selecting user interface elements. This is to ensure that 
the functionality of keyboard navigation is always enabled for accessibility.
\end{description}

\section{Module Hierarchy} \label{SecMH}

This section provides an overview of the module design. Modules are summarized
in a hierarchy decomposed by secrets in Table \ref{TblMH}. The modules listed
below, which are leaves in the hierarchy tree, are the modules that will
actually be implemented.

\begin{description}
\item [\refstepcounter{mnum} \mthemnum \label{mHH}:] Hardware-Hiding Module
\item ...
\end{description}


\begin{table}[h!]
\centering
\begin{tabular}{p{0.3\textwidth} p{0.6\textwidth}}
\toprule
\textbf{Level 1} & \textbf{Level 2}\\
\midrule

{Hardware-Hiding Module} & ~ \\
\midrule

\multirow{7}{0.3\textwidth}{Behaviour-Hiding Module} & ?\\
& Backend Controller Module\\
& ?\\
& ?\\
& ?\\
& ?\\
& ?\\ 
& ?\\
\midrule

\multirow{3}{0.3\textwidth}{Software Decision Module} & {?}\\
& ?\\
& ?\\
\bottomrule

\end{tabular}
\caption{Module Hierarchy}
\label{TblMH}
\end{table}

\section{Connection Between Requirements and Design} \label{SecConnection}

The design of the system is intended to satisfy the requirements developed in
the SRS. In this stage, the system is decomposed into modules. The connection
between requirements and modules is listed in Table~\ref{TblRT}.

\wss{The intention of this section is to document decisions that are made
  ``between'' the requirements and the design.  To satisfy some requirements,
  design decisions need to be made.  Rather than make these decisions implicit,
  they are explicitly recorded here.  For instance, if a program has security
  requirements, a specific design decision may be made to satisfy those
  requirements with a password.}

\section{Module Decomposition} \label{SecMD}

Modules are decomposed according to the principle of ``information hiding''
proposed by \citet{ParnasEtAl1984}. The \emph{Secrets} field in a module
decomposition is a brief statement of the design decision hidden by the
module. The \emph{Services} field specifies \emph{what} the module will do
without documenting \emph{how} to do it. For each module, a suggestion for the
implementing software is given under the \emph{Implemented By} title. If the
entry is \emph{OS}, this means that the module is provided by the operating
system or by standard programming language libraries.  \emph{\progname{}} means the
module will be implemented by the \progname{} software.

Only the leaf modules in the hierarchy have to be implemented. If a dash
(\emph{--}) is shown, this means that the module is not a leaf and will not have
to be implemented.

\subsection{Hardware Hiding Modules (\mref{mHH})}

\begin{description}
\item[Secrets:]The data structure and algorithm used to implement the virtual
  hardware.
\item[Services:]Serves as a virtual hardware used by the rest of the
  system. This module provides the interface between the hardware and the
  software. So, the system can use it to display outputs or to accept inputs.
\item[Implemented By:] OS
\end{description}

\subsection{Behaviour-Hiding Module}

\begin{description}
\item[Secrets:]The contents of the required behaviours.
\item[Services:]Includes programs that provide externally visible behaviour of
  the system as specified in the software requirements specification (SRS)
  documents. This module serves as a communication layer between the
  hardware-hiding module and the software decision module. The programs in this
  module will need to change if there are changes in the SRS.
\item[Implemented By:] --
\end{description}

\subsubsection{Input Format Module (\mref{mInput})}

\begin{description}
\item[Secrets:]The format and structure of the input data.
\item[Services:]Converts the input data into the data structure used by the
  input parameters module.
\item[Implemented By:] [Your Program Name Here]
\item[Type of Module:] [Record, Library, Abstract Object, or Abstract Data Type]
  [Information to include for leaf modules in the decomposition by secrets tree.]
\end{description}
\subsubsection{Authentication Service Module (\mref{mAUTH})}

\begin{description}
\item[Secrets:] The internal logic for credential verification, token structure and signing, token refresh rules, lockout thresholds, and trust boundaries to external identity stores. Includes error classification for authentication failures and audit-safe handling of sensitive data.
\item[Services:] Verifies user credentials, issues and refreshes authentication tokens, validates tokens on incoming requests, resolves current user identity (userID/role) for access control, and coordinates with Session Management to create/terminate sessions.
\item[Implemented By:] Reading4All Team
\item[Type of Module:] Library
  [Information to include for leaf modules in the decomposition by secrets tree.]
\end{description}

\subsubsection{Session Management Module (\mref{mSESS})}

\begin{description}
\item[Secrets:] The representation and storage of active sessions, expiry/refresh policies, session-history data structures (interaction entries and Reading4All data references), and eviction/cleanup strategies. Includes indexing strategies for fast token lookups and privacy rules for session-scoped logs.
\item[Services:] Creates, validates, refreshes, and deletes sessions; appends interaction entries; links session activity to Reading4All data records; retrieves filtered session history and recent data references for the current session.
\item[Implemented By:] Reading4All Team
\item[Type of Module:] Abstract Data Type
  [Information to include for leaf modules in the decomposition by secrets tree.]
\end{description}


\subsubsection{Backend Controller Module (\mref{mBCM})}

\begin{description}
\item[Secrets:] The internal application flow of data and user inputs, including how backend requests are processed and routed between services. 
\item[Services:]Coordinates backend responses by receiving requests from the frontend and directing it to the appropriate service such as the User Authentication or  AI Model, and returning the results.
\item[Implemented By:] Reading4All Team 
\item[Type of Module:] Abstract Object
  [Information to include for leaf modules in the decomposition by secrets tree.]
\end{description}


\subsubsection{Etc.}


\subsection{Software Decision Module}

\subsubsection{AIModelTraining Module}

\begin{description}
\item[Secrets:]
How to train an AI model to generate accurate alternative text for images using machine learning techniques.  
Hidden details include the choice of architecture, optimization process, and data handling strategy.

\item[Services:]
Provides a trained model that can process new image inputs and learn relationships between visual features and descriptive text.  
Produces saved model weights and training logs for later use by the inference module.

\item[Implemented By:] \progname{}
\item[Type of Module:] Abstract Object
\end{description}

\subsubsection{Classifier / Inference Module}

\begin{description}
\item[Secrets:]
The logic used to generate descriptive alternative text from trained model outputs.  
This includes internal decoding strategies and how the system interprets predictions.

\item[Services:]
Takes in new images (or preprocessed tensors) and outputs meaningful, readable descriptions for end users.  
Uses the trained model produced by the AIModelTraining module.

\item[Implemented By:] \progname{}
\item[Type of Module:] Abstract Data Type
\end{description}

\section{Traceability Matrix} \label{SecTM}

This section shows two traceability matrices: between the modules and the
requirements and between the modules and the anticipated changes.

% the table should use mref, the requirements should be named, use something
% like fref
\begin{table}[H]
\centering
\begin{tabular}{p{0.2\textwidth} p{0.6\textwidth}}
\toprule
\textbf{Req.} & \textbf{Modules}\\
\midrule
FR1 & \mref{mHH}, \mref{mInput}, \mref{mParams}, \mref{mControl}\\
FR2 & \mref{mInput}, \mref{mParams}\\
FR3 & \mref{mVerify}\\
FR4 & \mref{mOutput}, \mref{mControl}\\
FR5 & \mref{mOutput}, \mref{mODEs}, \mref{mControl}, \mref{mSeqDS}, \mref{mSolver}, \mref{mPlot}\\
FR6 & \mref{mOutput}, \mref{mODEs}, \mref{mControl}, \mref{mSeqDS}, \mref{mSolver}, \mref{mPlot}\\
\bottomrule
\end{tabular}
\caption{Trace Between Functional Requirements and Modules}
\label{TblFRT}
\end{table}

\begin{table}[H]
\centering
\begin{tabular}{p{0.2\textwidth} p{0.6\textwidth}}
\toprule
\textbf{Req.} & \textbf{Modules}\\
\midrule
LFR-AR1 & \mref{mHH}, \mref{mInput}, \mref{mParams}, \mref{mControl}\\
LFR-AR2 & \mref{mInput}, \mref{mParams}\\
LFR-AR3 & \mref{mVerify}\\
LFR-AR4 & \mref{mOutput}, \mref{mControl}\\
\bottomrule
\end{tabular}
\caption{Trace Between Look and Feel Requirements and Modules}
\label{TblFRT}
\end{table}

\begin{table}[H]
\centering
\begin{tabular}{p{0.2\textwidth} p{0.6\textwidth}}
\toprule
\textbf{Req.} & \textbf{Modules}\\
\midrule
LFR-SR1 & \mref{mHH}, \mref{mInput}, \mref{mParams}, \mref{mControl}\\
LFR-SR2 & \mref{mInput}, \mref{mParams}\\
LFR-SR3 & \mref{mVerify}\\
\bottomrule
\end{tabular}
\caption{Trace Between Style Requirements and Modules}
\label{TblSRT}
\end{table}


\begin{table}[H]
\centering
\begin{tabular}{p{0.2\textwidth} p{0.6\textwidth}}
\toprule
\textbf{Req.} & \textbf{Modules}\\
\midrule
UHR-EUR1 & \mref{mHH}, \mref{mInput}, \mref{mParams}, \mref{mControl}\\
UHR-EUR2 & \mref{mInput}, \mref{mParams}\\
UHR-EUR3 & \mref{mVerify}\\
UHR-EUR4 & \mref{mOutput}, \mref{mControl}\\
\bottomrule
\end{tabular}
\caption{Trace Between Usability and Humanity Requirements and Modules}
\label{TblURT}
\end{table}



\begin{table}[H]
\centering
\begin{tabular}{p{0.2\textwidth} p{0.6\textwidth}}
\toprule
\textbf{Req.} & \textbf{Modules}\\
\midrule
UHR-PIR1 & \mref{mHH}, \mref{mInput}, \mref{mParams}, \mref{mControl}\\
\bottomrule
\end{tabular}
\caption{Trace Between Personalization and Internationalization Requirements and Modules}
\label{TblPIRT}
\end{table}


\begin{table}[H]
\centering
\begin{tabular}{p{0.2\textwidth} p{0.6\textwidth}}
\toprule
\textbf{Req.} & \textbf{Modules}\\
\midrule
UHR-LR1 & \mref{mHH}, \mref{mInput}, \mref{mParams}, \mref{mControl}\\
\bottomrule
\end{tabular}
\caption{Trace Between Learning Requirements and Modules}
\label{TblPIRT}
\end{table}


\begin{table}[H]
\centering
\begin{tabular}{p{0.2\textwidth} p{0.6\textwidth}}
\toprule
\textbf{Req.} & \textbf{Modules}\\
\midrule
UHR-UPR1 & \mref{mHH}, \mref{mInput}, \mref{mParams}, \mref{mControl}\\
\bottomrule
\end{tabular}
\caption{Trace Between Understandability and Politeness Requirements and Modules}
\label{TblUPRT}
\end{table}



\begin{table}[H]
\centering
\begin{tabular}{p{0.2\textwidth} p{0.6\textwidth}}
\toprule
\textbf{Req.} & \textbf{Modules}\\
\midrule
UHR-AR1 & \mref{mHH}, \mref{mInput}, \mref{mParams}, \mref{mControl}\\
UHR-AR2 & \mref{mInput}, \mref{mParams}\\
\bottomrule
\end{tabular}
\caption{Trace Between Accessibility Requirements and Modules}
\label{TblART}
\end{table}


\begin{table}[H]
\centering
\begin{tabular}{p{0.2\textwidth} p{0.6\textwidth}}
\toprule
\textbf{Req.} & \textbf{Modules}\\
\midrule
PR-SL1 & \mref{mHH}, \mref{mInput}, \mref{mParams}, \mref{mControl}\\
PR-SL2 & \mref{mInput}, \mref{mParams}\\
PR-SCR1 & \mref{mVerify}\\
PR-SCR2 & \mref{mOutput}, \mref{mControl}\\
PR-SCR3 & \mref{mOutput}, \mref{mODEs}, \mref{mControl}, \mref{mSeqDS}, \mref{mSolver}, \mref{mPlot}\\
PR-SR-HA1 & \mref{mOutput}, \mref{mODEs}, \mref{mControl}, \mref{mSeqDS}, \mref{mSolver}, \mref{mPlot}\\
PR-SR-HA2 & \mref{mOutput}, \mref{mODEs}, \mref{mControl}, \mref{mSeqDS}, \mref{mSolver}, \mref{mPlot}\\
PR-SR-HA3 & \mref{mOutput}, \mref{mODEs}, \mref{mControl}, \mref{mSeqDS}, \mref{mSolver}, \mref{mPlot}\\
PR-PAR1 & \mref{mOutput}, \mref{mODEs}, \mref{mControl}, \mref{mSeqDS}, \mref{mSolver}, \mref{mPlot}\\
PR-PAR2 & \mref{mOutput}, \mref{mODEs}, \mref{mControl}, \mref{mSeqDS}, \mref{mSolver}, \mref{mPlot}\\
PR-PAR3 & \mref{mOutput}, \mref{mODEs}, \mref{mControl}, \mref{mSeqDS}, \mref{mSolver}, \mref{mPlot}\\
PR-RFT1 & \mref{mOutput}, \mref{mODEs}, \mref{mControl}, \mref{mSeqDS}, \mref{mSolver}, \mref{mPlot}\\
PR-RFT2 & \mref{mOutput}, \mref{mODEs}, \mref{mControl}, \mref{mSeqDS}, \mref{mSolver}, \mref{mPlot}\\
PR-CR1 & \mref{mOutput}, \mref{mODEs}, \mref{mControl}, \mref{mSeqDS}, \mref{mSolver}, \mref{mPlot}\\
PR-CR2 & \mref{mOutput}, \mref{mODEs}, \mref{mControl}, \mref{mSeqDS}, \mref{mSolver}, \mref{mPlot}\\
PR-SER1 & \mref{mOutput}, \mref{mODEs}, \mref{mControl}, \mref{mSeqDS}, \mref{mSolver}, \mref{mPlot}\\
PR-LR1 & \mref{mOutput}, \mref{mODEs}, \mref{mControl}, \mref{mSeqDS}, \mref{mSolver}, \mref{mPlot}\\
PR-LR2 & \mref{mOutput}, \mref{mODEs}, \mref{mControl}, \mref{mSeqDS}, \mref{mSolver}, \mref{mPlot}\\
\bottomrule
\end{tabular}
\caption{Trace Between Performance Requirements and Modules}
\label{TblFRT}
\end{table}


\begin{table}[H]
\centering
\begin{tabular}{p{0.2\textwidth} p{0.6\textwidth}}
\toprule
\textbf{Req.} & \textbf{Modules}\\
\midrule
OER-EP1 & \mref{mHH}, \mref{mInput}, \mref{mParams}, \mref{mControl}\\
OER-EP2 & \mref{mInput}, \mref{mParams}\\
OER-WE1 & \mref{mVerify}\\
OER-WE2 & \mref{mOutput}, \mref{mControl}\\
OER-IAS1 & \mref{mVerify}\\
OER-IAS2 & \mref{mOutput}, \mref{mControl}\\
OER-IAS3 & \mref{mOutput}, \mref{mControl}\\
OER-PR1 & \mref{mOutput}, \mref{mControl}\\
OER-RL1 & \mref{mOutput}, \mref{mControl}\\
OER-RL2 & \mref{mOutput}, \mref{mControl}\\
\bottomrule
\end{tabular}
\caption{Trace Between Operational and Environmental Requirements and Modules}
\label{TblOERT}
\end{table}


\begin{table}[H]
\centering
\begin{tabular}{p{0.2\textwidth} p{0.6\textwidth}}
\toprule
\textbf{Req.} & \textbf{Modules}\\
\midrule
MS-MNT1 & \mref{mHH}, \mref{mInput}, \mref{mParams}, \mref{mControl}\\
MS-MNT2 & \mref{mInput}, \mref{mParams}\\
MS-MNT3 & \mref{mVerify}\\
MS-SUP1 & \mref{mOutput}, \mref{mControl}\\
MS-AD1 & \mref{mVerify}\\
MS-AD2 & \mref{mOutput}, \mref{mControl}\\
MS-AD3 & \mref{mOutput}, \mref{mControl}\\
\bottomrule
\end{tabular}
\caption{Trace Between Maintainability and Support  Requirements and Modules}
\label{TblMSRT}
\end{table}



\begin{table}[H]
\centering
\begin{tabular}{p{0.2\textwidth} p{0.6\textwidth}}
\toprule
\textbf{AC} & \textbf{Modules}\\
\midrule
\acref{acHardware} & \mref{mHH}\\
\acref{acInput} & \mref{mInput}\\
\acref{acParams} & \mref{mParams}\\
\acref{acVerify} & \mref{mVerify}\\
\acref{acOutput} & \mref{mOutput}\\
\acref{acVerifyOut} & \mref{mVerifyOut}\\
\acref{acODEs} & \mref{mODEs}\\
\acref{acEnergy} & \mref{mEnergy}\\
\acref{acControl} & \mref{mControl}\\
\acref{acSeqDS} & \mref{mSeqDS}\\
\acref{acSolver} & \mref{mSolver}\\
\acref{acPlot} & \mref{mPlot}\\
\bottomrule
\end{tabular}
\caption{Trace Between Anticipated Changes and Modules}
\label{TblACT}
\end{table}

\section{Use Hierarchy Between Modules} \label{SecUse}

In this section, the uses hierarchy between modules is
provided. \citet{Parnas1978} said of two programs A and B that A {\em uses} B if
correct execution of B may be necessary for A to complete the task described in
its specification. That is, A {\em uses} B if there exist situations in which
the correct functioning of A depends upon the availability of a correct
implementation of B.  Figure \ref{FigUH} illustrates the use relation between
the modules. It can be seen that the graph is a directed acyclic graph
(DAG). Each level of the hierarchy offers a testable and usable subset of the
system, and modules in the higher level of the hierarchy are essentially simpler
because they use modules from the lower levels.

\wss{The uses relation is not a data flow diagram.  In the code there will often
be an import statement in module A when it directly uses module B.  Module B
provides the services that module A needs.  The code for module A needs to be
able to see these services (hence the import statement).  Since the uses
relation is transitive, there is a use relation without an import, but the
arrows in the diagram typically correspond to the presence of import statement.}

\wss{If module A uses module B, the arrow is directed from A to B.}

\begin{figure}[H]
\centering
%\includegraphics[width=0.7\textwidth]{UsesHierarchy.png}
\caption{Use hierarchy among modules}
\label{FigUH}
\end{figure}

%\section*{References}

\section{User Interfaces}

\wss{Design of user interface for software and hardware.  Attach an appendix if
needed. Drawings, Sketches, Figma}

\section{Design of Communication Protocols}

\wss{If appropriate}

\section{Timeline}

\wss{Schedule of tasks and who is responsible}

\wss{You can point to GitHub if this information is included there}

\bibliographystyle {plainnat}
\bibliography{../../../refs/References}

\newpage{}

\end{document}