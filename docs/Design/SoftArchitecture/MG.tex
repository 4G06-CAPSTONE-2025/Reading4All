\documentclass[12pt, titlepage]{article}

% ===============================
% Math packages (DO NOT REMOVE)
% ===============================
\usepackage{amsmath}
\usepackage{amssymb}
\usepackage{amsfonts}
\usepackage{bm}
\usepackage{mathtools}
% ===============================

\usepackage{fullpage}
\usepackage[round]{natbib}
\usepackage{multirow}
\usepackage{booktabs}
\usepackage{tabularx}
\usepackage{graphicx}
\usepackage{float}
\usepackage{hyperref}
\usepackage{enumitem}
\hypersetup{
  colorlinks,
  citecolor=blue,
  filecolor=black,
  linkcolor=red,
  urlcolor=blue
}

\input{../../Comments}
%% Common Parts

\newcommand{\progname}{ProgName} % PUT YOUR PROGRAM NAME HERE
\newcommand{\authname}{Team 22, READING4ALL
\\ Fiza Sehar
\\ Nawaal Fatima
\\ Dhruv Sardana
\\ Moly Mikhail
\\ Casey Francine Bulaclac } % AUTHOR NAMES                  

\usepackage{hyperref}
    \hypersetup{colorlinks=true, linkcolor=blue, citecolor=blue, filecolor=blue,
                urlcolor=blue, unicode=false}
    \urlstyle{same}
                                


\newcounter{acnum}
\newcommand{\actheacnum}{AC\theacnum}
\newcommand{\acref}[1]{AC\ref{#1}}

\newcounter{ucnum}
\newcommand{\uctheucnum}{UC\theucnum}
\newcommand{\uref}[1]{UC\ref{#1}}

\newcounter{mnum}
\newcommand{\mthemnum}{M\themnum}
\newcommand{\mref}[1]{M\ref{#1}}

\begin{document}

\title{Module Guide for \progname{}}
\author{\authname}
\date{\today}

\maketitle

\pagenumbering{roman}

\section{Revision History}

\begin{tabularx}{\textwidth}{p{4cm}p{2cm}X}
  \toprule {\bf Date} & {\bf Version} & {\bf Notes}\\
  \midrule
  November 13 2025 & 1.0 & Rev-1 Design Document\\
  January 21 2026 & 2.0 & Rev-0 Design Document\\
  \bottomrule
\end{tabularx}

\newpage

\section{Reference Material}

This section records information for easy reference.

\subsection{Abbreviations and Acronyms}

\renewcommand{\arraystretch}{1.2}
\begin{tabular}{l l}
  \toprule
  \textbf{symbol} & \textbf{description}\\
  \midrule
  AC & Anticipated Change\\
  DAG & Directed Acyclic Graph \\
  M & Module \\
  MG & Module Guide \\
  OS & Operating System \\
  R & Requirement\\
  SC & Scientific Computing \\
  SRS & Software Requirements Specification\\
  \progname & Explanation of program name\\
  UC & Unlikely Change \\
  WCAG 2.1 &  Web Content Accessibility Guidelines\\
  AODA & Accessibility for Ontarians with Disabilities Act \\
  UI & User Interface \\
  DOM & Document Object Model \\
  AI & Artificial Intelligence \\
  ML & Machine Learning \\
  ARIA & Accessible Rich Internet Applications \\
  \bottomrule
\end{tabular}\\

\newpage

\tableofcontents

\listoftables

\listoffigures

\newpage

\pagenumbering{arabic}

\section{Introduction}

Decomposing a system into modules is a commonly accepted approach to developing
software.  A module is a work assignment for a programmer or programming
team~\citep{ParnasEtAl1984}.  We advocate a decomposition
based on the principle of information hiding~\citep{Parnas1972a}.  This
principle supports design for change, because the ``secrets'' that each module
hides represent likely future changes.  Design for change is valuable in SC,
where modifications are frequent, especially during initial development as the
solution space is explored.

Our design follows the rules layed out by \citet{ParnasEtAl1984}, as follows:
\begin{itemize}
  \item System details that are likely to change independently should be the
    secrets of separate modules.
  \item Each data structure is implemented in only one module.
  \item Any other program that requires information stored in a module's data
    structures must obtain it by calling access programs belonging to
    that module.
\end{itemize}

After completing the first stage of the design, the Software Requirements
Specification (SRS), the Module Guide (MG) is
developed~\citep{ParnasEtAl1984}. The MG
specifies the modular structure of the system and is intended to allow both
designers and maintainers to easily identify the parts of the software.  The
potential readers of this document are as follows:

\begin{itemize}
  \item New project members: This document can be a guide for a new
    project member
    to easily understand the overall structure and quickly find the
    relevant modules they are searching for.
  \item Maintainers: The hierarchical structure of the module guide improves the
    maintainers' understanding when they need to make changes to the
    system. It is
    important for a maintainer to update the relevant sections of the document
    after changes have been made.
  \item Designers: Once the module guide has been written, it can be used to
    check for consistency, feasibility, and flexibility. Designers
    can verify the
    system in various ways, such as consistency among modules,
    feasibility of the
    decomposition, and flexibility of the design.
\end{itemize}

The rest of the document is organized as follows. Section
\ref{SecChange} lists the anticipated and unlikely changes of the software
requirements. Section \ref{SecMH} summarizes the module decomposition that
was constructed according to the likely changes. Section \ref{SecConnection}
specifies the connections between the software requirements and the
modules. Section \ref{SecMD} gives a detailed description of the
modules. Section \ref{SecTM} includes two traceability matrices. One checks
the completeness of the design against the requirements provided in the SRS. The
other shows the relation between anticipated changes and the modules. Section
\ref{SecUse} describes the use relation between modules.

\section{Anticipated and Unlikely Changes} \label{SecChange}

This section lists possible changes to the system. According to the likeliness
of the change, the possible changes are classified into two
categories. Anticipated changes are listed in Section \ref{SecAchange}, and
unlikely changes are listed in Section \ref{SecUchange}.

\subsection{Anticipated Changes} \label{SecAchange}

Anticipated changes are the source of the information that is to be hidden
inside the modules. Ideally, changing one of the anticipated changes will only
require changing the one module that hides the associated decision. The approach
adapted here is called design for change. The traceability matrix between the Anticipated
(Likely) Changes and Modules is highlighted in table~\ref{TblACT}. \texttt{AC7} is mapped
to multiple modules as each of these modules encapsulates the frontend subsystem and 
this anticipated change affects the entire frontend subsystem. 

\begin{description}

  \item[\refstepcounter{acnum} \actheacnum \label{acInfra}:] The specific
    infrastructure on which the software is running.
  \item[\refstepcounter{acnum} \actheacnum \label{acInputFormat}:]
    The format of the
    initial input data accepted by the system; in the future the
    system might extend to support PDFs and other formats.
  \item [\refstepcounter{acnum} \actheacnum \label{acMLModel}:] The
    machine learning model used for alt-text generation.
  \item [\refstepcounter{acnum} \actheacnum \label{acLang}:] The
    language used for alt-text generation, allowing the system to
    generate alt-text in multiple languages beyond English.
  \item [\refstepcounter{acnum} \actheacnum \label{acSession}:] The
    duration of time the user remains authenticated before being
    prompted to log in again.
  \item [\refstepcounter{acnum} \actheacnum \label{acInputTopic}:]
    The range of topic domains supported for alt-text generation.
  \item [\refstepcounter{acnum} \actheacnum \label{acPlatform}:] The
    platform that the system operates on can be extended beyond a web
    application to also include desktop or mobile applications.

\end{description}

\subsection{Unlikely Changes} \label{SecUchange}

The module design should be as general as possible. However, a general system is
more complex. Sometimes this complexity is not necessary. Fixing some design
decisions at the system architecture stage can simplify the software design. If
these decision should later need to be changed, then many parts of the design
will potentially need to be modified. Hence, it is not intended that these
decisions will be changed.\\

\noindent The following are changes that the system is unlikely to incur:
\begin{description}
  \item[\refstepcounter{ucnum} \uctheucnum \label{ucIO}:] The
    project's evaluation metrics as shown in Table 5 of the
    Appendix in our \citet{SRS} document will not change as these
    metrics are concrete and
    will be used to evaluate the effectiveness of the generated
    alternative text.
  \item [\refstepcounter{ucnum} \uctheucnum \label{ucIO}:] The
    primary output device for this system will remain as a screen,
    even if the system is later expanded to additional platforms
    (i.e., mobile) and not just the current web-based tool.
  \item [\refstepcounter{ucnum} \uctheucnum \label{ucIO}:] The system
    will always support keyboard input as its primary
    interaction method, with mouse input also supported for selecting
    user interface elements. This is to ensure that
    the functionality of keyboard navigation is always enabled for
    accessibility.

\end{description}

\section{Module Hierarchy} \label{SecMH}

This section provides an overview of the module design. Modules are summarized
in a hierarchy decomposed by secrets in Table \ref{TblMH}. The
modules listed in Table \ref{TblMHS}
below, which are leaves in the hierarchy tree, are the modules that will
actually be implemented.

\begin{table}[H]
  \centering
  \begin{tabular}{|p{4cm}|p{4cm}|p{4cm}| p{4cm}|}
    \hline
    \textbf{AI Model Modules} & \textbf{Backend Modules} &
    \textbf{Frontend Modules} & \textbf{System Information Modules}\\ \hline

    DataPreprocess Module & AuthenticationService Module &
    UserInterfaceInteractions Module & Logging Module\\ \hline
    WCAGCompliance Module & SessionManagement Module & MainScreen
    Module & ~\\ \hline
    ModelOutput Module & ImageValidation Module & ShowHistoryScreen
    Module & ~\\ \hline
    AIModelTraining Module & BackendController Module & LogInScreen
    Module & ~\\ \hline
    CaptionGeneration Module & ~ & UserInterfaceAccessibility Module
    & ~\\ \hline
    InternalMetricCompliance Module & ~ & ~ & ~\\ \hline
    FeedbackMetrics Module & ~ & ~ & ~\\ \hline
    FeedbackLoop Module & ~ & ~ & ~\\ \hline
    Feedback Module & ~ & ~ & ~\\ \hline

  \end{tabular}
  \caption{Module Hierarchy by Subsystem}
  \label{TblMHS}
\end{table}

\begin{table}[h!]
  \centering
  \begin{tabular}{p{0.3\textwidth} p{0.6\textwidth}}
    \toprule
    \textbf{Level 1} & \textbf{Level 2}\\
    \midrule

    {Hardware-Hiding Module [Section (\ref{hhmmain})]} & N/A \\
    \midrule

    \multirow{13}{0.3\textwidth}{Behaviour-Hiding Module [Section
    (\ref{bhmmain})]} &
    {DataPreprocess Module (\refstepcounter{mnum}\mthemnum
    \label{mod:preprocessing_table}) (see \ref{mod:preprocessing})}\\ &
    {WCAGCompliance Module (\refstepcounter{mnum}\mthemnum
    \label{mod:wcagcompliance_table}) (see \ref{mod:wcagcompliance})}\\ &
    {ModelOutput Module (\refstepcounter{mnum}\mthemnum
    \label{mod:modeloutput_table}) (see \ref{mod:modeloutput})}\\ &
    {AuthenticationService Module (\refstepcounter{mnum}\mthemnum
    \label{mod:auth_table} )(see \ref{mod:auth})}\\ &
    {SessionManagement Module (\refstepcounter{mnum}\mthemnum
    \label{mod:mess_table}) (see \ref{mod: mess})}\\ &
    {BackendController Module (\refstepcounter{mnum}\mthemnum
    \label{mod:bcm_table}) (see \ref{mod:bcm})}\\ &
    {Logging Module (\refstepcounter{mnum}\mthemnum
    \label{mod:logmod_table}) (see \ref{mod:logmod})}\\ &
    {ImageValidation Module (\refstepcounter{mnum}\mthemnum
    \label{mod:imgvalidmod_table}) (see \ref{mod:imgvalidmod})}\\ &
    {UserInterfaceInteractions Module
      (\refstepcounter{mnum}\mthemnum) \label{mod:uii_table} (see
    \ref{mod: uii})} \\ &
    {MainScreen Module (\refstepcounter{mnum}\mthemnum)
    \label{mod:msm_table} (see \ref{mod: msm})} \\ &
    {ShowHistoryScreen Module (\refstepcounter{mnum}\mthemnum)
    \label{mod:showhm_table} (see \ref{mod: showhm})} \\ &
    {LogInScreen Module (\refstepcounter{mnum}\mthemnum)
    \label{mod:lis_table} (see \ref{mod: lis})} \\ &
    {UserInterfaceAccessibility Module
      (\refstepcounter{mnum}\mthemnum) \label{mod:uia_table} (see
    \ref{mod: uia})} \\
    \midrule

    \multirow{6}{0.3\textwidth}{Software Decision Module [Section
    (\ref{mod:sdmmain})]} &
    {AIModelTraining Module (\refstepcounter{mnum}\mthemnum
    \label{mod:modeltraining_table}) (see \ref{mod:modeltraining})}\\ &
    {CaptionGeneration Module (\refstepcounter{mnum}\mthemnum
    \label{mod:captiongeneration_table}) (see \ref{mod:captiongeneration})}\\ &
    {InternalMetricCompliance (\refstepcounter{mnum}\mthemnum
      \label{mod:internalmetriccompliance_table}) (see
    \ref{mod:internalmetriccompliance})}\\ &
    {FeedbackMetrics Module (\refstepcounter{mnum}\mthemnum
    \label{mod:feedbackmetrics_table} )(see \ref{mod:feedbackmetrics})}\\ &
    {FeedbackLoop Module (\refstepcounter{mnum}\mthemnum
    \label{mod:feedbackloop_table}) (see \ref{mod:feedbackloop})}\\ &
    {Feedback Module (\refstepcounter{mnum}\mthemnum
    \label{mod:feedback_table}) (see \ref{mod:feedback})}\\
    \bottomrule
  \end{tabular}
  \caption{Module Hierarchy}
  \label{TblMH}
\end{table}

\section{Connection Between Requirements and Design} \label{SecConnection}

The design of the system is intended to satisfy the requirements developed in
the SRS. In this stage, the system is decomposed into modules. The connection
between requirements and modules is listed in Tables
\ref{TblFRT}-\ref{TblACT} in
section \ref{SecTM}.\\

The decompositon of the modules was separated by four main subsystems
including: Frontend Modules,
Backend Modules, AI Model Modules, and System Information Modules. \\

The front-end modules of the \textit{Reading4All} system collectively ensure that the system meets all Look and Feel, Usability, 
Humanity, and Accessibility requirements through a responsive design and 
and clear interaction patterns. The \texttt{MainScreen}, \texttt{LogInScreen}, and \texttt{ShowHistoryScreen} modules that are responsible for rendering 
the different screens of the user interface will use simple and clean layouts, 
WCAG-compliant contrast ratios, non-colour-dependent cues, and fully descriptive alt text for non-text elements. These display modules present 
only essential information and will remain intuitive to navigate. The \texttt{UserInterfaceInteractions} module will
ensure usability by providing immediate textual feedback and ensure to support efficient task 
completion with minimal steps. Lastly, within the frontend modules, the \texttt{UserInterfaceAccessibility} module will ensure compatibility with screen readers and keyboard navigation by 
managing ARIA landmarks, live-region announcements, and predictable focus order. Together, these modules will deliver an interface that is simple, memorable, 
inclusive, and fully aligned with WCAG 2.1 Level AA and AODA standards. It will allow all users to operate the system effectively and confidently.\\

In the backend subsystem, many modules play a role in ensuring that the functional,
performance, privacy and operational requirements specified are fulfilled. 
The \texttt{BackendController Module} coordinates the flow of data between the Frontend, AI modules and other Backend modules. It ensures that user's and their inputs are validated, and alternative text is generated by invoking the appropriate modules. 
The \texttt{BackendController} also ensures that users are presented with the appropriate information and never exposed to technical details. 
The \texttt{ImageValidation} module fulfills the requirements relating to validating user inputs. This module checks that inputted images are of the correct size and types before the alt text generation process begins. 
The \texttt{Logging} module records any system events and errors, helping fulfill requirements related to protecting user data and having a reliable system. The \texttt{Logging} module will help ensure that user data is always deleted when any system failure occurs. \\

Furthermore, within the backend subsystem, the \texttt{AuthenticationService} and \\\texttt{SessionManagement} modules together ensure secure and reliable user interaction with the Reading4All platform. 
The \texttt{AuthenticationService} module is responsible for verifying user credentials, issuing secure tokens, and validating user identity before granting access to system functionalities. 
It guarantees that only authenticated users can upload images or retrieve generated alt text, fulfilling the security and privacy requirements defined in the SRS. 
Supporting this, the \texttt{SessionManagement} module manages each user’s active session lifecycle by creating, validating, refreshing, and terminating session tokens. 
It also maintains a short-term record of user interactions during the active session, such as image uploads and model outputs, ensuring that users can view their recent activity while preserving system integrity and confidentiality. 
Together, these two backend modules ensure that all authentication, authorization, and session tracking processes are handled transparently and securely, thereby supporting reliability, data protection, and user trust across the Reading4All system.
\section{Module Decomposition} \label{SecMD}

Modules are decomposed according to the principle of ``information hiding''
proposed by \citet{ParnasEtAl1984}. The \emph{Secrets} field in a module
decomposition is a brief statement of the design decision hidden by the
module. The \emph{Services} field specifies \emph{what} the module will do
without documenting \emph{how} to do it. For each module, a suggestion for the
implementing software is given under the \emph{Implemented By} title. If the
entry is \emph{OS}, this means that the module is provided by the operating
system or by standard programming language libraries.
\emph{\progname{}} means the
module will be implemented by the \progname{} software.

Only the leaf modules in the hierarchy have to be implemented. If a dash
(\emph{--}) is shown, this means that the module is not a leaf and will not have
to be implemented.

\subsection{Hardware Hiding Modules \label{hhmmain}}
This subsection is N/A as the Reading4All system has no hardware components.

\subsection{Behaviour-Hiding Modules \label{bhmmain}}
This subsection includes modules that pertain to the required
behavior of the Reading4All system.
The modules are defined as their own subsection and follow the following format:

\begin{description}
  \item[Secrets:]This part describes the essential behavioral
    components that define a module's intended functionality.
  \item[Services:]This component defines the module's programs that
    implement the
    externally visible behaviors of the system, as outlined in the
    Software Requirements Specification (SRS). Any modifications to the
    SRS will likely require corresponding updates to the programs
    within this module.
  \item[Implemented By:] This part mentions if the Reading4All system
    or external systems will
    implement the specific module.
  \item[Type of Module:] This section defines the module type into a
    Record, Library, Abstract
    Object, or Abstract Data Type. This section also includes
    information for leaf modules in the decomposition by secrets tree.
\end{description}

\subsubsection{DataPreprocess Module (\mref{mod:preprocessing_table})}
\label{mod:preprocessing}
\begin{description}
  \item[Secrets:] Includes the steps used to prepare input images for
    further processing.
  \item[Services:] Normalizes, sorts, and standardizes the image
    input so it can be correctly used by the training and caption
    generation modules.
  \item[Implemented By:] Reading4All
  \item[Type of Module:] Library
\end{description}

\subsubsection{WCAGCompliance Module (\mref{mod:wcagcompliance_table})}
\label{mod:wcagcompliance}

\begin{description}
  \item[Secrets:] The rules used to decide whether a caption meets
    AODA (\cite{AODA}) and WCAG 2.1 AA accessibility standards (\citet{WCAG}).
  \item[Services:] Checks the generated text against WCAG 2.1 AA (\citet{WCAG})
    criteria and flags or adjusts any generated alt text that does
    not meet these standards.
  \item[Implemented By:] Reading4All
  \item[Type of Module:] Library
\end{description}

\subsubsection{ModelOutput Module (\mref{mod:modeloutput_table})}
\label{mod:modeloutput}

\begin{description}
  \item[Secrets:] The structure used to store and present the final
    text to the user via the BackendController as mentioned in
    Section \ref{mod:bcm}.
  \item[Services:] Produces and formats the final version of the
    accessibility-compliant text for output or download.
  \item[Implemented By:] Reading4All
  \item[Type of Module:] Record
\end{description}

\subsubsection{AuthenticationService Module (\mref{mod:auth_table})}
\label{mod:auth}
\begin{description}
  \item[Secrets:] The internal logic for credential verification,
    token structure and signing, token refresh rules, lockout
    thresholds, and interactions with external authentication providers.
    Includes error classification for authentication failures and
    audit-safe handling of sensitive data.
  \item[Services:] Verifies user credentials, issues and refreshes
    authentication tokens, validates tokens on incoming requests,
    resolves current user identity (userID/role) for access control.
    This module is used to sign up new users, log in existing users. 
    This is also used to protect user data and ensure only necessary sign up information
    is collected and stored. 
  \item[Implemented By:] Reading4All
  \item[Type of Module:] Library
\end{description}

\subsubsection{UserInterfaceInteractions Module (\mref{mod:uii_table})}
\label{mod: uii}
\begin{description}
  \item[Secrets:] The internal logic and mapping of user-triggered
    actions such as clicks and keyboard presses
    on the user interface into the backend and display screens modules.
  \item[Services:] Captures user actions from the UI and connects UI
    controls to application
    behavior by attaching event handlers that calls on other modules’
    access programs with
    validated parameters.
  \item[Implemented By:] Reading4All Team
  \item[Type of Module:] Library
\end{description}

\subsubsection{MainScreen Module (\mref{mod:msm_table})}
\label{mod: msm}
\begin{description}
  \item[Secrets:] The internal layout composition of the main screen.
    This includes how the upload panel and supporting UI elements are
    arranged and rendered into the \texttt{DOM}.
  \item[Services:] Renders the “Main” screen of the application. This
    screen displays the image-upload entry point for users.
    It calls on the UserInterfaceAccessibility module
    to assign the main landmark, set initial focus, and announce the
    screen to screen readers.
  \item[Implemented By:] Reading4All Team
  \item[Type of Module:] Library
\end{description}

\subsubsection{ShowHistoryScreen Module (\mref{mod:showhm_table})}
\label{mod: showhm}
\begin{description}
  \item[Secrets:] The internal layout composition of the show history
    screen when an authenticated user wants to see their
    previously generated alt-text history. This includes how the
    history items are formatted and displayed.
  \item[Services:] Renders the user’s current session history to the
    screen. It relies on the UserInterfaceAccessibility module
    to set the main landmark, focus the first history item (if any),
    and announce the screen.
  \item[Implemented By:] Reading4All Team
  \item[Type of Module:] Library
\end{description}

\subsubsection{LogInScreen Module (\mref{mod:lis_table})}
\label{mod: lis}
\begin{description}
  \item[Secrets:] The internal layout composition of the log in
    screen including username/email and password fields, and
    a log in button.
  \item[Services:] Renders the system’s Log In screen. It calls on
    the UserInterfaceAccessibility module
    to set focus on the fields on the log in form and announce the screen.
  \item[Implemented By:] Reading4All Team
  \item[Type of Module:] Library
\end{description}

\subsubsection{UserInterfaceAccessibility Module (\mref{mod:uia_table})}
\label{mod: uia}
\begin{description}
  \item[Secrets:] The internal mechanism for providing screen reader
    and keyboard navigation compatibility in the browser.
    This includes how ARIA (\citet{ARIA}) live regions are
    implemented, how focus is manipulated, and how keyboard
    navigation is enabled
    internally.
  \item[Services:] Provides accessibility capabilities to other user
    interface modules by: assigning and maintaining
    the correct main landmark on each screen; managing which element
    receives initial keyboard focus; sending
    announcements to screen readers using ARIA (\citet{ARIA}) live
    regions; and enabling keyboard navigation patterns.
  \item[Implemented By:] Reading4All Team
  \item[Type of Module:] Library
\end{description}
\subsubsection{SessionManagement Module}
\label{mod: mess}
\begin{description}
  \item[Secrets:] The representation and storage of active sessions,
    expiry/refresh policies, session-history data structures
    (interaction entries and Reading4All data references), and
    eviction/cleanup strategies. Includes indexing strategies for
    fast token lookups and privacy rules for session-scoped logs.
  \item[Services:] Creates, validates, refreshes, and deletes
    sessions; appends interaction entries; links session activity to
    Reading4All data records; retrieves filtered session history and
    recent data references for the current session.
  \item[Implemented By:] Reading4All
  \item[Type of Module:] Abstract Data Type
\end{description}

\subsubsection{BackendController Module (\mref{mod:bcm_table})}
\label{mod:bcm}
\begin{description}
  \item[Secrets:] The internal application flow of data and user
    inputs, including how backend requests are processed and routed
    between services.
  \item[Services:]Coordinates backend responses by receiving requests
    from the frontend and directing it to the appropriate service
    such as the User Authentication or  AI Model, and returning the results.
  \item[Implemented By:] Reading4All
  \item[Type of Module:] Library
\end{description}

\subsubsection{Logging Module (\mref{mod:logmod_table})}
\label{mod:logmod}
\begin{description}
  \item[Secrets:] The internal details of which events and errors are
    logged and the specific format used for log entries.
  \item[Services:] Records system events and errors received from the
    BackendController Module.
  \item[Implemented By:] Reading4All
  \item[Type of Module:] Library
\end{description}

\subsubsection{ImageValidation Module (\mref{mod:imgvalidmod_table})}
\label{mod:imgvalidmod}
\begin{description}
  \item[Secrets:] The internal details of how uploaded images are
    verified, including how file type/size are checked and whether a
    file is reachable.
  \item[Services:] Validates uploaded images to ensure that they meet
    system requirements before the alt text generation process.
  \item[Implemented By:] Reading4All
  \item[Type of Module:] Library
\end{description}

\subsection{Software Decision Modules}
\label{mod:sdmmain}
This subsection includes the modules' internal operations,
connections with other modules and decision making components.
Similar to Section \ref{bhmmain} the modules are defined as their own
subsection and follow the following format:

\begin{description}
  \item[Secrets:] Internal logic, data flow, and processing steps
    that define how the system produces and improves alternative text.
  \item[Services:] Implements the internal operations of the system,
    such as model training, caption generation, quality validation,
    feedback collection, and output production.
  \item[Implemented By:] This part mentions if the Reading4All system
    or external systems will
    implement the specific module.
  \item[Type of Module:] This section defines the module type into a
    Record, Library, Abstract
    Object, or Abstract Data Type. This section also includes
    information for leaf modules in the decomposition by secrets tree.
\end{description}

\subsubsection{AIModelTraining Module (\mref{mod:modeltraining_table})}
\label{mod:modeltraining}
\begin{description}
  \item[Secrets:] The internal design of how the model is built and
    improved. This includes how image and text data are linked, how
    training is divided into sets, and what hidden patterns the model
    learns. The specific model structure and training process are not
    exposed to other modules.
  \item[Services:] Trains the system to understand relationships
    between images and text so it can later describe new images
    accurately. Produces a trained model and supporting data (such as
    validation results) that are used by the CaptionGeneration module.
  \item[Implemented By:] Reading4All
  \item[Type of Module:] Abstract Object
\end{description}

\subsubsection{CaptionGeneration Module (\mref{mod:captiongeneration_table})}
\label{mod:captiongeneration}
\begin{description}
  \item[Secrets:] The process the system uses to generate descriptive
    text from an image. This includes how image features are
    interpreted and how words are selected and arranged into sentences.
  \item[Services:] Uses the trained model from the AIModelTraining
    module to produce alternative text
    for a given image. Passes the generated caption to the WCAG
    (\citet{WCAG}) and
    InternalMetricCompliance modules to ensure it meets accessibility
    and clarity standards.
  \item[Implemented By:] Reading4All
  \item[Type of Module:] Library
\end{description}

\subsubsection{InternalMetricCompliance Module
(\mref{mod:internalmetriccompliance_table})}
\label{mod:internalmetriccompliance}
\begin{description}
  \item[Secrets:] The internal measures chosen by Team 22 to judge
    how clear, concise or relevant the generated text is.
  \item[Services:] Reviews each caption using internal quality checks
    that complement the AODA (\citet{AODA}) and WCAG 2.1 AA
    (\citet{WCAG}) validation standards.
  \item[Implemented By:] Reading4All
  \item[Type of Module:] Library
\end{description}

\subsubsection{FeedbackMetrics Module (\mref{mod:feedbackmetrics_table})}
\label{mod:feedbackmetrics}
\begin{description}
  \item[Secrets:] How user feedback is collected, interpreted, and
    organized for later use in model improvement. The exact method
    for analyzing or weighting different types of feedback is hidden.
  \item[Services:] Receives direct input from users in the form of
    comments, ratings, or corrections about the generated alternative
    text. Summarizes this information into a structured format that
    can be used to guide model retraining and future system updates.
  \item[Implemented By:] Reading4All
  \item[Type of Module:] Library
\end{description}

\subsubsection{FeedbackLoop Module (\mref{mod:feedbackloop_table})}
\label{mod:feedbackloop}
\begin{description}
  \item[Secrets:] The rules and internal logic that determine how
    user feedback is used to update and improve the system’s model.
    The specific approach to retraining or fine-tuning is hidden.
  \item[Services:] Takes processed feedback data from the
    FeedbackMetrics module and uses it to retrain or adjust the model
    over time. This allows the system to gradually improve its
    accuracy and produce captions that better reflect user expectations.
  \item[Implemented By:] Reading4All
  \item[Type of Module:] Abstract Data Type
\end{description}

\subsubsection{Feedback Module (\mref{mod:feedback_table})}
\label{mod:feedback}
\begin{description}
  \item[Secrets:] How feedback information is stored and organized.
  \item[Services:] Stores the user’s feedback entries and makes them
    available to the FeedbackMetrics and FeedbackLoop modules for
    review and improvement purposes.
  \item[Implemented By:] Reading4All
  \item[Type of Module:] Record
\end{description}

\section{Traceability Matrix} \label{SecTM}

This section shows below how different requirements connect to our
modules. This can be seen in Table ~\ref{TblFRT},
Table ~\ref{TblLFRT}, Table ~\ref{TblLFRT}, Table ~\ref{TblSRT},
Table ~\ref{TblURT}, Table ~\ref{TblPIRT}, Table ~\ref{TblLRMT},
Table ~\ref{TblUPRT}, Table ~\ref{TblART}, Table ~\ref{TblPRT}, Table
~\ref{TblOERT}, Table ~\ref{TblMSRT}, Table ~\ref{TblSRT1},
Table ~\ref{TblCR}, Table ~\ref{TbltCompR}.

Finally, Table ~\ref{TblACT} shows how our modules connect to the
anticipated changes
two traceability matrices: between the modules and the
requirements and between the modules and the anticipated changes.

% the table should use mref, the requirements should be named, use something
% like fref
\begin{table}[H]
  \centering
  \begin{tabular}{p{0.2\textwidth} p{0.6\textwidth}}
    \toprule
    \textbf{Req.} & \textbf{Modules}\\
    \midrule
    FR1 & \mref{mod:preprocessing_table},  \mref{mod:bcm_table},
    \mref{mod:imgvalidmod_table}\\
    FR2 & \mref{mod:bcm_table}, \mref{mod:captiongeneration_table},
    \mref{mod:feedbackmetrics_table}, \mref{mod:feedbackloop_table},
    \mref{mod:feedback_table}\\
    FR3 & \mref{mod:wcagcompliance_table},
    \mref{mod:modeloutput_table}, \mref{mod:bcm_table}, \mref{mod:uia_table}\\
    FR4 & \mref{mod:bcm_table}\\
    FR5 &  \mref{mod:mess_table}, \mref{mod:bcm_table},
    \mref{mod:showhm_table}\\
    FR6 &  \mref{mod:auth_table}, \mref{mod:bcm_table}\\
    \bottomrule
  \end{tabular}
  \caption{Trace Between Functional Requirements and Modules}
  \label{TblFRT}
\end{table}

\begin{table}[H]
  \centering
  \begin{tabular}{p{0.2\textwidth} p{0.6\textwidth}}
    \toprule
    \textbf{Req.} & \textbf{Modules}\\
    \midrule
    LFR-AR1 & \mref{mod:uii_table}, \mref{mod:msm_table},
    \mref{mod:showhm_table}, \mref{mod:lis_table}, \mref{mod:uia_table}\\
    LFR-AR2 & \mref{mod:msm_table}, \mref{mod:uii_table},
    \mref{mod:showhm_table}, \mref{mod:lis_table}, \mref{mod:uia_table}\\
    LFR-AR3 & \mref{mod:msm_table}, \mref{mod:uii_table},
    \mref{mod:showhm_table}, \mref{mod:lis_table}, \mref{mod:uia_table}\\
    LFR-AR4 & \mref{mod:internalmetriccompliance_table},
    \mref{mod:msm_table}, \mref{mod:uii_table},
    \mref{mod:showhm_table}, \mref{mod:lis_table}. \mref{mod:uia_table}\\
    \bottomrule
  \end{tabular}
  \caption{Trace Between Look and Feel Requirements and Modules}
  \label{TblLFRT}
\end{table}

\begin{table}[H]
  \centering
  \begin{tabular}{p{0.2\textwidth} p{0.6\textwidth}}
    \toprule
    \textbf{Req.} & \textbf{Modules}\\
    \midrule
    LFR-SR1 & \mref{mod:uii_table}, \mref{mod:msm_table},
    \mref{mod:showhm_table}, \mref{mod:lis_table}, \mref{mod:uia_table}\\
    LFR-SR2 & \mref{mod:uii_table}, \mref{mod:msm_table},
    \mref{mod:showhm_table}, \mref{mod:lis_table}, \mref{mod:uia_table}\\
    LFR-SR3 & \mref{mod:uii_table}, \mref{mod:msm_table},
    \mref{mod:showhm_table}, \mref{mod:lis_table}, \mref{mod:uia_table}\\
    \bottomrule
  \end{tabular}
  \caption{Trace Between Style Requirements and Modules}
  \label{TblSRT}
\end{table}

\begin{table}[H]
  \centering
  \begin{tabular}{p{0.2\textwidth} p{0.6\textwidth}}
    \toprule
    \textbf{Req.} & \textbf{Modules}\\
    \midrule
    UHR-EUR1 & \mref{mod:wcagcompliance_table},
    \mref{mod:modeloutput_table}, \mref{mod:uii_table},
    \mref{mod:msm_table}, \mref{mod:showhm_table},
    \mref{mod:lis_table}. \mref{mod:uia_table}\\
    UHR-EUR2 & \mref{mod:uii_table}, \mref{mod:msm_table},
    \mref{mod:showhm_table}, \mref{mod:lis_table}, \mref{mod:uia_table}\\
    UHR-EUR3 & \mref{mod:uii_table}, \mref{mod:msm_table},
    \mref{mod:showhm_table}, \mref{mod:lis_table}, \mref{mod:uia_table}\\
    UHR-EUR4 & \mref{mod:uii_table}, \mref{mod:msm_table},
    \mref{mod:showhm_table}, \mref{mod:lis_table}, \mref{mod:uia_table}\\
    \bottomrule
  \end{tabular}
  \caption{Trace Between Usability and Humanity Requirements and Modules}
  \label{TblURT}
\end{table}

\begin{table}[H]
  \centering
  \begin{tabular}{p{0.2\textwidth} p{0.6\textwidth}}
    \toprule
    \textbf{Req.} & \textbf{Modules}\\
    \midrule
    UHR-PIR1 & \mref{mod:uii_table}, \mref{mod:msm_table},
    \mref{mod:uia_table}\\
    \bottomrule
  \end{tabular}
  \caption{Trace Between Personalization and Internationalization
  Requirements and Modules}
  \label{TblPIRT}
\end{table}

\begin{table}[H]
  \centering
  \begin{tabular}{p{0.2\textwidth} p{0.6\textwidth}}
    \toprule
    \textbf{Req.} & \textbf{Modules}\\
    \midrule
    UHR-LR1 & \mref{mod:uii_table}, \mref{mod:msm_table},
    \mref{mod:showhm_table}, \mref{mod:lis_table}, \mref{mod:uia_table}\\
    \bottomrule
  \end{tabular}
  \caption{Trace Between Learning Requirements and Modules}
  \label{TblLRMT}
\end{table}

\begin{table}[H]
  \centering
  \begin{tabular}{p{0.2\textwidth} p{0.6\textwidth}}
    \toprule
    \textbf{Req.} & \textbf{Modules}\\
    \midrule
    UHR-UPR1 & \mref{mod:uii_table}, \mref{mod:msm_table},
    \mref{mod:showhm_table}, \mref{mod:lis_table}, \mref{mod:uia_table}\\
    \bottomrule
  \end{tabular}
  \caption{Trace Between Understandability and Politeness
  Requirements and Modules}
  \label{TblUPRT}
\end{table}

\begin{table}[H]
  \centering
  \begin{tabular}{p{0.2\textwidth} p{0.6\textwidth}}
    \toprule
    \textbf{Req.} & \textbf{Modules}\\
    \midrule
    UHR-AR1 & \mref{mod:uii_table}, \mref{mod:msm_table},
    \mref{mod:showhm_table}, \mref{mod:lis_table}, \mref{mod:uia_table}\\
    UHR-AR2 & \mref{mod:uii_table}, \mref{mod:msm_table},
    \mref{mod:showhm_table}, \mref{mod:lis_table}, \mref{mod:uia_table}\\
    \bottomrule
  \end{tabular}
  \caption{Trace Between Accessibility Requirements and Modules}
  \label{TblART}
\end{table}

\begin{table}[H]
  \centering
  \begin{tabular}{p{0.2\textwidth} p{0.6\textwidth}}
    \toprule
    \textbf{Req.} & \textbf{Modules}\\
    \midrule
    PR-SL1 & \mref{mod:captiongeneration_table}\\
    PR-SL2 &  \mref{mod:uia_table} \\
    PR-SCR1 & \mref{mod:mess_table}\\
    PR-SCR2 & \mref{mod:internalmetriccompliance_table}\\
    PR-SCR3 & \mref{mod:uii_table}, \mref{mod:msm_table},
    \mref{mod:showhm_table}, \mref{mod:lis_table}, \mref{mod:uia_table}\\
    PR-SR-HA1 & \mref{mod:bcm_table}, \mref{mod:logmod_table},
    \mref{mod:mess_table}, \mref{mod:auth_table}\\
    PR-SR-HA2 &  \mref{mod:bcm_table}, \mref{mod:logmod_table}\\
    PR-SR-HA3 &  \mref{mod:bcm_table}, \mref{mod:imgvalidmod_table}\\
    PR-PAR1 & \mref{mod:captiongeneration_table}\\
    PR-PAR2 & \mref{mod:internalmetriccompliance_table}\\
    PR-PAR3 & \mref{mod:feedbackmetrics_table}\\
    PR-RFT1 & \mref{mod:imgvalidmod_table}, \mref{mod:bcm_table},
    \mref{mod:logmod_table}\\
    PR-RFT2 & \mref{mod:bcm_table}, \mref{mod:logmod_table}\\
    PR-CR1 & \mref{mod:mess_table}, \mref{mod:auth_table},
    \mref{mod:bcm_table}, \mref{mod:logmod_table}\\
    PR-CR2 & \mref{mod:auth_table}, \mref{mod:showhm_table}\\
    PR-SER1 & \mref{mod:modeltraining_table}\\
    PR-LR1 & \mref{mod:modeltraining_table}, \mref{mod:wcagcompliance_table}\\
    PR-LR2 & \mref{mod:uii_table}, \mref{mod:msm_table},
    \mref{mod:showhm_table}, \mref{mod:lis_table}, \mref{mod:uia_table}\\
    \bottomrule
  \end{tabular}
  \caption{Trace Between Performance Requirements and Modules}
  \label{TblPRT}
\end{table}

\begin{table}[H]
  \centering
  \begin{tabular}{p{0.2\textwidth} p{0.6\textwidth}}
    \toprule
    \textbf{Req.} & \textbf{Modules}\\
    \midrule
    OER-EP1 &  \mref{mod:preprocessing_table},
    \mref{mod:wcagcompliance_table}, \mref{mod:modeloutput_table},
    \mref{mod:auth_table}, \mref{mod:mess_table},
    \mref{mod:bcm_table}, \mref{mod:logmod_table},
    \mref{mod:imgvalidmod_table}, \mref{mod:uii_table},
    \mref{mod:msm_table},  \mref{mod:showhm_table},
    \mref{mod:lis_table}, \mref{mod:uia_table}, \mref{mod:modeltraining_table},
    \mref{mod:captiongeneration_table},
    \mref{mod:internalmetriccompliance_table}, \mref{mod:feedbackmetrics_table},
    \mref{mod:feedbackloop_table}, \mref{mod:feedback_table}\\
    OER-EP2 & \mref{mod:uii_table}\\
    OER-WE1 & \mref{mod:wcagcompliance_table}\\
    OER-WE2 & \mref{mod:preprocessing_table},
    \mref{mod:wcagcompliance_table}, \mref{mod:modeloutput_table},
    \mref{mod:auth_table}, \mref{mod:mess_table},
    \mref{mod:bcm_table}, \mref{mod:logmod_table},
    \mref{mod:imgvalidmod_table}, \mref{mod:uii_table},
    \mref{mod:msm_table},  \mref{mod:showhm_table},
    \mref{mod:lis_table}, \mref{mod:uia_table}, \mref{mod:modeltraining_table},
    \mref{mod:captiongeneration_table},
    \mref{mod:internalmetriccompliance_table}, \mref{mod:feedbackmetrics_table},
    \mref{mod:feedbackloop_table}, \mref{mod:feedback_table}\\
    OER-IAS1 & \mref{mod:preprocessing_table}\\
    OER-IAS2 & \mref{mod:imgvalidmod_table}\\
    OER-IAS3 & \mref{mod:internalmetriccompliance_table}\\
    OER-PR1 &  \mref{mod:preprocessing_table},
    \mref{mod:wcagcompliance_table}, \mref{mod:modeloutput_table},
    \mref{mod:auth_table}, \mref{mod:mess_table},
    \mref{mod:bcm_table}, \mref{mod:logmod_table},
    \mref{mod:imgvalidmod_table}, \mref{mod:uii_table},
    \mref{mod:msm_table},  \mref{mod:showhm_table},
    \mref{mod:lis_table}, \mref{mod:uia_table}, \mref{mod:modeltraining_table},
    \mref{mod:captiongeneration_table},
    \mref{mod:internalmetriccompliance_table}, \mref{mod:feedbackmetrics_table},
    \mref{mod:feedbackloop_table}, \mref{mod:feedback_table}\\
    OER-RL1 & \mref{mod:preprocessing_table},
    \mref{mod:modeltraining_table},
    \mref{mod:captiongeneration_table},
    \mref{mod:wcagcompliance_table}, \mref{mod:internalmetriccompliance_table}\\
    OER-RL2 & \mref{mod:preprocessing_table},
    \mref{mod:wcagcompliance_table}, \mref{mod:modeloutput_table},
    \mref{mod:auth_table}, \mref{mod:mess_table},
    \mref{mod:bcm_table}, \mref{mod:logmod_table},
    \mref{mod:imgvalidmod_table}, \mref{mod:uii_table},
    \mref{mod:msm_table},  \mref{mod:showhm_table},
    \mref{mod:lis_table}, \mref{mod:uia_table}, \mref{mod:modeltraining_table},
    \mref{mod:captiongeneration_table},
    \mref{mod:internalmetriccompliance_table}, \mref{mod:feedbackmetrics_table},
    \mref{mod:feedbackloop_table}, \mref{mod:feedback_table}\\
    \bottomrule
  \end{tabular}
  \caption{Trace Between Operational and Environmental Requirements and Modules}
  \label{TblOERT}
\end{table}

\begin{table}[H]
  \centering
  \begin{tabular}{p{0.2\textwidth} p{0.6\textwidth}}
    \toprule
    \textbf{Req.} & \textbf{Modules}\\
    \midrule
    MS-MNT1 & \mref{mod:preprocessing_table},
    \mref{mod:captiongeneration_table},
    \mref{mod:wcagcompliance_table}, \mref{mod:uii_table},
    \mref{mod:msm_table}, \mref{mod:showhm_table},
    \mref{mod:lis_table}, \mref{mod:uia_table}, \mref{mod:bcm_table}\\
    MS-MNT2 & \mref{mod:preprocessing_table},
    \mref{mod:wcagcompliance_table}, \mref{mod:modeloutput_table},
    \mref{mod:auth_table}, \mref{mod:mess_table},
    \mref{mod:bcm_table}, \mref{mod:logmod_table},
    \mref{mod:imgvalidmod_table}, \mref{mod:uii_table},
    \mref{mod:msm_table},  \mref{mod:showhm_table},
    \mref{mod:lis_table}, \mref{mod:uia_table}, \mref{mod:modeltraining_table},
    \mref{mod:captiongeneration_table},
    \mref{mod:internalmetriccompliance_table}, \mref{mod:feedbackmetrics_table},
    \mref{mod:feedbackloop_table}, \mref{mod:feedback_table}\\
    MS-MNT3 & \mref{mod:preprocessing_table},
    \mref{mod:wcagcompliance_table}, \mref{mod:modeloutput_table},
    \mref{mod:auth_table}, \mref{mod:mess_table},
    \mref{mod:bcm_table}, \mref{mod:logmod_table},
    \mref{mod:imgvalidmod_table}, \mref{mod:uii_table},
    \mref{mod:msm_table},  \mref{mod:showhm_table},
    \mref{mod:lis_table}, \mref{mod:uia_table}, \mref{mod:modeltraining_table},
    \mref{mod:captiongeneration_table},
    \mref{mod:internalmetriccompliance_table}, \mref{mod:feedbackmetrics_table},
    \mref{mod:feedbackloop_table}, \mref{mod:feedback_table}\\
    MS-SUP1 & \mref{mod:logmod_table}\\
    MS-AD1 & \mref{mod:modeltraining_table}\\
    MS-AD2 &  \mref{mod:mess_table},\mref{mod:bcm_table},
    \mref{mod:logmod_table}\\
    MS-AD3 & \mref{mod:preprocessing_table},
    \mref{mod:wcagcompliance_table}, \mref{mod:modeloutput_table},
    \mref{mod:auth_table}, \mref{mod:mess_table},
    \mref{mod:bcm_table}, \mref{mod:logmod_table},
    \mref{mod:imgvalidmod_table}, \mref{mod:uii_table},
    \mref{mod:msm_table},  \mref{mod:showhm_table},
    \mref{mod:lis_table}, \mref{mod:uia_table}, \mref{mod:modeltraining_table},
    \mref{mod:captiongeneration_table},
    \mref{mod:internalmetriccompliance_table}, \mref{mod:feedbackmetrics_table},
    \mref{mod:feedbackloop_table}, \mref{mod:feedback_table}\\
    \bottomrule
  \end{tabular}
  \caption{Trace Between Maintainability and Support Requirements and Modules}
  \label{TblMSRT}
\end{table}

\begin{table}[H]
  \centering
  \begin{tabular}{p{0.2\textwidth} p{0.6\textwidth}}
    \toprule
    \textbf{Req.} & \textbf{Modules}\\
    \midrule
    SR-AR1 & \mref{mod:bcm_table}, \mref{mod:auth_table}\\
    SR-AR2 & \mref{mod:bcm_table}, \mref{mod:mess_table}\\
    SR-IR1 & \mref{mod:bcm_table}\\
    SR-IR2 & \mref{mod:bcm_table}, \mref{mod:preprocessing_table}\\
    SR-PR1 & \mref{mod:bcm_table}\\
    SR-PR2 & \mref{mod:internalmetriccompliance_table}\\
    SR-AU1 & \mref{mod:logmod_table}\\
    SR-AU2 & \mref{mod:logmod_table}\\
    SR-IM1 & \mref{mod:imgvalidmod_table}\\
    SR-IM2 & \mref{mod:auth_table}\\
    \bottomrule
  \end{tabular}
  \caption{Trace Between all Security Requirements and Modules}
  \label{TblSRT1}
\end{table}

\begin{table}[H]
  \centering
  \begin{tabular}{p{0.2\textwidth} p{0.6\textwidth}}
    \toprule
    \textbf{Req.} & \textbf{Modules}\\
    \midrule
    CR1 & \mref{mod:captiongeneration_table}\\
    CR2 & \mref{mod:internalmetriccompliance_table}\\
    CR3 & \mref{mod:internalmetriccompliance_table}\\
    \bottomrule
  \end{tabular}
  \caption{Trace Between all Cultural Requirements and Modules}
  \label{TblCR}
\end{table}

\begin{table}[H]
  \centering
  \begin{tabular}{p{0.2\textwidth} p{0.6\textwidth}}
    \toprule
    \textbf{Req.} & \textbf{Modules}\\
    \midrule
    CR-LR1 & \mref{mod:wcagcompliance_table}\\
    CR-SCR1 & \mref{mod:bcm_table}\\
    CR-SCR2 & \mref{mod:logmod_table}\\
    \bottomrule
  \end{tabular}
  \caption{Trace Between all Compliance Requirements and Modules}
  \label{TbltCompR}
\end{table}

\begin{table}[H]
  \centering
  \begin{tabular}{p{0.2\textwidth} p{0.6\textwidth}}
    \toprule
    \textbf{AC} & \textbf{Modules}\\
    \midrule
    \acref{acInfra} & \mref{mod:bcm_table}\\
    \acref{acInputFormat} & \mref{mod:imgvalidmod_table} \\
    \acref{acMLModel} &  \mref{mod:modeltraining_table} \\
    \acref{acLang} & \mref{mod:modeloutput_table} \\
    \acref{acSession} & \mref{mod:mess_table}\\
    \acref{acInputTopic} & \mref{mod:captiongeneration_table}\\
    \acref{acPlatform} & \mref{mod:uii_table},  \mref{mod:msm_table}, \mref{mod:showhm_table}, \mref{mod:lis_table}, \mref{mod:uia_table} \\

    \bottomrule
  \end{tabular}
  \caption{Trace Between Anticipated Changes and Modules}
  \label{TblACT}
\end{table}

\section{Use Hierarchy Between Modules} \label{SecUse}

In this section, the uses hierarchy between modules is
provided. \citet{Parnas1978} said of two programs A and B that A {\em uses} B if
correct execution of B may be necessary for A to complete the task described in
its specification. That is, A {\em uses} B if there exist situations in which
the correct functioning of A depends upon the availability of a correct
implementation of B.  Figure \ref{FigUH} illustrates the use relation between
the modules. It can be seen that the graph is a directed acyclic graph
(DAG). Each level of the hierarchy offers a testable and usable subset of the
system, and modules in the higher level of the hierarchy are essentially simpler
because they use modules from the lower levels.


\begin{figure}[H]
  \centering
  \includegraphics[width=1.1\textwidth]{HierachyDiagramFinal.jpg}
  \caption{Use hierarchy among modules}
  \label{FigUH}
\end{figure}


%\section*{References}

\section{User Interfaces}

The \textit{Reading4All} interface is composed of multiple pages. The home page, shown in Figure \ref{HomePage}, 
serves as the primary entry point to the \textit{Reading4All} system. From this interface, users are introduced 
to the purpose of the application and are provided with clear navigation options to either sign up for a new account 
or log in to an existing one. The home page is designed to be minimal and accessible, ensuring that users can quickly 
understand the system’s functionality and proceed to authentication without unnecessary complexity.
The sign-up page, shown in Figure \ref{SignUpPage}, allows new users to create an account by providing the required 
registration information. Upon submission, the system validates the provided details and directs the user to either a
sign-up success page (Figure \ref{SignUpSuccessFigma}) or a sign-up failure page (Figure \ref{SignUpFailureFigma}),
clearly indicating whether the account creation process was successful.
The login page enables returning users to authenticate using their existing credentials. As shown in Figures \ref{LoginPageSuccessFigma} 
and \ref{LoginPageFailureFigma}, the system provides explicit feedback by redirecting users to either a login success page upon successful
authentication or a failure page when invalid credentials are detected.
The user is able to upload their images of diagrams through the interface shown in Figure \ref{UploadMain}. 
Furthermore, the system directs the user to a success or failure page accordingly, as highlighted Figure
\ref{UploadSuccess} and Figure \ref{UploadFail}, respectively. Once the user has successfully uploaded an 
image, they can initiate the alt text generation process by selecting the button seen in Figure \ref{UploadSuccess}.
Finally, the user is presented their generated alt text and an option to copy it, as shown in Figure \ref{AltTextGenFig}.
Moreover, the user is able to view their history and sign out at any time through the top bar of each page. The history page allows 
users to view their 10 latest alt text generations and can be seen in Figure \ref{HistoryPage}.

\begin{figure}[H]
  \centering
  \includegraphics[width=0.8\textwidth]{Homepage_Reading4ALL.png}
  \caption{Home Page UI}
  \label{HomePage}
\end{figure}

\begin{figure}[H]
  \centering
  \includegraphics[width=0.8\textwidth]{SignUpFigma.png}
  \caption{Sign Up Page UI}
  \label{SignUpPage}
\end{figure}

\begin{figure}[H]
  \centering
  \includegraphics[width=0.8\textwidth]{SignUpSuccessFigma.png}
  \caption{Sign Up Success Page UI}
  \label{SignUpSuccessFigma}
\end{figure}

\begin{figure}[H]
  \centering
  \includegraphics[width=0.8\textwidth]{SignUpFailureFigma.png}
  \caption{Sign Up Failure Page UI}
  \label{SignUpFailureFigma}
\end{figure}

\begin{figure}[H]
  \centering
  \includegraphics[width=0.8\textwidth]{LoginPageSuccessFigma.png}
  \caption{Login Page Success UI}
  \label{LoginPageSuccessFigma}
\end{figure}

\begin{figure}[H]
  \centering
  \includegraphics[width=0.8\textwidth]{LoginPageFailureFigma.png}
  \caption{Login Page Failure UI}
  \label{LoginPageFailureFigma}
\end{figure}

\begin{figure}[H]
  \centering
  \includegraphics[width=0.8\textwidth]{UploadFigma.png}
  \caption{Upload Page UI}
  \label{UploadMain}
\end{figure}

\begin{figure}[H]
  \centering
  \includegraphics[width=0.8\textwidth]{UploadSuccessFigma.png}
  \caption{Upload Success UI}
  \label{UploadSuccess}
\end{figure}

\begin{figure}[H]
  \centering
  \includegraphics[width=0.8\textwidth]{UploadFailureFigma.png}
  \caption{Upload Failure UI}
  \label{UploadFail}
\end{figure}

\begin{figure}[H]
  \centering
  \includegraphics[width=0.8\textwidth]{AltTextGenFigma.png}
  \caption{Alt Text Generated UI}
  \label{AltTextGenFig}
\end{figure}

\begin{figure}[H]
  \centering
  \includegraphics[width=0.8\textwidth]{HistoryPageFigma.png}
  \caption{History Page UI}
  \label{HistoryPage}
\end{figure}

\section{Design of Communication Protocols}
Not applicable for this project.

\section{Timeline}
The team will split the project into three main subgroups. These
three subgroups and the current
team members within each subgroup for Rev 0 are outlined below.
Design decisions across each subgroup
will be discussed and made together by the entire team.
\begin{enumerate}
  \item \textbf{AI:} Nawaal, Fiza, Francine
  \item \textbf{Backend:} Dhruv, Moly, Francine (backup)
  \item \textbf{Frontend} Dhruv, Moly, Francine (backup)
\end{enumerate}
The development schedules per subgroup for Rev 0 (including tentative dates past Rev 0 and moving towards Rev 1)
are outlined below in Tables~\ref{tab:ai-dev-schedule}-\ref{tab:backend-dev-schedule}. From September 2025 
up to the Proof of Concept demontration on November 26th, 2025, planning 
and elicitation for the Reading4All system was conducted. The official development stages began at the 
end of December 2025.   \\ \\ 
\textbf{AI Subgroup}
\begin{table}[H]
  \centering
  \caption{AI Subgroup Development Schedule}
  \label{tab:ai-dev-schedule}
  \begin{tabular}{|p{3cm}|p{5cm}|p{4cm}|p{4cm}|}
    \hline
    \textbf{Date} &
    \textbf{Task} &
    \textbf{Module Associated} &
    \textbf{Team Member(s) Responsible} \\ \hline

    12/25/2025 -- 01/05/2026 & Lock scope of data, set up
    environments, and acquire datasets & None &  All team members \\ \hline
    01/06/2026 -- 01/11/2026 & Data preprocessing, begin training on
    dataset, metadata labelling & None &  Nawaal, Fiza, Francine \\ \hline
    01/11/2026 -- 01/23/2026 & Model fine-tuning and training using
    Jing's dataset & AIModelTraining, CaptionGeneration &  Nawaal,
    Fiza, Francine \\ \hline
    01/23/2026 -- 01/30/2026 & Implement WCAG compliance checks, and
    internal metric compliance & InternalMetric-Compliance,
    WCAGCompliance &  Nawaal, Fiza, Francine \\ \hline
    02/02/2026 -- 02/06/2026 & \textbf{Rev0} &  &  All team members \\ \hline
    02/09/2026 -- 02/13/2026 & User testing for generated alt text &
    None &  All team members \\ \hline
    02/14/2026 -- 02/27/2026 & Implement feedback from user testing
    to AI model & FeedbackMetrics, FeedbackLoop, Feedback &  Nawaal,
    Fiza, Francine \\ \hline
    02/27/2026 -- 03/15/2026 & Evaluation of metrics, system
    integration with backend/frontend & FeedbackMetrics,
    FeedbackLoop, Feedback &  Nawaal, Fiza, Francine \\ \hline

  \end{tabular}
\end{table}

\noindent\textbf{Frontend Subgroup}

\begin{table}[H]
  \centering
  \caption{Frontend Subgroup Development Schedule}
  \label{tab:frontend-dev-schedule}
  \begin{tabular}{|p{3cm}|p{5cm}|p{4cm}|p{4cm}|}
    \hline
    \textbf{Date} &
    \textbf{Task} &
    \textbf{Module Associated} &
    \textbf{Team Member(s) Responsible} \\ \hline

    01/05/2026 -- 01/12/2026 & Create figma designs for each page &
    None & Dhruv, Moly, Francine \\ \hline
    01/12/2026 -- 01/21/2026 & Implement UI pages and test screen
    reader compatibility & LogInScreen, MainScreen,
    Show-HistoryScreen, UserInterface-Accessibility,
    UserInterface-Interactions &  Dhruv and Moly \\ \hline
    01/21/2026 -- 01/30/2026 & Integration with backend system and
    WCAG testing& None &  Dhruv, Moly, Francine \\ \hline
    02/02/2026 -- 02/06/2026 & \textbf{Rev0} & None &  All team
    members \\ \hline
    02/09/2026 -- 02/13/2026 & User testing & None &  All team members \\ \hline
    02/14/2026 -- 02/27/2026 & Revise UI if needed after testing &
    None &  Dhruv, Moly, Francine \\ \hline

  \end{tabular}
\end{table}

\noindent\textbf{Backend Subgroup}

\begin{table}[H]
  \centering
  \caption{Backend Subgroup Development Schedule}
  \label{tab:backend-dev-schedule}
  \begin{tabular}{|p{3cm}|p{5cm}|p{4cm}|p{4cm}|}
    \hline
    \textbf{Date} &
    \textbf{Task} &
    \textbf{Module Associated} &
    \textbf{Team Member(s) Responsible} \\ \hline

    01/05/2026 -- 01/12/2026 & Planning APIs and creating schemas &
    None &  Dhruv, Moly, Francine \\ \hline
    01/13/2026 -- 01/21/2026 & Implement APIs and manual testing &
    AuthenticationService, BackendController, SessionManagement,
    ImageValidation &  Dhruv, Moly, Francine \\ \hline
    01/21/2026 -- 01/30/2026 & API testing and integration with
    Frontend and AI systems & Logging  & Dhruv, Moly, Francine \\ \hline
    02/02/2026 -- 02/06/2026 & \textbf{Rev0} & None &  All team
    members \\ \hline
    02/09/2026 -- 02/13/2026 & User testing & None &  All team members \\ \hline
    02/14/2026 -- 02/27/2026 & Revise backend logic if needed after
    user tesing & None &  Dhruv, Moly, Francine \\ \hline

  \end{tabular}
\end{table}

\newpage
\section{Appendix A - Mathematical Representation for
\textit{Reading4All - AI Subsystem}}
This section provides an overall formal mathematical representation of the
\textit{Reading4All} system such that the system can be understood
independently of its implementation. Furthermore, specific attempts for the
AI trained under the \textit{AI Subgroup} are mathematically denoted with the
goal being to explicitly document all attempts made during the training
process and how intermediate representations are constructed.

\subsection{Reading4All - AI Subsystem Overview}

The \textit{Reading4All - AI Subsystem} is a multi-stage vision--language system
designed to automatically generate accessible alternative (alt) text for STEM
figures. The subsystem operates by progressively transforming a visual input
into increasingly abstract representations, culminating in natural-language
descriptions suitable for assistive technologies such as screen readers.

\paragraph{Input Space}
Let,
\[
I \in \mathbb{R}^{H \times W \times 3}
\]
denote a single RGB input image, where $H$ and $W$ correspond to the height and
width of the image in pixels, and the third dimension represents color channels (which are red, green and blue).

\paragraph{Output Space}
The final output of the AI Subsystem is a sequence of textual tokens:
\[
A = (a_1, a_2, \dots, a_T),
\]
where each $a_t$ belongs to a discrete vocabulary $\mathcal{V}$ and $T$ denotes
the length of the generated alt text.

\paragraph{System-Level Mapping}
At a high level, the \textit{Reading4All - AI Subsystem} implements a conditional mapping:
\[
\mathcal{F}: \mathbb{R}^{H \times W \times 3} \rightarrow \mathcal{V}^T,
\]
which is instantiated through multiple training scenarios described in the
subsequent sections.

\subsection{Mathematical Preliminaries and Shared Notation}

The following notation is used consistently throughout this appendix:

\begin{itemize}
    \item $I$: input image
    \item $P$: number of visual patches extracted from $I$
    \item $p$: patch side length in pixels
    \item $d$: hidden embedding dimension of transformer models
    \item $L$: number of transformer layers
    \item $\theta$: set of trainable parameters
    \item $\mathcal{L}$: loss function
\end{itemize}

All models used within the \textit{Reading4All - AI Subsystem} are based on transformer
architectures and employ attention mechanisms and residual connections.

\subsection{Vision Transformer Encoding}

All training scenarios begin by converting the input image $I$ into a latent
visual representation using a vision transformer encoder.

\subsubsection{Patch Extraction}

The image $I$ is partitioned into non-overlapping square patches of size
$p \times p$. The total number of patches is given by:
\[
P = \frac{H \cdot W}{p^2}.
\]

Each patch is flattened into a vector:
\[
\mathbf{x}_i \in \mathbb{R}^{3p^2}, \quad i = 1,\dots,P.
\]

\subsubsection{Patch Embedding}

Each flattened patch vector is projected into a $d$-dimensional embedding space
using a learned linear transformation:
\[
\mathbf{e}_i = \mathbf{W}_E \mathbf{x}_i + \mathbf{b}_E,
\]
where $\mathbf{W}_E \in \mathbb{R}^{d \times 3p^2}$ and
$\mathbf{b}_E \in \mathbb{R}^d$ are trainable parameters.

A special learnable classification token
$\mathbf{e}_{\text{CLS}} \in \mathbb{R}^d$ is prepended to the patch sequence,
yielding:
\[
\mathbf{Z}_0 =
[\mathbf{e}_{\text{CLS}}, \mathbf{e}_1, \mathbf{e}_2, \dots, \mathbf{e}_P].
\]

\subsubsection{Transformer Encoder Layers}

The embedded patch sequence is processed by $L$ stacked transformer layers. For
layer $\ell \in \{1,\dots,L\}$, the hidden representation is computed as:
\[
\mathbf{Z}_\ell =
\text{LayerNorm}\left(
\mathbf{Z}_{\ell-1} +
\text{FFN}\big(\text{MSA}(\mathbf{Z}_{\ell-1})\big)
\right).
\]

Multi-head self-attention (MSA) is defined by:
\[
\text{MSA}(\mathbf{Q}, \mathbf{K}, \mathbf{V}) =
\text{softmax}\left(
\frac{\mathbf{Q}\mathbf{K}^\top}{\sqrt{d}}
\right)\mathbf{V},
\]
where $\mathbf{Q}$, $\mathbf{K}$, and $\mathbf{V}$ are linear projections of
$\mathbf{Z}_{\ell-1}$.

The final image representation is taken as the output corresponding to the
classification token:
\[
\mathbf{h}_{\text{img}} = \mathbf{Z}_L^{(0)}.
\]

\subsection{Scenario 1: Caption-Centric Vision--Language Learning}

Scenario 1 implements a direct vision-to-language pipeline using a small BLIP-based
model and scientific caption supervision.

\subsubsection{Stage 1: PubLayNet Layout Classification [\cite{zhong2019publaynet}]}

\paragraph{Task Definition}
Given an input image $I$, the model predicts a dominant layout category
\[
y \in \{\text{text}, \text{title}, \text{list}, \text{table}, \text{figure}\}.
\]

\paragraph{Classification Head}
A linear classifier maps the image embedding $\mathbf{h}_{\text{img}}$ to
unnormalized logits:
\[
\mathbf{o} = \mathbf{W}_C \mathbf{h}_{\text{img}} + \mathbf{b}_C,
\]
where $\mathbf{W}_C \in \mathbb{R}^{5 \times d}$ and
$\mathbf{b}_C \in \mathbb{R}^5$.

\paragraph{Loss Function}
The layout classification loss is defined as the categorical cross-entropy:
\[
\mathcal{L}_{\text{layout}} =
- \log \frac{\exp(o_y)}{\sum_{k=1}^{5} \exp(o_k)}.
\]

At the time of this document submission, only the vision encoder parameters are updated during this stage, while the language model parameters remain frozen to prevent premature or unstable updates to the uninitialized decoder.

\subsubsection{Stage 2: SciCap Caption Generation [\cite{hsu-etal-2021-scicap-generating}]}

\paragraph{Caption Representation}
Let a reference scientific caption be represented as a token sequence:
\[
C = (c_1, c_2, \dots, c_T).
\]

\paragraph{Autoregressive Decoding}
The decoder models the conditional probability of each token given previous
tokens and the image representation:
\[
p(C \mid I) = \prod_{t=1}^{T} p(c_t \mid c_{<t}, \mathbf{h}_{\text{img}}).
\]

\paragraph{Cross-Attention}
Decoder hidden states attend to visual embeddings via cross-attention:
\[
\text{Attn}_{\text{cross}} =
\text{softmax}\left(
\frac{\mathbf{Q}\mathbf{K}^\top}{\sqrt{d}}
\right)\mathbf{V}.
\]

\paragraph{Loss Function}
The captioning objective is the negative log-likelihood:
\[
\mathcal{L}_{\text{caption}} =
- \sum_{t=1}^{T} \log p(c_t \mid c_{<t}, I).
\]

As per our initial attempt the vision encoder is frozen while only the decoder and cross-attention parameters are updated. Freezing the encoder stabilizes training and ensures that the visual features learned in Stage 1 are preserved.


\subsection{Scenario 2: Structured Reasoning-Based Alt Text Generation}

Scenario~2 introduces an explicit structured intermediate representation
between visual understanding and language generation.

\subsubsection{Stage 1: PubLayNet Layout Supervision [\cite{zhong2019publaynet}]}

This stage mirrors the layout supervision described in Scenario~1 and is used
to initialize a layout-aware vision encoder for subsequent stages.

\subsubsection{Stage 2: AI2D Structured Representation Learning [\cite{kembhavi2016diagram}]}

\paragraph{Structured Output Space}
Given an image $I$, the model generates a structured token sequence:
\[
S = (s_1, s_2, \dots, s_K),
\]
where each $s_k$ represents a symbolic component or relationship extracted
from the diagram.
Example of Structured Components: For instance, each $s_k$ can represent a symbolic component or relationship extracted from a diagram, e.g.,
\[
s_k = \text{``arrow: connects A $\rightarrow$ B''} \quad \text{or} \quad
s_k = \text{``circle: highlights key element''}.
\]

\paragraph{Conditional Modeling}
The structured representation is generated autoregressively:
\[
p(S \mid I) = \prod_{k=1}^{K} p(s_k \mid s_{<k}, I).
\]

\paragraph{Loss Function}
\[
\mathcal{L}_{\text{struct}} =
- \sum_{k=1}^{K} \log p(s_k \mid s_{<k}, I).
\]

At the time of the attempt during this stage, the vision encoder is frozen during this stage, and only the decoder cross-attention parameters are updated. This preserves the layout-aware visual features while allowing the model to learn structured symbolic representations from diagrams.

\subsubsection{Stage 3: T5-Based Alt Text Generation [\cite{2020t5}]}

\paragraph{Input Transformation}
The structured sequence $S$ is serialized into text and provided as input to a T5-based language model.

\paragraph{Alt Text Generation}
The T5 model produces alt text tokens autoregressively:
\[
p(A \mid S) = \prod_{t=1}^{T} p(a_t \mid a_{<t}, S),
\]
where $A = (a_1, a_2, \dots, a_T)$ is the generated alt text token sequence.

\paragraph{Loss Function}
The alt text generation objective is the negative log-likelihood:
\[
\mathcal{L}_{\text{alt}} =
- \sum_{t=1}^{T} \log p(a_t \mid a_{<t}, S).
\]

At the time of this document's submission and during this stage, the vision encoder remains frozen, the decoder cross-attention parameters from the previous stage are reused, and only the T5 model parameters are updated. Freezing the encoder preserves the learned visual and structured representations, while allowing the language model to specialize in generating descriptive alt text.

\subsection{End-to-End System Composition}

The final \textit{Reading4All - AI subsystem} is a choice between the two scenarios attempted so far:
\[
A =
\begin{cases}
\text{BLIP}_{\text{SciCap}}(\text{BLIP}_{\text{PubLayNet}}(I)), & \text{Scenario 1} \\
\text{T5}(\text{Pix2Struct}_{\text{AI2D}}(\text{Pix2Struct}_{\text{PubLayNet}}(I))), & \text{Scenario 2}
\end{cases}
\]\\
Where,
\begin{itemize}
  \item Scenario 1: end-to-end captioning;
  \item Scenario 2: structured reasoning with interpretable intermediate representations.
\end{itemize}  

\paragraph{Design Rationale}
Each scenario represents a distinct design choice within the \textit{Reading4All - AI subsystem} framework:

\begin{itemize}
    \item \textbf{Scenario 1:} Optimized for end-to-end caption generation using BLIP-based models, with direct supervision from scientific captions.
    \item \textbf{Scenario 2:} Introduces structured intermediate representations (e.g., $s_k =$ ``arrow: connects A $\rightarrow$ B'', ``circle: highlights key element'') to improve interpretability and accessibility while maintaining accurate alt text generation via a T5 language model.
\end{itemize}

To iterate, freezing certain parameters at each stage ensures stability, preserves previously learned representations, and prevents overfitting when training downstream components on limited data.

\newpage
\bibliographystyle {plainnat}
\bibliography{../../../refs/References}

\newpage{}

\end{document}