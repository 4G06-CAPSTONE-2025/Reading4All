\documentclass[12pt, titlepage]{article}

\usepackage{amsmath, mathtools}

\usepackage[round]{natbib}
\usepackage{amsfonts}
\usepackage{amssymb}
\usepackage{graphicx}
\usepackage{colortbl}
\usepackage{xr}
\usepackage{hyperref}
\usepackage{longtable}
\usepackage{xfrac}
\usepackage{tabularx}
\usepackage{float}
\usepackage{siunitx}
\usepackage{booktabs}
\usepackage{multirow}
\usepackage[section]{placeins}
\usepackage{caption}
\usepackage{fullpage}

\newcommand{\String}{\mathsf{String}}
\newcommand{\Real}{\ensuremath{\mathbb{R}}}
\newcommand{\Bool}{\ensuremath{\mathbb{B}}}


\hypersetup{
  bookmarks=true,     % show bookmarks bar?
  colorlinks=true,       % false: boxed links; true: colored links
  linkcolor=red,          % color of internal links (change box color
  % with linkbordercolor)
  citecolor=blue,      % color of links to bibliography
  filecolor=magenta,  % color of file links
  urlcolor=cyan          % color of external links
}

\usepackage{array}

\externaldocument{../../SRS/SRS}

\input{../../Comments}
%% Common Parts

\newcommand{\progname}{ProgName} % PUT YOUR PROGRAM NAME HERE
\newcommand{\authname}{Team 22, READING4ALL
\\ Fiza Sehar
\\ Nawaal Fatima
\\ Dhruv Sardana
\\ Moly Mikhail
\\ Casey Francine Bulaclac } % AUTHOR NAMES                  

\usepackage{hyperref}
    \hypersetup{colorlinks=true, linkcolor=blue, citecolor=blue, filecolor=blue,
                urlcolor=blue, unicode=false}
    \urlstyle{same}
                                


\begin{document}

\title{Module Interface Specification for \progname{}}

\author{\authname}

\date{\today}

\maketitle

\pagenumbering{roman}

\section{Revision History}

\begin{tabularx}{\textwidth}{p{3cm}p{2cm}X}
  \toprule {\bf Date} & {\bf Version} & {\bf Notes}\\
  \midrule
  Date 1 & 1.0 & Notes\\
  Date 2 & 1.1 & Notes\\
  \bottomrule
\end{tabularx}

~\newpage

\section{Symbols, Abbreviations and Acronyms}

See SRS Documentation at \wss{give url}

\wss{Also add any additional symbols, abbreviations or acronyms}

\newpage

\tableofcontents

\newpage

\pagenumbering{arabic}

\section{Introduction}

The following document details the Module Interface Specifications for
\wss{Fill in your project name and description}

Complementary documents include the System Requirement Specifications
and Module Guide.  The full documentation and implementation can be
found at \url{...}.  \wss{provide the url for your repo}

\section{Notation}

\wss{You should describe your notation.  You can use what is below as
a starting point.}

The structure of the MIS for modules comes from \citet{HoffmanAndStrooper1995},
with the addition that template modules have been adapted from
\cite{GhezziEtAl2003}.  The mathematical notation comes from Chapter 3 of
\citet{HoffmanAndStrooper1995}.  For instance, the symbol := is used for a
multiple assignment statement and conditional rules follow the form $(c_1
\Rightarrow r_1 | c_2 \Rightarrow r_2 | ... | c_n \Rightarrow r_n )$.

The following table summarizes the primitive data types used by \progname.

\begin{center}
  \renewcommand{\arraystretch}{1.2}
  \noindent
  \begin{tabular}{l l p{7.5cm}}
    \toprule
    \textbf{Data Type} & \textbf{Notation} & \textbf{Description}\\
    \midrule
    character & char & a single symbol or digit\\
    integer & $\mathbb{Z}$ & a number without a fractional component
    in (-$\infty$, $\infty$) \\
    natural number & $\mathbb{N}$ & a number without a fractional
    component in [1, $\infty$) \\
    real & $\mathbb{R}$ & any number in (-$\infty$, $\infty$)\\
    \bottomrule
  \end{tabular}
\end{center}

\noindent
The specification of \progname \ uses some derived data types:
sequences, strings, and
tuples. Sequences are lists filled with elements of the same data type. Strings
are sequences of characters. Tuples contain a list of values, potentially of
different types. In addition, \progname \ uses functions, which
are defined by the data types of their inputs and outputs. Local functions are
described by giving their type signature followed by their specification.

\section{Module Decomposition}

The following table is taken directly from the Module Guide document
for this project.

\begin{table}[h!]
  \centering
  \begin{tabular}{p{0.3\textwidth} p{0.6\textwidth}}
    \toprule
    \textbf{Level 1} & \textbf{Level 2}\\
    \midrule

    {Hardware-Hiding} & ~ \\
    \midrule

    \multirow{3}{0.3\textwidth}{Behaviour-Hiding Module} &
    DataPreprocess Module\\
    & WCAGCompliance Module\\
    & ModelOutput Model \\
    \midrule

    \multirow{4}{0.3\textwidth}{Software Decision Module} &
    AIModelTraining Module\\
    & CaptionGeneration Module\\
    & FeedbackMetrics Module\\
    & FeedbackIntegration Module\\
    \bottomrule

  \end{tabular}
  \caption{Module Hierarchy}
  \label{TblMH}
\end{table}

\newpage
~\newpage

%----------------------------------------------------------

\section{MIS of DataPreprocess Module} \label{datapp}

\subsection{Module}
DataPP

\subsection{Uses}
Receives image input data from front-end or input-handling modules. Passes preprocessed data to the AIModelTraining Module (ADD REF HERE).

\subsection{Syntax}

\subsubsection{Exported Constants}
None.

\subsubsection{Exported Access Programs}

\begin{center}
  \begin{tabular}{p{3cm} p{2cm} p{4cm} p{6cm}}
    \hline
    \textbf{Name} & \textbf{In} & \textbf{Out} & \textbf{Exceptions} \\
    \hline
    filterInput() & Image & Filtered image data & InvalidInputException \\
    \hline
  \end{tabular}
\end{center}

\subsection{Semantics}

\subsubsection{State Variables}
None.

\subsubsection{Environment Variables}
None.

\subsubsection{Assumptions}
Input images are valid files provided through the system interface.

\subsubsection{Access Routine Semantics}

\noindent filterInput():
\begin{itemize}
  \item transition: None
  \item output: Returns image data compatible with the AIModelTraining module.
  \item exception: Raises InvalidInputException if file type or format is unsupported.
\end{itemize}

\subsubsection{Local Functions}
None.

\section{MIS of WCAGCompliance Module} \label{wcagc}

\subsection{Module}
WCAGComp

\subsection{Uses}
Takes generated captions from CaptionGeneration Module and validates them against WCAG 2.1 AA criteria (ADD REF HERE).

\subsection{Syntax}
\subsubsection{Exported Constants}
None.

\subsubsection{Exported Access Programs}

\begin{center}
  \begin{tabular}{p{4cm} p{2cm} p{4cm} p{6cm}}
    \hline
    \textbf{Name} & \textbf{In} & \textbf{Out} & \textbf{Exceptions} \\
    \hline
    checkCompliance() & Caption text & Compliance report & ValidationException \\
    \hline
  \end{tabular}
\end{center}

\subsection{Semantics}

\subsubsection{State Variables}
None.

\subsubsection{Environment Variables}
None.

\subsubsection{Assumptions}
Assumes caption text is non-empty and properly formatted.

\subsubsection{Access Routine Semantics}

\noindent checkCompliance():
\begin{itemize}
  \item transition: None
  \item output: Returns a compliance report summarizing WCAG validation results.
  \item exception: Raises ValidationException if text input is invalid.
\end{itemize}

\subsubsection{Local Functions}
None.


\section{MIS of ModelOutput Module} \label{modout}

\subsection{Module}
ModOut

\subsection{Uses}
Receives validated captions from WCAGCompliance (Section \ref{wcagc}) or FeedbackMetrics module and sends final output to the user-facing interface.(ADD REF HERE).

\subsection{Syntax}
\subsubsection{Exported Constants}
None.

\subsubsection{Exported Access Programs}

\begin{center}
  \begin{tabular}{p{4cm} p{2cm} p{4cm} p{6cm}}
    \hline
    \textbf{Name} & \textbf{In} & \textbf{Out} & \textbf{Exceptions} \\
    \hline
    generateOutput() & Final caption & Displayed text & OutputException \\
    \hline
  \end{tabular}
\end{center}

\subsection{Semantics}
\subsubsection{State Variables}
None.

\subsubsection{Environment Variables}
Connection to the user interface or backend display module.

\subsubsection{Assumptions}
Assumes output channel (backend or UI) is active.

\subsubsection{Access Routine Semantics}

\noindent generateOutput():
\begin{itemize}
  \item transition: Sends final text to the appropriate system interface.
  \item output: Returns confirmation of successful output delivery.
  \item exception: Raises OutputException if communication with UI fails.
\end{itemize}

\subsubsection{Local Functions}
None.


\section{MIS of AIModelTraining Module} \label{aitrain}

\subsection{Module}
AITrain

\subsection{Uses}
Receives preprocessed data from DataPreprocess Module and produces a trained model used by CaptionGeneration Module.

\subsection{Syntax}
\subsubsection{Exported Constants}
None.

\subsubsection{Exported Access Programs}

\begin{center}
  \begin{tabular}{p{4cm} p{2cm} p{4cm} p{6cm}}
    \hline
    \textbf{Name} & \textbf{In} & \textbf{Out} & \textbf{Exceptions} \\
    \hline
    trainModel() & Preprocessed data & Trained model & TrainingException \\
    \hline
  \end{tabular}
\end{center}

\subsection{Semantics}
\subsubsection{State Variables}
trainedModel: Stores the trained AI model instance.

\subsubsection{Environment Variables}
None.

\subsubsection{Assumptions}
Assumes sufficient computational resources and valid training data.

\subsubsection{Access Routine Semantics}

\noindent trainModel():
\begin{itemize}
  \item transition: Updates trainedModel with a newly trained instance.
  \item output: Returns reference to the trained model.
  \item exception: Raises TrainingException if model fails to converge or data is invalid.
\end{itemize}

\subsubsection{Local Functions}
splitData(): Splits input data into training and validation sets.

\section{MIS of CaptionGeneration Module} \label{capgen}

\subsection{Module}
CapGen

\subsection{Uses}
Utilizes the trained model from AIModelTraining Module to generate image captions.

\subsection{Syntax}
\subsubsection{Exported Constants}
None.

\subsubsection{Exported Access Programs}

\begin{center}
  \begin{tabular}{p{4cm} p{2cm} p{4cm} p{6cm}}
    \hline
    \textbf{Name} & \textbf{In} & \textbf{Out} & \textbf{Exceptions} \\
    \hline
    generateCaption() & Image & Caption text & GenerationException \\
    \hline
  \end{tabular}
\end{center}

\subsection{Semantics}
\subsubsection{State Variables}
None.

\subsubsection{Environment Variables}
None.

\subsubsection{Assumptions}
The AI model has been successfully trained prior to use.

\subsubsection{Access Routine Semantics}

\noindent generateCaption():
\begin{itemize}
  \item transition: None
  \item output: Returns generated alternative text for input image.
  \item exception: Raises GenerationException if model inference fails.
\end{itemize}

\subsubsection{Local Functions}
None.


\section{MIS of FeedbackMetrics Module} \label{modmetric}

\subsection{Module}
ModMetric

\subsection{Uses}
Compares generated captions with target thresholds and passes performance scores to FeedbackIntegration Module.

\subsection{Syntax}
\subsubsection{Exported Constants}
None.

\subsubsection{Exported Access Programs}

\begin{center}
  \begin{tabular}{p{4cm} p{2cm} p{4cm} p{6cm}}
    \hline
    \textbf{Name} & \textbf{In} & \textbf{Out} & \textbf{Exceptions} \\
    \hline
    evaluateCaption() & Caption text & Metric scores & EvaluationException \\
    \hline
  \end{tabular}
\end{center}

\subsection{Semantics}
\subsubsection{State Variables}
None.

\subsubsection{Environment Variables}
None.

\subsubsection{Assumptions}
Assumes evaluation metrics are predefined and accessible.

\subsubsection{Access Routine Semantics}

\noindent evaluateCaption():
\begin{itemize}
  \item transition: None
  \item output: Returns performance metrics [Put in appendix Jing's metrics].
  \item exception: Raises EvaluationException if metrics are unavailable.
\end{itemize}

\subsubsection{Local Functions}
None.

\section{MIS of FeedbackIntegration Module} \label{modint}

\subsection{Module}
ModInt

\subsection{Uses}
Uses feedback from FeedbackMetrics to improve model parameters in AIModelTraining.

\subsection{Syntax}
\subsubsection{Exported Constants}
None.

\subsubsection{Exported Access Programs}

\begin{center}
  \begin{tabular}{p{4cm} p{2cm} p{4cm} p{6cm}}
    \hline
    \textbf{Name} & \textbf{In} & \textbf{Out} & \textbf{Exceptions} \\
    \hline
    updateModel() & Feedback metrics & Updated model & UpdateException \\
    \hline
  \end{tabular}
\end{center}

\subsection{Semantics}
\subsubsection{State Variables}
modelState: Holds the current model parameters for retraining.

\subsubsection{Environment Variables}
None.

\subsubsection{Assumptions}
Assumes valid feedback metrics are received from FeedbackMetrics Module.

\subsubsection{Access Routine Semantics}

\noindent updateModel():
\begin{itemize}
  \item transition: Updates modelState with new optimized parameters.
  \item output: Returns confirmation of successful update.
  \item exception: Raises UpdateException if optimization fails.
\end{itemize}

\subsubsection{Local Functions}
None.

%----------------------------------------------------------

\section{MIS of \wss{Module Name}} \label{Module} \wss{Use labels for
cross-referencing}

\wss{You can reference SRS labels, such as R\ref{R_Inputs}.}

\wss{It is also possible to use \LaTeX for hypperlinks to external documents.}

\subsection{Module}

\texttt{BackendController}

\subsection{Uses}
Serves as the coordinator between the frontend, backend and machine learning modules. 
\begin{itemize}
  \item Forwards frontend requests to the appropriate backend modules, such authentication or image validation.
  \item Sends validated inputs to the ML Model for alt-text generation. 
  \item Retrieves generated alt-text and returns it to the frontend modules. 
\end{itemize}

\subsection{Syntax}

\subsubsection{Exported Constants}
\texttt{SESSION\_TIMEOUT}: $\mathbb{N}$ - Specifies the amount of time in minutes a users logged-in session can remain active for before requiring reauthentication.\\ \\
\texttt{API\_RESPONSE\_TIMEOUT}: $\mathbb{N}$ - Specifies the amount of time in seconds that the controller waits for a response from other modules before timing out.\\

\subsubsection{Exported Access Programs}

\begin{center}
\begin{tabular}{p{3cm} p{4cm} p{3cm} p{2cm}}
\hline
\textbf{Name} & \textbf{In} & \textbf{Out} & \textbf{Exceptions} \\
\hline
validateUser & userToken & Boolean & InvalidTokenException \\
\hline
validateImage & image, userToken & Boolean & InvalidImageException\\
\hline
sendToModel & image, userToken & None & AltTextGenerationError\\
\hline 
getAltText & image, userToken & altText & AltTextNotFoundException\\
\hline 
returnToFrontend & altText or errorMsg, userID & None& FrontendUnreachableException \\
\end{tabular}
\end{center}

\subsection{Semantics}

\subsubsection{State Variables}

\wss{Not all modules will have state variables.  State variables give the module
a memory.}

\subsubsection{Environment Variables}

\wss{This section is not necessary for all modules.  Its purpose is to capture
  when the module has external interaction with the environment, such as for a
  device driver, screen interface, keyboard, file, etc.}
  \begin{itemize}
    \item Frontend Modules: Receives user requests such as image uploads and returns responses such alt-text generation or error messages.
    \item User Authentication Modules: Sends the user's session token to module to verify users identity. 
    \item Image Validation Modules: Sends the user's inputted image to be confirm it meets the required format and size, prior to alt-text generation.
    \item Machine Learning Modules: Sends users validated image to ML model to begin alt text generation.
  \end{itemize}

\subsubsection{Assumptions}

\wss{Try to minimize assumptions and anticipate programmer errors via
  exceptions, but for practical purposes assumptions are sometimes appropriate.}
\begin{itemize}
  \item Frontend, User authentication, Image validation and machine learning modules are correctly operating and reachable. 
  \item Network connection is available for API communication. 
\end{itemize}

\subsubsection{Access Routine Semantics}

\noindent \texttt{validateUser(userToken)}:
\begin{itemize}
\item transition: Sends the \texttt{userToken} to the User Authentication Module for Verification.
\item output: Returns \texttt{True} if the user is valid, otherwise \texttt{False} is returned.
\item exception: \texttt{InvalidTokenException} is raised if token is expired or cannot be validated. 
\end{itemize}



\noindent \texttt{validateImage(image,userToken)}:
\begin{itemize}
\item transition: Sends the uploaded image and associated user token to the Image Validation Module for verification of file type and size.
\item output:  Returns \texttt{True} if the image was successfully validated, otherwise \texttt{False} is returned.
\item exception:  \texttt{InvalidImageException} is raised if the image fails the validation process and does not meet system requirements.
\end{itemize}



\noindent \texttt{sendToModel(image,userToken)}:
\begin{itemize}
\item transition: Sends the validated image and associated user token to the Machine Learning Module to initiate alt-text generation.  
\item output:  None
\item exception: \texttt{AltTextGenerationError} is raised if the Machine Learning Module fails to process image or generate alt text. 
\end{itemize}


\noindent \texttt{returnToFrontend(alt Text or errorMsg, userToken)}:
\begin{itemize}
\item transition: Sends the generated alt text or corresponding error message to the Frontend Module. 
\item output:  None
\item exception: \texttt{FrontendUnreachableException} is raised if the Frontend Module cannot be reached. 
\end{itemize}


\newpage
\section{MIS of AIModelTraining Module} \label{aitrain}

\subsection{Module}
AITrain

\subsection{Uses}
Receives preprocessed data from the DataPreprocess module and produces a trained model used by the Classifier/Inference module for alt-text generation.  

\subsection{Syntax}
\subsubsection{Exported Constants}
None.

\subsubsection{Exported Access Programs}

\begin{center}
  \begin{tabular}{p{4cm} p{4cm} p{4cm} p{3cm}}
    \toprule
    \textbf{Name} & \textbf{In} & \textbf{Out} & \textbf{Exceptions} \\
    \midrule
    initializeModel & configPath:\,String & None & ConfigFileNotFoundException \\
    trainModel & datasetSpec:\,DatasetSpec & modelWeights:\,ModelWeights & TrainingException \\
    saveCheckpoint & epoch:\,$\mathbb{N}$, metrics:\,MetricsReport & None & WriteErrorException \\
    evaluateModel & validationSet:\,DatasetSpec & metrics:\,MetricsReport & EvaluationErrorException \\
    \bottomrule
  \end{tabular}
\end{center}

\subsection{Semantics}

\subsubsection{State Variables}
\begin{itemize}
  \item trainedModel:\,ModelWeights — stores most recently trained weights.
  \item trainingLogs:\,MetricsLog — records metrics for each epoch.
\end{itemize}

\subsubsection{Environment Variables}
\begin{itemize}
  \item Filesystem — used to read/write checkpoints and logs.
  \item GPU/CPU — executes model computations.
\end{itemize}

\subsubsection{Assumptions}
\begin{itemize}
  \item Dataset inputs are valid and preprocessed.
  \item Sufficient compute (GPU/CPU) resources available.
  \item The configuration file exists and is readable.
\end{itemize}

\subsubsection{Access Routine Semantics}

\noindent \texttt{initializeModel(configPath)}:
\begin{itemize}
  \item transition: Loads model architecture, optimizer, and training parameters.
  \item output: None.
  \item exception: \texttt{ConfigFileNotFoundException} if file is missing or invalid.
\end{itemize}

\noindent \texttt{trainModel(datasetSpec)}:
\begin{itemize}
  \item transition: Performs training; updates \texttt{trainedModel}.
  \item output: Returns \texttt{ModelWeights}.
  \item exception: \texttt{TrainingException} if training fails or dataset is invalid.
\end{itemize}

\noindent \texttt{saveCheckpoint(epoch, metrics)}:
\begin{itemize}
  \item transition: Saves current weights and metrics to storage.
  \item output: None.
  \item exception: \texttt{WriteErrorException} if write operation fails.
\end{itemize}

\noindent \texttt{evaluateModel(validationSet)}:
\begin{itemize}
  \item transition: Evaluates performance; appends metrics to log.
  \item output: Returns \texttt{MetricsReport}.
  \item exception: \texttt{EvaluationErrorException} if validation set missing or corrupt.
\end{itemize}

\subsubsection{Local Functions}
None.


\newpage
\section{MIS of Classifier / Inference Module} \label{classify}

\subsection{Module}
ClassifierInference

\subsection{Uses}
Uses trained weights from \hyperref[aitrain]{AIModelTraining Module} to generate alt-text descriptions for new images.  

\subsection{Syntax}
\subsubsection{Exported Constants}
None.

\subsubsection{Exported Access Programs}

\begin{center}
  \begin{tabular}{p{4cm} p{4cm} p{4cm} p{3cm}}
    \toprule
    \textbf{Name} & \textbf{In} & \textbf{Out} & \textbf{Exceptions} \\
    \midrule
    loadModel & modelWeights:\,ModelWeights & None & ModelLoadException \\
    predictAltText & image:\,TensorImage & altText:\,AltText & InferenceErrorException \\
    decodeOutput & rawPredictions:\,Logits & altText:\,AltText & DecodeErrorException \\
    getConfidenceScore & altText:\,AltText & confidence:\,$\mathbb{R}$ & None \\
    \bottomrule
  \end{tabular}
\end{center}

\subsection{Semantics}

\subsubsection{State Variables}
\begin{itemize}
  \item loadedModel:\,ModelWeights — active weights for inference.
\end{itemize}

\subsubsection{Environment Variables}
\begin{itemize}
  \item GPU/CPU — runs inference computations.
\end{itemize}

\subsubsection{Assumptions}
\begin{itemize}
  \item \texttt{modelWeights} correspond to a valid trained model.
  \item \texttt{image} is preprocessed to expected input dimensions.
\end{itemize}

\subsubsection{Access Routine Semantics}

\noindent \texttt{loadModel(modelWeights)}:
\begin{itemize}
  \item transition: Loads model weights into memory.
  \item output: None.
  \item exception: \texttt{ModelLoadException} if weights are corrupted or missing.
\end{itemize}

\noindent \texttt{predictAltText(image)}:
\begin{itemize}
  \item transition: Performs inference and generates raw predictions.
  \item output: Returns generated alt text.
  \item exception: \texttt{InferenceErrorException} if computation fails.
\end{itemize}

\noindent \texttt{decodeOutput(rawPredictions)}:
\begin{itemize}
  \item transition: Converts raw outputs into readable descriptions.
  \item output: Returns decoded \texttt{AltText}.
  \item exception: \texttt{DecodeErrorException} if decoding fails.
\end{itemize}

\noindent \texttt{getConfidenceScore(altText)}:
\begin{itemize}
  \item transition: None.
  \item output: Returns confidence score in range [0,1].
  \item exception: None.
\end{itemize}

\subsubsection{Local Functions}
None.



\wss{A module without environment variables or state variables is unlikely to
  have a state transition.  In this case a state transition can only occur if
the module is changing the state of another module.}

\wss{Modules rarely have both a transition and an output.  In most cases you
will have one or the other.}

\subsubsection{Local Functions}

\wss{As appropriate} \wss{These functions are for the purpose of specification.
  They are not necessarily something that is going to be implemented
  explicitly.  Even if they are implemented, they are not exported; they only
  have local scope.}
  \section{MIS of Authentication Service} \label{Module:AuthenticationService}

  \subsection{Module}
  
  \texttt{AuthenticationService}
  
  \subsection{Uses}
  This module is responsible for managing and validating user authentication across the Reading4All system. It ensures that only verified users can upload, edit, or retrieve images and generated alt text.  
  \begin{itemize}
    \item Validates user session tokens received from frontend modules.
    \item Manages the current logged-in user’s credentials and authentication status.
    \item Communicates with \texttt{SessionManagement} to track and store active sessions.
    \item Provides user authentication results to \texttt{BackendController} for access control.
  \end{itemize}
  
  \subsection{Syntax}
  
  \subsubsection{Exported Constants}
  \texttt{TOKEN\_EXPIRY\_TIME}: $\mathbb{N}$ - Duration in minutes after which an authentication token expires. \\ \\
  \texttt{MAX\_LOGIN\_ATTEMPTS}: $\mathbb{N}$ - Maximum number of failed login attempts allowed before temporary account lockout. \\
  
  \subsubsection{Exported Access Programs}
  
  \begin{center}
  \begin{tabular}{p{3cm} p{4cm} p{3cm} p{2cm}}
  \hline
  \textbf{Name} & \textbf{In} & \textbf{Out} & \textbf{Exceptions} \\
  \hline
  validateToken & userToken & Boolean & InvalidTokenException \\
  \hline
  signInUser & credentials & userToken & AuthenticationFailedException \\
  \hline
  signOutUser & userToken & Boolean & SessionNotFoundException \\
  \hline
  getCurrentUser & userToken & userID & UserNotAuthenticatedException \\
  \hline
  refreshToken & userToken & newToken & TokenRefreshException \\
  \hline
  \end{tabular}
  \end{center}
  \subsection{Semantics}
  
  \subsubsection{State Variables}
  \begin{itemize}
    \item \texttt{currentUser}: Stores information of the currently authenticated user (userID, role, status).
    \item \texttt{activeTokens}: List of active tokens issued during authenticated sessions.
  \end{itemize}
  
  \subsubsection{Environment Variables}
  \begin{itemize}
    \item Frontend Module: Receives user login and logout requests, sends authentication tokens.
    \item Session Management Module: Manages active sessions and stores validated tokens.
    \item Backend Controller Module: Requests user validation before allowing access to backend operations.
  \end{itemize}
  
  \subsubsection{Assumptions}
  \begin{itemize}
    \item The database or session store used for token verification is operational.
    \item Network connection to backend and session services is stable.
    \item All tokens follow the system-defined structure and encryption standard.
  \end{itemize}
  
  \subsubsection{Access Routine Semantics}
  
  \noindent \texttt{validateToken(userToken)}:
  \begin{itemize}
  \item transition: Compares provided token against valid session records in \texttt{SessionManagement}.
  \item output: Returns \texttt{True} if token is valid and active, otherwise \texttt{False}.
  \item exception: \texttt{InvalidTokenException} is raised if the token has expired or does not exist.
  \end{itemize}
  
  \noindent \texttt{signInUser(credentials)}:
  \begin{itemize}
  \item transition: Verifies credentials (e.g., username and password) and issues a unique token for session creation.
  \item output: Returns a valid \texttt{userToken} on successful authentication.
  \item exception: \texttt{AuthenticationFailedException} is raised if credentials are invalid or user does not exist.
  \end{itemize}
  
  \noindent \texttt{signOutUser(userToken)}:
  \begin{itemize}
  \item transition: Removes the associated token from the active token list and ends the user’s session.
  \item output: Returns \texttt{True} on successful logout.
  \item exception: \texttt{SessionNotFoundException} is raised if the token is invalid or session not found.
  \end{itemize}
  
  \noindent \texttt{getCurrentUser(userToken)}:
  \begin{itemize}
  \item transition: Queries active sessions to identify the user associated with the token.
  \item output: Returns \texttt{userID} if token corresponds to an active session.
  \item exception: \texttt{UserNotAuthenticatedException} is raised if no active session exists for the token.
  \end{itemize}
  
  \noindent \texttt{refreshToken(userToken)}:
  \begin{itemize}
  \item transition: Reissues a new token with an updated expiry time after verifying the old token’s validity.
  \item output: Returns \texttt{newToken}.
  \item exception: \texttt{TokenRefreshException} is raised if old token is expired or tampered.
  \end{itemize}
  
  \subsubsection{Local Functions}
  \begin{itemize}
    \item \texttt{generateToken(userID)} — Creates a secure, encrypted token for the authenticated user.
    \item \texttt{encryptData(data)} — Encrypts sensitive authentication information before storage.
    \item \texttt{validateCredentials(credentials)} — Confirms user credentials against secure storage.
  \end{itemize}

  \section{MIS of Session Management} \label{Module:SessionManagement}

  \subsection{Module}
  \texttt{SessionManagement}
  
  \subsection{Uses}
  The Session Management module is responsible for creating, storing, and maintaining active user sessions within the Reading4All system. It also tracks the session-specific history of user interactions, referencing corresponding entries in the Reading4All data record. Other modules, such as the Authentication Service and Backend Controller, rely on it for validating active tokens and retrieving user activity logs within the current session.
  \begin{itemize}
      \item Creates new sessions upon successful authentication.
      \item Stores mappings between users and their associated session tokens.
      \item Tracks and records user interactions (e.g., uploads, alt-text generation) linked to Reading4All data records.
      \item Provides access to current session history and recent data records.
      \item Invalidates sessions on logout or timeout and clears associated session history.
  \end{itemize}
  
  \subsection{Syntax}

  \subsubsection{Exported Constants}
  \texttt{MAX\_SESSION\_DURATION}: $\mathbb{N}$ - Specifies the maximum duration, in minutes, that a session remains active before expiration. \\ \\
  \texttt{SESSION\_REFRESH\_INTERVAL}: $\mathbb{N}$ - The interval, in minutes, at which session validity is checked or refreshed. \\ \\
  \texttt{MAX\_SESSION\_HISTORY}: $\mathbb{N}$ - Maximum number of interaction entries stored per session to prevent unbounded growth. \\
  
  \subsubsection{Exported Access Programs}
  
  \begin{center}
  \begin{tabular}{p{3.3cm} p{4.5cm} p{3cm} p{3cm}}
  \hline
  \textbf{Name} & \textbf{In} & \textbf{Out} & \textbf{Exceptions} \\
  \hline
  createSession & userID & sessionToken & SessionCreationException \\
  \hline
  storeSession & userID, sessionToken & Boolean & SessionStorageException \\
  \hline
  validateSession & userToken & Boolean & InvalidSessionException \\
  \hline
  deleteSession & userToken & Boolean & SessionNotFoundException \\
  \hline
  getSession & userToken & sessionData & SessionNotFoundException \\
  \hline
  appendInteraction & userToken, interactionEntry & Boolean & SessionNotFoundException \\
  \hline
  getSessionHistory & userToken, filter & interactionList & SessionNotFoundException \\
  \hline
  linkDataRecord & userToken, dataRecordID, metadata & Boolean & SessionNotFoundException \\
  \hline
  getRecentRecords & userToken, limit & dataRecordRefs & SessionNotFoundException \\
  \hline
  \end{tabular}
  \end{center}
  
  \subsection{Semantics}

  \subsubsection{State Variables}
  \begin{itemize}
      \item \texttt{activeSessions}: Mapping of \texttt{userID} $\rightarrow$ \texttt{sessionToken} representing currently active sessions.
      \item \texttt{sessionExpiry}: Mapping of \texttt{sessionToken} $\rightarrow$ expiry time.
      \item \texttt{sessionHistory}: Mapping of \texttt{sessionToken} $\rightarrow$ ordered list of interaction entries for that session.
      \item \texttt{sessionDataRefs}: Mapping of \texttt{sessionToken} $\rightarrow$ set of dataRecordIDs referencing Reading4All data entries.
  \end{itemize}
  
  \subsubsection{Environment Variables}
  \begin{itemize}
      \item \textbf{Authentication Service:} Initiates creation, validation, and deletion of sessions.
      \item \textbf{Backend Controller:} Updates session history after each major user action (e.g., validateImage, sendToModel, getAltText).
      \item \textbf{Reading4All Data:} Serves as the persistent storage layer for user images and generated alt-text records, referenced by session data IDs.
  \end{itemize}
  
  \subsubsection{Assumptions}
  \begin{itemize}
      \item The system clock is synchronized to ensure accurate session expiry times.
      \item Tokens and user IDs are unique and securely generated.
      \item A fast-access storage service (e.g., in-memory cache or Redis) is available for storing session and interaction data.
      \item Reading4All data provides stable and retrievable \texttt{dataRecordID}s for cross-module referencing.
  \end{itemize}
  
  \subsubsection{Access Routine Semantics}
  
  \noindent \texttt{createSession(userID)}:
  \begin{itemize}
      \item transition: Generates a new session token, initializes empty history and data references, and sets the expiry time.
      \item output: Returns the created \texttt{sessionToken}.
      \item exception: Raises \texttt{SessionCreationException} if session creation fails.
  \end{itemize}
  
  \noindent \texttt{validateSession(userToken)}:
  \begin{itemize}
      \item transition: None.
      \item output: Returns \texttt{True} if the session exists and is active, otherwise \texttt{False}.
      \item exception: Raises \texttt{InvalidSessionException} if the token is expired or invalid.
  \end{itemize}
  
  \noindent \texttt{appendInteraction(userToken, interactionEntry)}:
  \begin{itemize}
      \item transition: Appends the provided \texttt{interactionEntry} (with timestamp and type) to \texttt{sessionHistory[userToken]}. If the maximum history size is exceeded, the oldest entry is removed.
      \item output: Returns \texttt{True} upon success.
      \item exception: Raises \texttt{SessionNotFoundException} if the token is not associated with an active session.
  \end{itemize}
  
  \noindent \texttt{linkDataRecord(userToken, dataRecordID, metadata?)}:
  \begin{itemize}
      \item transition: Associates the provided \texttt{dataRecordID} (from Reading4All data) with the user’s current session for future retrieval.
      \item output: Returns \texttt{True} upon success.
      \item exception: Raises \texttt{SessionNotFoundException} if no active session exists.
  \end{itemize}
  
  \noindent \texttt{getSessionHistory(userToken, filter?)}:
  \begin{itemize}
      \item transition: None.
      \item output: Returns a filtered list of interaction entries (e.g., uploads, alt-text retrieval) for the specified session.
      \item exception: Raises \texttt{SessionNotFoundException} if the token is invalid or no session history exists.
  \end{itemize}
  
  \noindent \texttt{getRecentRecords(userToken, limit)}:
  \begin{itemize}
      \item transition: None.
      \item output: Returns up to the specified number of recent \texttt{dataRecordID} references associated with the current session.
      \item exception: Raises \texttt{SessionNotFoundException} if no active session exists.
  \end{itemize}
  
  \noindent \texttt{deleteSession(userToken)}:
  \begin{itemize}
      \item transition: Removes the token from active sessions and clears its associated history and data references.
      \item output: Returns \texttt{True} if deletion is successful.
      \item exception: Raises \texttt{SessionNotFoundException} if the token is invalid or not found.
  \end{itemize}
  
  \subsubsection{Local Functions}
  \begin{itemize}
      \item \texttt{generateSessionToken(userID)}: Creates a secure, unique token for the given user.
      \item \texttt{updateExpiry(sessionToken)}: Extends the expiry time of a session on continued activity.
      \item \texttt{cleanExpiredSessions()}: Removes all expired sessions and their histories from storage.
      \item \texttt{applyHistoryFilter(history, filter)}: Applies filtering parameters (e.g., type, time range, status) to session history queries.
  \end{itemize}

\section{MIS of Image Validation Module } \label{ImageValid}

\wss{You can reference SRS labels, such as R\ref{R_Inputs}.}

\wss{It is also possible to use \LaTeX for hypperlinks to external documents.}

\subsection{Module}
  \texttt{ImageValidation}

\subsection{Uses}
This module is responsible for verifying that uploaded images meet the systems requirements for format and size before the alt-text generation process begins. 

\subsection{Syntax}

\subsubsection{Exported Constants}
  \texttt{MAX\_IMAGE\_SIZE}: $\mathbb{N}$ - Specifies the maximum file size in (MB) allowed for uploaded images.\\
  \texttt{MIN\_IMAGE\_SIZE}: $\mathbb{N}$ - Specifies the minimum file size in (MB) allowed for uploaded images.\\
  \texttt{ACCEPTED\_IMAGE\_TYPES}: \texttt{List of Strings}- Specifies the accepted image file types. 
\subsubsection{Exported Access Programs}

\begin{center}
\begin{tabular}{p{4 cm} p{4cm} p{2cm} p{4cm}}
\hline
\textbf{Name} & \textbf{In} & \textbf{Out} & \textbf{Exceptions} \\
\hline
validateImageFile & image, userToken & \texttt{boolean} & InvalidImageTypeException \\
\hline
\end{tabular}
\end{center}

\subsection{Semantics}

\subsubsection{State Variables}
None

\subsubsection{Environment Variables}
\begin{itemize}
  \item \textbf{Backend Controller:} Sends uploaded images to be validated and receives the results after validation is completed. 
\end{itemize}



\subsubsection{Assumptions}

\wss{Try to minimize assumptions and anticipate programmer errors via
  exceptions, but for practical purposes assumptions are sometimes appropriate.}

  \begin{itemize}
  \item The Backend Controller module provides the Image Validation module with an image file that is reachable for validation. 
  \item The user has successfully been validated through the user authentication process. 
\end{itemize}
\subsubsection{Access Routine Semantics}

\noindent \texttt{validateImageFile(image, userToken)}
\begin{itemize}
\item transition: Verifies that the inputted image meets size and format requirements.
\item output: Returns \texttt{True} if the image meets size and format requirements, otherwise \texttt{False} is returned.
\item exception: \texttt{InvalidImageException} is raised if the image cannot be reached and therefore cannot be validated. 
\end{itemize}

\wss{A module without environment variables or state variables is unlikely to
  have a state transition.  In this case a state transition can only occur if
  the module is changing the state of another module.}

\wss{Modules rarely have both a transition and an output.  In most cases you
  will have one or the other.}

\subsubsection{Local Functions}
\begin{itemize}
  \item \texttt{verifyFileType(image)}: Checks that the images file type matches a type found in \texttt{ACCEPTED\_IMAGE\_TYPES}.
  \item \texttt{verifyFileSize(image)}: Checks that the images size falls between \texttt{MIN\_IMAGE\_SIZE} and \texttt{MAX\_IMAGE\_SIZE}.
  \item \texttt{verifyFileReachable(image)}: Checks that the image can be accessed and opened successfully. 
\end{itemize}


\section{MIS of Logger Module} \label{LogModule} \wss{Use labels for
  cross-referencing}

\wss{You can reference SRS labels, such as R\ref{R_Inputs}.}

\wss{It is also possible to use \LaTeX for hypperlinks to external documents.}

\subsection{Module}
\texttt{Logger}

\subsection{Uses}
The module handles the recording of key system events and any errors encountered. It logs actions such as image uploads,alt-text generation and user validations, as well as system issues like timeouts. This logging will enables developers to debug for effectively and ensure system reliability and security during failures. 

\subsection{Syntax}

\subsubsection{Exported Constants}
  \texttt{LOG\_FILE\_PATH}: String - Specifies the location of where the logs should be stored.\\
  \texttt{LOG\_TYPE}: List of Strings - Specifies the possible log types such as an \texttt{Event} and \texttt{Error}.\\

\subsubsection{Exported Access Programs}

\begin{center}
\begin{tabular}{p{2cm} p{6cm} p{2cm} p{4cm}}
\hline
\textbf{Name} & \textbf{In} & \textbf{Out} & \textbf{Exceptions} \\
\hline
logEvent & eventMessage, logType & None & LogWriteException\\
\hline
\end{tabular}
\end{center}

\subsection{Semantics}

\subsubsection{State Variables}

None

\subsubsection{Environment Variables}

\wss{This section is not necessary for all modules.  Its purpose is to capture
  when the module has external interaction with the environment, such as for a
  device driver, screen interface, keyboard, file, etc.}
\begin{itemize}
  \item The Backend Controller Module sends any key system events and error messages to be logged. 
  \item File System is used to store the logs written by this module.
\end{itemize}

\subsubsection{Assumptions}

\wss{Try to minimize assumptions and anticipate programmer errors via
  exceptions, but for practical purposes assumptions are sometimes appropriate.}
\begin{itemize}
  \item The file system is available and logs can be saved to it. 
  \item The LOG\_FILE\_PATH location can be accessed. 
  \item Log messages are provided in string format by the Backend Controller Module.
\end{itemize}
\subsubsection{Access Routine Semantics}

\noindent \texttt{logEvent(eventMessage, logType)}:
\begin{itemize}
\item transition: Creates an log entry containing the time of event occur, type and message. 
\item output: An updated log file with new event or error entry added. 
\item exception: \texttt{LogWriteException} is raised if the log file cannot be accessed or written to. 
\end{itemize}

\wss{A module without environment variables or state variables is unlikely to
  have a state transition.  In this case a state transition can only occur if
  the module is changing the state of another module.}

\wss{Modules rarely have both a transition and an output.  In most cases you
  will have one or the other.}

\subsubsection{Local Functions}

\wss{As appropriate} \wss{These functions are for the purpose of specification.
  They are not necessarily something that is going to be implemented
  explicitly.  Even if they are implemented, they are not exported; they only
  have local scope.}


\section{MIS of User Interface Interactions Module}\label{UI_Interactions}

\subsection{Module}

interactionsUI

\subsection{Uses}
BackendController \\
mainScreen \\ 
logInScreen \\ 
showHistory 

\subsection{Syntax}

\subsubsection{Exported Constants}

\texttt{SUPPORTED\_IMG\_TYPES = \{"PNG", "JPEG"\}}\\
\texttt{OUTPUT\_FILE\_TYPE = \{".txt"\}}\\
\texttt{MAX\_IMG\_SIZE = 10MB}

\subsubsection{Exported Access Programs}

\begin{center}
\begin{tabular}{p{2.5cm} p{4cm} p{4cm} p{3cm}}
\hline
\textbf{Name} & \textbf{In} & \textbf{Out} & \textbf{Exceptions} \\
\hline
uploadImage & imageFilePath:String & Success:\Bool & FileNotFound, FileTypeNotSupported, FileTooLarge\\
downloadText & altText:String & altTextFile:String & NetworkError \\
copyText & altText:String & Success:\Bool & ClipboardError \\
editText & altText:String, editedText:String & Success:\Bool & NetworkError \\
signIn & token:String & Success:\(\mathbb{B}\) & AuthError \\
signOut & - & Success:\(\mathbb{B}\) & AuthError \\
showHistory & - & - & NetworkError \\

\hline
\end{tabular}
\end{center}

\subsection{Semantics}

\subsubsection{State Variables}

None

\subsubsection{Environment Variables}

\begin{itemize}
  \item Clipboard - system's clipboard interface for copy function.
  \item HTTPClient - used to call backend services.
  \item Keyboard - receives key presses for keyboard navigation. 
\end{itemize}


\subsubsection{Assumptions}
  
  \begin{itemize}
    \item Clipboard access is enabled by the user for \texttt{copyText}.
    \item User allows system to access files for uploading images and downloading alt text. 
    \item \texttt{uploadImage} is called before any other access program.
  \end{itemize}


\subsubsection{Access Routine Semantics}

\noindent \textbf{uploadImage(imageFilePath)}:
\begin{itemize}
\item transition: \texttt{HTTPClient} sends the image bytes into the backend service which stores the image for that session, and communicates 
with the AI model for alt text generation. 
\item output: returns \texttt{true} iff the backend server acknowledges and is able to process the image, returns \texttt{false} otherwise.
\item exception: FileNotFound, FileTypeNotSupported (not in \texttt{SUPPORT\_IMG\_TYPES}), and FileTooLarge (not within \texttt{MAX\_IMG\_SIZE})
\end{itemize}

\noindent \textbf{downloadText(altText)}:
\begin{itemize}
\item transition: None (read-only fetch).
\item output: a \texttt{.txt} payload that contains the generated alt text. 
\item exception: NetworkError.
\end{itemize}

\noindent \textbf{copyText(altText)}:
\begin{itemize} 
\item transition: Clipboard := altText
\item output: returns \texttt{true} iff clipboard write succeeds, returns \texttt{false} otherwise. 
\item exception: ClipboardError.
\end{itemize}

\noindent \textbf{editText(altText,editedText)}:
\begin{itemize}
\item transition: \texttt{HTTPClient} sends update to backend and stored altText for uploaded image is replaced with editedText.
\item output: return \texttt{true} iff server acknowledges the update, returns \texttt{false} otherwise.
\item exception: NetworkError.
\end{itemize}

\noindent \textbf{signIn(token)}:
\begin{itemize}
\item transition: if \texttt{token} is valid according to BackendController then user session becomes active.
Otherwise no state change.
\item output: return \texttt{true} iff user session is active, returns \texttt{false} otherwise.
\item exception: Authentication Error.
\end{itemize}

\noindent \textbf{signOut()}:
\begin{itemize}
\item transition: terminates current active user session if any exists.
Otherwise no state change.
\item output: return \texttt{true} iff user session is was terminated, returns \texttt{false} otherwise.
\item exception: Authentication Error.
\end{itemize}

\noindent \textbf{showHistory()}:
\begin{itemize}
\item transition: requests the showHistoryScreen module to render the History Screen using the current stored alt text session history.
\item output: None.
\item exception: Authentication Error.
\end{itemize}

\subsubsection{Local Functions}

\begin{itemize}
  \item \textbf{isSupportedImage(filePath:String) \texttt{->} \Bool}: returns \texttt{true} if 
  the file type $\in$ \texttt{SUPPORTED\_IMG\_TYPES}.
  \item \textbf{checkImgSize(filePath:String) \texttt{->} \Bool}: returns \texttt{true} if 
  the size of the image $\le$ \texttt{MAX\_IMG\_SIZE}.
  \item \textbf{saveAsTxt(altText: String) \texttt{->} filePath:String}: wraps generated alt text
   into a \texttt{.txt} file for \texttt{downloadText()}.
  \item \textbf{buildUploadPayload(filePath: String) \texttt{->} Binary}: reads the file and constructs the binary payload for the 
  \texttt{uploadImage} function to send to the BackendController module.
\end{itemize}


\section{MIS of Main Screen Module}\label{mainScreen} 

\subsection{Module}

mainScreen

\subsection{Uses}

accessibilityUI\\
interactionsUI

\subsection{Syntax}

\subsubsection{Exported Constants}
None

\subsubsection{Exported Access Programs}

\begin{center}
\begin{tabular}{p{2.5cm} p{4cm} p{4cm} p{3cm}}
\hline
\textbf{Name} & \textbf{In} & \textbf{Out} & \textbf{Exceptions} \\
\hline
renderMain & - & - & RenderError, AccessibilityError \\
\hline
\end{tabular}
\end{center}

\subsection{Semantics}

\subsubsection{State Variables}

None

\subsubsection{Environment Variables}

\texttt{DOM} - active document element into which UI is rendered.

\subsubsection{Assumptions}

\begin{itemize}
  \item Accessibility services are provided by the \texttt{accessibilityUI} module for screen reader announcements,
  landmark management, initial focus, and keyboard navigation helpers. This module will provide notifications to
  the \texttt{accessibilityUI} module. 
\end{itemize}
\subsubsection{Access Routine Semantics}

\noindent \textbf{renderMain()}:
\begin{itemize}
  \item transition: update DOM to reflect main screen containing the upload page, then call on \texttt{accessibilityUI} to 
  handle compatibility with screen readers and keyboard navigation.
  \item output: None
  \item exception: Render Error, Accessibility Error.
\end{itemize}


\subsubsection{Local Functions}

\begin{itemize}
  \item \textbf{ensureMainLandmark()  \texttt{->} \Bool}: calls \texttt{accessibilityUI}
  to mark/set the main landmark.
  \item \textbf{setInitialFocus()\texttt{->} \Bool}: calls \texttt{accessibilityUI}
   to move focus to the main region.
  \item \textbf{announceMainLoaded() \texttt{->} \Bool}: calls \texttt{accessibilityUI}
   to announce screen load.

\end{itemize}

\section{MIS of Show History Screen Module}\label{showHistory} 

\subsection{Module}

showHistoryScreen

\subsection{Uses}

accessibilityUI\\
interactionsUI

\subsection{Syntax}

\subsubsection{Exported Constants}
None

\subsubsection{Exported Access Programs}

\begin{center}
\begin{tabular}{p{2.5cm} p{4cm} p{4cm} p{3cm}}
\hline
\textbf{Name} & \textbf{In} & \textbf{Out} & \textbf{Exceptions} \\
\hline
showHistory & - & - & RenderError, AccessibilityError\\
\hline
\end{tabular}
\end{center}

\subsection{Semantics}

\subsubsection{State Variables}

None

\subsubsection{Environment Variables}

\texttt{DOM} - active document element into which UI is rendered.

\subsubsection{Assumptions}

\begin{itemize}
  \item If history is shown, a valid and authenticated session already exists.
  \item \texttt{accessibilityUI} is used for announcements and focus.
  \item History content to display is already prepared by the \texttt{interactionsUI} module through
  the \texttt{BackendController} module.
\end{itemize}

\subsubsection{Access Routine Semantics}

\noindent \textbf{showHistory()}:
\begin{itemize} 
  \item transition: update DOM to reflect user's current session history, then call on \texttt{accessibilityUI} to 
  handle compatibility with screen readers and keyboard navigation.
  \item output: None
  \item exception: Render Error, Accessibility Error.
\end{itemize}

\subsubsection{Local Functions}

\begin{itemize}
  \item \textbf{ensureMainLandmark()  \texttt{->} \Bool}: calls \texttt{accessibilityUI}
  to mark/set the main landmark.
  \item \textbf{setInitialFocus()\texttt{->} \Bool}: calls \texttt{accessibilityUI}
   to move focus to the main region.
  \item \textbf{announceHistoryLoaded() \texttt{->} \Bool}: calls \texttt{accessibilityUI}
   to announce history screen load.
\end{itemize}

\section{MIS of Log In Screen Module}\label{logInScreen} 


\subsection{Module}

logInScreen

\subsection{Uses}

accessibilityUI\\
interactionsUI

\subsection{Syntax}

\subsubsection{Exported Constants}
None

\subsubsection{Exported Access Programs}

\begin{center}
\begin{tabular}{p{2.5cm} p{4cm} p{4cm} p{3cm}}
\hline
\textbf{Name} & \textbf{In} & \textbf{Out} & \textbf{Exceptions} \\
\hline
renderLogIn & - & - & RenderError, AccessibilityError \\
\hline
\end{tabular}
\end{center}

\subsection{Semantics}

\subsubsection{State Variables}

None

\subsubsection{Environment Variables}

\texttt{DOM} - active document element into which UI is rendered.

\subsubsection{Assumptions}

\begin{itemize}
  \item \texttt{accessibilityUI} is used for announcements and focus.
\end{itemize}

\subsubsection{Access Routine Semantics}


\noindent \textbf{renderLogIn()}:
\begin{itemize}
\item transition: update DOM to reflect log in screen with username and password fields, 
then call on \texttt{accessibilityUI} to 
handle compatibility with screen readers and keyboard navigation.
\item output: None 
\item exception: Render Error, Authentication Error.
\end{itemize}

\subsubsection{Local Functions}
\begin{itemize}
  \item \textbf{ensureMainLandmark()  \texttt{->} \Bool}: calls \texttt{accessibilityUI}
  to mark/set the main landmark.
  \item \textbf{setInitialFocus(username:String)\texttt{ -> }\Bool}:
  calls \texttt{accessibilityUI} to move keyboard focus to the username
  input field after the log in screen is rendered.
  \item \textbf{announceLogInLoaded() \texttt{->} \Bool}: calls \texttt{accessibilityUI}
   to announce log in screen load.
\end{itemize}

\section{MIS of User Interface Accessibility Module}\label{UI_accessibility}
  
  \subsection{Module}
  
  accessibilityUI
  
  \subsection{Uses}
  mainScreen\\
  logInScreen\\
  showHistoryScreen
  \subsection{Syntax}
  
  \subsubsection{Exported Constants}
  
  \subsubsection{Exported Access Programs}
  
  \begin{center}
  \begin{tabular}{p{4cm} p{4cm} p{4cm} p{2cm}}
  \hline
  \textbf{Name} & \textbf{In} & \textbf{Out} & \textbf{Exceptions} \\
  \hline
  ensureMainLandmark & rootSel:String & Success:\(\mathbb{B}\) & AccessibilityError\\
  setInitialFocus & targetSel:String & Success:\(\mathbb{B}\) & AccessibilityError\\
  announce & msg:String & Success:\(\mathbb{B}\) & AccessibilityError\\
  applyKeyboardNav & scopeSel:String & Success:\(\mathbb{B}\) & AccessibilityError\\
  \hline
  \end{tabular}
  \end{center}
  
  \subsection{Semantics}
  
  \subsubsection{State Variables}

  None
  
  \subsubsection{Environment Variables}
    \texttt{DOM} - the active HTML document containing interactive elements \\ 
    \texttt{ScreenReaderAPI} - hidden live region nodes used to deliver announcements to screen readers 
    for compatibility\\
    \texttt{Keyboard} - delivers keypress events to use for keyboard navigation\\
  
  \subsubsection{Assumptions}

    \begin{itemize}
      \item The page allows insertion of valid ARIA live region nodes
      \item \texttt{mainScreen}, \texttt{logInScreen}, \texttt{showHistoryScreen} modules calls \texttt{accessibilityUI} 
      to announce changes on the user interface. 
    \end{itemize}
  
  \subsubsection{Access Routine Semantics}
  
  \noindent \textbf{ensureMainLandmark(rootSel:String)}:
  \begin{itemize}
  \item transition: ensure that the element selected by \texttt{rootSel} is the page’s 
  main landmark (i.e. \texttt{<main>}).
  \item output: returns \texttt{true} if main landmark was present after the call, returns \texttt{false}
  otherwise.
  \item exception: AccessibilityError.
  \end{itemize}

  \noindent \textbf{setInitialFocus(targetSel:String)}:
  \begin{itemize}
  \item transition: move keyboard focus to the element matching \texttt{targetSel} for keyboard navigation; 
  if not focusable, make it focusable for the interaction.
  \item output: returns \texttt{true} if focus was moved, returns \texttt{false}
  otherwise.
  \item exception: AccessibilityError.
  \end{itemize}

  \noindent \textbf{announce(msg:String)}:
  \begin{itemize}
  \item transition: update the ARIA live region with \texttt{msg} 
  so screen readers can announce the change.
  \item output: returns \texttt{true} if live region was updated, returns \texttt{false}
  otherwise.
  \item exception: AccessibilityError.
  \end{itemize}

  \noindent \textbf{applyKeyboardNav(scopeSel:String)}:
  \begin{itemize}
  \item transition: attach handlers within \texttt{scopeSel} (container that gets arrow-key navigation behaviour) 
  to enable roving-tabindex and arrow-key navigation (i.e., ArrowLeft/Right/Up/Down, Home/End, Enter/Space activation) 
  without changing application state.
  \item output: returns \texttt{true} if handlers are attached successfully, returns \texttt{false}
  otherwise.
  \item exception: AccessibilityError.
  \end{itemize}
  
  
  \subsubsection{Local Functions}
  \begin{itemize}
    \item \textbf{getLiveRegion(type:String)  \texttt{->} Node}: returns the hidden live region for announcements;
    calls on \texttt{createLiveRegion(type: String)} if no live region to create one.
    \item \textbf{createLiveRegion(type:String) \texttt{->} Node}: creates and returns the hidden live region for announcements.
    \item \textbf{isFocusable(node:Node) \texttt{ -> }\Bool}: returns \texttt{true} if \texttt{node} is focusable. 
    \item \textbf{applyRovingTabindex(container:Node) \texttt{->} \Bool}: ensures exactly one child is tabbable 
    and others are reachable via arrow keys.
    \item \textbf{moveFocus(container:Node, dir:String) \texttt{->} \Bool}: shifts focus to the next or previous roving 
    item given \texttt{dir} $\in$ \texttt{{ "next","prev","home","end" }}.
  \end{itemize}


\newpage

\bibliographystyle {plainnat}
\bibliography {../../../refs/References}

\newpage

\section{Appendix} \label{Appendix}

\wss{Extra information if required}

\newpage{}

\section*{Appendix --- Reflection}

\wss{Not required for CAS 741 projects}

The information in this section will be used to evaluate the team members on the
graduate attribute of Problem Analysis and Design.

\input{../../Reflection.tex}

\begin{enumerate}
  \item What went well while writing this deliverable?
  \item What pain points did you experience during this deliverable, and how
    did you resolve them?
  \item Which of your design decisions stemmed from speaking to your client(s)
    or a proxy (e.g. your peers, stakeholders, potential users)? For those that
    were not, why, and where did they come from?
  \item While creating the design doc, what parts of your other documents (e.g.
    requirements, hazard analysis, etc), it any, needed to be changed, and why?
  \item What are the limitations of your solution?  Put another way, given
    unlimited resources, what could you do to make the project
    better? (LO\_ProbSolutions)
  \item Give a brief overview of other design solutions you considered.  What
    are the benefits and tradeoffs of those other designs compared
    with the chosen
    design?  From all the potential options, why did you select the
    documented design?
    (LO\_Explores)
\end{enumerate}


\end{document} 
