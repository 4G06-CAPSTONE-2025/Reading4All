\documentclass[12pt, titlepage]{article}

\usepackage{booktabs}
\usepackage{tabularx}
\usepackage{hyperref}
\usepackage{float}

\hypersetup{
    colorlinks,
    citecolor=blue,
    filecolor=black,
    linkcolor=red,
    urlcolor=blue
}
\usepackage[round]{natbib}

%% Comments

\usepackage{color}

\newif\ifcomments\commentstrue %displays comments
%\newif\ifcomments\commentsfalse %so that comments do not display

\ifcomments
\newcommand{\authornote}[3]{\textcolor{#1}{[#3 ---#2]}}
\newcommand{\todo}[1]{\textcolor{red}{[TODO: #1]}}
\else
\newcommand{\authornote}[3]{}
\newcommand{\todo}[1]{}
\fi

\newcommand{\wss}[1]{\authornote{magenta}{SS}{#1}} 
\newcommand{\plt}[1]{\authornote{cyan}{TPLT}{#1}} %For explanation of the template
\newcommand{\an}[1]{\authornote{cyan}{Author}{#1}}

%% Common Parts

\newcommand{\progname}{ProgName} % PUT YOUR PROGRAM NAME HERE
\newcommand{\authname}{Team \#, Team Name
\\ Student 1 name
\\ Student 2 name
\\ Student 3 name
\\ Student 4 name} % AUTHOR NAMES                  

\usepackage{hyperref}
    \hypersetup{colorlinks=true, linkcolor=blue, citecolor=blue, filecolor=blue,
                urlcolor=blue, unicode=false}
    \urlstyle{same}
                                


\begin{document}

\title{System Verification and Validation Plan for \progname{}} 
\author{\authname}
\date{\today}
	
\maketitle

\pagenumbering{roman}

\section*{Revision History}

\begin{tabularx}{\textwidth}{p{3cm}p{2cm}X}
\toprule {\bf Date} & {\bf Version} & {\bf Notes}\\
\midrule
Date 1 & 1.0 & Notes\\
Date 2 & 1.1 & Notes\\
\bottomrule
\end{tabularx}

~\\
\wss{The intention of the VnV plan is to increase confidence in the software.
However, this does not mean listing every verification and validation technique
that has ever been devised.  The VnV plan should also be a \textbf{feasible}
plan. Execution of the plan should be possible with the time and team available.
If the full plan cannot be completed during the time available, it can either be
modified to ``fake it'', or a better solution is to add a section describing
what work has been completed and what work is still planned for the future.}

\wss{The VnV plan is typically started after the requirements stage, but before
the design stage.  This means that the sections related to unit testing cannot
initially be completed.  The sections will be filled in after the design stage
is complete.  the final version of the VnV plan should have all sections filled
in.}

\newpage

\tableofcontents

\listoftables
\wss{Remove this section if it isn't needed}

\listoffigures
\wss{Remove this section if it isn't needed}

\newpage

\section{Symbols, Abbreviations, and Acronyms}

\renewcommand{\arraystretch}{1.2}
\begin{tabular}{l l} 
  \toprule		
  \textbf{symbol} & \textbf{description}\\
  \midrule 
  T & Test\\
  \bottomrule
\end{tabular}\\

\wss{symbols, abbreviations, or acronyms --- you can simply reference the SRS
  \citep{SRS} tables, if appropriate}

\wss{Remove this section if it isn't needed}

\newpage

\pagenumbering{arabic}

This document ... \wss{provide an introductory blurb and roadmap of the
  Verification and Validation plan}

\section{General Information}

\subsection{Summary}

\wss{Say what software is being tested.  Give its name and a brief overview of
  its general functions.}

The software being tested is called Reading4All. This software will utilize artificial intelligence (AI)/
machine learning (ML) techniques to provide detailed, context-informed alternative text for complex technical images, specifically those found in post-secondary Science, Technology, Engineering and Mathematics (STEM)
course materials. Reading4All will allow users to upload images and automatically produce corresponding alternative text, that meet the outlined criteria. The system is intended for use by McMaster University students and faculty, therefore it will include user validation through McMaster's sign-on, ensuring that only verified users can access the Reading4All system.  
In addition, the system will allow users to edit the generated alternative text, view a history of uploaded images and their alternative text within that session and download the final outputs in their desired formats.


\subsection{Objectives}

\wss{State what is intended to be accomplished.  The objective will be around
  the qualities that are most important for your project.  You might have
  something like: ``build confidence in the software correctness,''
  ``demonstrate adequate usability.'' etc.  You won't list all of the qualities,
  just those that are most important.}


\wss{You should also list the objectives that are out of scope.  You don't have 
the resources to do everything, so what will you be leaving out.  For instance, 
if you are not going to verify the quality of usability, state this.  It is also 
worthwhile to justify why the objectives are left out.}

\wss{The objectives are important because they highlight that you are aware of 
limitations in your resources for verification and validation.  You can't do everything, 
so what are you going to prioritize?  As an example, if your system depends on an 
external library, you can explicitly state that you will assume that external library 
has already been verified by its implementation team.}


The objective of this Verification and Validation (VnV) plan is to build confidence in the correctness, accessibility and usability of the Reading4All system. 
The plan focuses on ensuring that the system generates, accurate, detailed and contextually appropriate alternative text for complex STEM images, as this is the main functionality of the software. 
It also aims to verify that the systems interface is accessible an usable for individuals with visual impairments, who are one of the main users of the software. The last object is to demonstrate effective and accessible usability in secondary features
such as editing the outputted alternative text, viewing session history and file downloading. 
Verifying and validating these objectives is essential to ensure that users can benefit from the Reading4All system and 
it can effectively fulfill its goal of making STEM diagrams more accessible.
\\

Some objectives are out of scope from this VnV plan due to the time and resource limitations of the project.
The external libraries that might be used in the development of the system, including PyTorch, TensorFlow, Scikit-Learn, Pandas and frontend frameworks, will not be verified by our team, as they will be assumed to have been tested and validated by their implementation teams. 
The McMaster sign-on authentication service will also not be tested, as its maintained and run by the university.

\subsection{Challenge Level and Extras}

\wss{State the challenge level (advanced, general, basic) for your project.
Your challenge level should exactly match what is included in your problem
statement.  This should be the challenge level agreed on between you and the
course instructor.  You can use a pull request to update your challenge level
(in TeamComposition.csv or Repos.csv) if your plan changes as a result of the
VnV planning exercise.}

\wss{Summarize the extras (if any) that were tackled by this project.  Extras
can include usability testing, code walkthroughs, user documentation, formal
proof, GenderMag personas, Design Thinking, etc.  Extras should have already
been approved by the course instructor as included in your problem statement.
You can use a pull request to update your extras (in TeamComposition.csv or
Repos.csv) if your plan changes as a result of the VnV planning exercise.}
\\
The challenge level of the project is set at a \textit{general} level that will include two additional 
components (extras). The first additional component will be a Norman's Principles report that will 
evaluate the design of our final product to ensure that it optimizes usability and accessibility. The second additional 
component will be a user manual that will include comprehensive documentation to guide users in using the final 
product effectively. 

\subsection{Relevant Documentation}

\wss{Reference relevant documentation.  This will definitely include your SRS
  and your other project documents (design documents, like MG, MIS, etc).  You
  can include these even before they are written, since by the time the project
  is done, they will be written.  You can create BibTeX entries for your
  documents and within those entries include a hyperlink to the documents.}

\wss{Don't just list the other documents.  You should explain why they are relevant and 
how they relate to your VnV efforts.} 
\\
The System Verification and Validation (VnV) plan will reference the following documents to aid in 
the project's assessment and testing:
\begin{enumerate}
  \item \textbf{Software Requirements Specification} (\citet{SRS}): This document outlines the key components 
for the VnV plan as it details the functional and non-functional requirements of the product. Ensuring that our testing 
satisfies these requirements is essential to meeting the goals of the project.
  \item \textbf{Design Document - Module Guide} (\citet{MG}): This document outlines how the system is divided into separate module and their respective functions. 
  This structure helps the VnV Plan by making it easier to test, trace, and confirm that each module works correctly and meets the requirements.
  \item \textbf{Design Document - Module Interface Specification} (\citet{MIS}): This document outlines how the modules of the system works and how they interact with each other.
  This aids in our VnV plan as it defines how to validate the testing of interactions between the individual parts of the system.
\end{enumerate}

\section{Plan}

\wss{Introduce this section.  You can provide a roadmap of the sections to
  come.}

\subsection{Verification and Validation Team}

\wss{Your teammates.  Maybe your supervisor.
  You should do more than list names.  You should say what each person's role is
  for the project's verification.  A table is a good way to summarize this information.}

\begin{table}[H]
    \centering
    \caption{Verification and Validation Responsibility Breakdown}
    \label{tab:data-dictionary-reading4all}
    \begin{tabular}{ |p{3.0cm}|p{3.8cm}|p{7.3cm}| }
      \hline
      \textbf{Name} & \textbf{ Focus Area } & \textbf{Responsibility} \\
      \hline
      Moly Mikhail and Fiza Sehar & Functional Requirements Tester & Completes manuel testing on the software to ensure that functional requirements are met. Also oversees any written unit tests to ensure they correctly test the system and intended functionality \\
      \hline
      Nawaal Fatima & User Testing Preparation & Coordinates the user testing sessions, and oversees the planning and execution of the sessions. \\
      \hline 
      Casey Francine Bulaclac & Usability Requirements Tester & Completes manuel testing to ensure all the defined usability requirements are met. Also references WCAG 2.1 crtieria is met, prior to automated testing. \\
      \hline 
      Dhruv Sardana & Look and Feel Requirements & Evaluates the systems design and interface to ensure the outlined requreiments are met. Also references WCAG 2.1 criteria is met, prior to automated testing. \\
      \hline 
      Ms. Jingchuan Sui (Supervisor) & Overall Usability and Quality of Alternative Text Reviewer & Reviews and provides feedback to team on the overall usability of the system and the quality of the generated alternative text. Also helps team complete automating evaluation of the system against WCAG 2.1 criteria. \\ 
      \hline
    \end{tabular}
  \end{table}
  
  
\subsection{SRS Verification}

\wss{List any approaches you intend to use for SRS verification.  This may
  include ad hoc feedback from reviewers, like your classmates (like your
  primary reviewer), or you may plan for something more rigorous/systematic.}

\wss{If you have a supervisor for the project, you shouldn't just say they will
read over the SRS.  You should explain your structured approach to the review.
Will you have a meeting?  What will you present?  What questions will you ask?
Will you give them instructions for a task-based inspection?  Will you use your
issue tracker?}

The verification of the Reading4All SRS will be completed through supervisor feedback, user testing,
team review and a peer review. \\

\textbf{Team Review}:
As the Reading4All team has already reviewed the outlined requirements in the SRS document, an team review will 
consist of each team member individually going through each requirement and trying to validate it manually. 
Furthermore, the team review will also ensure that any previously determined numerical constraints are met and realistic.  \\


\textbf{Supervisor Feedback}
We will dedicate one of our weekly scheduled meetings with our supervisor to verifying our SRS document. 
We will first highlight functional and non functional requirements and ensure we have not missed any.
This will ensure our supervisor has a good understanding of the specific requirements we have outlined and can keep them in 
mind when completing a task based inspection. During this inspection, we will provide our supervisor with a checklist that they can review while trying to 
use all the main functionalities within the system. \\

\textbf{User Testing}
User Testing will be used to verify the non-functional requirements of the system. Participants will being given access to the system 
and asked to complete a predefined list of tasks such as uploading multiple images, viewing their session history and downloading the output. The team will
be present during these sessions to observe and note any challenges the users face. The participants will also be asked follow-up questions to gain additional feedback on their overall experience. 



\wss{Maybe create an SRS checklist?}

\subsection{Design Verification}

\wss{Plans for design verification}

\wss{The review will include reviews by your classmates}

\wss{Create a checklists?}
\\
The verification of our design will rely on reviews conducted by our fellow classmates and supervisor who will
evaluate our design documents including our Module Guide (MG) and Module Interface Specification (MIS). They will also aid in performing
consistency and traceability checks to ensure clear alignment between our requirements, design, and implementation. \\[1ex]
In a formal review, our supervisor, Ms. Jing, will verify that the user interface design supports accessibility in accordance with WCAG 2.1 standards. 
Additionally, our professor, Dr. Smith, will assess the technical aspects of the system, focusing on the machine learning architecture to ensure it is well designed.

The Verification and Validation plan will be verified to ensure its completeness, and alignment with previous 
documents such as our SRS, HA and development plan. This will be completed through a team and peer review. \\

\textbf{Team Review}\\
The team will perform a review to ensure that the unit testing described within the document effectively tests the system and achieves the defined objectives. 
This review will involve the team collaboratively going through the document, identifying areas for improvement and making the necessary changes before completing further validation activities.\\

\textbf{Peer Review}\\
A peer review will be completed by Team 10 (One of a Kind) to provide feedback on our V\&V plan. They will review the document to ensure that the sections are consistent and aligned, as well as to identify any missing unit tests.

\subsection{Verification and Validation Plan Verification}

\wss{The verification and validation plan is an artifact that should also be
verified.  Techniques for this include review and mutation testing.}

\wss{The review will include reviews by your classmates}

\wss{Create a checklists?}

\subsection{Implementation Verification}

\wss{You should at least point to the tests listed in this document and the unit
  testing plan.}

\wss{In this section you would also give any details of any plans for static
  verification of the implementation.  Potential techniques include code
  walkthroughs, code inspection, static analyzers, etc.}

\wss{The final class presentation in CAS 741 could be used as a code
walkthrough.  There is also a possibility of using the final presentation (in
CAS741) for a partial usability survey.}

\subsection{Automated Testing and Verification Tools}

\wss{What tools are you using for automated testing.  Likely a unit testing
  framework and maybe a profiling tool, like ValGrind.  Other possible tools
  include a static analyzer, make, continuous integration tools, test coverage
  tools, etc.  Explain your plans for summarizing code coverage metrics.
  Linters are another important class of tools.  For the programming language
  you select, you should look at the available linters.  There may also be tools
  that verify that coding standards have been respected, like flake9 for
  Python.}



\wss{If you have already done this in the development plan, you can point to
that document.}

\wss{The details of this section will likely evolve as you get closer to the
  implementation.}
\\
The AI Generated Alternative (alt) Text tool will make use of the following automated testing and verification tools: 
\begin{itemize}
  \item \textbf{Automated Accessibility Testing}: Automated web accessibility testing tools such as the WAVE Web Accessibility Evaluation Tool will be used to 
  verify that the tool adheres to the WCAG 2.1 guidelines. These tools will automatically scan for accessibility issues such as missing alternative text and color contrast violations.
  \item \textbf{Unit Testing Framework, Linters, Coverage Tools, and Continuous Integration}: These tools have been detailed and outlined in the Expected Technologies section
  of our Development Plan (\citet{DP}) document under Table 3. 
\end{itemize}

\subsection{Software Validation}

\wss{If there is any external data that can be used for validation, you should
  point to it here.  If there are no plans for validation, you should state that
  here.}

\wss{You might want to use review sessions with the stakeholder to check that
the requirements document captures the right requirements.  Maybe task based
inspection?}

\wss{For those capstone teams with an external supervisor, the Rev 0 demo should 
be used as an opportunity to validate the requirements.  You should plan on 
demonstrating your project to your supervisor shortly after the scheduled Rev 0 demo.  
The feedback from your supervisor will be very useful for improving your project.}

\wss{For teams without an external supervisor, user testing can serve the same purpose 
as a Rev 0 demo for the supervisor.}

\wss{This section might reference back to the SRS verification section.}
\\
The AI Generated Alternative (alt) Text tool will be validated through the following: 
\begin{itemize}
  \item \textbf{User Testing}: Stakeholders of this project including students experiencing visual 
  and cognitive disabilities will perform realistic tasks such as uploading an image and generating alt text to validate 
  that it meets their needs and requirements. 
  \item \textbf{Formal Reviews}: Reviews will be conducted with the stakeholders of the project and our supervisor, Ms. Jing, to validate that the requirements
  outlined in our SRS document are correct and captures the needs for the tool. Additionally, the Rev 0 demo with Ms. Jing will serve as a validation review and allow the team 
  to receive feedback to improve the tool. 
\end{itemize} 

\section{System Tests}

\wss{There should be text between all headings, even if it is just a roadmap of
the contents of the subsections.}

\subsection{Tests for Functional Requirements}

\wss{Subsets of the tests may be in related, so this section is divided into
  different areas.  If there are no identifiable subsets for the tests, this
  level of document structure can be removed.}

\wss{Include a blurb here to explain why the subsections below
  cover the requirements.  References to the SRS would be good here.}

\subsubsection{Area of Testing1}

\wss{It would be nice to have a blurb here to explain why the subsections below
  cover the requirements.  References to the SRS would be good here.  If a section
  covers tests for input constraints, you should reference the data constraints
  table in the SRS.}
		
\paragraph{Title for Test}

\begin{enumerate}

\item{test-id1\\}

Control: Manual versus Automatic
					
Initial State: 
					
Input: 
					
Output: \wss{The expected result for the given inputs.  Output is not how you
are going to return the results of the test.  The output is the expected
result.}

Test Case Derivation: \wss{Justify the expected value given in the Output field}
					
How test will be performed: 
					
\item{test-id2\\}

Control: Manual versus Automatic
					
Initial State: 
					
Input: 
					
Output: \wss{The expected result for the given inputs}

Test Case Derivation: \wss{Justify the expected value given in the Output field}

How test will be performed: 

\end{enumerate}

\subsubsection{Area of Testing2}

...

\subsection{Tests for Nonfunctional Requirements}

\wss{The nonfunctional requirements for accuracy will likely just reference the
  appropriate functional tests from above.  The test cases should mention
  reporting the relative error for these tests.  Not all projects will
  necessarily have nonfunctional requirements related to accuracy.}

\wss{For some nonfunctional tests, you won't be setting a target threshold for
passing the test, but rather describing the experiment you will do to measure
the quality for different inputs.  For instance, you could measure speed versus
the problem size.  The output of the test isn't pass/fail, but rather a summary
table or graph.}

\wss{Tests related to usability could include conducting a usability test and
  survey.  The survey will be in the Appendix.}

\wss{Static tests, review, inspections, and walkthroughs, will not follow the
format for the tests given below.}

\wss{If you introduce static tests in your plan, you need to provide details.
How will they be done?  In cases like code (or document) walkthroughs, who will
be involved? Be specific.}

\subsubsection{Area of Testing1}
		
\paragraph{Title for Test}

\begin{enumerate}

\item{test-id1\\}

Type: Functional, Dynamic, Manual, Static etc.
					
Initial State: 
					
Input/Condition: 
					
Output/Result: 
					
How test will be performed: 
					
\item{test-id2\\}

Type: Functional, Dynamic, Manual, Static etc.
					
Initial State: 
					
Input: 
					
Output: 
					
How test will be performed: 

\end{enumerate}

\subsubsection{Area of Testing2}

...

\subsection{Traceability Between Test Cases and Requirements}

\wss{Provide a table that shows which test cases are supporting which
  requirements.}

\section{Unit Test Description}

\wss{This section should not be filled in until after the MIS (detailed design
  document) has been completed.}

\wss{Reference your MIS (detailed design document) and explain your overall
philosophy for test case selection.}  

\wss{To save space and time, it may be an option to provide less detail in this section.  
For the unit tests you can potentially layout your testing strategy here.  That is, you 
can explain how tests will be selected for each module.  For instance, your test building 
approach could be test cases for each access program, including one test for normal behaviour 
and as many tests as needed for edge cases.  Rather than create the details of the input 
and output here, you could point to the unit testing code.  For this to work, you code 
needs to be well-documented, with meaningful names for all of the tests.}

\subsection{Unit Testing Scope}

\wss{What modules are outside of the scope.  If there are modules that are
  developed by someone else, then you would say here if you aren't planning on
  verifying them.  There may also be modules that are part of your software, but
  have a lower priority for verification than others.  If this is the case,
  explain your rationale for the ranking of module importance.}

\subsection{Tests for Functional Requirements}

\wss{Most of the verification will be through automated unit testing.  If
  appropriate specific modules can be verified by a non-testing based
  technique.  That can also be documented in this section.}

\subsubsection{Module 1}

\wss{Include a blurb here to explain why the subsections below cover the module.
  References to the MIS would be good.  You will want tests from a black box
  perspective and from a white box perspective.  Explain to the reader how the
  tests were selected.}

\begin{enumerate}

\item{test-id1\\}

Type: \wss{Functional, Dynamic, Manual, Automatic, Static etc. Most will
  be automatic}
					
Initial State: 
					
Input: 
					
Output: \wss{The expected result for the given inputs}

Test Case Derivation: \wss{Justify the expected value given in the Output field}

How test will be performed: 
					
\item{test-id2\\}

Type: \wss{Functional, Dynamic, Manual, Automatic, Static etc. Most will
  be automatic}
					
Initial State: 
					
Input: 
					
Output: \wss{The expected result for the given inputs}

Test Case Derivation: \wss{Justify the expected value given in the Output field}

How test will be performed: 

\item{...\\}
    
\end{enumerate}

\subsubsection{Module 2}

...

\subsection{Tests for Nonfunctional Requirements}

\wss{If there is a module that needs to be independently assessed for
  performance, those test cases can go here.  In some projects, planning for
  nonfunctional tests of units will not be that relevant.}

\wss{These tests may involve collecting performance data from previously
  mentioned functional tests.}

\subsubsection{Module ?}
		
\begin{enumerate}

\item{test-id1\\}

Type: \wss{Functional, Dynamic, Manual, Automatic, Static etc. Most will
  be automatic}
					
Initial State: 
					
Input/Condition: 
					
Output/Result: 
					
How test will be performed: 
					
\item{test-id2\\}

Type: Functional, Dynamic, Manual, Static etc.
					
Initial State: 
					
Input: 
					
Output: 
					
How test will be performed: 

\end{enumerate}

\subsubsection{Module ?}

...

\subsection{Traceability Between Test Cases and Modules}

\wss{Provide evidence that all of the modules have been considered.}
				
\bibliographystyle{plainnat}

\bibliography{../../refs/References}

\newpage

\section{Appendix}

This is where you can place additional information.

\subsection{Symbolic Parameters}

The definition of the test cases will call for SYMBOLIC\_CONSTANTS.
Their values are defined in this section for easy maintenance.

\subsection{Usability Survey Questions?}

\wss{This is a section that would be appropriate for some projects.}

\newpage{}
\section*{Appendix --- Reflection}

\wss{This section is not required for CAS 741}

The information in this section will be used to evaluate the team members on the
graduate attribute of Lifelong Learning.

The purpose of reflection questions is to give you a chance to assess your own
learning and that of your group as a whole, and to find ways to improve in the
future. Reflection is an important part of the learning process.  Reflection is
also an essential component of a successful software development process.  

Reflections are most interesting and useful when they're honest, even if the
stories they tell are imperfect. You will be marked based on your depth of
thought and analysis, and not based on the content of the reflections
themselves. Thus, for full marks we encourage you to answer openly and honestly
and to avoid simply writing ``what you think the evaluator wants to hear.''

Please answer the following questions.  Some questions can be answered on the
team level, but where appropriate, each team member should write their own
response:

\begin{enumerate}
  \item What went well while writing this deliverable? 
  \item What pain points did you experience during this deliverable, and how
    did you resolve them?
  \item What knowledge and skills will the team collectively need to acquire to
  successfully complete the verification and validation of your project?
  Examples of possible knowledge and skills include dynamic testing knowledge,
  static testing knowledge, specific tool usage, Valgrind etc.  You should look to
  identify at least one item for each team member.
  \item For each of the knowledge areas and skills identified in the previous
  question, what are at least two approaches to acquiring the knowledge or
  mastering the skill?  Of the identified approaches, which will each team
  member pursue, and why did they make this choice?
\end{enumerate}

\end{document}