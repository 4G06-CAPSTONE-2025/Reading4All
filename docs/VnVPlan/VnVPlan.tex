\documentclass[12pt, titlepage]{article}

\usepackage{booktabs}
\usepackage{tabularx}
\usepackage{hyperref}
\hypersetup{
    colorlinks,
    citecolor=blue,
    filecolor=black,
    linkcolor=red,
    urlcolor=blue
}
\usepackage[round]{natbib}

\input{../Comments}
%% Common Parts

\newcommand{\progname}{ProgName} % PUT YOUR PROGRAM NAME HERE
\newcommand{\authname}{Team 22, READING4ALL
\\ Fiza Sehar
\\ Nawaal Fatima
\\ Dhruv Sardana
\\ Moly Mikhail
\\ Casey Francine Bulaclac } % AUTHOR NAMES                  

\usepackage{hyperref}
    \hypersetup{colorlinks=true, linkcolor=blue, citecolor=blue, filecolor=blue,
                urlcolor=blue, unicode=false}
    \urlstyle{same}
                                


\begin{document}

\title{System Verification and Validation Plan for \progname{}} 
\author{\authname}
\date{\today}
	
\maketitle

\pagenumbering{roman}

\section*{Revision History}

\begin{tabularx}{\textwidth}{p{3cm}p{2cm}X}
\toprule {\bf Date} & {\bf Version} & {\bf Notes}\\
\midrule
Date 1 & 1.0 & Notes\\
Date 2 & 1.1 & Notes\\
\bottomrule
\end{tabularx}

~\\
\wss{The intention of the VnV plan is to increase confidence in the software.
However, this does not mean listing every verification and validation technique
that has ever been devised.  The VnV plan should also be a \textbf{feasible}
plan. Execution of the plan should be possible with the time and team available.
If the full plan cannot be completed during the time available, it can either be
modified to ``fake it'', or a better solution is to add a section describing
what work has been completed and what work is still planned for the future.}

\wss{The VnV plan is typically started after the requirements stage, but before
the design stage.  This means that the sections related to unit testing cannot
initially be completed.  The sections will be filled in after the design stage
is complete.  the final version of the VnV plan should have all sections filled
in.}

\newpage

\tableofcontents

\listoftables
\wss{Remove this section if it isn't needed}

\listoffigures
\wss{Remove this section if it isn't needed}

\newpage

\section{Symbols, Abbreviations, and Acronyms}

\renewcommand{\arraystretch}{1.2}
\begin{tabular}{l l} 
  \toprule		
  \textbf{symbol} & \textbf{description}\\
  \midrule 
  T & Test\\
  \bottomrule
\end{tabular}\\

\wss{symbols, abbreviations, or acronyms --- you can simply reference the SRS
  \citep{SRS} tables, if appropriate}

\wss{Remove this section if it isn't needed}

\newpage

\pagenumbering{arabic}

This document ... \wss{provide an introductory blurb and roadmap of the
  Verification and Validation plan}

\section{General Information}

\subsection{Summary}

\wss{Say what software is being tested.  Give its name and a brief overview of
  its general functions.}

\subsection{Objectives}

\wss{State what is intended to be accomplished.  The objective will be around
  the qualities that are most important for your project.  You might have
  something like: ``build confidence in the software correctness,''
  ``demonstrate adequate usability.'' etc.  You won't list all of the qualities,
  just those that are most important.}

\wss{You should also list the objectives that are out of scope.  You don't have 
the resources to do everything, so what will you be leaving out.  For instance, 
if you are not going to verify the quality of usability, state this.  It is also 
worthwhile to justify why the objectives are left out.}

\wss{The objectives are important because they highlight that you are aware of 
limitations in your resources for verification and validation.  You can't do everything, 
so what are you going to prioritize?  As an example, if your system depends on an 
external library, you can explicitly state that you will assume that external library 
has already been verified by its implementation team.}

\subsection{Challenge Level and Extras}

\wss{State the challenge level (advanced, general, basic) for your project.
Your challenge level should exactly match what is included in your problem
statement.  This should be the challenge level agreed on between you and the
course instructor.  You can use a pull request to update your challenge level
(in TeamComposition.csv or Repos.csv) if your plan changes as a result of the
VnV planning exercise.}

\wss{Summarize the extras (if any) that were tackled by this project.  Extras
can include usability testing, code walkthroughs, user documentation, formal
proof, GenderMag personas, Design Thinking, etc.  Extras should have already
been approved by the course instructor as included in your problem statement.
You can use a pull request to update your extras (in TeamComposition.csv or
Repos.csv) if your plan changes as a result of the VnV planning exercise.}

\subsection{Relevant Documentation}

\wss{Reference relevant documentation.  This will definitely include your SRS
  and your other project documents (design documents, like MG, MIS, etc).  You
  can include these even before they are written, since by the time the project
  is done, they will be written.  You can create BibTeX entries for your
  documents and within those entries include a hyperlink to the documents.}

\citet{SRS}

\wss{Don't just list the other documents.  You should explain why they are relevant and 
how they relate to your VnV efforts.}

\section{Plan}

\wss{Introduce this section.  You can provide a roadmap of the sections to
  come.}

\subsection{Verification and Validation Team}

\wss{Your teammates.  Maybe your supervisor.
  You should do more than list names.  You should say what each person's role is
  for the project's verification.  A table is a good way to summarize this information.}

\subsection{SRS Verification}

\wss{List any approaches you intend to use for SRS verification.  This may
  include ad hoc feedback from reviewers, like your classmates (like your
  primary reviewer), or you may plan for something more rigorous/systematic.}

\wss{If you have a supervisor for the project, you shouldn't just say they will
read over the SRS.  You should explain your structured approach to the review.
Will you have a meeting?  What will you present?  What questions will you ask?
Will you give them instructions for a task-based inspection?  Will you use your
issue tracker?}

\wss{Maybe create an SRS checklist?}

\subsection{Design Verification}

\wss{Plans for design verification}

\wss{The review will include reviews by your classmates}

\wss{Create a checklists?}

\subsection{Verification and Validation Plan Verification}

\wss{The verification and validation plan is an artifact that should also be
verified.  Techniques for this include review and mutation testing.}

\wss{The review will include reviews by your classmates}

\wss{Create a checklists?}

\subsection{Implementation Verification}

\wss{You should at least point to the tests listed in this document and the unit
  testing plan.}

\wss{In this section you would also give any details of any plans for static
  verification of the implementation.  Potential techniques include code
  walkthroughs, code inspection, static analyzers, etc.}

\wss{The final class presentation in CAS 741 could be used as a code
walkthrough.  There is also a possibility of using the final presentation (in
CAS741) for a partial usability survey.}

\subsection{Automated Testing and Verification Tools}

\wss{What tools are you using for automated testing.  Likely a unit testing
  framework and maybe a profiling tool, like ValGrind.  Other possible tools
  include a static analyzer, make, continuous integration tools, test coverage
  tools, etc.  Explain your plans for summarizing code coverage metrics.
  Linters are another important class of tools.  For the programming language
  you select, you should look at the available linters.  There may also be tools
  that verify that coding standards have been respected, like flake9 for
  Python.}

\wss{If you have already done this in the development plan, you can point to
that document.}

\wss{The details of this section will likely evolve as you get closer to the
  implementation.}

\subsection{Software Validation}

\wss{If there is any external data that can be used for validation, you should
  point to it here.  If there are no plans for validation, you should state that
  here.}

\wss{You might want to use review sessions with the stakeholder to check that
the requirements document captures the right requirements.  Maybe task based
inspection?}

\wss{For those capstone teams with an external supervisor, the Rev 0 demo should 
be used as an opportunity to validate the requirements.  You should plan on 
demonstrating your project to your supervisor shortly after the scheduled Rev 0 demo.  
The feedback from your supervisor will be very useful for improving your project.}

\wss{For teams without an external supervisor, user testing can serve the same purpose 
as a Rev 0 demo for the supervisor.}

\wss{This section might reference back to the SRS verification section.}

\section{System Tests}

\wss{There should be text between all headings, even if it is just a roadmap of
the contents of the subsections.}

\subsection{Tests for Functional Requirements}

\wss{Subsets of the tests may be in related, so this section is divided into
  different areas.  If there are no identifiable subsets for the tests, this
  level of document structure can be removed.}

\wss{Include a blurb here to explain why the subsections below
  cover the requirements.  References to the SRS would be good here.}

\subsubsection{Area of Testing1}

\wss{It would be nice to have a blurb here to explain why the subsections below
  cover the requirements.  References to the SRS would be good here.  If a section
  covers tests for input constraints, you should reference the data constraints
  table in the SRS.}
		
\paragraph{Title for Test}

\begin{enumerate}

\item{test-id1\\}

Control: Manual versus Automatic
					
Initial State: 
					
Input: 
					
Output: \wss{The expected result for the given inputs.  Output is not how you
are going to return the results of the test.  The output is the expected
result.}

Test Case Derivation: \wss{Justify the expected value given in the Output field}
					
How test will be performed: 
					
\item{test-id2\\}

Control: Manual versus Automatic
					
Initial State: 
					
Input: 
					
Output: \wss{The expected result for the given inputs}

Test Case Derivation: \wss{Justify the expected value given in the Output field}

How test will be performed: 

\end{enumerate}

\subsubsection{Area of Testing2}

...

\subsection{Tests for Nonfunctional Requirements}

\wss{The nonfunctional requirements for accuracy will likely just reference the
  appropriate functional tests from above.  The test cases should mention
  reporting the relative error for these tests.  Not all projects will
  necessarily have nonfunctional requirements related to accuracy.}

\wss{For some nonfunctional tests, you won't be setting a target threshold for
passing the test, but rather describing the experiment you will do to measure
the quality for different inputs.  For instance, you could measure speed versus
the problem size.  The output of the test isn't pass/fail, but rather a summary
table or graph.}

\wss{Tests related to usability could include conducting a usability test and
  survey.  The survey will be in the Appendix.}

\wss{Static tests, review, inspections, and walkthroughs, will not follow the
format for the tests given below.}

\wss{If you introduce static tests in your plan, you need to provide details.
How will they be done?  In cases like code (or document) walkthroughs, who will
be involved? Be specific.}

\subsubsection{Area of Testing1}
		
\paragraph{Title for Test}

\begin{enumerate}

\item{test-id1\\}

Type: Functional, Dynamic, Manual, Static etc.
					
Initial State: 
					
Input/Condition: 
					
Output/Result: 
					
How test will be performed: 
					
\item{test-id2\\}

Type: Functional, Dynamic, Manual, Static etc.
					
Initial State: 
					
Input: 
					
Output: 
					
How test will be performed: 

\end{enumerate}

\subsubsection{Area of Testing2}

...

\subsection{Traceability Between Test Cases and Requirements}

\wss{Provide a table that shows which test cases are supporting which
  requirements.}

\section{Unit Test Description}
\subsection{Unit Testing Scope}

The purpose of unit testing for \textit{Reading4All} is to verify the correctness, robustness, and accessibility compliance of individual components prior to system integration. Each module is tested independently using automated test scripts (PyTest) and deterministic input fixtures.

\textbf{Modules in Scope:}
\begin{itemize}
  \item Image Upload \& Validation Module
  \item Alt-Text Generation Module
  \item Accessibility and UI Compliance Module
  \item Security and Privacy Module
\end{itemize}

\textbf{Modules Out of Scope:}
Third-party OCR engines, pretrained vision/language models, and external McMaster authentication services are assumed to be validated independently. Only the thin wrappers and internal interactions with these APIs are tested here.

\subsection{Tests for Functional Requirements}

\subsubsection{Module 1 \textemdash{} Image Upload and Validation}

\textbf{Goal:} Ensure uploaded images meet all input constraints for format, size, and type, and that invalid files are rejected gracefully.

\begin{enumerate}
\item{UT1-UploadValidImage\\}
Type: Functional, Dynamic, Automatic\\
Initial State: Application running; no active uploads.\\
Input: Valid PNG image, 1 MB in size.\\
Output: File accepted; confirmation message displayed; metadata stored in temporary session.\\
Test Case Derivation: Confirms compliance with input constraints (JPEG/PNG $\leq$ 10 MB).\\
How test will be performed: Run automated pytest verifying HTTP 200 response and valid JSON schema.

\item{UT2-UploadInvalidFileType\\}
Type: Functional, Dynamic, Automatic\\
Initial State: No uploads.\\
Input: Unsupported file type (e.g., \texttt{.pdf}).\\
Output: Error message ``Unsupported file format'' returned; no data stored.\\
Test Case Derivation: Validates enforcement of file type constraint and secure rejection.\\
How test will be performed: Send POST request with invalid MIME type; assert error 400 and log entry.

\item{UT3-UploadOversizedFile\\}
Type: Functional, Dynamic, Automatic\\
Initial State: No uploads.\\
Input: PNG file $>$ 10 MB.\\
Output: Upload rejected with clear error; no file stored.\\
Test Case Derivation: Confirms handling of maximum size threshold.\\
How test will be performed: Simulate multipart upload; verify memory cleanup and error alert.
\end{enumerate}

\subsubsection{Module 2 \textemdash{} Alt-Text Generation}

\textbf{Goal:} Validate that image inference and text generation components produce deterministic, relevant, and correctly formatted alternative text.

\begin{enumerate}
\item{UT4-GenerateAltText\\}
Type: Functional, Dynamic, Automatic\\
Initial State: Valid image uploaded and accessible to model service.\\
Input: Image containing labeled diagram.\\
Output: Non-empty descriptive string within 3--8~s latency window.\\
Test Case Derivation: Confirms compliance with generation timing and content sufficiency.\\
How test will be performed: Mock ML service; assert response schema and timing $<$ T\_ALT\_GEN\_SMALL.

\item{UT5-EditAltText\\}
Type: Functional, Dynamic, Manual\\
Initial State: Alt-text successfully generated.\\
Input: User edits description and saves.\\
Output: Edited text replaces old version in session storage.\\
Test Case Derivation: Ensures edit functionality modifies session data only.\\
How test will be performed: Selenium automation of UI; assert saved value persists on reload.

\item{UT6-HandleEmptyAltText\\}
Type: Functional, Dynamic, Automatic\\
Initial State: Image uploaded yields no model output.\\
Input: Blank model return.\\
Output: Error message and retry option; no text stored.\\
Test Case Derivation: Confirms graceful failure and user notification.\\
How test will be performed: Patch model API to return empty string; validate error log.
\end{enumerate}

\subsubsection{Module 3 \textemdash{} Accessibility and UI Compliance}

\textbf{Goal:} Validate that all UI components meet accessibility and usability criteria (keyboard navigation, zoom, color contrast, alt text labeling).

\begin{enumerate}
\item{UT7-KeyboardNavigation\\}
Type: Functional, Dynamic, Automatic\\
Initial State: Application home screen loaded.\\
Input: Simulated Tab and Enter key presses.\\
Output: All focusable elements reachable; no trap detected.\\
Test Case Derivation: Confirms WCAG 2.1 Success Criterion 2.1.1.\\
How test will be performed: Automated Axe/WAVE accessibility scan with keyboard simulation.

\item{UT8-ContrastValidation\\}
Type: Functional, Static, Automatic\\
Initial State: Deployed UI snapshot available.\\
Input: CSS stylesheet.\\
Output: All color pairs $\geq$ 4.5:1 contrast ratio.\\
Test Case Derivation: Confirms MIN\_CONTRAST\_RATIO threshold met.\\
How test will be performed: Run Lighthouse CI contrast-check script.

\item{UT9-ZoomResilience\\}
Type: Functional, Manual\\
Initial State: Browser window at 100\%.\\
Input: Zoom increased to 200\%.\\
Output: Interface remains fully visible and interactive.\\
Test Case Derivation: Ensures compliance with MAX\_ZOOM\_PERCENTAGE.\\
How test will be performed: Manual inspection + screen-reader pass.
\end{enumerate}

\subsubsection{Module 4 \textemdash{} Security and Privacy}

\textbf{Goal:} Verify that all authentication, encryption, and data-deletion procedures uphold confidentiality and integrity requirements.

\begin{enumerate}
\item{UT10-LoginAuthentication\\}
Type: Functional, Dynamic, Automatic\\
Initial State: No user session.\\
Input: Valid McMaster SSO credentials.\\
Output: Access granted; session token stored.\\
Test Case Derivation: Ensures access restriction and session linking.\\
How test will be performed: Mock OAuth SSO; assert 200 and JWT valid.

\item{UT11-RejectUnauthorizedAccess\\}
Type: Functional, Dynamic, Automatic\\
Initial State: No valid session token.\\
Input: API request to /generate endpoint.\\
Output: HTTP 401 Unauthorized.\\
Test Case Derivation: Confirms secure access control.\\
How test will be performed: Post request without auth header; verify denial and log entry.

\item{UT12-TemporaryFileDeletion\\}
Type: Functional, Dynamic, Automatic\\
Initial State: Completed alt-text generation.\\
Input: Wait $>$ 60~s.\\
Output: Uploaded file deleted from temporary directory.\\
Test Case Derivation: Verifies privacy compliance (FILE\_DELETE\_TIME).\\
How test will be performed: Check directory contents before/after timeout.
\end{enumerate}

\subsection{Tests for Non-Functional Requirements}

\subsubsection{Module 5 \textemdash{} Performance and Reliability}

\textbf{Goal:} Ensure responsiveness, stability, and error handling meet defined thresholds.

\begin{enumerate}
\item{UT13-LatencyBenchmark\\}
Type: Dynamic, Automatic\\
Input/Condition: Upload 5 images $\leq$ 2~MB each concurrently.\\
Output/Result: Mean response $\leq$ 8~s; no timeouts.\\
How test will be performed: Stress-test script measuring T\_ALT\_GEN\_LARGE; record average latency.

\item{UT14-FaultRecovery\\}
Type: Dynamic, Automatic\\
Input/Condition: Force backend process crash.\\
Output/Result: Recovery $\leq$ 5~s; no user data loss.\\
How test will be performed: Docker restart test; verify persistence logs.
\end{enumerate}

\subsubsection{Module 6 \textemdash{} Usability and Accessibility Metrics}

\textbf{Goal:} Quantify user interaction quality and accessibility performance.

\begin{enumerate}
\item{UT15-UsabilitySurvey\\}
Type: Manual, Empirical\\
Input/Condition: Ten participants complete key tasks.\\
Output/Result: Median usability rating $\geq 3$/4; no responses $<2$.\\
How test will be performed: Controlled observation using evaluation rubric.

\item{UT16-ScreenReaderCompatibility\\}
Type: Functional, Dynamic, Manual\\
Input/Condition: Generate alt text and read using NVDA, JAWS, VoiceOver.\\
Output/Result: All screen readers announce output correctly.\\
How test will be performed: Manual auditory confirmation + accessibility log capture.
\end{enumerate}

\subsection{Traceability Between Test Cases and Modules}

\begin{center}
\begin{tabular}{|l|l|l|}
\hline
\textbf{Test ID} & \textbf{Module / Feature Tested} & \textbf{Supported Requirement(s)}\\
\hline
UT1--UT3   & Image Upload and Validation          & FR1, PR-RFT1 \\
UT4--UT6   & Alt-Text Generation                  & FR2, FR4, PR-PAR1, PR-PAR2 \\
UT7--UT9   & Accessibility and UI Compliance      & LFR-AR1--AR4, UHR-EUR1--4 \\
UT10--UT12 & Security and Privacy                 & SR-AR1, SR-PR1, PR-SCR1--PR-SCR3 \\
UT13--UT14 & Performance and Reliability          & PR-SL1--2, PR-RFT2 \\
UT15--UT16 & Usability and Accessibility Metrics  & UHR-LR1, UHR-AR1, OER-IAS1 \\
\hline
\end{tabular}
\end{center}
\newpage

\bibliographystyle{plainnat}

\bibliography{../../refs/References}

\newpage

\section{Appendix}

This is where you can place additional information.

\subsection{Symbolic Parameters}

The definition of the test cases will call for SYMBOLIC\_CONSTANTS.
Their values are defined in this section for easy maintenance.

\subsection{Usability Survey Questions?}

\wss{This is a section that would be appropriate for some projects.}

\newpage{}
\section*{Appendix --- Reflection}

\wss{This section is not required for CAS 741}

The information in this section will be used to evaluate the team members on the
graduate attribute of Lifelong Learning.

\input{../Reflection.tex}

\textbf{Fiza Sehar}
\begin{enumerate}
\item What went well while writing this deliverable?

We effectively organized the verification and validation framework for Reading4All by referencing both functional and non-functional requirements from the SRS. We collaborated to design clear, structured unit tests, ensuring traceability and consistency across modules.

\item What pain points did you experience during this deliverable, and how did you resolve them?

This section was not required, but we completed it in detail before realizing that, which limited our focus on other sections. This caused minor conflicts regarding time and task distribution, which we resolved through open communication and by redistributing responsibilities for future deliverables. We decided to created internal rubrics and clearer priorities to stay organized and avoid similar issues moving forward.

\end{enumerate}

\end{document}