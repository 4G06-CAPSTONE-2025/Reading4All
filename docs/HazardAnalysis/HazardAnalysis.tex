\documentclass{article}

\usepackage{booktabs}
\usepackage{tabularx}
\usepackage{hyperref}

\hypersetup{
  colorlinks=true,       % false: boxed links; true: colored links
  linkcolor=red,          % color of internal links (change box color
  % with linkbordercolor)
  citecolor=green,        % color of links to bibliography
  filecolor=magenta,      % color of file links
  urlcolor=cyan           % color of external links
}

\title{Hazard Analysis\\\progname}

\author{\authname}

\date{}

%% Comments

\usepackage{color}

\newif\ifcomments\commentstrue %displays comments
%\newif\ifcomments\commentsfalse %so that comments do not display

\ifcomments
\newcommand{\authornote}[3]{\textcolor{#1}{[#3 ---#2]}}
\newcommand{\todo}[1]{\textcolor{red}{[TODO: #1]}}
\else
\newcommand{\authornote}[3]{}
\newcommand{\todo}[1]{}
\fi

\newcommand{\wss}[1]{\authornote{magenta}{SS}{#1}} 
\newcommand{\plt}[1]{\authornote{cyan}{TPLT}{#1}} %For explanation of the template
\newcommand{\an}[1]{\authornote{cyan}{Author}{#1}}

%% Common Parts

\newcommand{\progname}{ProgName} % PUT YOUR PROGRAM NAME HERE
\newcommand{\authname}{Team \#, Team Name
\\ Student 1 name
\\ Student 2 name
\\ Student 3 name
\\ Student 4 name} % AUTHOR NAMES                  

\usepackage{hyperref}
    \hypersetup{colorlinks=true, linkcolor=blue, citecolor=blue, filecolor=blue,
                urlcolor=blue, unicode=false}
    \urlstyle{same}
                                


\begin{document}

\maketitle
\thispagestyle{empty}

~\newpage

\pagenumbering{roman}

\begin{table}[hp]
  \caption{Revision History} \label{TblRevisionHistory}
  \begin{tabularx}{\textwidth}{llX}
    \toprule
    \textbf{Date} & \textbf{Developer(s)} & \textbf{Change}\\
    \midrule
    Date1 & Name(s) & Description of changes\\
    Date2 & Name(s) & Description of changes\\
    ... & ... & ...\\
    \bottomrule
  \end{tabularx}
\end{table}

~\newpage

\tableofcontents

~\newpage

\pagenumbering{arabic}

\wss{You are free to modify this template.}

\section{Introduction}

\wss{You can include your definition of what a hazard is here.}

\section{Scope and Purpose of Hazard Analysis}

\wss{You should say what \textbf{loss} could be incurred because of the
hazards.}

\section{System Boundaries and Components}

\wss{Dividing the system into components will help you brainstorm the hazards.
  You shouldn't do a full design of the components, just get a feel
  for the major
  ones.  For projects that involve hardware, the components will
  typically include
  each individual piece of hardware.  If your software will have a
  database, or an
important library, these are also potential components.}

\subsection{System Overview}
The system is a web-based AI tool that generates alternative text
(alt text) for uploaded images or figures
and integrated with screen readers to improve accessibility for
visually impaired users.
\subsection{System Boundaries}
\begin{itemize}
  \item \textbf{Internal Components:} Alternative Text Generation
    Machine Learning (ML) Model, User Interface, Session History Manager
  \item \textbf{External Components:} McMaster Authentication System,
    External AI/ML Frameworks, Screen Reader Software
\end{itemize}
\subsection{System Components}
\begin{enumerate}
  \item \textbf{Alt Text Generation ML Model}
    \begin{itemize}
      \item \textbf{Purpose:} To automatically generate accurate and
        descriptive alternative text for uploaded images using machine learning.
      \item \textbf{Key Functions:}
        \begin{itemize}
          \item Process image inputs received from the backend and
            extract key visual features.
          \item Generate contextually relevant text descriptions.
          \item Return the generated alternative text to the backend
            to display to the user interface.
        \end{itemize}
    \end{itemize}
  \item \textbf{User Interface}
    \begin{itemize}
      \item \textbf{Purpose:} To serve as the primary interaction
        point between the user and the system and allow users to
        upload images and view generated alt text.
      \item \textbf{Key Functions:}
        \begin{itemize}
          \item Enable users to upload images through an accessible
            web interface.
          \item Display generated alt text and allow users to edit,
            copy, and download the text.
          \item Provide features that are accessible and complies
            with the Web Content Accessibility Guidelines (WCAG) 2.1 standards.
          \item Communicates user requests and display outputs from the backend.
        \end{itemize}
    \end{itemize}
  \item \textbf{Session History Manager}
    \begin{itemize}
      \item \textbf{Purpose:} To ensure the continuity of the current
        session and manage user data during active use of the system.
      \item \textbf{Key Functions:}
        \begin{itemize}
          \item Track unique user sessions throughout interaction
            with the web application.
          \item Store previously uploaded images and generated
            alternative text for the current session to allow users
            to view history.
        \end{itemize}
    \end{itemize}
\end{enumerate}
\section{Critical Assumptions}
This section documents the assumptions made during the hazard analysis of
\progname{} (Reading4All). The number of assumptions is kept to a minimum
to reduce the chance of overlooking potential hazards. Where assumptions
are made, they set clear boundaries for the analysis and define the
conditions under which the system is expected to operate safely.

\begin{itemize}
  \item \textbf{Assumption 1: Input Integrity.} All image files provided
    to the system are assumed to be valid image formats and not corrupted
    or maliciously constructed to exploit parsing vulnerabilities.

  \item \textbf{Assumption 2: Standards Stability.} Accessibility
    guidelines (WCAG 2.1, AODA) are assumed to remain stable for the
    operational lifetime of the system.

  \item \textbf{Assumption 3: Model Performance.} The machine learning
    models used for visual recognition and natural language generation are
    assumed to operate within validated ranges of accuracy and reliability.

  \item \textbf{Assumption 4: Human Oversight.} It is assumed that
    alternative text generated by the system will undergo review by human
    instructors, teaching assistants, or accessibility specialists before
    being used in educational contexts.
\end{itemize}

Violations of these assumptions may introduce additional hazards outside
the current scope of this analysis. Such cases would require
re-evaluation of risks and system design updates.

\section{Failure Mode and Effect Analysis}

\wss{Include your FMEA table here. This is the most important part of
this document.}
\wss{The safety requirements in the table do not have to have the prefix SR.
  The most important thing is to show traceability to your SRS. You
  might trace to
  requirements you have already written, or you might need to add new
requirements.}
\wss{If no safety requirement can be devised, other mitigation strategies can be
  entered in the table, including strategies involving providing additional
documentation, and/or test cases.}

\section{Safety and Security Requirements}

\wss{Newly discovered requirements.  These should also be added to the SRS.  (A
rationale design process how and why to fake it.)}

\section{Roadmap}

\wss{Which safety requirements will be implemented as part of the
  capstone timeline?
Which requirements will be implemented in the future?}

\newpage{}

\section*{Appendix --- Reflection}

\wss{Not required for CAS 741}

The purpose of reflection questions is to give you a chance to assess your own
learning and that of your group as a whole, and to find ways to improve in the
future. Reflection is an important part of the learning process.  Reflection is
also an essential component of a successful software development process.  

Reflections are most interesting and useful when they're honest, even if the
stories they tell are imperfect. You will be marked based on your depth of
thought and analysis, and not based on the content of the reflections
themselves. Thus, for full marks we encourage you to answer openly and honestly
and to avoid simply writing ``what you think the evaluator wants to hear.''

Please answer the following questions.  Some questions can be answered on the
team level, but where appropriate, each team member should write their own
response:

\begin{enumerate}
  \item What went well while writing this deliverable?
  \item What pain points did you experience during this deliverable, and how
    did you resolve them?
  \item Which of your listed risks had your team thought of before this
    deliverable, and which did you think of while doing this deliverable? For
    the latter ones (ones you thought of while doing the Hazard Analysis), how
    did they come about?
  \item Other than the risk of physical harm (some projects may not have any
    appreciable risks of this form), list at least 2 other types of risk in
    software products. Why are they important to consider?
\end{enumerate}

\textbf{Nawaal Fatima  - Reflection}
\begin{enumerate}
  \item \textbf{What went well while writing this deliverable?}\newline
    I think the easiest part was coming up with the assumptions
    themselves. From my summer work experience with Ms. Sui, I was already familiar with the main areas where Reading4All
    could run into issues (inputs, standards, model accuracy, and
    human review), it felt pretty natural to turn those into clear
    assumptions. It also helped that the section didn’t need to be
    super long, so I could keep it focused and to the point.

  \item \textbf{What pain points did you experience during this
    deliverable, and how did you resolve them?}\newline
    The tricky part was figuring out how much detail to put in
    without overcomplicating things. At first, I thought about
    writing assumptions that basically ruled out certain failures,
    but I realized that would go against the whole idea of hazard
    analysis. To fix that, I stuck to assumptions that made sense for
    defining boundaries without pretending hazards don’t exist.
    Another small challenge was making the wording simple enough - so
    I rewrote a couple of the assumptions to sound clearer and less
    ``technical report'' heavy.
\end{enumerate}

\end{document}
