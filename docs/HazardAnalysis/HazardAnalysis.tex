\documentclass{article}

\usepackage{booktabs}
\usepackage{tabularx}
\usepackage{hyperref}

\hypersetup{
    colorlinks=true,       % false: boxed links; true: colored links
    linkcolor=red,          % color of internal links (change box color with linkbordercolor)
    citecolor=green,        % color of links to bibliography
    filecolor=magenta,      % color of file links
    urlcolor=cyan           % color of external links
}

\title{Hazard Analysis\\\progname}

\author{\authname}

\date{}

\input{../Comments}
%% Common Parts

\newcommand{\progname}{ProgName} % PUT YOUR PROGRAM NAME HERE
\newcommand{\authname}{Team 22, READING4ALL
\\ Fiza Sehar
\\ Nawaal Fatima
\\ Dhruv Sardana
\\ Moly Mikhail
\\ Casey Francine Bulaclac } % AUTHOR NAMES                  

\usepackage{hyperref}
    \hypersetup{colorlinks=true, linkcolor=blue, citecolor=blue, filecolor=blue,
                urlcolor=blue, unicode=false}
    \urlstyle{same}
                                


\begin{document}

\maketitle
\thispagestyle{empty}

~\newpage

\pagenumbering{roman}

\begin{table}[hp]
\caption{Revision History} \label{TblRevisionHistory}
\begin{tabularx}{\textwidth}{llX}
\toprule
\textbf{Date} & \textbf{Developer(s)} & \textbf{Change}\\
\midrule
Date1 & Name(s) & Description of changes\\
Date2 & Name(s) & Description of changes\\
... & ... & ...\\
\bottomrule
\end{tabularx}
\end{table}

~\newpage

\tableofcontents

~\newpage

\pagenumbering{arabic}


\section{Introduction}

\subsection{Problem Statement}
Students with disabilities confront major challenges to reading technical diagrams, which are often given in the form of static images that screen readers cannot interpret. Manually developing alternative (alt) text is possible, but it is resource-intensive, inconsistent, and not scalable for huge volumes of course content. This generates disparities in access to learning materials in postsecondary education.

To address this issue, this project proposes creating an AI/ML-powered application that automatically generates clear and detailed alt text for technical diagrams. The tool intends to increase accessibility, assure compliance with AODA regulations, and promote greater inclusion in higher education.

\subsection{Hazard Analysis Introduction}
A hazard is a system state or combination of conditions that, when paired with specific environmental or contextual factors, can result in an unwanted or adverse event in the system or its surroundings. Hazards in software and AI development encompass not only physical threats but also usability, ethical, and technological issues.

Hazards in this project may develop during the design, training, and implementation of the alt-text generating model. These risks include technical flaws in generated text, misinterpretations that may mislead learners, ethical concerns regarding bias in the dataset, and compatibility issues with assistive technology.

This hazard analysis identifies and evaluates such risks to guarantee that the system provides consistent accessibility gains while not mistakenly introducing new obstacles.

\section{Scope and Purpose of Hazard Analysis}
The objective of this hazard analysis is to analyze the potential hazards and damages related to the AI/ML alt-text generating tool throughout its development and operation. The most significant hazards are:

\begin{itemize}
    \item \textbf{Technical Flaws:} Incorrect or inadequate alt text may mislead students, impair comprehension, or violate accessibility guidelines. This could result in a loss of confidence in the tool and poor results in education.
    \item \textbf{User Insensitivity:} Alt text that is excessively technical, unsophisticated, or inconsistent may not meet the needs of students. This may result in a loss of usefulness, reducing participation among students, teachers, and accessibility professionals.
    \item \textbf{External Dependencies:} Relying on third-party libraries, APIs, or screen reader compatibility can lead to external failures. Interruptions in these functions may result in a loss of functionality or downtime.
\end{itemize}

The purpose of this hazard study is to systematically identify these hazards, estimate their potential impact, and develop mitigation strategies. By doing this, we aim to minimize losses associated with time, trust, resources, and accessibility outcomes, ensuring that the tool contributes positively to equitable education.

\section{System Boundaries and Components}

\wss{Dividing the system into components will help you brainstorm the hazards.
You shouldn't do a full design of the components, just get a feel for the major
ones.  For projects that involve hardware, the components will typically include
each individual piece of hardware.  If your software will have a database, or an
important library, these are also potential components.}

\section{Critical Assumptions}

\wss{These assumptions that are made about the software or system.  You should
minimize the number of assumptions that remove potential hazards.  For instance,
you could assume a part will never fail, but it is generally better to include
this potential failure mode.}

\section{Failure Mode and Effect Analysis}

\wss{Include your FMEA table here. This is the most important part of this document.}
\wss{The safety requirements in the table do not have to have the prefix SR.
The most important thing is to show traceability to your SRS. You might trace to
requirements you have already written, or you might need to add new
requirements.}
\wss{If no safety requirement can be devised, other mitigation strategies can be
entered in the table, including strategies involving providing additional
documentation, and/or test cases.}

\section{Safety and Security Requirements}

\wss{Newly discovered requirements.  These should also be added to the SRS.  (A
rationale design process how and why to fake it.)}

\section{Roadmap}

\wss{Which safety requirements will be implemented as part of the capstone timeline?
Which requirements will be implemented in the future?}

\newpage{}

\section*{Appendix --- Reflection}

\wss{Not required for CAS 741}

\input{../Reflection.tex}

\begin{enumerate}
    \item What went well while writing this deliverable? 
    \item What pain points did you experience during this deliverable, and how
    did you resolve them?
    \item Which of your listed risks had your team thought of before this
    deliverable, and which did you think of while doing this deliverable? For
    the latter ones (ones you thought of while doing the Hazard Analysis), how
    did they come about?
    \item Other than the risk of physical harm (some projects may not have any
    appreciable risks of this form), list at least 2 other types of risk in
    software products. Why are they important to consider?
\end{enumerate}

\end{document}