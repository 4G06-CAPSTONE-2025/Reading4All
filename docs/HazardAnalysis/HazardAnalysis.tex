\documentclass{article}

\usepackage{booktabs}
\usepackage{tabularx}
\usepackage{hyperref}
\usepackage{longtable}
\usepackage{array}
\usepackage{booktabs}
\usepackage{enumitem}
\usepackage{graphicx}  
\usepackage{makecell}
\usepackage{multicol}

\usepackage{pdflscape}
\usepackage[margin=1.5in]{geometry}  

\hypersetup{ 
    colorlinks=true,       % false: boxed links; true: colored links
    linkcolor=red,          % color of internal links (change box color with linkbordercolor)
    citecolor=green,        % color of links to bibliography
    filecolor=magenta,      % color of file links
    urlcolor=cyan           % color of external links
}

\title{Hazard Analysis\\\progname}

\author{\authname}

\date{}

\input{../Comments}
%% Common Parts

\newcommand{\progname}{ProgName} % PUT YOUR PROGRAM NAME HERE
\newcommand{\authname}{Team 22, READING4ALL
\\ Fiza Sehar
\\ Nawaal Fatima
\\ Dhruv Sardana
\\ Moly Mikhail
\\ Casey Francine Bulaclac } % AUTHOR NAMES                  

\usepackage{hyperref}
    \hypersetup{colorlinks=true, linkcolor=blue, citecolor=blue, filecolor=blue,
                urlcolor=blue, unicode=false}
    \urlstyle{same}
                                


\begin{document}

\maketitle
\thispagestyle{empty}

~\newpage

\pagenumbering{roman}

\begin{table}[hp]
  \caption{Revision History} \label{TblRevisionHistory}
  \begin{tabularx}{\textwidth}{llX}
    \toprule
    \textbf{Date} & \textbf{Developer(s)} & \textbf{Change}\\
    \midrule
    October 10th 2025 & All Team Members & Initial Draft of HA \\
    November 24th 2025 & Casey Francine Bulaclac & Implement Peer Review \#155 \\
    \bottomrule
  \end{tabularx}
\end{table}

~\newpage

\tableofcontents

~\newpage

\pagenumbering{arabic}


\section{Introduction}

\subsection{Problem Statement}
Students with disabilities face major challenges to reading technical diagrams, which are often given in the form of static images that screen readers cannot interpret. Manually developing alternative (alt) text is possible, but it is resource-intensive, inconsistent, and not scalable for huge volumes of course content. This generates disparities in access to learning materials in postsecondary education.

To address this issue, this project proposes creating an Artificial Intelligence (AI) / Machine Learning (ML) powered application that automatically generates clear and detailed alt text for technical diagrams. The tool intends to increase accessibility, assure compliance with Accessibility for Ontarians with Disabilities Act (AODA) regulations, and promote greater inclusion in higher education.

\subsection{Hazard Analysis Introduction}
A hazard is a system state or combination of conditions that, when paired with specific environmental or contextual factors, can result in an unwanted or adverse event in the system or its surroundings. Hazards in software and AI development encompass not only physical threats but also usability, ethical, and technological issues.

Hazards in this project may develop during the design, training, and implementation of the alt-text generating model. These risks include technical flaws in generated text, misinterpretations that may mislead learners, ethical concerns regarding bias in the dataset, and compatibility issues with assistive technology.

This hazard analysis identifies and evaluates such risks to guarantee that the system provides consistent accessibility gains while not mistakenly introducing new obstacles.

\section{Scope and Purpose of Hazard Analysis}
The objective of this hazard analysis is to analyze the potential hazards and damages related to the AI/ML alt-text generating tool throughout its development and operation. The most significant hazards are:

\begin{itemize}
    \item \textbf{Technical Flaws:} Incorrect or inadequate alt text may mislead students, impair comprehension, or violate accessibility guidelines. This could result in a loss of confidence in the tool and poor results in education.
    \item \textbf{User Insensitivity:} Alt text that is excessively technical, unsophisticated, or inconsistent may not meet the needs of students. This may result in a loss of usefulness, reducing participation among students, teachers, and accessibility professionals.
    \item \textbf{External Dependencies:} Relying on third-party libraries, Application Programming Interfaces (APIs), or screen reader compatibility can lead to external failures. Interruptions in these functions may result in a loss of functionality or downtime.
\end{itemize}

The purpose of this hazard study is to systematically identify these hazards, estimate their potential impact, and develop mitigation strategies. By doing this, we aim to minimize losses associated with time, trust, resources, and accessibility outcomes, ensuring that the tool contributes positively to equitable education.

\section{System Boundaries and Components}

\subsection{System Overview}
The system is a web-based AI tool that generates alt text
for uploaded images or figures
and integrated with screen readers to improve accessibility for
visually impaired users.
\subsection{System Boundaries}
\begin{itemize}
    \item \textbf{Internal Components:} Alternative Text Generation ML Model, User Interface (UI), Session History Manager, Backend Server
    \item \textbf{External Components:} McMaster Authentication System, External AI/ML Frameworks, Screen Reader Software
\end{itemize}
\subsection{System Components}
\begin{enumerate}
  \item \textbf{Alt Text Generation ML Model}
    \begin{itemize}
      \item \textbf{Purpose:} To automatically generate accurate and
        descriptive alt text for uploaded images using machine learning.
      \item \textbf{Key Functions:}
        \begin{itemize}
          \item Process image inputs received from the backend and
            extract key visual features.
          \item Generate contextually relevant text descriptions.
          \item Return the generated alt text to the backend
            to display to the user interface.
        \end{itemize}
    \end{itemize}
  \item \textbf{User Interface}
    \begin{itemize}
      \item \textbf{Purpose:} To serve as the primary interaction
        point between the user and the system and allow users to
        upload images and view generated alt text.
      \item \textbf{Key Functions:}
        \begin{itemize}
          \item Enable users to upload images through an accessible
            web interface.
          \item Display generated alt text and allow users to edit,
            copy, and download the text.
          \item Provide features that are accessible and complies
            with the Web Content Accessibility Guidelines (WCAG) 2.1 standards.
          \item Communicates user requests and display outputs from the backend.
        \end{itemize}
    \end{itemize}
  \item \textbf{Session History Manager}
    \begin{itemize}
      \item \textbf{Purpose:} To ensure the continuity of the current
        session and manage user data during active use of the system.
      \item \textbf{Key Functions:}
        \begin{itemize}
          \item Track unique user sessions throughout interaction
            with the web application.
          \item Store previously uploaded images and generated
            alt text for the current session to allow users
            to view history.
        \end{itemize}
    \end{itemize}
    \item \textbf{Backend Server}
    \begin{itemize}
        \item \textbf{Purpose:} To securely manage communication between the user interface, session manager, and the machine learning model.
        This is done by validating inputs, ensuring safety-related checks, and coordinating all data processing flows.
        \item \textbf{Key Functions:}
        \begin{itemize}
            \item Receive and process image upload requests from the user interface.
            \item Handle safety-related operations such as user authentication and
            authorization of session actions.
            \item Forward images to the Alt Text Generation ML Model for processing.
            \item Retrieve generated alt text from the model and return it to the user interface. 
            \item Interact with the Session Manager to store or retrieve session data such as previous uploads and generated results.
            \item Manage error handling procedures to ensure reliability and prevent invalid states while operating the system.
        \end{itemize}
    \end{itemize}
\end{enumerate}
\section{Critical Assumptions}
This section documents the assumptions made during the hazard analysis of
\progname{} (Reading4All). The number of assumptions is kept to a minimum
to reduce the chance of overlooking potential hazards. Where assumptions
are made, they set clear boundaries for the analysis and define the
conditions under which the system is expected to operate safely.

\begin{itemize}
  \item \textbf{Assumption 1: Input Integrity.} All image files provided
    to the system are assumed to be valid image formats and not corrupted
    or maliciously constructed to exploit parsing vulnerabilities.

  \item \textbf{Assumption 2: Standards Stability.} Accessibility
    guidelines (WCAG 2.1, AODA) are assumed to remain stable for the
    operational lifetime of the system.

  \item \textbf{Assumption 3: Model Performance.} The machine learning
    models used for visual recognition and natural language generation are
    assumed to operate within validated ranges of accuracy and reliability.

  \item \textbf{Assumption 4: Human Oversight.} It is assumed that
    alt text generated by the system will undergo review by human
    instructors, teaching assistants, or accessibility specialists before
    being used in educational contexts.
\end{itemize}

Violations of these assumptions may introduce additional hazards outside
the current scope of this analysis. Such cases would require
re-evaluation of risks and system design updates.

\section{Failure Mode and Effect Analysis}
The \autoref{tab:fmea-reading4all} below outlines the hazards found within our system, as well as effects of failure, possible causes and the recommended actions. 

\begin{landscape}
    \renewcommand{\arraystretch}{1.2}
    \setlength{\tabcolsep}{5pt}
    \small                         
    
    \begin{longtable}{|p{1.2cm}|p{2.0cm}|p{3.0cm}|p{3.2cm}|p{3.2cm}|p{4.0cm}|p{2.6cm}|}
    \caption{Failure Mode and Effect Analysis (FMEA) for Reading4All}
    \label{tab:fmea-reading4all}\\
    \hline
    \textbf{HA ID} & \textbf{Component} & \textbf{Failure Mode} & \textbf{Effects of Failure} &
    \textbf{Possible Causes} & \textbf{Recommended Action / Mitigation} &
    \textbf{SRS Ref.} \\
    \hline
    \endfirsthead
    \multicolumn{7}{r}{\small \textit{Table \thetable\ (continued)}}\\
    \hline
    \textbf{HA ID} & \textbf{Component} & \textbf{Failure Mode} & \textbf{Effects of Failure} &
    \textbf{Possible Causes} & \textbf{Recommended Action / Mitigation} &
    \textbf{SRS Ref.} \\
    \hline
    \endhead
    \hline
    \multicolumn{7}{r}{\small \textit{Table continues on next page}}\\
    \hline
    \endfoot
    \hline
    \endlastfoot   
    
    HA-1 & Frontend UI &
    All user interface components cannot be controlled using keyboard &
    Users using assistive technology cannot operate all features &
    Missing Accessible Rich Internet Applications (ARIA) roles, incorrect tab order &
    WCAG~2.1 Level~AA evaluation; fix focus order and ARIA; add additional tests for accessibility needs &
    UHR-AR~1, UHR-AR~2 \\ \hline
    
    HA-2 & Frontend UI &
    Alt text output not compatible with screen readers &
    Screen readers skip generated alt text &
    Missing ARIA labels &
    Validate with NVDA/JAWS/VoiceOver; use \texttt{aria-label} with user interface components  &
    FR~3, UHR-AR~1 \\ \hline
    
    HA-3 & Backend Controller &
    Accepts unsupported/corrupted files &
    System crash or user confusion &
    Missing images/type validation or file-size limits &
    Validate uploads; enforce size/type checks; detailed error messages &
    FR~1, PR-RFT~1, SR-IM~1 \\ \hline
    
    HA-4 & Backend Controller &
    Timeout during image analysis &
    User perceives failure and is frustrated with system; repeated submissions; increased load on backend controller &
    Increased model latency; missing timeout handling &
    Add timeouts/retries for better user handling; progress indicator &
    PR-SL~1, PR-SL~2 \\ \hline
    
    HA-5 & Alt Text Generation Model &
    Generates offensive/biased or Personal Interest Information (PII)–leaking text &
    Ethical/privacy risk; loss of user trust &
    Lack of Model Training and Accuracy; no output filtering &
    Add additional filters and checks for PII Data and offensive texts &
    SR-PR~2 \\ \hline
    
    HA-6 & Alt Text Generation Model &
    No output or empty alt text &
    Users cannot use result; reduced learning impact &
    Interface failure or Application Programmable Interface (API) crash &
    Retry option; “No Text Generated” label; clear user feedback &
    PR-RFT~2 \\ \hline
    
    HA-7 & Session Storage Component &
    Images metadata or generated alt text not deleted after processing &
    Privacy exposure; increased storage/cost &
    Cleanup jobs fail or not configured &
    Auto-delete temp files; periodic cleanup; log storage usage; manual deletion triggered by deletion failure alarms &
    SR-PR~1 \\ \hline
    
    HA-8 & Session Storage Component &
    Session history not found &
    Loss of user trust and session data causing frustration &
    Session key mismatch; write errors &
    Atomic writes; bind session to SSO token &
    FR~5, SR-AR~2 \\ \hline
    
    HA-9 & Alt Text Model Output Training Evaluation &
    Evaluation metrics and scale linked to wrong model output &
    Inaccurate metrics &
    Race condition; wrong foreign key &
    Immutable IDs; transactional writes; enforce referential integrity &
    PR-PAR~1 \\ \hline
    
    HA-10 & McMaster SSO (Access Control) &
    Session bypass &
    Unauthorized access &
    Token reuse; improper validation &
    Validate tokens on the server &
    SR-AR~1, SR-AR~2 \\ \hline

    \end{longtable}
    
\end{landscape}
    
    
\section{Safety and Security Requirements}
\begin{enumerate}[label=PR-SR-HA \arabic*., wide=0pt, leftmargin=*]
  \item \emph{The system must notify the user when a timeout occurs during alt text generation.}\\[2mm]
    {\bf Rationale:} Users should be informed when the alt text generation exceeds the expected amount of time. If users are not notified they may send multiple requests, leading the server to be overloaded; this would also lead to user frustration.\\
    {\bf Fit Criterion:} When a timeout occurs, the system displays a message indicating the timeout and a ''Retry" option. The message must follow accessibility guidelines and be compatible with screen readers.\\
    {\bf Priority:} Medium\\
    {\bf Hazard Analysis Connected}: HA4
  \item \emph{The system must safely exit when a timeout occurs and ensure that no user data or incomplete alt text is stored or shown to the user.}\\[2mm]
    {\bf Rationale:} Safely exiting during a timeout prevents users from seeing incomplete alt text and mistaking it for a complete output, which may cause confusion. Leaving user data stored would also be a security violation. \\
    {\bf Fit Criterion:} When a timeout occurs, the system must stop processing and delete the users data and any incomplete alt text that has been generated.\\
    {\bf Priority:} High\\
    {\bf Hazard Analysis Connected}: HA4
  \item \emph{The system must ensure that messages notifying the user of failure, do not reveal any system code or data. }\\[2mm]
    {\bf Rationale:} This will prevent internal data from being shown to users, which may lead to system and user security issues. \\
    {\bf Fit Criterion:} All error messages shown to the user only display the necessary information and do not contain any technical information.\\
    {\bf Priority:} High \\
    {\bf Hazard Analysis Connected}: HA4, HA7
\end{enumerate}

\section{Roadmap}

All safety and security requirements outlined in our SRS, as well as newly discovered requirements will be implemented as part of the capstone timeline. This includes the following requirements, which can be found in our SRS document: 
\begin{multicols}{4}
\begin{itemize}
  \item SR-AR 1
  \item SR-AR 2 
  \item SR-IR 1
  \item SR-IR 2
  \item SR-PR 1
  \item SR-PR 2
  \item SR-AU 1
  \item SR-AU 2
  \item SR-IM 1
  \item SR-IM 2
  \item PR-SCR 1
  \item PR-SCR 2
  \item PR-SCR 3
  \item PR-SCR-HA 1
  \item PR-SCR-HA 2
  \item PR-SCR-HA 3
\end{itemize}
\end{multicols}
\newpage{}

\section*{Appendix --- Reflection}

\input{../Reflection.tex}\\

\textbf{Group - Reflection}
\begin{enumerate}
     \item \textbf{Which of your listed risks had your team thought of before this
    deliverable, and which did you think of while doing this deliverable? For
    the latter ones (ones you thought of while doing the Hazard Analysis), how
    did they come about?}

    Prior to completing this deliverable we had considered the privacy hazard, specifically relating to user's data. 
    We also had discussed the need to add programmable labels when developing our system. Furthermore, completing this deliverable
    revealed many hazards, the most significant being the possibility of the system timing out and the issues that may cause. 
    It revealed other hazards relating to the model evaluation process, which is essential to our system. 


    \item \textbf{Other than the risk of physical harm (some projects may not have any
    appreciable risks of this form), list at least 2 other types of risk in
    software products. Why are they important to consider?}

    One main risk in software products is in security and privacy. Situations such as unauthorized access and data leaks can harm users of the software and 
    can violate laws. This can also lead to damage of the developer's reputation, as well as fines and lawsuits. Another risk in software products is failure to comply with 
    certain required regulations. For example, violating licenses or accessibility rules can result in the delay or blocking of deployments, and may also lead 
    to legal issues. 
\end{enumerate}

\textbf{Moly Mikhail - Reflection}
\begin{enumerate}
    \item \textbf{What went well while writing this deliverable?}
    Completing part 6, where we outlined the Safety and Security requirements, 
    went well for several reasons. Firstly, completing the previous section 5 helped us directly derive the new Safety and
    Security requirements that needed to be outlined. Additionally, working on this document after finishing the SRS
    document made the process much easier, as we had gained a lot of practice with determining and writing requirements from the SRS. 
    \item \textbf{What pain points did you experience during this deliverable, and how
    did you resolve them?}
    One pain point we faced during this deliverable was the dependency on other sections within the 
    HA document and the SRS document. For example, to complete section 5 in the HA document, we needed
    the requirements in SRS to be completed so they can be referenced as needed. This was challenging as 
    developing our SRS requirements was a long process that required significant work and detail; 
    meanwhile, the HA Failure Mode and Effect analysis also needed extensive work.  
    This made it hard to manage our time effectively, as the two documents couldn’t be worked on in parallel at times. 
\end{enumerate}

\textbf{Nawaal Fatima  - Reflection}
\begin{enumerate}
  \item \textbf{What went well while writing this deliverable?}\newline
    I think the easiest part was coming up with the assumptions
    themselves. From my summer work experience with Ms. Sui, I was already familiar with the main areas where Reading4All
    could run into issues (inputs, standards, model accuracy, and
    human review), it felt pretty natural to turn those into clear
    assumptions. It also helped that the section didn’t need to be
    super long, so I could keep it focused and to the point.

  \item \textbf{What pain points did you experience during this
    deliverable, and how did you resolve them?}\newline
    The tricky part was figuring out how much detail to put in
    without overcomplicating things. At first, I thought about
    writing assumptions that basically ruled out certain failures,
    but I realized that would go against the whole idea of hazard
    analysis. To fix that, I stuck to assumptions that made sense for
    defining boundaries without pretending hazards don’t exist.
    Another small challenge was making the wording simple enough - so
    I rewrote a couple of the assumptions to sound clearer and less
    ``technical report'' heavy.
\end{enumerate}
\textbf{Fiza Sehar  - Reflection}
\begin{enumerate}
    \item What went well while writing this deliverable? \\[1ex]
    Sections 1 and 2 were more manageable to write because the project's goals 
    and problem statement were clearly specified in the SRS. This made 
    it easy to explain the goal and scope of the hazard analysis. Understanding 
    accessibility standards like AODA and WCAG 2.1 enabled us to clearly link 
    identified concerns (technical, ethical, and usability) to specific compliance 
    requirements. Our team was already familiar with much of the contextual 
    groundwork, notably AI-generated alt text and educational accessibility, so 
    the authoring process was straightforward.

    
    \item What pain points did you experience during this deliverable, and how
    did you resolve them?\\[1ex]
    One big problem was blending technical detail with explanations that were 
    more accessible. Because this document combines software engineering and 
    accessibility compliance, it was challenging to prevent repetition between 
    the SRS and Hazard Analysis. We also found it difficult to make the hazard 
    descriptions concise while maintaining clarity for non-technical readers. 
    To address this, we altered the structure several times and used clear subheadings 
    and bullet points to distinguish between hazard types (Technical flaws, user insensitivity, 
    and external dependencies).
\end{enumerate}
    
\textbf{Dhruv Sardana  - Reflection}
\begin{enumerate}
  \item \textbf{What went well while writing this deliverable?}\newline
    While writing this deliverable, I was responsible for the FMEA Table. This gave me an opportunity to 
    understand the system in depth and identify potential failure modes. It helped me identify different risks and 
    it helped other team members understand requirements better and helped refine the SRS document. 
    It also helped me otherstand the mitigation strategies for the identified risks.

  \item \textbf{What pain points did you experience during this
    deliverable, and how did you resolve them?}\newline
    Some of the pain points I faced while writing this deliverable was the dependency on other sections within the document.
    Especially referring to SRS document and accompanying them with system components. 
    This was challenging as the SRS document was being worked on in parallel and requirements were constantly being refined
\end{enumerate}

\textbf{Casey Francine Bulaclac - Reflection}
\begin{enumerate}
    \item \textbf{What went well while writing this deliverable? }
    Writing this deliverable was organized as the team split the sections beforehand, so we had clear instructions on how and who is completing which sections. It 
    was also helpful completing the SRS before this document as it helped define certain sections such as the System Boundaries and Components. 

    \item \textbf{What pain points did you experience during this deliverable, and how
    did you resolve them?}
    One challenge I ran into with this deliverable were keeping the language and details aligned with the SRS. Another challenge was having to finish one section before I could move on 
    to others as some sections depended on previous ones. To address this, I double-checked the SRS and kept clear, timely communication with my team about what needed to be done and by when.
\end{enumerate}

\end{document}
