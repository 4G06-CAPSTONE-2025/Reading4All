% THIS DOCUMENT IS FOLLOWS THE VOLERE TEMPLATE BY Suzanne Robertson and James Robertson
% ONLY THE SECTION HEADINGS ARE PROVIDED
%
% Initial draft from https://github.com/Dieblich/volere
%
% Risks are removed because they are covered by the Hazard Analysis
\documentclass[12pt]{article}

\usepackage{booktabs}
\usepackage{tabularx}
\usepackage{enumitem}
\usepackage{hyperref}
\hypersetup{
    bookmarks=true,         % show bookmarks bar?
      colorlinks=true,      % false: boxed links; true: colored links
    linkcolor=red,          % color of internal links (change box color with linkbordercolor)
    citecolor=green,        % color of links to bibliography
    filecolor=magenta,      % color of file links
    urlcolor=cyan           % color of external links
}

\newcommand{\lips}{\textit{Insert your content here.}}

%% Comments

\usepackage{color}

\newif\ifcomments\commentstrue %displays comments
%\newif\ifcomments\commentsfalse %so that comments do not display

\ifcomments
\newcommand{\authornote}[3]{\textcolor{#1}{[#3 ---#2]}}
\newcommand{\todo}[1]{\textcolor{red}{[TODO: #1]}}
\else
\newcommand{\authornote}[3]{}
\newcommand{\todo}[1]{}
\fi

\newcommand{\wss}[1]{\authornote{magenta}{SS}{#1}} 
\newcommand{\plt}[1]{\authornote{cyan}{TPLT}{#1}} %For explanation of the template
\newcommand{\an}[1]{\authornote{cyan}{Author}{#1}}

%% Common Parts

\newcommand{\progname}{ProgName} % PUT YOUR PROGRAM NAME HERE
\newcommand{\authname}{Team \#, Team Name
\\ Student 1 name
\\ Student 2 name
\\ Student 3 name
\\ Student 4 name} % AUTHOR NAMES                  

\usepackage{hyperref}
    \hypersetup{colorlinks=true, linkcolor=blue, citecolor=blue, filecolor=blue,
                urlcolor=blue, unicode=false}
    \urlstyle{same}
                                


\begin{document}

\title{Software Requirements Specification for \progname: subtitle describing software} 
\author{\authname}
\date{\today}
	
\maketitle

~\newpage

\pagenumbering{roman}

\tableofcontents

~\newpage

\section*{Revision History}

\begin{tabularx}{\textwidth}{p{3cm}p{2cm}X}
\toprule {\textbf{Date}} & {\textbf{Version}} & {\textbf{Notes}}\\
\midrule
Date 1 & 1.0 & Notes\\
Date 2 & 1.1 & Notes\\
\bottomrule
\end{tabularx}

~\\

~\newpage
\section{Introduction}

\subsection{Purpose of Document}
This Software Requirements Specification (SRS) defines the functional and non-functional requirements for the \progname{} system, \textbf{Reading4All}. The document explains the project's aims, context, restrictions, and expected behavior in order to ensure that developers, accessibility professionals, instructors, and project stakeholders all have a shared understanding.

This SRS provides a formal reference for Reading4All's design, implementation, testing, and validation. It establishes a clear link between user requirements, accessibility standards (such as AODA and WCAG 2.1), and system features that enable reliable alternative text generation for technical diagrams.

\subsection{Scope of Project}
Reading4All is an artificial intelligence/machine learning tool that provides detailed and contextually aware alternative text for complicated technical graphics, notably those found in postsecondary STEM course materials. The system combines machine vision and natural language generation models to assess diagram content, identify key parts, and provide text descriptions that are compatible with screen readers and assistive technology.

\noindent \textbf{Core objectives:}
\begin{itemize}
    \item Automate the creation of comprehensive and accurate alternative text for technical diagrams.
    \item Maintain compliance with the \textbf{Accessibility for Ontarians with Disabilities Act (AODA)} and the \textbf{WCAG 2.1} accessibility standards.
    \item Improve the learning experiences and inclusion of students with visual challenges in higher education.
    \item Reduce the instructor effort and institutional costs related to manual alt-text creation.
\end{itemize}

\noindent \textbf{Final deliverable includes:}
\begin{itemize}
    \item A web-based or locally hosted interface for uploading images and creating alternative-texts.
    \item A backend service that combines AI models for diagram analysis and generation of languages.
    \item Structured output formats are suitable with systems for learning management and screen readers.
\end{itemize}

\subsection{Characteristics of Intended Reader}
This document is intended for:
\begin{itemize}
    \item \textbf{Developers} who are responsible for implementing and testing the system.
    \item \textbf{Accessibility Professionals} to ensure AODA and WCAG compliance.
    \item \textbf{Academic Stakeholders} including educators, instructional designers, and content creators.
    \item \textbf{Supervisors and Assessors} who assess project completion and design quality.
\end{itemize}

Readers should have an overall knowledge of software engineering, web development, and fundamental machine learning techniques. Accessibility reviewers should understand digital accessibility principles and standards.

\subsection{Organization of Document}
This SRS is organized as follows:
\begin{itemize}
    \item \textbf{Section 1: Introduction} – Overview of the document's aim and intended readership.
    \item \textbf{Section 2: Overall Description} – System context, product functionality, user characteristics, and design restrictions.
    \item \textbf{Section 3: Specific Requirements} – Includes both the functional and non-functional criteria.
    \item \textbf{Section 4: Appendices} – Additional information, such as references, citations, and definitions.
\end{itemize}

\subsection{System Context}
Reading4All serves as a link between professionals who post technical images and students who use screen readers to receive the resulting descriptions.

\noindent \textbf{System Environment:}
\begin{itemize}
    \item \textbf{Inputs:} Technical schematics or graphics (JPEG, PNG, or PDF).
    \item \textbf{Processing:} A computer vision model extracts diagram content, while a language model generates descriptive text.
    \item \textbf{Outputs:} Structured textual descriptions in formats that are accessible such as HTML alt attributes, text files, and ARIA labels.
\end{itemize}

\subsection{Document Conventions}
This document adheres to the layout and guidelines specified in the McMaster University SFWRENG 4X03/4HC3 Capstone SRS template. All measurements are in SI units. Mathematical or symbolic notation will be introduced in subsequent parts if needed.

\subsection{Reference Material}
Relevant Standards and Reference Documents:
\begin{itemize}
    \item Accessibility for Ontarians with Disabilities Act (AODA, 2005)
    \item Web Content Accessibility Guidelines (WCAG 2.1)
    \item ISO/IEC 25010:2011 Software Quality Model
    \item \textit{Drasil SRS Template}, McMaster University
\end{itemize}
\lips
\section{Stakeholders}
\subsection{Client}
\lips
\subsection{Customer}
\lips
\subsection{Other Stakeholders}
\lips
\subsection{Hands-On Users of the Project}
\lips
\subsection{Personas}
\lips
\subsection{Priorities Assigned to Users}
\lips
\subsection{User Participation}
\lips
\subsection{Maintenance Users and Service Technicians}
\lips

\section{Mandated Constraints}
\subsection{Solution Constraints}
\lips
\subsection{Implementation Environment of the Current System}
\lips
\subsection{Partner or Collaborative Applications}
\lips
\subsection{Off-the-Shelf Software}
\lips
\subsection{Anticipated Workplace Environment}
\lips
\subsection{Schedule Constraints}
\lips
\subsection{Budget Constraints}
\lips
\subsection{Enterprise Constraints}
\lips

\section{Naming Conventions and Terminology}
\subsection{Glossary of All Terms, Including Acronyms, Used by Stakeholders
involved in the Project}
\lips

\section{Relevant Facts And Assumptions}
\subsection{Relevant Facts}
\lips
\subsection{Business Rules}
\lips
\subsection{Assumptions}
\lips

\section{The Scope of the Work}
\subsection{The Current Situation}
\lips
\subsection{The Context of the Work}
\lips
\subsection{Work Partitioning}
\lips
\subsection{Specifying a Business Use Case (BUC)}
\lips

\section{Business Data Model and Data Dictionary}
\subsection{Business Data Model}
\lips
\subsection{Data Dictionary}
\lips

\section{The Scope of the Product}
\subsection{Product Boundary}
\lips
\subsection{Product Use Case Table}
\lips
\subsection{Individual Product Use Cases (PUC's)}
\lips

\section{Functional Requirements}
\subsection{Functional Requirements}
\lips

\section{Look and Feel Requirements}
\subsection{Appearance Requirements}
\lips
\subsection{Style Requirements}
\lips

\section{Usability and Humanity Requirements}
\subsection{Ease of Use Requirements}
\lips
\subsection{Personalization and Internationalization Requirements}
\lips
\subsection{Learning Requirements}
\lips
\subsection{Understandability and Politeness Requirements}
\lips
\subsection{Accessibility Requirements}
\lips

\section{Performance Requirements}
\subsection{Speed and Latency Requirements}
\lips
\subsection{Safety-Critical Requirements}
\lips
\subsection{Precision or Accuracy Requirements}
\lips
\subsection{Robustness or Fault-Tolerance Requirements}
\lips
\subsection{Capacity Requirements}
\lips
\subsection{Scalability or Extensibility Requirements}
\lips
\subsection{Longevity Requirements}
\lips
\section{Operating Environment Requirements}

The following operating environment requirements specify the physical, wider, interfacing, and productization contexts within which the Reading4All system must function reliably and efficiently.

\subsection{Expected Physical Environment}

\begin{enumerate}[label=OER-EP\arabic*., wide=0pt, leftmargin=*]
  \item \emph{The system shall be operable on standard computing devices (laptops, desktops, or servers) running common operating systems such as Windows, macOS, or Linux.}\\[2mm]
    {\bf Rationale:} Reading4All is intended for academic and institutional use, where users employ varied operating systems. Ensuring cross-platform support maximizes accessibility and ease of adoption.\\
    {\bf Fit Criterion:} The system runs successfully on all major operating systems without compatibility errors.\\
    {\bf Priority:} High

  \item \emph{The system shall operate effectively under normal ambient temperature and indoor lighting conditions typical of classrooms or office environments.}\\[2mm]
    {\bf Rationale:} As the system is designed for digital use within educational or professional environments, no special hardware or environmental setup is required.\\
    {\bf Fit Criterion:} The tool performs consistently in standard room conditions (10°C–35°C, normal lighting).\\
    {\bf Priority:} Low
\end{enumerate}

\subsection{Wider Environment Requirements}

\begin{enumerate}[label=OER-WE\arabic*., wide=0pt, leftmargin=*]
  \item \emph{The system shall comply with accessibility and data privacy regulations, including AODA and WCAG 2.1 standards.}\\[2mm]
    {\bf Rationale:} Compliance ensures that the system promotes digital inclusion while protecting user data and adhering to institutional accessibility mandates.\\
    {\bf Fit Criterion:} Independent evaluation confirms AODA and WCAG 2.1 Level AA compliance; no accessibility blockers are reported during user testing.\\
    {\bf Priority:} High

  \item \emph{The system shall function within typical academic network environments with stable internet connectivity.}\\[2mm]
    {\bf Rationale:} Reading4All relies on cloud-based inference models for AI text generation; stable network access is essential for accurate and timely processing.\\
    {\bf Fit Criterion:} The system maintains successful API communication under standard university Wi-Fi and Ethernet conditions.\\
    {\bf Priority:} Medium
\end{enumerate}

\subsection{Requirements for Interfacing with Adjacent Systems}

\begin{enumerate}[label=OER-IAS\arabic*., wide=0pt, leftmargin=*]
  \item \emph{The system shall integrate with learning management systems (LMS) such as Avenue2Learn and Moodle for image uploads and description retrieval.}\\[2mm]
    {\bf Rationale:} Integration with LMS platforms simplifies the workflow for instructors and students, allowing automatic alternative text generation directly from course content.\\
    {\bf Fit Criterion:} Reading4All successfully connects to at least one LMS through API endpoints for image retrieval and text output.\\
    {\bf Priority:} High

  \item \emph{The system shall support interoperability with assistive technologies such as screen readers (e.g., NVDA, JAWS, and VoiceOver).}\\[2mm]
    {\bf Rationale:} Compatibility with screen readers ensures that the generated descriptions are correctly interpreted and read aloud to visually impaired users.\\
    {\bf Fit Criterion:} Descriptions produced by Reading4All are correctly parsed and read by major screen readers without formatting errors.\\
    {\bf Priority:} High

  \item \emph{The system shall support importing diagrams from common file formats (PNG, JPG, SVG, PDF).}\\[2mm]
    {\bf Rationale:} Supporting standard file formats ensures flexibility for users across disciplines and sources of visual content.\\
    {\bf Fit Criterion:} The system can process all listed file formats and produce corresponding alternative text.\\
    {\bf Priority:} Medium
\end{enumerate}

\subsection{Productization Requirements}

\begin{enumerate}[label=OER-PR\arabic*., wide=0pt, leftmargin=*]
  \item \emph{The system shall be deployable as both a web application and an API service for institutional integration.}\\[2mm]
    {\bf Rationale:} Dual deployment ensures accessibility for both individual users (web) and organizations (API integration into accessibility workflows).\\
    {\bf Fit Criterion:} A hosted web app and a functional REST API are accessible and verified through institutional testing.\\
    {\bf Priority:} High

  \item \emph{The system shall store configuration and model parameters in a secure and scalable manner.}\\[2mm]
    {\bf Rationale:} Proper configuration management ensures reproducibility of model outputs and secure handling of sensitive accessibility data.\\
    {\bf Fit Criterion:} Configuration files are encrypted and securely stored; model versions are tracked and retrievable.\\
    {\bf Priority:} Medium
\end{enumerate}

\subsection{Release Requirements}

\begin{enumerate}[label=OER-RL\arabic*., wide=0pt, leftmargin=*]
  \item \emph{All major functionalities (image analysis, text generation, and accessibility validation) must be implemented, tested, and verified prior to release.}\\[2mm]
    {\bf Rationale:} Ensures the system delivers on its core promise of reliable and inclusive alt-text generation before public or institutional rollout.\\
    {\bf Fit Criterion:} Verification and Validation (V\&V) documentation confirms that each functional requirement has been tested successfully.\\
    {\bf Priority:} High

  \item \emph{The system must be ready for release by March 17th, 2025, in alignment with the McMaster University SFWRENG 4G06 Capstone demonstration schedule.}\\[2mm]
    {\bf Rationale:} Adherence to this timeline ensures timely completion for academic evaluation and presentation to stakeholders.\\
    {\bf Fit Criterion:} The final deliverable is fully functional and deployed for the Capstone demonstration date.\\
    {\bf Priority:} Medium
\end{enumerate}


\section{Maintainability and Support Requirements}

The following are defined as maintainability and support requirements for the Reading4All system. These ensure that the tool can be efficiently maintained, supported, and adapted to meet future accessibility standards and technological advancements.

\subsection{Maintenance Requirements}

\begin{enumerate}[label=MS-MNT\arabic*., wide=0pt, leftmargin=*]
  \item \emph{The system shall be designed with modular components to enable updates to individual subsystems such as the vision model, language model, or accessibility checker without affecting the entire architecture.}\\[2mm]
    {\bf Rationale:} Modularization ensures that AI models or accessibility modules can be updated independently as new datasets or techniques become available, reducing downtime and maintenance complexity.\\
    {\bf Fit Criterion:} Developers can update or replace a subsystem (e.g., the text generator or image classifier) without requiring a full system redeployment.\\
    {\bf Priority:} High

  \item \emph{The system shall maintain comprehensive developer documentation, including setup instructions, API usage guides, and dataset handling protocols.}\\[2mm]
    {\bf Rationale:} Clear and complete documentation enables new developers to understand and maintain the system without relying on the original development team.\\
    {\bf Fit Criterion:} Documentation is verified through internal onboarding tests—new developers can successfully deploy and test the system using only the provided materials.\\
    {\bf Priority:} High

  \item \emph{The system shall include automated testing and CI/CD integration to verify core functionality after each update.}\\[2mm]
    {\bf Rationale:} Automated testing ensures that future changes to AI models, APIs, or user interfaces do not break existing functionality. This is especially critical for accessibility validation and compliance features.\\
    {\bf Fit Criterion:} Every merge or model update triggers an automated test suite that validates performance, accuracy, and WCAG/AODA compliance.\\
    {\bf Priority:} Medium
\end{enumerate}

\subsection{Supportability Requirements}

\begin{enumerate}[label=MS-SUP\arabic*., wide=0pt, leftmargin=*]
  \item \emph{The system shall include a feedback and issue-reporting feature accessible through the web interface.}\\[2mm]
    {\bf Rationale:} Allowing instructors and accessibility specialists to report errors or suggest improvements supports long-term refinement and ensures usability across diverse course materials.\\
    {\bf Fit Criterion:} Users can submit structured feedback (e.g., incorrect alt text, model inaccuracy) through a built-in form, and submissions are logged in a support database.\\
    {\bf Priority:} Medium

  \item \emph{The system shall allow maintainers to monitor API usage and model performance metrics over time.}\\[2mm]
    {\bf Rationale:} Continuous monitoring supports proactive troubleshooting, helps identify recurring accessibility errors, and allows performance optimization.\\
    {\bf Fit Criterion:} System logs include usage analytics and error rates accessible to maintainers via a dashboard or exported logs.\\
    {\bf Priority:} Medium
\end{enumerate}

\subsection{Adaptability Requirements}

\begin{enumerate}[label=MS-AD\arabic*., wide=0pt, leftmargin=*]
  \item \emph{The system shall support updates to comply with new accessibility standards and policies, including future revisions of WCAG and AODA.}\\[2mm]
    {\bf Rationale:} Accessibility guidelines evolve over time, and compliance with new standards is essential for ongoing institutional use.\\
    {\bf Fit Criterion:} The system’s compliance modules (e.g., alt-text validation rules) can be updated within one development cycle to align with revised guidelines.\\
    {\bf Priority:} High

  \item \emph{The system shall support integration with emerging accessibility tools and APIs used in academic institutions.}\\[2mm]
    {\bf Rationale:} Academic technology ecosystems change frequently; interoperability ensures that Reading4All remains relevant and compatible with new assistive platforms.\\
    {\bf Fit Criterion:} The system successfully connects to at least one new accessibility or LMS API added after initial release.\\
    {\bf Priority:} Medium

  \item \emph{The system shall be adaptable to different machine learning models for image captioning and language generation.}\\[2mm]
    {\bf Rationale:} As AI research advances, using improved models can enhance the accuracy and inclusivity of generated descriptions. The architecture must accommodate future ML updates.\\
    {\bf Fit Criterion:} New models can be integrated into the existing inference pipeline with only minor configuration changes (e.g., via YAML or JSON model config).\\
    {\bf Priority:} Medium
\end{enumerate}



\section{Security Requirements}
\subsection{Access Requirements}
\lips
\subsection{Integrity Requirements}
\lips
\subsection{Privacy Requirements}
\lips
\subsection{Audit Requirements}
\lips
\subsection{Immunity Requirements}
\lips

\section{Cultural Requirements}
\subsection{Cultural Requirements}
\lips

\section{Compliance Requirements}
\subsection{Legal Requirements}
\lips
\subsection{Standards Compliance Requirements}
\lips

\section{Open Issues}
\lips

\section{Off-the-Shelf Solutions}
\subsection{Ready-Made Products}
\lips
\subsection{Reusable Components}
\lips
\subsection{Products That Can Be Copied}
\lips
\section{Off-the-Shelf Solutions}

This section identifies existing tools, reusable components, and research products that can support or inspire the development of Reading4All. These solutions address various aspects of the system’s vision analysis, language generation, and accessibility workflows.

\subsection{Ready-Made Products}

\begin{itemize}
    \item \textbf{Google Cloud Vision API:} A commercial computer vision service capable of detecting objects, text, and structural elements within images. It can assist in preliminary diagram segmentation or optical character recognition for Reading4All’s visual analysis component.

    \item \textbf{Microsoft Azure Cognitive Services:} Offers image captioning and scene-description APIs that generate textual summaries of visual content. These can serve as a performance benchmark for Reading4All’s AI-generated academic alt text.

    \item \textbf{OpenAI GPT-4V (Vision):} A multimodal AI model capable of interpreting images and generating context-aware captions. It can be evaluated for use in academic diagrams and as a reference for Reading4All’s natural-language generation pipeline.
\end{itemize}

\subsection{Reusable Components}

\begin{itemize}
    \item \textbf{Hugging Face Transformers:} An open-source library offering pretrained multimodal models (e.g., BLIP-2, CLIP, and ViT-GPT2). These can be fine-tuned to identify structural and semantic relationships within STEM diagrams.

    \item \textbf{Albumentations:} A Python library for data augmentation, enhancing dataset diversity for diagram recognition tasks. It can be reused in Reading4All’s preprocessing pipeline to improve robustness.

    \item \textbf{Pandas + Matplotlib AltText Plugin:} An open-source extension that generates descriptive alt text for statistical plots. Its modular logic can guide Reading4All’s diagram-specific description component.
\end{itemize}

\subsection{Products That Can Be Copied}

\begin{itemize}
    \item \textbf{Chart2Text:} A benchmark system that automatically converts data visualizations such as bar charts and line graphs into natural-language summaries. Its approach to aligning visual features with linguistic structures provides a strong foundation for Reading4All’s diagram-description workflow.

    \item \textbf{SciA11y:} A research initiative by the Allen Institute for AI that generates accessible figure descriptions for scientific papers. Its methods for extracting captions, metadata, and contextual relationships between visual and textual elements align closely with Reading4All’s academic accessibility objectives.
\end{itemize}
\section{Tasks}

\subsection{Project Planning}

\begin{itemize}
    \item \textbf{Development Approach}  
    The Reading4All team will follow an iterative agile workflow emphasizing continuous improvement and regular feedback from stakeholders. Development will progress through a series of short sprints that each target measurable goals, allowing flexibility as requirements evolve. The general process will involve:
    \begin{enumerate}
        \item Requirement analysis and refinement of functional specifications
        \item Sprint planning and backlog prioritization
        \item Incremental coding and integration
        \item Verification, validation, and unit testing
        \item Periodic reviews with the project supervisor and accessibility lead
        \item System deployment and documentation finalization
    \end{enumerate}

    \item \textbf{Key Tasks}
    \begin{itemize}
        \item Confirm system architecture and interface specifications  
        \item Establish a shared GitHub repository with branching standards  
        \item Configure CI/CD automation through GitHub Actions  
        \item Implement the vision module for diagram segmentation and labeling  
        \item Integrate the language generation component for alt-text synthesis  
        \item Conduct user testing with instructors and accessibility specialists  
        \item Evaluate outputs for WCAG 2.1 compliance and descriptive accuracy  
    \end{itemize}

    \item \textbf{Resource Estimates}  
    The project involves a five-member team responsible for design, development, testing, and reporting. Shared tools will include:
    \begin{itemize}
        \item Cloud computing resources (GPU-based inference environment)  
        \item Version control and issue tracking using GitHub Projects  
        \item Datasets of academic diagrams and verified alternative texts  
    \end{itemize}

    \item \textbf{Key Considerations}
    \begin{itemize}
        \item Early integration testing to avoid dependency conflicts  
        \item Dataset bias evaluation to ensure inclusive model outputs  
        \item Consistent communication with the faculty advisor and accessibility consultant  
        \item Risk reduction through checkpoint testing and documentation reviews  
    \end{itemize}

    \item \textbf{Documentation Process}
    \begin{itemize}
        \item Maintain all documentation collaboratively within the GitHub Wiki  
        \item Use meaningful commit messages and structured pull requests  
        \item Require automated testing to pass before merging any code  
        \item Apply peer review and version tagging for every major deliverable  
        \item Archive final reports and test results for supervisor review  
    \end{itemize}
\end{itemize}

\subsection{Planning of the Development Phases}

\textbf{Deliverables and Tentative Schedule}

\begin{itemize}
    \item Problem Statement, Proof of Concept, and Development Plan — \textbf{Week 4 (September 24, 2025)}  
    \item Software Requirements Specification (SRS) and Hazard Analysis — \textbf{Week 6 (October 8, 2025)}  
    \item Verification and Validation Plan — \textbf{Week 8 (October 22, 2025)}  
    \item Design Document (Revision 0) — \textbf{Week 10 (November 5, 2025)}  
    \item Proof of Concept Demonstration — \textbf{Weeks 11–12 (November 12–19, 2025)}  
    \item Design Document (Revision 1) — \textbf{Week 16 (January 21, 2026)}  
    \item Revision 0 Demonstration — \textbf{Weeks 18–19 (February 4–11, 2026)}  
    \item Verification and Validation Report — \textbf{Week 22 (March 4, 2026)}  
    \item Final Demonstration (Revision 1) — \textbf{Week 24 (March 18, 2026)}  
    \item Capstone EXPO — \textbf{Week 26 (April 1, 2026)}  
    \item Final Documentation Submission — \textbf{Week 26 (April 1, 2026)}  
\end{itemize}

\textbf{Sprint and Review Cycle}
\begin{itemize}
    \item Each sprint will run for approximately two weeks, concluding with a progress review.  
    \item Development milestones will be validated through automated testing and peer evaluation.  
    \item Supervisor and accessibility advisor feedback will guide milestone adjustments and task prioritization.  
\end{itemize}

By adopting this structured development plan, the team will ensure that Reading4All evolves into a reliable and maintainable system capable of producing accurate, accessible, and pedagogically useful alt text for academic diagrams.


\section{Migration to the New Product}
\subsection{Requirements for Migration to the New Product}
\lips
\subsection{Data That Has to be Modified or Translated for the New System}
\lips

\section{Costs}
\lips
\section{User Documentation and Training}
\subsection{User Documentation Requirements}
\lips
\subsection{Training Requirements}
\lips

\section{Waiting Room}
\lips

\section{Ideas for Solution}
\lips

\newpage{}
\section*{Appendix --- Reflection}

The purpose of reflection questions is to give you a chance to assess your own
learning and that of your group as a whole, and to find ways to improve in the
future. Reflection is an important part of the learning process.  Reflection is
also an essential component of a successful software development process.  

Reflections are most interesting and useful when they're honest, even if the
stories they tell are imperfect. You will be marked based on your depth of
thought and analysis, and not based on the content of the reflections
themselves. Thus, for full marks we encourage you to answer openly and honestly
and to avoid simply writing ``what you think the evaluator wants to hear.''

Please answer the following questions.  Some questions can be answered on the
team level, but where appropriate, each team member should write their own
response:

\textbf{Group Reflection  - Reflection}
\begin{enumerate}
  \item What went well while writing this deliverable? 
  
  Throughout the process, our team maintained organization and had good communication.  It was simpler to preserve flow and prevent redundancy because we had a clear framework and consistent formatting from earlier deliverables.  We were also able to maintain our efficiency and keep the material consistent with our previous work.  In order review sections, define expectations, and make sure the deliverable satisfied all requirements, we also had meetings with our TA and our team.
  \item What pain points did you experience during this deliverable, and how did
  you resolve them?

  One difficulty was avoiding redundancy by keeping formatting constant and clearly distinguishing design components from implementation specifics. In order to guarantee accuracy and uniformity throughout all sections, we addressed this by going over the course templates and conducting team editing sessions to make sure the formatting followed a consistent style. We also helped each other out by sharing tips and tricks to improve clarity and presentation.

  \item How many of your requirements were inspired by speaking to your
  client(s) or their proxies (e.g. your peers, stakeholders, potential users)?
  
  Key design and functionality components that were influenced by internal meetings, conversations with our supervisor, pertinent compliance guidelines, and my internship's prior experience working with AI included the accessible interface, the quality of the alt text that was generated, and adherence to AODA and WCAG 2.1 standards.
  \item Which of the courses you have taken, or are currently taking, will help
  your team to be successful with your capstone project.

  The Software Requirements course from third year was extremely useful in creating this deliverable.  It created a solid foundation for organizing requirement documents, developing precise specifications, and grasping the foundations of software documentation and traceability, all of which significantly aided our work on this project.

  \item What knowledge and skills will the team collectively need to acquire to
  successfully complete this capstone project?  Examples of possible knowledge
  to acquire include domain specific knowledge from the domain of your
  application, or software engineering knowledge, mechatronics knowledge or
  computer science knowledge.  Skills may be related to technology, or writing,
  or presentation, or team management, etc.  You should look to identify at
  least one item for each team member.}\newline
  
  \item \textbf{For each of the knowledge areas and skills identified in the previous
  question, what are at least two approaches to acquiring the knowledge or
  mastering the skill?  Of the identified approaches, which will each team
  member pursue, and why did they make this choice?}\newline

\end{enumerate}


\textbf{Moly Mikhail  - Reflection}
\begin{enumerate}
  \item \textbf{What went well while writing this deliverable?} \newline
  I believe writing this deliverable many things went well. I really enjoyed getting to think about the different
  non-functional requirements. I found that have different sections of non-functional requirements encouraged me to think about different 
  aspects of the system and things we will have to keep in mind during development. For example, prior to writing 
  this deliverable, we hadn’t considered personalization and internationalization requirements; however, having to
  complete that section led us to add important functionality of allowing the user to decide which way
  to store the alternative text. 
  
\item \textbf{What pain points did you experience during this
    deliverable, and how did you resolve them?} \newline
  One pain point I experience writing this deliverable dealt with completing the product boundary. 
  Initially, I was confused on The Scope of the Product section and what was expected. 
  To resolve this, I researched the Volere Requirements Specification Template and looked into the section. 
  However, I was still confused and what was expected of the section. Finally, during our meeting with
  our TA I was able to clarify the expectations for this section and I was able to complete the section. 
 
   \item \textbf{How many of your requirements were inspired by
      speaking to your client(s) or their proxies (e.g., your peers,
    stakeholders, potential users)?} \newline
  I believe many of our non-functional requirements, specifically look and feel 
  requirements, as well as usability and humanity requirements. Through our
  conversations with our supervisor Jing, we learned a lot of the accessibility
  requirements for website applications. For example, one specific requirement 
  that was derived from our conversations was that the system cannot use color alone to convey any messages
  or information. I believe without having this conversation, this is a requirement that would not have been discovered. 
 
 \item \textbf{Which of the courses you have taken, or are currently
      taking, will help your team be successful with your capstone
    project?} \newline
  I believe many courses that I have taken, and some that I’m currently taking will 
  contribute to the success of our capstone project. I completed
  SFWRENG 4HC3 - Human Computer Interfaces, which has taught me many 
  important design principles, such as Normans Design Principles. Furthermore, 
  completing COMPSCI 4AL3 - Applications of Machine Learning, also will be a lot of help when 
  completing our capstone. This course introduced me to developing machine learning models and 
  will be directly applicable. Finally, taking COMPSCI 3RA3 - Software Requirements and Security 
  Considerations will also help our team be successful. 
  
\end{enumerate}

\textbf{Casey Francine Bulaclac - Reflection}
\begin{enumerate}
  \item \textbf{What went well while writing this deliverable?} \newline
  Having discussed the project thoroughly as a team and with our supervisor helped in writing this deliverable as the team
  was very knowledgeable about the needs for the project. 
  This deliverable went much smoother than the last due to stronger operational procedures, and better organization in how we structured and completed the SRS. 
  The team communicated well and were clear of the goals for this deliverable.
  \item \textbf{What pain points did you experience during this
  deliverable, and how did you resolve them?} \newline
  One pain point in writing the SRS was figuring out what each of the many sections entailed in the Volere's template. The template 
  is very thorough and needed many details, in which some sections seem to overlap which can be confusing. Another pain point was ensuring traceability
  between our goals in the project and the requirements. To resolve this, I made sure to ask the TA for feedback and clarification about specific sections.
  Additionally, communicating with each team member and ensuring our requirements aligned to the goals of the project was very helpful in aiding to ensure
  traceability.
\item \textbf{How many of your requirements were inspired by
      speaking to your client(s) or their proxies (e.g., your peers,
    stakeholders, potential users)?} \newline
  Many, if not most, requirements were inspired through speaking with our supervisor, who had the most knowledge and experience with our project's 
  potential users and stakeholders. In this project, it is important to understand our target users as we are designing for accessibility, so it was critical 
  in making our requirements. 
 \item \textbf{Which of the courses you have taken, or are currently
      taking, will help your team be successful with your capstone
    project?} \newline
  In this deliverable, the course that was most beneficial was Software Requirements and Security Considerations (SFWRENG 3RA3) as we learned how to create effected SRS documents.
  A course I've taken that will help thoroughly in ensuring our user interface is accessible is Human Computer Interfaces (SFWRENG 4HC3) as the course taught us principles of good design. Lastly,
  another course I took that contribute to the success of our project is Applications of Machine Learning (SFWRENG 4AL3) as this project heavily involves machine learning
  in generating alternative text. 
\end{enumerate}

\end{document}