% THIS DOCUMENT IS FOLLOWS THE VOLERE TEMPLATE BY Suzanne Robertson
% and James Robertson
% ONLY THE SECTION HEADINGS ARE PROVIDED
%
% Initial draft from https://github.com/Dieblich/volere
%
% Risks are removed because they are covered by the Hazard Analysis
\documentclass[12pt]{article}

\usepackage{booktabs}
\usepackage{tabularx}
\usepackage{enumitem}
\usepackage{hyperref}
\usepackage{enumitem}
\usepackage{graphicx}
\usepackage{float}
\usepackage{placeins}
\usepackage{hanging}

\hypersetup{
  bookmarks=true,         % show bookmarks bar?
  colorlinks=true,      % false: boxed links; true: colored links
  linkcolor=red,          % color of internal links (change box color
  % with linkbordercolor)
  citecolor=green,        % color of links to bibliography
  filecolor=magenta,      % color of file links
  urlcolor=cyan           % color of external links
}

\newcommand{\lips}{\textit{Insert your content here.}}

\input{../Comments}
%% Common Parts

\newcommand{\progname}{ProgName} % PUT YOUR PROGRAM NAME HERE
\newcommand{\authname}{Team 22, READING4ALL
\\ Fiza Sehar
\\ Nawaal Fatima
\\ Dhruv Sardana
\\ Moly Mikhail
\\ Casey Francine Bulaclac } % AUTHOR NAMES                  

\usepackage{hyperref}
    \hypersetup{colorlinks=true, linkcolor=blue, citecolor=blue, filecolor=blue,
                urlcolor=blue, unicode=false}
    \urlstyle{same}
                                


\begin{document}

\title{Software Requirements Specification for \progname}
\author{\authname}
\date{\today}

\maketitle

~\newpage

\pagenumbering{roman}

\tableofcontents
~\newpage

\section*{Symbolic Constants}
\begin{tabularx}{\textwidth}{|X|X|}
\toprule {\textbf{Name}} & {\textbf{Value}}\\
\midrule
T\_ALT\_GEN\_SMALL & 3 seconds(s) \\
T\_ALT\_GEN\_LARGE & 8 s \\
T\_UI\_RESP & 300 miliseconds (ms) \\
R\_SUFFICIENCY & 85\% \\
R\_LENGTH & 90\% \\
R\_USABILITY\_MEDIAN & 3 (rating) \\
R\_USABILITY\_MIN & 2 (rating) \\
T\_ERROR\_HANDLE & 2 s \\
T\_RECOVERY & 5 s \\
CAP\_CONCURRENT & 2 requests \\
CAP\_STORAGE & 500 images/day \\
MAINT\_TIME & 2 person-days/quarter \\
COMPAT\_VERSIONS & 2 releases \\
IMG\_SIZE\_BIG & 10 MEGABYTES (MB) \\
IMG\_SIZE\_SMALL & 2 MEGABYTES \\
TLS\_VERSION & 1.2 \\ 
FILE\_DELETE\_TIME & 60 s \\
FILE\_TYPES & .png, .jpg, .jpeg, .svg, .webp \\
NETWORK\_SOURCE\_POLICY & McMaster SSO tokens or IP ranges only\\
TEAM\_SIZE & 5 students \\
HOURS\_RESEARCH & 40 hours \\
HOURS\_BACKEND & 120 hours \\
HOURS\_FRONTEND & 80 hours \\
HOURS\_TESTING & 60 hours \\
HOURS\_DOCS & 30 hours \\
HOURS\_TOTAL & 330 hours \\
HOURS\_PROJECT & 1,320 person-hours \\
COST\_PER\_HOUR & \$20/hour \\
COST\_TOTAL & \$26,400 CAD \\
COST\_ACTUAL & \$0 CAD \\
COST\_INCENTIVE\_MIN & \$100 CAD \\
COST\_INCENTIVE\_MAX & \$150 CAD \\
MAX\_ZOOM\_PERCENTAGE & 200\%\\
MIN\_CONTRAST\_RATIO & 4.5:1\\
MAX\_UPLOAD\_STEPS & 5 steps\\
MAX\_MINUTES & 5 minutes\\
USERS\_SUCCESS\_PERCENT & 80\%\\
MAX\_ERROR\_RECOVER & 2 seconds \\
LEARNING\_PERCENT & 90\% \\
MAX\_LEARNING\_MINUTES & 5 minutes \\
MIN\_COMPENSATION\_DOLLARS & \$100 CAD \\ 
MAX\_COMPENSATION\_DOLLARS & \$150 CAD \\ 
LATEST\_RELEASES\_NUM & 3 \\
MOST\_COMMON\_SR & 3 \\
SFWR\_RELEASES & 2 \\


\bottomrule
\end{tabularx}


~\newpage

\section*{Revision History}

\begin{tabularx}{\textwidth}{p{3cm}p{2cm}X}
  \toprule {\textbf{Date}} & {\textbf{Version}} & {\textbf{Notes}}\\
  \midrule
  October 10 2025 & 1.0 & Initial Version of SRS\\
  \bottomrule
\end{tabularx}

~\\

~\newpage
\section{Introduction}

\subsection{Purpose of Document}
This Software Requirements Specification (SRS) defines the functional
and non-functional requirements for the \progname{} system,
\textbf{Reading4All}. The document explains the project's aims,
context, restrictions, and expected behavior in order to ensure that
developers, accessibility professionals, instructors, and project
stakeholders all have a shared understanding.

This SRS provides a formal reference for Reading4All's design,
implementation, testing, and validation. It establishes a clear link
between user requirements, accessibility standards, and system features that enable reliable alternative (alt) text
generation for technical diagrams.

\subsection{Scope of Project}
Reading4All is an artificial intelligence (AI)/machine learning (ML) tool that
provides detailed and contextually aware alternative text for
complicated technical graphics, notably those found in postsecondary
Science, Technology, Engineering, and Mathematics (STEM) course materials. The system combines machine learning and natural
language generation models to assess diagram content, identify key
parts, and provide text descriptions that are compatible with screen
readers and assistive technology.

\noindent \textbf{Core objectives:}
\begin{itemize}
  \item Automate the creation of comprehensive and accurate
    alternative text for technical diagrams.
  \item Maintain compliance with the \textbf{Accessibility for
    Ontarians with Disabilities Act (AODA)} and the \textbf{Web Content Accessibility Guidelines (WCAG) 2.1}
    accessibility standards.
  \item Improve the learning experiences and inclusion of students
    with disabilities in higher education.
  \item Reduce the instructor effort and institutional costs related
    to manual alt text creation.
\end{itemize}

\noindent \textbf{Final deliverable includes:}
\begin{itemize}
  \item A web-based or locally hosted interface for uploading images
    and creating alternative-texts.
  \item A backend service that combines AI models for diagram
    analysis and alt text generation.
  \item Generated alt text that are compatible with existing accessibility tools.
\end{itemize}

\subsection{Characteristics of Intended Reader}
This document is intended for:
\begin{itemize}
  \item \textbf{Developers} who are responsible for implementing and
    testing the system.
  \item \textbf{Accessibility Professionals} to ensure AODA and WCAG compliance.
  \item \textbf{Academic Stakeholders} including educators,
    instructional designers, and content creators.
  \item \textbf{Supervisors and Assessors} who assess project
    completion and design quality.
\end{itemize}

\noindent Readers should have an overall knowledge of software engineering, web
development, and fundamental machine learning techniques.
Accessibility reviewers should understand digital accessibility
principles and standards.

\subsection{Organization of Document}
This Software Requirements Specification (SRS) is divided into
twenty-six sections following the IEEE standard structure.
\begin{itemize}
  \item \textbf{Sections 1–5:} Establish the foundation of the
    document, including the project scope, stakeholders, assumptions,
    and terminology.
  \item \textbf{Sections 6–10:} Define the overall system context,
    business model, product scope, and high-level functional requirements.
  \item \textbf{Sections 11–17:} Specify detailed non-functional
    requirements such as usability, performance, maintainability,
    security, and compliance.
  \item \textbf{Sections 18–26:} Present supporting material
    including open issues, off-the-shelf components, project tasks,
    migration plans, documentation, and solution ideas.
\end{itemize}
This organization ensures logical flow from project background to
detailed requirements and supporting documentation.

\subsection{System Context}

\textbf{Inputs:} Technical diagrams or schematics provided for alt text generation. \\[2mm]
\textbf{Processing:} The system analyzes visual elements using a
vision model and generates textual descriptions through a language
model. \\[2mm]
\textbf{Outputs:} Descriptive alternative text formatted for screen
readers.

\subsection{Document Conventions}

This document follows the structure and formatting guidelines of the
McMaster University SFWRENG 4G06 Capstone SRS template.
All section numbers, requirement identifiers, and tables conform to
the IEEE SRS format.
Standard SI units are used where applicable.
Technical terms, acronyms, and variables are presented in
\texttt{monospaced} or \textit{italic} text for clarity.

\subsection{Reference Material}
Relevant Standards and Reference Documents:
\begin{itemize}
  \item Accessibility for Ontarians with Disabilities Act (AODA, 2005)
  \item Web Content Accessibility Guidelines (WCAG 2.1)
  \item \textit{SRS-Volere Template}, McMaster University
\end{itemize}

\section{Stakeholders}
The project stakeholders consist of people who have a need or
interest, whether direct or indirect, for alternative text generation
for visual and idle
content such as images or diagrams. These stakeholders will
influence and be affected by the
project's development decisions and progress. To meet user
needs, it is vital to understand the stakeholder roles and expectations.
\par First, this section introduces the client, customer and other
stakeholders involved in this project. Then, the product users are
described, specifically the hands-on users of the project. Finally,
personas, priority levels and anticipated
participation levels are listed for each stakeholder.
\subsection{Client}
This project's client is Ms. Jingchuan Sui who works as a Media Lab
Specialist Supervisor at the Faculty of Engineering, McMaster
University. As this project's supervisor, her main role is to provide
guidance and voice any concerns during the development phase with her
technical and domain expertise. She will be the main source for
setting requirements while also being directly involved in the
development of this project, providing feedback and opinions on the
human-computer interface components.
\subsection{Customer}
The customers of this product are McMaster users, specifically,
McMaster students, staff and teaching instructors who are directly
involved in learning from course content or making them accessible as per the AODA.
 In other
words, McMaster University stakeholders who benefit directly from
accessible course content. For
example, primary customers can include students who use a screen reader for
learning purposes or
a teaching assistant who is making a course's content AODA compliant.
Furthermore, under Ms. Sui's
position at McMaster University, she can also be considered a
customer, as she aids in course content remediation and thus, is also
one of the intended end-users.\\
For development, Group 22 is tailoring the solution to the
McMaster demographic in line with Ms. Sui’s requirements. However,
the product has the potential to support any users who require
alternative text generation for visual content. Feedback from
McMaster stakeholders will be prioritized to maintain a clear and
manageable scope.
\subsection{Other Stakeholders}
This subsection discusses other groups that are indirectly impacted
or who contribute to the ecosystem of accessibility, content
creation, and AODA compliance.
\subsubsection {Faculty of Engineering Instructors}
This group consists of professors and lecturers responsible for
creating and maintaining
course content. They may benefit from automated alternative text generation
to ensure their teaching materials are accessible.
\subsubsection {Teaching Assistants (TAs)}
As part of their work, TAs are often responsible for preparing,
modifying, and uploading course
content. They are stakeholders as they could use the system to
simplify accessibility compliance.
\subsubsection {Accessibility Services Office at McMaster}
This group includes staff members who oversee accessibility
compliance and provide
accommodations for students. They have a strong interest in ensuring
tools meet AODA standards.
\subsubsection {McMaster IT Services / Media Production Services}
These teams may be involved in system integration, technical support,
and maintenance of the product within the university’s digital infrastructure.
\subsubsection {Students with Accessibility Needs}
These students are those who may not be primary testers but are
indirectly impacted
by improved accessibility of course materials.
\subsection{Hands-On Users of the Project}
\subsubsection{Students with Accessibility Needs}
In some cases, students who use screen readers may provide feedback
loops to improve generated alternative text. While they are customers, they
may also be “hands-on” users if they test or adjust alt text themselves.
\subsubsection{Teaching Staff}
This group consists of TAs and instructors. As mentioned above, TAs
Frequently upload, adapt, and remediate course content. They would be
interacting directly with the tool to generate and refine alt text.
On the other hand, some instructors (especially those who prepare
their own slides, diagrams, or assignments) would use the system to
add or edit alternative text.
\subsection{Personas}
\subsubsection*{Persona: Alice Bayes}
\textbf{Age:} 27 \\
\textbf{Job Title:} Teaching Assistant at McMaster University\\
\textbf{Education:} Bachelor's in History\\
\textbf{Work Environment:} Alice works under several teaching
instructors to help deliver course content to students. She is in
charge of marking assignments and has recently been tasked with
auditing then remediating any inaccessible learning content. \\
\textbf{Professional Background:} Alice graduated two years ago, and
as part of her undergraduate career, she has experience in working
with students with disabilities. She is trained on making content
accessible and AODA compliant.\\[2mm]
\textbf{Need:} With so many courses to grade student work for, Alice
needs a tool that can easily and quickly generate content for her to
use as alternative text while she can ensure that her boss' teaching
content meets AODA compliance.\\
\textbf{Challenges:} Balancing her work and life has been difficult
as there are multiple images per document, and several documents per
course. She is overwhelmed with the amount of grading she has to do
on top of manually writing alternative text for over 50 images.

\subsubsection*{Persona: Chetan Dakshesh}
\textbf{Age:} 20 \\
\textbf{Job Title:} Student at McMaster University.\\
\textbf{Education:} He is currently pursuing a Bachelor's in
Electrical Engineering\\
\textbf{Work Environment:} Chetan has a super busy course load with
six courses and volleyball club!\\
\textbf{Professional Background:} N/A\\[2mm]
\textbf{Need:} With so many courses and volleyball practice to keep
up with, Chetan is finding it hard to keep track of course content.
Furthermore, through his screen reader, he has picked up that there
is no alternative text generated for several diagrams in a course he
is taking. These diagrams are vital to his learning experience but he
has little clue on what they indicate. \\
\textbf{Challenges:} Using large language models (LLMs) such as
Chat-GPT doesn't work for him as the text generated is too generic
and lacks substance. Chetan needs a tool that can effectively
describe the diagram to him while staying relevant to the course material.

\subsubsection*{Persona: Eyad Fahim}
\textbf{Age:} 40 \\
\textbf{Job Title:} Professor at McMaster University\\
\textbf{Education:} Doctor of Philosophy (PhD) in Engineering\\
\textbf{Work Environment:} Eyad works on a fast paced work
environment, connecting with over 100 students.\\
\textbf{Professional Background:} With over 20 years of experience both in
the workforce and academic, Eyad loves to teach the next generation
of leaders about various engineering techniques.\\[2mm]
\textbf{Need:} With the goal to
celebrate students of experiences, Eyad is looking for help to make
his teaching content accessible for all.\\
\textbf{Challenges:} Eyad needs a fast tool that can help close the gap in the
accessibility knowledge he lacks. He wants to ensure all students can
learn from his materials with little to no barriers, including
alternative text but he has no idea how to get started.
\subsection{Priorities Assigned to Users}
\textbf{Primary users:}
\begin{itemize}
  \item Students with accessibility needs
  \item Ms. Jingchuan Sui
  \item Teaching Staff
\end{itemize}
\textbf{Secondary users:}
\begin{itemize}
  \item Accessibility Services Office at McMaster
  \item McMaster IT/Media Production Services
\end{itemize}
\subsection{User Participation}
During the development process, the requirements will be gathered
mainly from Ms. Sui. During testing phase, Group 22 will conduct
usability testing to ensure AODA compliance and to further refine the product.
\subsection{Maintenance Users and Service Technicians}
For this project, maintenance activities may involve updating
alternative text generation models, fixing bugs, or upgrading dependencies.
\subsubsection*{Expected Maintenance Users and Roles}
\begin{itemize}
  \item \textbf{McMaster IT Services / Media Production Services} \\
    These teams may oversee deployment, integration with
    institutional systems, and technical support. They require access
    to configuration tools, diagnostic information, and documentation
    for updates or troubleshooting.
  \item \textbf{Accessibility Services Office Staff} \\
    Although initially secondary stakeholders, some staff may
    contribute to iterative refinement of alt text generation
    accuracy or compliance updates. Their participation may prompt
    system adjustments or patches.
  \item \textbf{Development Team (Group 22) or Future Maintainance Team} \\
    During initial deployment and handover, the development team or a
    designated successor group may perform updates to improve
    usability, resolve technical issues, or adapt to new
    accessibility standards.
\end{itemize}

\section{Mandated Constraints}
\subsection{Solution Constraints}
\begin{enumerate}[label=MD-SL \arabic*., wide=0pt, leftmargin=*]
  \item \emph{The solution design must comply with at least the Level
    AA of the Web Content Accessibility Guidelines (WCAG) 2.1 standards}\\[2mm]
    {\bf Rationale:} This ensures that the solution demonstrates
    inclusivity for users with visual, auditory, or cognitive impairments. \\
    {\bf Fit Criterion:} The solution must past all tests using WCAG
    automated testing tools and manual tests.\\
    {\bf Priority:} High.
  \item \emph{The solution must be implemented as a web tool}\\[2mm]
    {\bf Rationale:} A web tool will allow for automated testing
    against the WCAG standards which ensures accesibility for users
    and allow users to upload images/figures to generate alternative text.\\
    {\bf Fit Criterion:} The web tool must be functional and allow
    users to generate alternative text by uploading
    images and figures.\\
    {\bf Priority:} High.
  \item \emph{The solution must support common image formats (e.g.
        Joint Photographic Experts Group (JPEG), Portable Network Graphic
    (PNG), etc.)}\\[2mm]
    {\bf Rationale:} The web tool will enable users to upload images
    or figures to generate the alternative text, therefore the
    solution must be able
    to handle the different types of image formats.\\
    {\bf Fit Criterion:} The product must successfully process at
    least one image of each required format including JPEG and PNG
    images and figures.\\
    {\bf Priority:} High.
\end{enumerate}

\subsection{Implementation Environment of the Current System}
\begin{enumerate}[label=MD-IE \arabic*., wide=0pt, leftmargin=*]
  \item \emph{The product must be able to run on standard of laptop
      environments, including operating systems (OS) such as
    macOS, Windows, and Linux}\\[2mm]
    {\bf Rationale:} This ensures that the product is compatible with
    the latest and major operating systems to allow the product to be
    accessible to users, regardless of their laptop environment.\\
    {\bf Fit Criterion:} The product must successfully install and
    operate on the latest LATEST\_RELEASES\_NUM releases of macOS, Windows, and Linux,
    verified through installation and functionality testing on each OS.\\
    {\bf Priority:} High.
\end{enumerate}
\subsection{Partner or Collaborative Applications}
\begin{enumerate}[label=MD-PA \arabic*., wide=0pt, leftmargin=*]
  \item \emph{The product must be compatible with other accessibility
    tools (e.g. screen readers, screen magnifiers, dictation software)}\\[2mm]
    {\bf Rationale:} This is to ensure that the product does not
    limit or intefere with other accessibility tools that
    meets the users' needs. \\
    {\bf Fit Criterion:} The product must operate simultaneously with
    at least one other accessibility tool,
    verified through interoperability testing. \\
    {\bf Priority:} High.
\end{enumerate}
\subsection{Off-the-Shelf Software}
There are a number of existing AI generated alternative text
off-the-shelf software in the market today. The following
highlights a few of these tools, including their functions, benefits,
and limitations:
\begin{enumerate}
  \item \textbf{Azure AI Vision Image Analysis}: This service by
    Microsoft can extract a wide variety of
    visual features from images. Image Analysis offers image
    captioning models that generate one-sentence descriptions of an
    image's visual content.
    Limitations of this product is that it only generates one simple
    sentence, and that the image
    captions are only available in English.
  \item \textbf{ALTTEXT.AI}: This service allows users to upload
    images and generate alternative text. The website supports
    over 100 languages and many modern image formats. A significant
    limitation of this project
    is that it doesn't guarantee compliance with WCAG which limits
    accessibility.
  \item \textbf{accessiBe}: This service is an accessibility platform
    built for developers and engineers that plugs into their SDLC to detect and
    remediate WCAG issues at code level. The tools offers AI alt text
    descriptions for images and allows users to review
    and edit the alt text. A limitiation of this tool is that it uses
    overlays that sit on top of a website to fix issues at run-time.
    This is an issue
    because overlays can conflict with assistive technologies and
    miss context-specific WCAG requirements creating a false sense of
    real accessibility compliance.
\end{enumerate}
\subsection{Anticipated Workplace Environment}
The anticipated workplace environment for this product is academic
settings such as universities, where students may
require alternative text to interpret images and figures within their
coursework and study materials.
\subsection{Schedule Constraints}
\begin{enumerate}[label=MD-SC \arabic*., wide=0pt, leftmargin=*]
  \item \emph{The final product must be completed and tested by the
    end of the academic term (April 2026)}\\[2mm]
    {\bf Rationale:} This is to ensure that the final product is
    functional and meets all requirements
    at the end of the academic year.\\
    {\bf Fit Criterion:} All deliverables are submitted, and the
    final product is tested and operable by April 2026. \\
    {\bf Priority:} High.
\end{enumerate}
\subsection{Budget Constraints}
\begin{enumerate}[label=MD-BC \arabic*., wide=0pt, leftmargin=*]
  \item \emph{The project budget must include compensation for user
      testers, set at maximum MAX\_COMPENSATION\_DOLLARS canadian dollars per participant for two rounds
    of usability testing.}\\[2mm]
    {\bf Rationale:} This is to ensure that user testers are
    compensated for their meaningful feedback,
    and that our testing aligns with ethical practices.\\
    {\bf Fit Criterion:} There must be record of participants being
    compensated between the range of MIN\_COMPENSATION\_DOLLARS and MAX\_COMPENSATION\_DOLLARS for
    two rounds of testing. \\
    {\bf Priority:} High.
\end{enumerate}
\subsection{Enterprise Constraints}
\begin{enumerate}[label=MD-EC \arabic*., wide=0pt, leftmargin=*]
  \item \emph{The product must comply with the Accessibility for
    Ontarians with Disabilities Act (AODA)}\\[2mm]
    {\bf Rationale:} This ensures that the product meets the legal
    requirements in Ontario and guarantees that
    the product is accessible to users with diverse needs.\\
    {\bf Fit Criterion:} AODA requires compliance with WCAG
    standards, which ensures that the product meets AODA regulations.
    Compliance must be verified through both automated WCAG testing
    tools and manual accessibility testing.\\
    {\bf Priority:} High.
\end{enumerate}

\section{Naming Conventions and Terminology}
\subsection{Glossary of All Terms, Including Acronyms, Used by Stakeholders
involved in the Project}

\paragraph*{Accuracy}
The degree to which generated descriptions capture the image’s content correctly.

\paragraph*{AI (artificial intelligence)}
Techniques that enable computers to perform tasks that normally require human intelligence.

\paragraph*{Alt Text (Alternative Text)}
Textual description of non-text content such as images that allow accessibility tools such as screen readers to convey the content.

\paragraph*{AODA (Accessibility for Ontarians with Disabilities Act)}
Ontario law aimed at improving accessibility for people with disabilities by removing and preventing barriers when designing.

\paragraph*{API (Application Programming Interface)}
Rules and protocols that allows different software programs to communicate with each other.

\paragraph*{Backend}
Server components handling processes of an application that users don't see.

\paragraph*{Benchmarking}
The comparing of performance or quality of one's system against known systems or datasets.

\paragraph*{Contrast Ratio}
Luminance difference between text and background required by WCAG 2.1.

\paragraph*{Dataset Bias}
Systematic skew in training data that can harm fairness or accuracy of the model.

\paragraph*{Edge Case}
Uncommon input or scenario that the system must handle safely.

\paragraph*{FIPPA (Freedom of Information and Protection of Privacy Act)}
Ontario privacy law affecting the university data in the Authentication process.

\paragraph*{Frontend}
User interface in the browser that handles input, feedback, and accessibility features.

\paragraph*{Git/Github}
Version control and collaboration platform. 

\paragraph*{HTTP/HTTPS}
Web protocols in which HTTPS adds a transport layer security encryption for integrity and privacy.

\paragraph*{Issue (Github)}
Tracked unit of task, bug, or feature with discussion and linkage to commits in Github.

\paragraph*{JAWS (Job Access with Speech)}
A screen reader software available on Windows.

\paragraph*{JSON (JavaScript Object Notation) / YAML (Yet Another Markup Language)}
Human-readable data formats used for configs and API payloads.

\paragraph*{Latency}
Time from user action such as uploading an image to a system response or alt text generation

\paragraph*{Low Vision}
Reduced level of vision that interferes with daily activities and is to be considered in designing the user interface and testing.

\paragraph*{Manual Accessibility Testing}
Human review or testing of user interface and alt text.

\paragraph*{Modularity}
Separating user interface, vision, language, and validation for maintainability.

\paragraph*{NVDA (NonVisual Desktop Access)}
A free screen reader available on Windows.

\paragraph*{OCR (Optical Character Recognition)}
Extracts embedded text in images or diagrams.

\paragraph*{PII (Personally Identifiable Information)}
Data that identifies a person and must not appear in outputs or logs for security.

\paragraph*{Screen Magnifier}
Assistive technology to enlarge screen content.

\paragraph*{Screen Reader}
Assistive technology that reads text aloud.

\paragraph*{Session History}
 Record of user uploads and generated alt text during the current session.

 \paragraph*{Stakeholder}
 Anyone affected by or influencing the system.

 \paragraph*{Technical Diagram}
 An informational visual used in post-secondary course materials. 

 \paragraph*{TLS (Transport Layer Security)}
 Protocol providing encryption and integrity.

 \paragraph*{WCAG 2.1 (Web Content Accessibility Guidelines)}
 International standard for accessible web content.

 \paragraph*{WCAG Levels (A/AA/AAA)}
 Different conformance tiers to WCAG 2.1 where Level AA is the target for the project.


\section{Relevant Facts And Assumptions}
\subsection{Relevant Facts}
\begin{itemize}
  \item This project is being developed for a Software Engineering
    Capstone course with a fixed timeline.
  \item The solution is targeted primarily for the last LATEST\_RELEASES\_NUM of laptop and/or desktop
    environments, but can later be extended for mobile platforms use.
\end{itemize}
\subsection{Business Rules}
The business rules established among the team are as follows:
\begin{itemize}
  \item \textbf{Adherence to Project Schedule}: All deliverables and
    milestones must be
    completed according to the established project schedule. Any
    anticipated delays must be communicated
    in advance.
  \item \textbf{Pull Request Requirement}: All pull requests made by
    a team member must be reviewed by
    three other members before being merged into the \texttt{main}
    branch. The reviewers must provide approval
    or feedback within 24 hours of the pull request.
  \item \textbf{Team Communication Standard}: All team members must
    communicate respectfully and professionally during
    all discussions, meetings, and written communication.
  \item \textbf{Testing Requirements}: All code contributions must
    include appropriate unit, intergration, and
    functionality tests to ensure correctness and reliability.
    Accessibility testing must also be performed for all
    product features.
\end{itemize}
\subsection{Assumptions}
The following assumptions are made when using the product:
\begin{itemize}
  \item Users will be operating on the LATEST\_RELEASES\_NUM latest releases of
    browsers including Chrome, Safari, and Firefox.
  \item Users will have access to stable internet connection when
    using the product.
  \item Users will have basic knowledge of installing and enabling web tools.
  \item Users will be using the MOST\_COMMON\_SR most commonly used and latest
    versions of screen readers.
\end{itemize}
\section{The Scope of the Work}
\subsection{The Current Situation}
Currently, alternative text generation tools are able to provide
sufficient descriptions for simple images and figures. However, for
more complex visuals
such as engineering diagrams, the generated alt text is often
misleading, incomplete, or inefficient at conveying the intended meaning.\\
Accurate alternative text is particularly essential for individuals
with visual or cognitive impairments, as it enables fair access to
academic content. Without
reliable descriptions, students may experience barriers to learning
and miss critical information conveyed in diagrams and figures.\\
The current limitations of existing generated alternative text tools
are as follows:
\begin{itemize}
  \item \textbf{Inaccurate Alternative Text}: Generated alt text may
    emphasize unimportant details and overlook key elements,
    resulting in misleading or confusing interpretations.
  \item \textbf{Oversimplification of Complex Figures}: Current tools
    frequently oversimplify technical or academic diagrams,
    failing to capture essential details required for learning.
  \item \textbf{High Manual Effort}: In many cases, subject matter
    experts must manually create alt text,
    which is time-intensive and not scalable across large volumes of
    academic content.
\end{itemize}

\subsection{The Context of the Work}
The product will be in the form of a web tool that integrates into
existing accessibility workflows
by providing accurate descriptions from images that can be read aloud
by screen readers. The product
will complement existing screen readers by ensuring accurate
generated alternative text from
uploaded images and figures of academic work are available.
\autoref{fig:work-context} shows
how the product will integrate with existing screen readers.

\begin{figure}[H] % or [htbp] + \FloatBarrier below
  \centering
  \includegraphics[width=\textwidth]{images/work-context-diagram.png}
  \caption{Work Context Diagram}
  \label{fig:work-context}
\end{figure}
\FloatBarrier   % prevents later floats from jumping before this point

\subsection{Work Partitioning}
\autoref{tab:work-partition} shows the work partitioning for
completing the project. It includes major events,
their inputs and outputs, and the summary of the event.

\begin{table}[H]
  \centering
  \caption{Work Partition for the System}
  \label{tab:work-partition}
  \begin{tabular}{ |p{3cm}|p{3cm}|p{3cm}|p{4cm}| }
    \hline
    \textbf{Event Name} & \textbf{Input} & \textbf{Output} & \textbf{Summary} \\
    \hline
    Login & Username, Password &  & User logs in using their McMaster account \\
    \hline
    Upload \mbox{Images/Figures} & PNG/JPEG files & Uploaded File
    Reference & User uploads their files to generate alternative text \\
    \hline
    Optical Character Recognition \mbox{OCR Text} \mbox{Extraction} & Uploaded
    \mbox{Images/Figures} & Detected Text & System reads the text
    embedded in the uploaded files \\
    \hline
    Generate \mbox{Alternative Text} & Uploaded
    \mbox{Images/Figures}, Extracted OCR, Model Parameters &
    Generated Alt Text & System analyzes the image and extracted OCR data to
    generate accurate alternative text \\
    \hline
    View History & \mbox{User Login,} Stored Uploads and Generated
    Alt Text & List of Previously Generated Alt Text & The system
    retrieves and displays a user’s history of uploaded images along
    with their associated generated alt text \\
    \hline
  \end{tabular} 
\end{table}

\subsection{Specifying a Business Use Case (BUC)}
The project has one primary business use case, which aims to achieve
the goal of providing users with visual and cognitive impairments an efficient
and accessible way to generate accurate alternative text for academic
images and figures.\\[1ex]
\textbf{Preconditions:}
\begin{itemize}
  \item The user has access to the web tool
  \item The user has files containing diagrams or images requiring
    alternative text
\end{itemize}
\textbf{Scenario:}
\begin{enumerate}
  \item The user logs into the system using their McMaster student credentials
  \item The user uploads one or more files (PNG, JPEG) containing diagrams
  \item The AI model analyzes the uploaded file(s), performs OCR to
    extract any visible text, and generates
    alternative text describing each image/figure accurately
  \item Screen readers use the generated alternative text to read
    aloud and convey the uploaded image
  \item The generated alternative text can be edited, copied, or
    dowloaded as a \texttt{.txt} file by the user if needed
  \item The user can view previously uploaded files and generated
    alternative text for future reference
\end{enumerate}
\textbf{Postcondition:}
\begin{itemize}
  \item The user obtains accurate and accessible alternative text
    that complies with AODA and WCAG 2.1 standards.
\end{itemize}

\section{Business Data Model and Data Dictionary}
This section describes the structure and organization of the data flow throughout the system. It explains how the system’s components interact with the stored data, ensuring a consistent and well-defined understanding of the information processed by the tool.

\subsection{Business Data Model}
The following diagram (Figure~\ref{fig:business_data_model}) illustrates the relationships between key components of the system as well as their interactions with external components.

\begin{figure}[h!]
    \centering
    % Export the draw.io diagram below as PNG and save it to this path:
    \includegraphics[width=0.85\textwidth]{images/business-data-model.png}
    \caption{Business Data Model of alt text Generation System}
    \label{fig:business_data_model}
\end{figure}

\subsection{Data Dictionary}
\begin{table}[H]
    \centering
    \caption{Data Dictionary for the Reading4All System}
    \label{tab:data-dictionary-reading4all}
    \begin{tabular}{ |p{3.6cm}|p{7.4cm}|p{3.0cm}| }
      \hline
      \textbf{Name} & \textbf{Content} & \textbf{Type} \\
      \hline
      Reading4All (container) &
      Logical package that contains the core components of the system (Frontend UI, Backend Controller, Model Service, Session Storage Component, Evaluation Service). &
      Package \\
      \hline
      Frontend User Interface (UI) &
      Uploads images, shows generated alt text, and captures tester evaluations. Provides keyboard/ARIA navigation and WCAG~2.1 Level~AA compliant views. &
      Module \\
      \hline
      Backend Controller &
      Orchestrates requests: validates inputs (images/type/size), routes to the Model Service, manages temporary/session storage, records events, and returns results to the user interface (UI). &
      Module \\
      \hline
      Model Service &
      Machine-learning inference service that generates alt text from uploaded images and returns text plus timing/quality metadata. Exposed via an internal Application Programmable Interface (API). &
      External Service / Module \\
      \hline
      Session Storage Component &
      Short term temporary storage for uploaded image bytes and short-lived artifacts used during inference; configured to auto-delete shortly after processing. &
      Storage \\
      \hline
      Evaluation Service &
      Collects and persists pilot-testing ratings (sufficiency, length, accessibility/usability, learning impact) and optional notes for analysis. &
      Module / Dataset \\
      \hline
      McMaster Single Sign On (SSO) &
      Institutional authentication provider used to restrict access to McMaster users; issues and validates user sessions for the web application. &
      External Service Provider \\
      \hline
      Web Tool (deployment) &
      Deployed application entry point that exposes the Reading4All system to end users over web protocols such as Hypertext Transfer Protocol Secure (HTTPS); integrates Single Sign On (SSO) and serves the User/Application Interfaces. &
      Package / Deployment Target \\
      \hline
    \end{tabular}
  \end{table}
  

\section{The Scope of the Product}
\subsection{Product Boundary}
\autoref{fig:product-boundary} below shows the components within the system and how they connect. The components that this project will aim on building include a user interface, an alternative text generation ML model, a session history manager. Furthermore, these components will utilize or communicate with a screen reader software, McMaster Authentication system and external AI/ML Frameworks. 
\label{tab:product-boundary} 
\begin{figure}[H]
    \centering
    \includegraphics[width=1.0\textwidth]{images/Product_Boundary_Diagram.jpg}
    \caption{Product Boundary Diagram}
    \label{fig:product-boundary}
\end{figure}

\subsection{Product Use Case Table}
\autoref{tab:product-use} summarizes the main product use cases
for the systems. For each use case it describes the actors involved,
inputs and outputs to the system and the related requirements.
\begin{table}[H]
  \centering
  \caption{Product Use Case Table}
  \label{tab:product-use}
  \begin{tabular}{|p{1.3cm}|p{2.5cm}|p{3cm}|p{4cm}|p{2.6cm}|}
    \hline
    \textbf{PUC\#} & \textbf{PUC Name} & \textbf{Actor(s)} &
    \textbf{Input \& Output(s)} & \textbf{Requirement} \\
    \hline
    PUC 1 & Login Using McMaster Credentials &  McMaster Student and
    Faculty , McMaster Authentication System & User Credentials
    (input), Authentication Results (output) & FR 6, SR-AR 1\\
    \hline
    PUC 2 & Upload Image & McMaster Student and Faculty & JPEG or PNG
    (input), Image upload status (output) & FR 1, UHR-EUR 3 \\
    \hline
    PUC 3 & Generate Alternative Text &  McMaster Student and
    Faculty, Alternative Text Generation Model & Uploaded image
    (input), Generated alternative text  & FR 2, FR 3\\
    \hline
    PUC 4 & Copy or Download Text & McMaster Student and Faculty &
    User decision to copy or download (input), text copied to
    clipboard or downloaded & UHR-PIR 1\\
    \hline
    PUC 5 & View History of Inputted Images and their Alternative
    Text & McMaster Student and Faculty & User request to view
    history (input), Display of previously inputted images and their
    generated text within a session & FR 5\\
    \hline
  \end{tabular}
\end{table}

\subsection{Individual Product Use Cases (PUC's)}
\textbf{PUC 1: Login Using McMaster Credentials }
\begin{quote}
  \textbf{Trigger:} User selects "Login" and is directed to McMaster
  sign in page.\\
  \textbf{Preconditions:}
  \begin{itemize}
    \item The user is registered person in McMaster system and has
      valid credentials.
  \end{itemize}
  \textbf{Actors:} McMaster Student or Faculty, McMaster
  Authentication System.\\
  \textbf{Outcome:} McMaster validates user's credentials are
  validated and they are given access to system.\\
  \textbf{Input:} McMaster System username and password. \\
  \textbf{Output:} User enters system or an error message is displayed.
\end{quote}
\textbf{PUC 2: Upload Image }
\begin{quote}
\textbf{Trigger:} User selects "Upload Image" and chooses a file.\\
\textbf{Preconditions:}
\begin{itemize}
  \item User successfully logged into the system.
\end{itemize}
\textbf{Actors:} McMaster Student or Faculty \\
\textbf{Outcome:} The selected image is uploaded and stored for later text generation.\\
\textbf{Input:} Image file (JPEG or PNG) \\
\textbf{Output:} A confirmation message is displayed if the image was successfully uploaded, or an error message otherwise is displayed.
\end{quote}
\textbf{PUC 3: Generate Alternative Text}
\begin{quote}
  \textbf{Trigger:} User selects \textit{Generate Alternative Text}
  for an uploaded image.\\
  \textbf{Preconditions:}
  \begin{itemize}
    \item A valid image has been successfully uploaded to the system.
  \end{itemize}
  \textbf{Actors:} McMaster Student or Faculty.\\
  \textbf{Outcome:} The system generates a descriptive alternative
  text for the uploaded image.\\
  \textbf{Input:} User selection to generate alternative text\\
  \textbf{Output:} Generated alternative text is displayed to the user.
\end{quote}
\textbf{PUC 4: Edit Generated Alternative Text}
\begin{quote}
\textbf{Trigger:} User decides to modify the generated alternative text before copying or downloading it.\\
\textbf{Preconditions:}
\begin{itemize}
  \item System has successfully generated alternative text for an uploaded image.
\end{itemize}
\textbf{Actors:} McMaster Student or Faculty, System User Interface \\
\textbf{Outcome:} The user edits and saves the updated alternative text. \\
\textbf{Input:} User edits to the generated alternative text.\\
\textbf{Output:} The edited alternative text is displayed to the user and stored within the session, to be copied or downloaded later.
\end{quote}
\textbf{PUC 5: Copy or Download Generated Alternative Text }
\begin{quote}
\textbf{Trigger:} User selects copy or download .txt after generating alternative text.\\
\textbf{Preconditions:}
\begin{itemize}
  \item System has successfully generated alternative text for an uploaded image.
  \item User is satisfied with generated alternative text and has made any desired changes. 
\end{itemize}
\textbf{Actors:} McMaster Student or Faculty. \\
\textbf{Outcome:} The user receives the alternative text through their preferred method. \\
\textbf{Input:} User decision to copy or download.\\
\textbf{Output:} Text is copied or downloaded as .txt file on the users device.
\end{quote}
\textbf{PUC 6: View History of Uploaded Images and Generated Alternative Text}
\begin{quote}
  \textbf{Trigger:} User selects the view history option.\\
  \textbf{Preconditions:}
  \begin{itemize}
    \item User is logged in with an active session.
    \item User has previously uploaded at least one image and
      generated text within the session.
  \end{itemize}
  \textbf{Actors:} McMaster Student or Faculty.\\
  \textbf{Outcome:} The user views a list of their images and the
  corresponding generated alternative text within the current session. \\
  \textbf{Input:} User request to view session history.\\
  \textbf{Output:} Display of uploaded images and their corresponding
  generated alternative text.\\
\end{quote}

\section{Functional Requirements}
\subsection{Functional Requirements}
\begin{enumerate}[label=FR \arabic*., wide=0pt, leftmargin=*]
  \item \emph{The system must accept technical diagrams in the format
    of JPEG and PNG.}\\[2mm]
    {\bf Rationale:} The system must process JPEG/PNG images in order
    to output alternative text. \\
    {\bf Fit Criterion:} The system successfully takes as accepts
    JPEG/PNG images and provides feedback to users when an invalid
    file type is inputted.  \\
    {\bf Priority:} High
  \item \emph{The system shall generate alternative text of uploaded
    images.}\\[2mm]
    {\bf Rationale:} The main purpose of the system is to make
    scientific diagrams more accessible by generating better
    alternative-text. \\
    {\bf Fit Criterion:} For a set of test diagrams, the alternative
    text generated must meet the pre-determined criteria.\\
    {\bf Priority:} High
  \item \emph{The system shall output alternative text in a format
    compatible with screen readers.}\\[2mm]
    {\bf Rationale:} Students with disabilities utilize screen
    readers to access digital content; therefore, the alternative
    text must be displayed in away that enables screen readers to
    read it correctly. Furthermore, if the alternative text output
    format is not compatible with screen readers, then students
    cannot benefit from the application output.\\
    {\bf Fit Criterion:} The alternative text output must be readable
    by at least MOST\_COMMON\_SR commonly used screen readers.\\
    {\bf Priority:} High
  \item \emph{The system shall allow users to edit the outputted
    alternative texts.}\\[2mm]
    {\bf Rationale:} Providing users with an option to edit the
    outputted text, enables them to adjust the output to better meet
    their needs if needed.\\
    {\bf Fit Criterion:} Users can add or delete text in any part of
    the outputted alternative text and save their changes.\\
    {\bf Priority:} High
  \item \emph{The system shall store and display all inputted images
    and their generated alternative texts within a session.}\\[2mm]
    {\bf Rationale:} Storing previously inputted images and their
    generated alternative texts, allows users to easily review or
    reuse them without re-uploading. \\
    {\bf Fit Criterion:} Users can see view all previously inputted
    images with their generated alternative texts during the same session. \\
    {\bf Priority:} Medium
  \item \emph{The system must validate users during login to confirm
    they are McMaster University students.}\\[2mm]
    {\bf Rationale:} User verification will ensure that only McMaster
    University students have access to the system, ensuring that the
    system is used by the intended users.  \\
    {\bf Fit Criterion:} Users can only gain access to the system
    features after their McMaster University credentials are
    successfully validated.  \\
    {\bf Priority:} High
\end{enumerate}

\section{Look and Feel Requirements}
\subsection{Appearance Requirements}
\begin{enumerate}[label=LFR-AR \arabic*., wide=0pt, leftmargin=*]
  \item \emph{The system must allow all text on the interface to be resized up to MAX\_ZOOM\_PERCENTAGE, without any loss of functionality or content. }\\[2mm] 
    {\bf Rationale:} Allowing text resizing will enable users with low vision to more easily utilize the system. This also ensures the system meets WCAG 2.1 Success Criterion 1.4.4 Resize Text.
    User verification will ensure that only McMaster University students have access to the system, ensuring that the system is used by the intended users.  \\
    {\bf Fit Criterion:} All text, excluding any captions and images of text can be enlarged to MAX\_ZOOM\_PERCENTAGE on a standard browser zoom (ex. Google Chrome) without any overlapping, hidden content, or broken features.  \\
    {\bf Priority:} High
  \item \emph{The system must not use color as the only method to
    provide information, indicate actions or prompt user input.}\\[2mm]
    {\bf Rationale:} Users with color vision deficiencies or other
    visual impairments may not detect color differences accurately.
    This also ensures the system meets WCAG 2.1 Success Criterion
    1.4.1 Use of Color.\\
    {\bf Fit Criterion:} Any use of color communicates information to
    the user or requests information from he user must be appear with text.  \\
    {\bf Priority:} High
  \item \emph{The system must ensure sufficient contrasts of text and images of text.}\\[2mm] 
    {\bf Rationale:} Sufficient color contrast is important as it enables users with low vision or color vision deficiencies to easily read any system text. This also ensures the system meets WCAG Success Criterion 1.4.3 Contrast (Minimum)\\
    {\bf Fit Criterion:}  All text and images of text in the system interfaces has a contrast ratio of at least MIN\_CONTRAST\_RATIO.  \\
    {\bf Priority:} High
  \item \emph{The system must provide alternative text for all
    non-text content.}\\[2mm]
    {\bf Rationale:} Users with visual impairment often use screen
    readers to navigate through software systems; therefore, it is
    essential that all images have sufficient alternative text, so
    that the purpose of the images can understood. This also ensures
    the system meets WCAG Success Criterion 1.1.1 Non-text Content  .\\
    {\bf Fit Criterion:} All decorative images and non-text elements
    have alternative text that communicate their meaning. \\
    {\bf Priority:} High
\end{enumerate}

\subsection{Style Requirements}
\begin{enumerate}[label=LFR-SR \arabic*., wide=0pt, leftmargin=*]
  \item \emph{The system interface must follow a simple and modern
    design style.}\\[2mm]
    {\bf Rationale:} A simple interface will improve the systems
    usability as it better highlights the system's features, while
    also ensuring the system is visually appealing.\\
    {\bf Fit Criterion:} The system uses a clean layout with a
    maximum of three colors, consistent font styles and sizes, as
    well as only has key design elements that support usability. \\
    {\bf Priority:} High
\item \emph{The system interface must use McMaster University branding while maintaining accessibility standards and a modern style.}\\[2mm] 
    {\bf Rationale:} As the system is targeted towards McMaster University students, using the schools branding will build trust with users and ensure the system aligns with McMaster's identify. However, using McMaster branding must not interfere with usability and accessibility criteria..\\
    {\bf Fit Criterion:} The system interface includes McMaster University' official logo and meets the WCAG 2.1 contrast and non-text content success criteria. \\
    {\bf Priority:} High
\item \emph{The system interface must follow Don Norman's principles of designing an interface.}\\[2mm] 
    {\bf Rationale:} Applying Norman's principles, including visibility, feedback, constraints, mapping, consistency, and affordance, will make the system intuitive and user-friendly.\\
    {\bf Fit Criterion:} Evaluation against Norman's principles reveals no issues with any interface elements.\\
    {\bf Priority:} High
\end{enumerate}

\section{Usability and Humanity Requirements}
\subsection{Ease of Use Requirements}
\begin{enumerate}[label=UHR-EUR \arabic*., wide=0pt, leftmargin=*]
\item \emph{The system interface must allow users to efficiently use the system features.}\\[2mm] 
    {\bf Rationale:} It is important the users can quickly access and use the system features, as they may be generating multiple alternative text outputs in a single session. \\
    {\bf Fit Criterion:} Users can upload images to the system and generate alternative text in MAX\_UPLOAD\_STEPS or fewer. \\
    {\bf Priority:} High
\item \emph{The system interface must be easy for users to remember how to use after not using it for some time.}\\[2mm] 
    {\bf Rationale:} Users should be able to quickly recall how to use the system without needing to relearn the features. An intuitive design will make it easier for returning users to find and use key features.  \\
    {\bf Fit Criterion:} Users who have not used the system in a month, can successfully login, upload an image and generate alternative text within MAX\_MINUTES, without needing any assistance. \\
    {\bf Priority:} Medium
\item \emph{The system interface must provide users with clear and immediate feedback for all actions.}\\[2mm] 
    {\bf Rationale:} Providing the users with feedback ensures they understand the outcome of their actions and whether they are using the system correctly. This reduces the Gulf of Evaluation as it makes users more confident while using the system. \\
    {\bf Fit Criterion:} The system provides textual feedback within 1 second after a user interaction, such as uploading an image.  \\
    {\bf Priority:} High 
\item \emph{The system interface must provide clear instructions, prevent common errors and allow users to easily correct them.}\\[2mm] 
    {\bf Rationale:} Providing easy to follow instructions will help ensure that users can easily use the system features and prevent errors. Additionally, if a user makes a mistake, they should easily be able to revert it.  \\
    {\bf Fit Criterion:} In user testing, at least USERS\_SUCCESS\_PERCENT of users can complete tasks without errors. When a user error occurs, the system explains the issue and how to recover within MAX\_ERROR\_RECOVER.\\
    {\bf Priority:} High 
\end{enumerate}

\subsection{Personalization and Internationalization Requirements}
\begin{enumerate}[label=UHR-PIR \arabic*., wide=0pt, leftmargin=*]
\item \emph{The system interface must allow users to choose how generated alternative text is stored or copied.}\\[2mm] 
    {\bf Rationale:} Providing users with the option to either copy generated text or download it as file, helps tailor the output to the users specific needs.  \\
    {\bf Fit Criterion:} After generating the alternative text users can choose to copy or download as .txt" from the interface and system successfully completes the chosen option.\\
    {\bf Priority:} High
\end{enumerate}

\subsection{Learning Requirements}
\begin{enumerate}[label=UHR-LR \arabic*., wide=0pt, leftmargin=*]
\item \emph{The system must be easy for low-vision users to learn and operate with screen readers.}\\[2mm] 
    {\bf Rationale:} The system should be intuitive for users with low vision to use without prior training. Additionally, the system being highly compatible with screen readers, allows users to more easily navigate and use the system.  \\
    {\bf Fit Criterion:} In user testing, at least LEARNING\_PERCENT of first time users with low vision using a screen reader can upload an image and generate alternative text within MAX\_LEARNING\_MINUTES without assistance. \\
    {\bf Priority:} High
\end{enumerate}

\subsection{Understandability and Politeness Requirements}
\begin{enumerate}[label=UHR-LR \arabic*., wide=0pt, leftmargin=*]
  \item \emph{The system must only display essential information and
    hide all technical details.}\\[2mm]
    {\bf Rationale:} The system should only communicate the
    information needed to use the system. Displaying any technical
    details may cause the user to be confused and make the system
    less usable.   \\
    {\bf Fit Criterion:} In user testing, users do not encounter any
    technical terms, code outputs or information that is not relevant
    to them.  \\
    {\bf Priority:} High
\end{enumerate}

\subsection{Accessibility Requirements}
\begin{enumerate}[label=UHR-AR \arabic*., wide=0pt, leftmargin=*]
  \item \emph{The system must meet the WCAG 2.1 Level AA
    accessibility standards.}\\[2mm]
    {\bf Rationale:} The Accessibility for Ontarians with
    Disabilities Act (AODA) requires organizations to meet WCAG 2.0
    Level AA for web tools. Therefore, meeting WCAG 2.1 Level AA
    ensures the system meets AODA standards and is accessible for
    users with disabilities.  \\
    {\bf Fit Criterion:} The system will be evaluated using an
    accessibility testing tool such as Pope Tech and Wave Web Aim to
    ensure WCAG 2.1 criteria is met.\\
    {\bf Priority:} High
  \item \emph{The system must accept keyboard input for navigation.}\\[2mm]
    {\bf Rationale:} Many users, including those with disabilities,
    use keyboard inputs to navigate through applications, the system
    must support this as a way to navigate.\\
    {\bf Fit Criterion:} Users can navigate to all the main functions
    and areas of the system using their keyboard. \\
    {\bf Priority:} High
\end{enumerate}

\section{Performance Requirements}

\subsection{Speed and Latency Requirements}
\begin{enumerate}[label=PR-SL \arabic*., wide=0pt, leftmargin=*]
\item \emph{The tool shall generate alt text for uploaded images within a reasonable time frame.}\\[2mm] 
    {\bf Rationale:} Ensures users, including those using assistive technologies, do not experience delays that hinder accessibility.\\
    {\bf Fit Criterion:} The system shall return generated alt text within T\_ALT\_GEN\_SMALL for images $\leq$ IMG\_SIZE\_SMALL and within T\_ALT\_GEN\_LARGE for images $\leq$ IMG\_SIZE\_BIG under normal load conditions.\\
    {\bf Priority:} High

\item \emph{The web tool shall load and render accessibility components efficiently.}\\[2mm] 
    {\bf Rationale:} Improves user experience and responsiveness for screen-reader users and keyboard navigation.\\
    {\bf Fit Criterion:} All interactive elements shall respond within T\_UI\_RESP of user input under typical conditions.\\
    {\bf Priority:} Medium
\end{enumerate}

\subsection{Safety-Critical Requirements}
\begin{enumerate}[label=PR-SCR \arabic*., wide=0pt, leftmargin=*]
\item \emph{The tool shall ensure that no personally identifiable data from uploaded images is stored or shared without consent.}\\[2mm] 
    {\bf Rationale:} Protects user privacy and adheres to ethical AI standards.\\
    {\bf Fit Criterion:} Uploaded images are deleted from temporary storage for the user session unless explicitly saved by the user. The images will not be stored in any database without user consent.\\
    {\bf Priority:} High

\item \emph{The tool shall not produce alt text containing offensive, biased, or harmful language beyond the context of image.}\\[2mm] 
    {\bf Rationale:} Ensures ethical AI output and inclusivity.\\
    {\bf Fit Criterion:} 0\% of generated outputs shall contain content flagged by moderation filters as offensive or biased.\\
    {\bf Priority:} High

\item \emph{The interface shall adhere to WCAG 2.1 Level AA accessibility guidelines to prevent stress or strain on users’ eyes and ensure comfortable interaction.}\\[2mm] 
    {\bf Rationale:} Provides a visually safe, inclusive experience for all users, including those with visual or cognitive impairments.\\
    {\bf Fit Criterion:} Verified through front-end accessibility testing that confirms conformance with WCAG 2.1 Level AA success criteria.\\
    {\bf Priority:} High
\end{enumerate}

\begin{enumerate}[label=PR-SR-HA \arabic*., wide=0pt, leftmargin=*]
  \item \emph{The system must notify the user when a timeout occurs during alternative text generation.}\\[2mm]
    {\bf Rationale:} Users should be informed when the alternative text generation exceeds the expected amount of time. If users are not notified they may send multiple requests, leading the server to be overloaded; this would also lead to user frustration.\\
    {\bf Fit Criterion:} When a timeout occurs, the system displays a message indicating the timeout and a "Retry" option. The message must follow accessibility guidelines and be compatible with screen readers.\\
    {\bf Priority:} Medium\\
    {\bf Hazard Analysis Connected}: HA4
  \item \emph{The system must safely exit when a timeout occurs and ensure that no user data or incomplete alternative text is stored or shown to the user.}\\[2mm]
    {\bf Rationale:} Safely exiting during a timeout prevents users from seeing incomplete alternative text and mistaking it for a complete output, which may cause confusion. Leaving user data stored would also be a security violation. \\
    {\bf Fit Criterion:} When a timeout occurs, the system must stop processing and delete the users data and any incomplete alternative text that has been generated.\\
    {\bf Priority:} High\\
    {\bf Hazard Analysis Connected}: HA4.
  \item \emph{The system must ensure that messages notifying the user of failure, do not reveal any system code or data. }\\[2mm]
    {\bf Rationale:} This will prevent internal data from being shown to users, which may lead to system and user security issues. \\
    {\bf Fit Criterion:} All error messages shown to the user only display the necessary information and do not contain any technical information.\\
    {\bf Priority:} High \\
    {\bf Hazard Analysis Connected}: HA4, HA7.
\end{enumerate}


\subsection{Precision or Accuracy Requirements}
Please refer to Appendix – Evaluation Metrics Summary (\ref{sec:evaluation-metrics}) to understand the evaluation scales and metrics.
\begin{enumerate}[label=PR-PAR \arabic*., wide=0pt, leftmargin=*]
\item \emph{The generated alt text shall adequately describe the image content with minimal omissions or irrelevant details.}\\[2mm] 
    {\bf Rationale:} Ensures the description fulfills its accessibility purpose.\\
    {\bf Fit Criterion:} At least R\_SUFFICIENCY of outputs rated “Sufficient” or better on the sufficiency scale by testers.\\
    {\bf Priority:} High

\item \emph{The alt text shall maintain appropriate length and readability.}\\[2mm] 
    {\bf Rationale:} Prevents overly short or verbose outputs that reduce usability.\\
    {\bf Fit Criterion:} $\geq$ R\_LENGTH of outputs rated “Proper Length” on the user-testing scale.\\
    {\bf Priority:} Medium

\item \emph{The overall accessibility and usability of the alt text shall be acceptable to testers.}\\[2mm] 
    {\bf Rationale:} Evaluates real-world effectiveness of generated descriptions.\\
    {\bf Fit Criterion:} Median user rating $\geq$ R\_USABILITY\_MEDIAN (“Mostly Accessible/Usable”) on the 0–3 or 0–4 scales; no outputs below R\_USABILITY\_MIN.\\
    {\bf Priority:} Medium
\end{enumerate}

\subsection{Robustness or Fault-Tolerance Requirements}
\begin{enumerate}[label=PR-RFT \arabic*., wide=0pt, leftmargin=*]
\item \emph{The system shall gracefully handle unsupported or corrupted image inputs.}\\[2mm] 
    {\bf Rationale:} Prevents crashes and maintains system stability.\\
    {\bf Fit Criterion:} Invalid files trigger a clear error message within T\_ERROR\_HANDLE without interrupting service.\\
    {\bf Priority:} High

\item \emph{The backend shall recover automatically from isolated process failures.}\\[2mm] 
    {\bf Rationale:} Ensures continued operation without developer intervention.\\
    {\bf Fit Criterion:} System recovers within T\_RECOVERY after fault detection.\\
    {\bf Priority:} High
\end{enumerate}

\subsection{Capacity Requirements}
\begin{enumerate}[label=PR-CR \arabic*., wide=0pt, leftmargin=*]
\item \emph{The system shall support limited concurrent usage suitable for a proof-of-concept deployment.}\\[2mm] 
    {\bf Rationale:} Demonstrates feasibility and reliability for initial testing without production-level scaling.\\
    {\bf Fit Criterion:} Supports at least CAP\_CONCURRENT simultaneous requests with response times $\leq$ 10 seconds.\\
    {\bf Priority:} Medium

\item \emph{Storage shall accommodate pilot testing datasets.}\\[2mm] 
    {\bf Rationale:} Ensures smooth prototype validation without capacity issues.\\
    {\bf Fit Criterion:} The system can temporarily store metadata for up to CAP\_STORAGE without data loss.\\
    {\bf Priority:} Low
\end{enumerate}

\subsection{Scalability or Extensibility Requirements}
\begin{enumerate}[label=PR-SER \arabic*., wide=0pt, leftmargin=*]
\item \emph{The architecture shall allow integration of improved ML models or multilingual capabilities in future phases.}\\[2mm] 
    {\bf Rationale:} Enables progressive enhancement and future accessibility expansion.\\
    {\bf Fit Criterion:} New models or language modules can be incorporated without restructuring existing components.\\
    {\bf Priority:} Medium
\end{enumerate}

\subsection{Longevity Requirements}
\begin{enumerate}[label=PR-LR \arabic*., wide=0pt, leftmargin=*]
\item \emph{The codebase shall be maintainable and adaptable to updates in WCAG guidelines, software libraries, and ML frameworks.}\\[2mm] 
    {\bf Rationale:} Ensures long-term usability and compliance even after the pilot phase.\\
    {\bf Fit Criterion:} Minor updates or migrations require $\leq$ MAINT\_TIME.\\
    {\bf Priority:} Medium

\item \emph{The prototype shall maintain compatibility with at least the previous SFWR\_RELEASES software releases.}\\[2mm] 
    {\bf Rationale:} Ensures sustainability of the pilot for educational and testing purposes.\\
    {\bf Fit Criterion:} Verified through annual testing on supported Python versions.\\
    {\bf Priority:} Low
\end{enumerate}
\section{Operational and Environmental Requirements}

The Reading4All system must operate reliably in typical university and institutional settings, both digitally and physically.

\subsection{Expected Physical Environment}

\begin{enumerate}[label=OER-EP\arabic*., wide=0pt, leftmargin=*]
  \item \emph{The system should be compatible with standard devices such as laptops, desktops, and cloud servers running Windows, macOS, Linux, or standard cloud platforms like AWS, GCP, or Azure.}\\[2mm]
    {\bf Rationale:} This ensures that it will work properly in a variety of academic and institutional settings.\\
    {\bf Fit Criterion:} The system operates without compatibility issues across all supported platforms.\\
    {\bf Priority:} High

  \item \emph{The system should function normally in common indoor conditions such as classrooms or offices with standard room temperatures and lighting.}\\[2mm]
    {\bf Rationale:} Since no specialized hardware setup is required, it should perform consistently in regular academic and office environments.\\
    {\bf Fit Criterion:} The system performs reliably under standard indoor temperature (10°C–35°C) and lighting conditions.\\
    {\bf Priority:} Low
\end{enumerate}

\subsection{Wider Environment Requirements}

\begin{enumerate}[label=OER-WE\arabic*., wide=0pt, leftmargin=*]
  \item \emph{The system should adhere to all applicable accessibility and privacy regulations, including AODA, WCAG 2.1 (Level AA), and institutional privacy policies like FIPPA.}\\[2mm]
    {\bf Rationale:} This ensures that the system complies with legal and ethical standards regarding inclusion and data protection.\\
    {\bf Fit Criterion:} Independent evaluation confirms compliance with AODA and WCAG 2.1 Level AA.\\
    {\bf Priority:} High

  \item \emph{The system shall operate effectively within standard
    academic network environments with stable internet connectivity.}\\[2mm]
    {\bf Rationale:} Reading4All depends on
    models for text generation, which may require reliable network access.\\
    {\bf Fit Criterion:} The system maintains consistent API
    communication over typical university Wi-Fi or Ethernet with
    upload speeds of at least IMG\_SIZE\_BIG Mbps.\\
    {\bf Priority:} Medium
\end{enumerate}
 
\subsection{Requirements for Interfacing with Adjacent Systems}

\begin{enumerate}[label=OER-IAS\arabic*., wide=0pt, leftmargin=*]
  \item \emph{The system shall support interoperability with
      assistive technologies such as screen readers (e.g., NVDA, JAWS,
    and VoiceOver).}\\[2mm]
    {\bf Rationale:} Screen-reader compatibility ensures generated
    text can be read aloud for visually impaired users.\\
    {\bf Fit Criterion:} Descriptions produced by Reading4All are
    correctly parsed and read by major screen readers without
    formatting issues.\\
    {\bf Priority:} High

  \item \emph{The system should support common image formats such as JPG, JPEG, and PNG to be compatible with the majority of course materials.}\\[2mm]
    {\bf Rationale:} Supporting standard image formats ensures instructors can use materials from a variety of academic sources.\\
    {\bf Fit Criterion:} The system processes each supported image format correctly and generates accurate alternative text.\\
    {\bf Priority:} Medium

  \item \emph{The system shall optionally interface with automated
    accessibility validation tools (e.g., WAVE or Axe).}\\[2mm]
    {\bf Rationale:} Integration with automated validators aids
    instructors in verifying alt text accessibility compliance.\\
    {\bf Fit Criterion:} Validation reports are successfully
    generated and accessible through the user interface.\\
    {\bf Priority:} Low
\end{enumerate}

\subsection{Productization Requirements}

\begin{enumerate}[label=OER-PR\arabic*., wide=0pt, leftmargin=*]
  \item \emph{The system shall be deployable as both a web
    tool for institutional integration.}\\[2mm]
    {\bf Rationale:} This ensures accessibility for
    individual users and organizations integrating accessibility workflows.\\
    {\bf Fit Criterion:} A hosted web tool is
    accessible and validated through institutional testing.\\
    {\bf Priority:} High
\end{enumerate}

\subsection{Release Requirements}

\begin{enumerate}[label=OER-RL\arabic*., wide=0pt, leftmargin=*]
  \item \emph{Core features such as image analysis, text generation, and accessibility validation should be implemented, tested, and verified prior to any release.}\\[2mm]
    {\bf Rationale:} Ensuring these core components are functional before release guarantees system reliability and completeness.\\
    {\bf Fit Criterion:} Verification and validation documentation confirms that all functional requirements have been met.\\
    {\bf Priority:} High

  \item \emph{The system must be ready for release by the end of March 2026,
      aligned with the McMaster University SFWRENG 4G06 Capstone final
    demonstration schedule.}\\[2mm]
    {\bf Rationale:} Aligns release timing with Capstone evaluation
    and stakeholder presentation.\\
    {\bf Fit Criterion:} The final deliverable is fully functional,
    accessible, and deployed for the 2026 demonstration.\\
    {\bf Priority:} Medium
\end{enumerate}

\section{Maintainability and Support Requirements}

Reading4All must be easy to maintain, update, and support after deployment, with structures that allow for future innovations.

\subsection{Maintenance Requirements}

\begin{enumerate}[label=MS-MNT\arabic*., wide=0pt, leftmargin=*]
  \item \emph{The system will be divided into modular components including the front-end interface, image analysis, language generation, and accessibility verification to ensure that each one can be updated or replaced without affecting the others.}\\[2mm]
    {\bf Rationale:} Modular design simplifies maintenance by isolating potential issues and enabling component-specific updates without impacting the entire system.\\
    {\bf Fit Criterion:} Each subsystem (e.g., the image analysis or text generation module) can be modified or redeployed independently without causing failures in other modules.\\
    {\bf Priority:} High

  \item \emph{Every source file must include concise comments, and higher-level documentation such as installation guides and API references should be maintained in GitHub Wiki or README files.}\\[2mm]
    {\bf Rationale:} Well-documented source code and external references make it easier for new developers to understand and maintain the project.\\
    {\bf Fit Criterion:} All functions and classes include docstrings, and complete setup instructions are verified through internal onboarding tests.\\
    {\bf Priority:} Medium

  \item \emph{Automated CI/CD pipelines should run unit, integration, and accessibility tests after each merge to ensure that updates do not disrupt existing functionality.}\\[2mm]
    {\bf Rationale:} Continuous testing prevents regressions and ensures that all code changes maintain system stability and accessibility compliance.\\
    {\bf Fit Criterion:} Each code merge triggers automated tests confirming successful execution of core functions and compliance checks.\\
    {\bf Priority:} High
\end{enumerate}

\subsection{Supportability Requirements}

\begin{enumerate}[label=MS-SUP\arabic*., wide=0pt, leftmargin=*]

  \item \emph{The system must record key metrics such as API calls, latency, error rates, and model confidence scores to support debugging and performance improvements.}\\[2mm]
    {\bf Rationale:} Tracking performance data allows maintainers to identify bottlenecks, reduce errors, and optimize processing efficiency.\\
    {\bf Fit Criterion:} Logged metrics are stored securely and reviewed periodically through a monitoring dashboard or exported reports.\\
    {\bf Priority:} Medium
\end{enumerate}

\subsection{Adaptability Requirements}

\begin{enumerate}[label=MS-AD\arabic*., wide=0pt, leftmargin=*]
  \item \emph{The system shall support integration of new AI models
    or components through a standardized interface schema.}\\[2mm]
    {\bf Rationale:} A consistent interface allows easy replacement
    or upgrading of AI components.\\
    {\bf Fit Criterion:} New modules adhere to the existing data
    structures and pass automated compatibility checks.\\
    {\bf Priority:} High

  \item \emph{The system shall allow configuration updates without
    modifying source code.}\\[2mm]
    {\bf Rationale:} Enables rapid adaptation to new accessibility or
    institutional requirements.\\
    {\bf Fit Criterion:} All configurable parameters are stored in
    editable files.\\
    {\bf Priority:} Medium

  \item \emph{Every academic term, significant updates should be tracked and documented to maintain change records and ensure reproducibility.}\\[2mm]
    {\bf Rationale:} Periodic documentation ensures version transparency and supports academic continuity for future development teams.\\
    {\bf Fit Criterion:} Version history includes changelogs, revision notes, and verification summaries for each update cycle.\\
    {\bf Priority:} Medium
\end{enumerate}




\section{Security Requirements}

\subsection{Access Requirements}
\begin{enumerate}[label=SR-AR \arabic*., wide=0pt, leftmargin=*]
\item \emph{The system shall restrict access exclusively to McMaster University users through institutional Single Sign-On (SSO) authentication.}\\[2mm] 
    {\bf Rationale:} Restricting access ensures only authorized users within McMaster can use the system during the pilot phase, reducing the risk of unauthorized use or data exposure.\\
    {\bf Fit Criterion:} All users must log in using verified McMaster SSO credentials before accessing the platform. Unauthenticated requests are automatically rejected.\\
    {\bf Priority:} High

\item \emph{All actions performed by users shall be tied to their authenticated session.}\\[2mm] 
    {\bf Rationale:} Linking actions to a user’s authenticated identity enables traceability and controlled access to system features.\\
    {\bf Fit Criterion:} Each upload or alt text generation event is associated with a unique McMaster user ID through SSO session tracking.\\
    {\bf Priority:} Medium
\end{enumerate}

\subsection{Integrity Requirements}
\begin{enumerate}[label=SR-IR \arabic*., wide=0pt, leftmargin=*]
\item \emph{All communication between the frontend, backend, and machine learning services shall use encrypted HTTPS (TLS TLS\_VERSION or higher).}\\[2mm] 
    {\bf Rationale:} Encryption prevents interception and tampering of sensitive data such as authentication tokens or image files.\\
    {\bf Fit Criterion:} All HTTP requests must be redirected to HTTPS; unencrypted requests are rejected by the web server.\\
    {\bf Priority:} High

\item \emph{Uploaded images shall remain unmodified during processing and analysis.}\\[2mm] 
    {\bf Rationale:} Preserving file integrity ensures consistent and accurate generation of alt text.\\
    {\bf Fit Criterion:} File hash comparison verifies that image files remain identical throughout the upload and analysis process.\\
    {\bf Priority:} High
\end{enumerate}

\subsection{Privacy Requirements}
\begin{enumerate}[label=SR-PR \arabic*., wide=0pt, leftmargin=*]
\item \emph{Uploaded images shall be deleted after each session.}\\[2mm] 
    {\bf Rationale:} Protects user privacy and ensures compliance with institutional data governance policies.\\
    {\bf Fit Criterion:} Uploaded files are stored temporarily in memory or on a secure local directory and deleted within FILE\_DELETE\_TIME after alt text generation.\\
    {\bf Priority:} High

\item \emph{Generated alt text shall not contain personally identifiable information (PII) or sensitive content.}\\[2mm] 
    {\bf Rationale:} Prevents disclosure of private information and ensures responsible AI usage.\\
    {\bf Fit Criterion:} The model output is passed through a content moderation filter that rejects or flags any alt text containing PII or inappropriate language.\\
    {\bf Priority:} Medium
\end{enumerate}

\subsection{Audit Requirements}
\begin{enumerate}[label=SR-AU \arabic*., wide=0pt, leftmargin=*]
\item \emph{System usage logs shall record authentication events, uploads, and generation activities for accountability and debugging.}\\[2mm] 
    {\bf Rationale:} Audit logs enable traceability, assist in debugging, and ensure ethical research practices.\\
    {\bf Fit Criterion:} Logs record timestamps, user IDs, and non-sensitive metadata while excluding image or generated text content.\\
    {\bf Priority:} Medium

\item \emph{Access to audit logs shall be restricted to authorized project administrators.}\\[2mm] 
    {\bf Rationale:} Limits access to potentially sensitive operational data and protects user confidentiality.\\
    {\bf Fit Criterion:} Logs are stored in a restricted-access directory with read permissions granted only to administrators.\\
    {\bf Priority:} Medium
\end{enumerate}

\subsection{Immunity Requirements}
\begin{enumerate}[label=SR-IM \arabic*., wide=0pt, leftmargin=*]
\item \emph{The system shall validate and sanitize all uploaded files to prevent malicious or unsupported file types.}\\[2mm] 
    {\bf Rationale:} Protects against injection attacks, corrupted uploads, or execution of non-image files.\\
    {\bf Fit Criterion:} Only files with valid image types FILE\_TYPES are accepted; unsupported or script files are automatically rejected.\\
    {\bf Priority:} High

\item \emph{The system shall block access from networks or domains outside McMaster University’s infrastructure.}\\[2mm] 
    {\bf Rationale:} Restricting network access minimizes exposure to external threats during the proof-of-concept phase.\\
    {\bf Fit Criterion:} Requests must originate from verified McMaster SSO tokens or IP ranges associated with university networks defined in NETWORK\_SOURCE\_POLICY.\\
    {\bf Priority:} High
\end{enumerate}

\section{Cultural Requirements}
The following list conists of cultural requirements the system shall follow:
\begin{itemize}
  \item[\textbf{CR 1.}] \textit{The system shall generate alternative
      text using neutral and inclusive language appropriate for
    academic environments.}\\
    \textbf{Rationale:} Ensures that generated content is respectful
    to diverse cultural and educational backgrounds.\\
    \textbf{Fit Criterion:} Generated alt text contains no culturally
    biased, exclusionary, or inappropriate terminology.\\
    \textbf{Priority:} High

  \item[\textbf{CR 2.}] \textit{The system shall avoid using
      culturally specific references unless the visual content
    explicitly requires it.}\\
    \textbf{Rationale:} Prevents misinterpretation and maintains
    accessibility for a wide audience.\\
    \textbf{Fit Criterion:} Alt text focuses on visual description
    and context without unnecessary cultural assumptions.\\
    \textbf{Priority:} Medium

  \item[\textbf{CR 3.}] \textit{The system shall use professional and
    educationally appropriate tone in all generated content.}\\
    \textbf{Rationale:} Maintains usability across academic
    departments and contexts.\\
    \textbf{Fit Criterion:} Outputs remain formal, non-colloquial,
    and context-relevant.\\
    \textbf{Priority:} Medium
\end{itemize}

\section{Compliance Requirements}
\subsection{Legal Requirements}
\begin{itemize}
  \item[\textbf{CR-LR 1.}] \textit{The system shall comply with AODA
    standards for alternative text generation.}\\
    \textbf{Rationale:} Ensures the tool supports institutional
    accessibility requirements and legal obligations.\\
    \textbf{Fit Criterion:} All generated alt text meets WCAG 2.1
    Level AA criteria for accuracy, clarity, and relevance.\\
    \textbf{Priority:} High
\end{itemize}
\subsection{Standards Compliance Requirements}
\begin{itemize}
  \item[\textbf{CR-SCR 1.}] \textit{The system shall follow
      institutional privacy and data-handling guidelines for uploaded
    teaching materials.}\\
    \textbf{Rationale:} Prevents unauthorized distribution or
    mishandling of academic content.\\
    \textbf{Fit Criterion:} No files are stored beyond active use
    unless explicitly authorized; logs exclude proprietary content.\\
    \textbf{Priority:} High

  \item[\textbf{CR-SCR 2.}] \textit{The system shall provide verifiable
    documentation or statements of compliance upon request.}\\
    \textbf{Rationale:} Facilitates audits, approvals, and
    integration into university workflows.\\
    \textbf{Fit Criterion:} A compliance overview document or help
    section is available to stakeholders.\\
    \textbf{Priority:} Medium
\end{itemize}

\section{Open Issues}
This section outlines unresolved questions and decisions that may
impact the overall success of the system.
The following items require additional research, testing, or
discussion to ensure the project’s successful completion.
\begin{itemize}
  \item The ML/AI model architecture the team will use to generate
    alternative text will need research and testing
    to ensure optimal accuracy and correctness.
  \item The optimal length of the generated alternative text requires
    further research to determine how many characters provide an
    accurate description without causing confusion or distracting
    from the main idea of the diagram.
  \item Some users have pre-defined keyboard shortcuts, while others use a standard
  menu. Therefore, more research needs to be conducted on the most optimal and efficient
  keyboard navigation type with the least accessibility barriers. 
  \item As mentioned in \autoref{tab:evaluation-metrics-summary} of the Appendix, all these metrics 
  need to be researched to further understand them. 
\end{itemize}

\section{Off-the-Shelf Solutions}
\subsection{Ready-Made Products}
\lips
\subsection{Reusable Components}
\lips
\subsection{Products That Can Be Copied}
\lips
\section{New Problems}

\subsection{Effects on the Current Environment}

The current academic and developmental environment may undergo a number of changes as a result of the implementation of the Reading4All system. Among these effects are:

\begin{enumerate}
    \item During analysis and description creation, the tool may use more memory and processing power, which could momentarily impair the functionality of other concurrently running applications.
    \item To incorporate Reading4All into current authoring or content preparation workflows, extra setup or dependencies might be needed.
\end{enumerate}

\subsection{Effects on the Installed Systems}

\begin{enumerate}
    \item It might be necessary to assess whether current departmental or institutional systems are compatible with Reading4All. The necessary integrations might not be fully supported by older software environments.
    \item Testing is necessary to make sure that routine document or media uploads continue to work as intended because the integration process may cause changes in system behaviour.
    \item In order to handle new descriptive text outputs and associated accessibility metadata, more storage or indexing might be required.
\end{enumerate}

\subsection{Reusable Components}
Inspirations can be taken from existing software components and libraries including: 
\begin{itemize}
  \item \textbf{Hugging Face Transformers:} An open-source library
    offering pretrained multimodal models (e.g., BLIP-2, CLIP, and
    ViT-GPT2). These can be fine-tuned to identify structural and
    semantic relationships within STEM diagrams.

  \item \textbf{Albumentations:} A Python library for data
    augmentation, enhancing dataset diversity for diagram recognition
    tasks. It can be reused in Reading4All’s preprocessing pipeline
    to improve robustness.

  \item \textbf{Pandas + Matplotlib AltText Plugin:} An open-source
    extension that generates descriptive alt text for statistical
    plots. Its modular logic can guide Reading4All’s diagram-specific
    description component.
\end{itemize}

\subsection{Follow-Up Problems}

\begin{enumerate}
    \item Reading4All will require recurring updates as accessibility guidelines change in order to stay compliant and guarantee its ongoing efficacy.
    \item Continuous user feedback might point out problems or areas that need work that weren't noticed during the first deployment.
    \item To guarantee consistent performance and integration with institutional systems, long-term upkeep and assistance will be needed.
\end{enumerate}


\section{Tasks}

\subsection{Project Planning}

\begin{itemize}
    \item \textbf{Development Approach}  
    The Reading4All team will adopt an iterative and adaptive workflow that promotes continuous improvement and consistent stakeholder engagement. Development will proceed through a series of short sprints, each focused on specific deliverables while allowing flexibility as project requirements evolve.  
    The general development process will include:
    \begin{enumerate}
        \item Analyzing requirements and refining functional specifications  
        \item Prioritizing the backlog and conducting sprint planning  
        \item Incremental coding and component integration  
        \item Performing unit testing, validation, and verification  
        \item Holding review sessions with the accessibility lead and project supervisor  
        \item Completing final deployment and documentation preparation  
    \end{enumerate}

    \item \textbf{Key Tasks}  
    The main project activities will involve:
    \begin{itemize}
      \item Confirm system architecture and interface specifications
      \item Establish a shared GitHub repository with branching standards
      \item Configure CI/CD automation through GitHub Actions
      \item Implement the vision module for diagram segmentation and labeling
      \item Integrate the language generation component for alt text synthesis
      \item Conduct user testing with instructors and accessibility specialists
      \item Evaluate outputs for WCAG 2.1 compliance and descriptive accuracy
    \end{itemize}

  \item \textbf{Resource Estimates}
    The project involves a TEAM\_SIZE member team responsible for design,
    development, testing, and reporting. Shared tools will include:
    \begin{itemize}
        \item Cloud-based GPU environments for AI model inference and experimentation  
        \item GitHub Projects for version control, issue tracking, and sprint management  
        \item Academic diagram datasets paired with validated alt-text examples for training and evaluation  
    \end{itemize}

    \item \textbf{Key Considerations}  
    The following factors are critical to ensuring efficient development and accessibility compliance:
    \begin{itemize}
        \item Detecting and resolving dependency conflicts through early integration testing  
        \item Monitoring and mitigating dataset bias to ensure inclusive model performance  
        \item Maintaining consistent communication with the accessibility consultant and faculty supervisor  
        \item Reducing project risks through incremental reviews, frequent documentation updates, and checkpoint testing  
    \end{itemize}

    \item \textbf{Documentation Process}  
    All project documentation will be collaboratively created and maintained by team members.  
    The documentation process will follow these practices:
    \begin{itemize}
        \item Storing all documents in the shared GitHub Wiki to promote transparency and accessibility  
        \item Using clear and concise commit messages and structured pull requests  
        \item Reviewing automated test results before code integration  
        \item Applying version tagging and peer review for all major deliverables  
        \item Archiving finalized reports and test outcomes for submission and supervisor review  
    \end{itemize}
\end{itemize}

\subsection{Planning of the Development Phases}

\textbf{Deliverables and Tentative Schedule}

\begin{center}
  \begin{tabularx}{\textwidth}{Xc}
    \toprule
    \textbf{Deliverables} & \textbf{Due Date} \\
    \midrule
    Problem Statement, Proof of Concept, and Development Plan & Week  04 \\
    Software Requirements Specifications and Hazards Analysis (Revision 0) & Week 06 \\
    Verification \& Validation Plan (Revision 0) & Week 08 \\
    Design Document (Rev-1) & Week 10 \\
    Proof of Concept Demonstration & Week 11 + 12 \\
    Design Document (Revision 0) & Week 16 \\
    Project Demonstration (Revision 0) & Week 18 + 19 \\
    Verification \& Validation Report (Revision 0) & Week 22 \\
    Final Demonstration (Revision 1) & Week 24 \\
    Final Documentation & Week 26 \\
    Capstone EXPO & Week 26 \\
    \bottomrule
  \end{tabularx}
\end{center}

\textbf{Sprint and Review Cycle}

\begin{itemize}
    \item Each sprint will last approximately two weeks and conclude with a structured progress review.  
    \item Milestones will be verified through peer review, supervisor evaluation, and automated testing.  
    \item Task reprioritization and milestone adjustments will be based on feedback from the accessibility advisor and project supervisor.  
\end{itemize}

\noindent By adopting this structured development plan, the team will ensure
that Reading4All evolves into a reliable and maintainable system
capable of producing accurate, accessible, and pedagogically useful
alt text for academic diagrams.

\section{Migration to the New Product}
\subsection{Requirements for Migration to the New Product}
\begin{enumerate}[label=MNP-RMNP  \arabic*., wide=0pt, leftmargin=*]
  \item \emph{The system shall support a
      phased implementation to allow gradual adoption while minimizing
    disruptions.}\\
    \textbf{Rationale:} Reduces organizational risk and allows
    controlled testing during rollout.\\
    \textbf{Fit Criterion:} Each phase is deployed and validated
    independently before progressing to the next.\\
    \textbf{Priority:} High

  \item \emph{The organization shall operate
      the new system in parallel with the old product for a defined
    transition period.}\\
    \textbf{Rationale:} Ensures continuity and confirms correct
    operation before full cutover.\\
    \textbf{Fit Criterion:} Parallel operation lasts until all
    critical functions pass acceptance testing.\\
    \textbf{Priority:} High

  \item \emph{The system shall provide
    procedures and tools for manual backup during transition.}\\
    \textbf{Rationale:} Maintains operational stability during migration.\\
    \textbf{Fit Criterion:} Backup processes are documented, tested,
    and accessible to staff.\\
    \textbf{Priority:} Medium

  \item \emph{The transition plan shall
    identify and schedule major components and release phases.}\\
    \textbf{Rationale:} Guides project planning and resource allocation.\\
    \textbf{Fit Criterion:} A migration timeline with milestones and
    dependencies is documented.\\
    \textbf{Priority:} Medium
\end{enumerate}

\subsection{Data That Has to be Modified or Translated for the New System}
This section does not apply to this project as there is no current
system to replace, thus, no data at all.

\section{Costs}

The total cost of developing this project is primarily based on the effort contributed by the student development team and faculty supervisors. As the project utilizes open-source technologies and university-hosted infrastructure, no direct monetary expenditure is incurred. The project is scheduled to be completed within the academic term (Minimum Viable Product (MVP) ready by April 2026), and resource allocation is focused on efficient time management and workload balancing rather than financial cost.

\subsection{Metrics for Estimation}
\begin{enumerate}[label=C-ME \arabic*., wide=0pt, leftmargin=*]
\item \emph{Number of image input/output workflows supported by the tool.}\\[2mm] 
    {\bf Rationale:} Defines the scope of functionality that affects the development and testing workload.\\
    {\bf Priority:} High

\item \emph{Number of core functional requirements (e.g., image upload, alt text generation, user authentication, evaluation metrics).}\\[2mm] 
    {\bf Rationale:} Determines the complexity of the implementation and testing processes.\\
    {\bf Priority:} High

\item \emph{Number of non-functional requirements (e.g., accessibility compliance, latency, privacy, and scalability).}\\[2mm] 
    {\bf Rationale:} Reflects the additional design and validation effort beyond core functionality.\\
    {\bf Priority:} Medium

\item \emph{Number of deliverables and milestones within the development timeline.}\\[2mm] 
    {\bf Rationale:} Ensures that project progress and deliverables are measurable and time-bound.\\
    {\bf Priority:} Medium

\item \emph{Team size and individual role distribution (frontend, backend, model integration, documentation).}\\[2mm] 
    {\bf Rationale:} Defines the workload balance and collaborative structure of the development process.\\
    {\bf Priority:} High
\end{enumerate}

\subsection{Estimation Approach}
\begin{enumerate}[label=C-EA \arabic*., wide=0pt, leftmargin=*]
\item \emph{Each deliverable has been estimated based on the effort required to implement, test, and document it within the academic term.}\\[2mm] 
    {\bf Rationale:} Provides a structured approach to time allocation aligned with academic deadlines.\\
    {\bf Fit Criterion:} Estimates are derived from prior experience with similar web-based machine learning projects and adjusted for accessibility integration.\\
    {\bf Priority:} High

\item \emph{Time allocation accounts for model fine-tuning, accessibility testing, and usability evaluation.}\\[2mm] 
    {\bf Rationale:} Incorporates all major tasks necessary to ensure compliance with WCAG standards and project objectives.\\
    {\bf Fit Criterion:} Schedule includes model optimization, frontend validation, and pilot user testing within the project timeline.\\
    {\bf Priority:} Medium
\end{enumerate}

\subsection{Cost Breakdown}
\begin{enumerate}[label=C-CB \arabic*., wide=0pt, leftmargin=*]
\item \emph{Development Effort}\\[2mm] 
    {\bf Rationale:} Represents the primary cost driver, measured in person-hours contributed by the student team and supervisor.\\
    {\bf Fit Criterion:} Based on a team of TEAM\_SIZE student developers and one faculty supervisor:
    \begin{itemize}
        \item Initial research, planning, and requirement analysis: HOURS\_RESEARCH per team member.
        \item Model integration and backend implementation: HOURS\_BACKEND per team member.
        \item Frontend development and accessibility compliance: HOURS\_FRONTEND per team member.
        \item Testing, debugging, and refinement: HOURS\_TESTING per team member.
        \item Documentation and presentation preparation: HOURS\_DOCS per team member.
    \end{itemize}
    {\bf Total Estimated Effort:} HOURS\_TOTAL per team member.\\
    {\bf Priority:} High

\item \emph{Tools and Software}\\[2mm] 
    {\bf Rationale:} All software components used in the project are open-source or free for academic use.\\
    {\bf Fit Criterion:} No direct licensing or software procurement costs are incurred.\\
    {\bf Priority:} High

\item \emph{Testing Environment}\\[2mm] 
    {\bf Rationale:} Defines the testing setup and potential costs associated with user testing incentives.\\
    {\bf Fit Criterion:} Testing is conducted on McMaster-hosted or open-source platforms. User testing sessions may include incentives between COST\_INCENTIVE\_MIN–COST\_INCENTIVE\_MAX.\\
    {\bf Priority:} Medium

\item \emph{Hardware and Infrastructure}\\[2mm] 
    {\bf Rationale:} Evaluates the physical and computational resources used during development.\\
    {\bf Fit Criterion:} No additional hardware purchases required beyond student laptops and academic cloud credits.\\
    {\bf Priority:} Low
\end{enumerate}

\subsection{Estimated Cost}
\begin{enumerate}[label=C-EC \arabic*., wide=0pt, leftmargin=*]
\item \emph{Total Development Effort}\\[2mm] 
    {\bf Rationale:} Quantifies total workload for the entire team in measurable units.\\
    {\bf Fit Criterion:} Approximately HOURS\_PROJECT person-hours across team members (TEAM\_SIZE × HOURS\_TOTAL).\\
    {\bf Priority:} High

\item \emph{Notional Monetary Cost}\\[2mm] 
    {\bf Rationale:} Provides an academic-equivalent cost estimate for effort valuation.\\
    {\bf Fit Criterion:} Assuming an average rate of COST\_PER\_HOUR, total notional cost is approximately COST\_TOTAL.\\
    {\bf Priority:} Medium

\item \emph{Actual Financial Cost}\\[2mm] 
    {\bf Rationale:} Since the project leverages university and open-source resources, there are no direct expenses.\\
    {\bf Fit Criterion:} The actual monetary cost is COST\_ACTUAL; all expenditure is in research and development effort.\\
    {\bf Priority:} High
\end{enumerate}
\section{User Documentation and Training}
\subsection{User Documentation Requirements}
\begin{enumerate}
  \item User Manual
    \begin{itemize}
      \item \textbf{Purpose}: The user manual will serve as a user
        guide and provide detailed information and instructions on
        the final product and how to use it effectively.
      \item \textbf{Target Audience}: Academic students, instructors,
        and other professionals.
      \item \textbf{Content}: Web tool navigation and instructions,
        usage examples, product purposes, and best practices.
    \end{itemize}
\end{enumerate}
\subsection{Training Requirements}
Users of the final product will require minimal to no training as we
aim to ensure that the tool is as accessible, simple, and intuitive
as possible. For any additional guidance,
a user manual will be created along with any relevant tutorials on
how to use the features within the web tool.

\section{Waiting Room}
This section lists potential ideas and features that are out of scope
for the current project, however, may be valuable for future updates.
\begin{itemize}
  \item Support for multilingual alternative text generation (e.g.,
    French and Spanish).
  \item A browser extension that automatically generates alternative
    text on websites or learning platforms (e.g. \textit{D2L}) using our model.
  \item Compatibility with mobile platforms to extend accessibility
    across users' preferred devices.
  \item Allowing users to report issues or feedback anonymously directly through the web tool 
  that can enable user-driven improvements while maintaining privacy.

\end{itemize}

\section{Ideas for Solution}
This section discusses potential ways to achieve some of the functionality discussed throughout this report, 
including image upload and processing, alternative text generation and session history. 
These ideas have been documented so they can be referenced later during development.\\

\begin{enumerate}
  \item \textbf{Image Upload and Processing}:
  This functionality can be achieved through a front-end interface,
  where users are prompted to upload an image using an upload button or
  by dragging their file into the drop box. Furthermore, to minimize
  errors, this will only allow JPEG and PNG image files.
  Once the image has been uploaded, it will be displayed to the user
  with the image file name, so users can confirm the correct file was chosen.
  If the upload fails, the system will display an error message
  explaining the issue. Furthermore, this can be achieved using the
  HTML5 File API, which supports reading and processing file data,
  specifically obtained through input or drag and drop. 
  \item 
  \textbf{Alternative Text Generation}:
  This functionality can be implemented using a vision-language model
  (VLM), which combines natural language models with computer vision.
  The model can learn from both images and text to solve various
  problems. The model can be trained using sample technical diagrams,
  paired with examples of descriptive alternative text, allow it to
  generate accurate and high quality descriptions for new images.
  \item 
  \textbf{Session History}:
  After the user is satisfied with the generated alternative text, the system will store the image and its final description in the browser's session storage as a JSON record. This allows the data to be stored temporarily and can easily be displayed to the user when they request their history.   
\end{enumerate}

\newpage
\section*{Appendix --- Evaluation Metrics Summary}
\label{sec:evaluation-metrics}
The following table summarizes the evaluation metrics that will be used to assess the quality and effectiveness of the alternative text generated by the Reading4All system. Each metric includes its scale type, acceptable range, and a brief description of its purpose.
\begin{table}[H]
    \centering
    \caption{Evaluation Metrics Summary}
    \label{tab:evaluation-metrics-summary}
    \begin{tabular}{ |p{3.5cm}|p{3cm}|p{3cm}|p{4cm}| }
      \hline
      \textbf{Metric Name} & \textbf{Scale Type} & \textbf{Acceptable Range} & \textbf{Summary Description} \\
      \hline
      Sufficiency of Description
        & Categorical (1--3)
        & $\geq$ 3 (Sufficient)
        & Does the alt text convey enough information to achieve the intended objective? \\
      \hline
      Length Appropriateness
        & Categorical (1--3)
        & $\geq$ 3 (Proper Length)
        & Is the alt text concise yet complete (not too short or overly verbose)? \\
      \hline
      Accessibility / Usability
        & Numerical (0--3)
        & $\geq$ 2 (Acceptable)
        & Assistive-technology compatibility and clarity; aligns with WCAG 2.1 Level~AA use. \\
      \hline
      Learning Impact
        & Numerical (0--3)
        & $\geq$ 2 (Positive)
        & Does the alt text support or enhance user understanding in learning contexts? \\
      \hline
      Qualitative Feedback Notes
        & Textual
        & N/A
        & Free-form comments on clarity, tone, and suggested improvements. \\
      \hline
    \end{tabular}
  \end{table}
  

\section*{Appendix --- Reflection}

\input{../Reflection.tex}

\begin{enumerate}
  \item What went well while writing this deliverable? 
  \item What pain points did you experience during this deliverable, and how did
  you resolve them?
  \item How many of your requirements were inspired by speaking to your
  client(s) or their proxies (e.g. your peers, stakeholders, potential users)?
  \item Which of the courses you have taken, or are currently taking, will help
  your team to be successful with your capstone project.
  \item What knowledge and skills will the team collectively need to acquire to
  successfully complete this capstone project?  Examples of possible knowledge
  to acquire include domain specific knowledge from the domain of your
  application, or software engineering knowledge, mechatronics knowledge or
  computer science knowledge.  Skills may be related to technology, or writing,
  or presentation, or team management, etc.  You should look to identify at
  least one item for each team member.
  \item For each of the knowledge areas and skills identified in the previous
  question, what are at least two approaches to acquiring the knowledge or
  mastering the skill?  Of the identified approaches, which will each team
  member pursue, and why did they make this choice?
\end{enumerate}

\textbf{Casey Francine Bulaclac - Reflection}
\begin{enumerate}
  \item What went well while writing this deliverable? \\[1ex]
  Having discussed the project thoroughly as a team and with our supervisor helped in writing this deliverable as the team
  was very knowledgeable about the needs for the project. 
  This deliverable went much smoother than the last due to stronger operational procedures, and better organization in how we structured and completed the SRS. 
  The team communicated well and were clear of the goals for this deliverable.
  \item What pain points did you experience during this deliverable, and how did
  you resolve them?\\[1ex]
  One pain point in writing the SRS was figuring out what each of the many sections entailed in the Volere's template. The template 
  is very thorough and needed many details, in which some sections seem to overlap which can be confusing. Another pain point was ensuring traceability
  between our goals in the project and the requirements. To resolve this, I made sure to ask the TA for feedback and clarification about specific sections.
  Additionally, communicating with each team member and ensuring our requirements aligned to the goals of the project was very helpful in aiding to ensure
  traceability.
  \item How many of your requirements were inspired by speaking to your
  client(s) or their proxies (e.g. your peers, stakeholders, potential users)? \\[1ex]
  Many, if not most, requirements were inspired through speaking with our supervisor, who had the most knowledge and experience with our project's 
  potential users and stakeholders. In this project, it is important to understand our target users as we are designing for accessibility, so it was critical 
  in making our requirements. 
  \item Which of the courses you have taken, or are currently taking, will help
  your team to be successful with your capstone project.\\[1ex]
  In this deliverable, the course that was most beneficial was Software Requirements and Security Considerations (SFWRENG 3RA3) as we learned how to create effected SRS documents.
  A course I've taken that will help thoroughly in ensuring our user interface is accessible is Human Computer Interfaces (SFWRENG 4HC3) as the course taught us principles of good design. Lastly,
  another course I took that contribute to the success of our project is Applications of Machine Learning (SFWRENG 4AL3) as this project heavily involves machine learning
  in generating alternative text.
  \item What knowledge and skills will the team collectively need to acquire to
  successfully complete this capstone project?  Examples of possible knowledge
  to acquire include domain specific knowledge from the domain of your
  application, or software engineering knowledge, mechatronics knowledge or
  computer science knowledge.  Skills may be related to technology, or writing,
  or presentation, or team management, etc.  You should look to identify at
  least one item for each team member. \\[1ex]
  
  \item For each of the knowledge areas and skills identified in the previous
  question, what are at least two approaches to acquiring the knowledge or
  mastering the skill?  Of the identified approaches, which will each team
  member pursue, and why did they make this choice?
\end{enumerate}

\textbf{Nawaal Fatima - Reflection}
\begin{enumerate}
  \item \textbf{What went well while writing this deliverable?} \newline
    Our group divided the work efficiently, which made the writing
    process smoother. I found that once we agreed on the structure,
    it became easier to contribute my part because everyone had a
    clear understanding of what they were responsible for.
    We all came together towards the end and reviewed everyone's
    parts, ensuring coherency and consistency in our writing.
    Communication also went well, and we were able to clarify
    uncertainties quickly through discussions. Personally, I felt
    more confident writing my section because I understood how my
    contribution fit into the overall deliverable.

  \item \textbf{What pain points did you experience during this
    deliverable, and how did you resolve them?} \newline
    One challenge I experienced was making sure my writing aligned
    with the tone and level of detail the rest of the team was using.
    At first, it was hard to tell how formal or detailed certain
    sections should be. I resolved this by checking in with my teammates'
    writing and reading over their parts so that my section matched in style.
    Another minor pain point was managing time alongside other
    coursework, but planning out smaller chunks helped me stay on track.
    Furthermore, my computer glitched during an update and I had to
    rewrite all of my sections which was super frustrating and
    delayed some teammate's sections. In the future I will ensure to
    commit all drafts to a remote branch before updating my computer.

  \item \textbf{How many of your requirements were inspired by
      speaking to your client(s) or their proxies (e.g., your peers,
    stakeholders, potential users)?} \newline
    A noticeable portion of the requirements came from talking to
    Ms. Sui and from my experiences working with her over the past
    three years. Even though we don't always interact with the customers
    directly, speaking to Ms. Sui and imagining how potential users
    would interact with the system helped shape several of the
    requirements. I'd estimate that roughly half of the requirements
    were influenced by those conversations or by feedback from people
    who could represent the end users.

  \item \textbf{Which of the courses you have taken, or are currently
      taking, will help your team be successful with your capstone
    project?} \newline
    Several courses connect directly to this project. Software
    engineering and requirements-focused courses helped with
    understanding how to draft clear specifications and think about
    users' needs. Any design or project-based courses gave me
    experience working in teams and coordinating deliverables.
    Courses that covered testing, human-computer interfaces,
    documentation, and development
    processes also helped ensure we follow good practices
    throughout the capstone.
\end{enumerate}

\newpage
\section*{References}

\begin{hangparas}{2em}{1}
  Microsoft, ``Azure AI Vision---Image Analysis,'' Microsoft Docs/Learn.
  Accessed: Oct.~9,~2025. [Online]. Available:
  \url{https://learn.microsoft.com/en-us/azure/ai-services/computer-vision/overview-image-analysis?tabs=4-0}
  \\
  \par
  AltText.ai, ``AltText.ai---Automatic Image Alt Text Generation,'' AltText.ai.
  Accessed: Oct.~9,~2025. [Online]. Available: \url{https://alttext.ai/}
  \\
  \par
  accessiBe, ``Building a more accessible web together'', accessiBe.
  Accessed: Oct.~9,~2025. [Online]. Available:
  \url{https://www.accessibe.com/about}
\end{hangparas}

\end{document}
