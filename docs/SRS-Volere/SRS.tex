% THIS DOCUMENT IS FOLLOWS THE VOLERE TEMPLATE BY Suzanne Robertson and James Robertson
% ONLY THE SECTION HEADINGS ARE PROVIDED
%
% Initial draft from https://github.com/Dieblich/volere
%
% Risks are removed because they are covered by the Hazard Analysis
\documentclass[12pt]{article}

\usepackage{booktabs}
\usepackage{tabularx}
\usepackage{hyperref}
\usepackage{enumitem}
\usepackage{float}
\usepackage{graphicx}



\hypersetup{
    bookmarks=true,         % show bookmarks bar?
      colorlinks=true,      % false: boxed links; true: colored links
    linkcolor=red,          % color of internal links (change box color with linkbordercolor)
    citecolor=green,        % color of links to bibliography
    filecolor=magenta,      % color of file links
    urlcolor=cyan           % color of external links
}

\newcommand{\lips}{\textit{Insert your content here.}}

\input{../Comments}
%% Common Parts

\newcommand{\progname}{ProgName} % PUT YOUR PROGRAM NAME HERE
\newcommand{\authname}{Team 22, READING4ALL
\\ Fiza Sehar
\\ Nawaal Fatima
\\ Dhruv Sardana
\\ Moly Mikhail
\\ Casey Francine Bulaclac } % AUTHOR NAMES                  

\usepackage{hyperref}
    \hypersetup{colorlinks=true, linkcolor=blue, citecolor=blue, filecolor=blue,
                urlcolor=blue, unicode=false}
    \urlstyle{same}
                                


\begin{document}

\title{Software Requirements Specification for \progname: subtitle describing software} 
\author{\authname}
\date{\today}
	
\maketitle

~\newpage

\pagenumbering{roman}

\tableofcontents

~\newpage

\section*{Revision History}

\begin{tabularx}{\textwidth}{p{3cm}p{2cm}X}
\toprule {\textbf{Date}} & {\textbf{Version}} & {\textbf{Notes}}\\
\midrule
Date 1 & 1.0 & Notes\\
Date 2 & 1.1 & Notes\\
\bottomrule
\end{tabularx}

~\\

~\newpage
\section{Purpose of the Project}
\subsection{User Business}
\lips
\subsection{Goals of the Project}
\lips
\section{Stakeholders}
\subsection{Client}
\lips
\subsection{Customer}
\lips
\subsection{Other Stakeholders}
\lips
\subsection{Hands-On Users of the Project}
\lips
\subsection{Personas}
\lips
\subsection{Priorities Assigned to Users}
\lips
\subsection{User Participation}
\lips
\subsection{Maintenance Users and Service Technicians}
\lips

\section{Mandated Constraints}
\subsection{Solution Constraints}
\lips
\subsection{Implementation Environment of the Current System}
\lips
\subsection{Partner or Collaborative Applications}
\lips
\subsection{Off-the-Shelf Software}
\lips
\subsection{Anticipated Workplace Environment}
\lips
\subsection{Schedule Constraints}
\lips
\subsection{Budget Constraints}
\lips
\subsection{Enterprise Constraints}
\lips

\section{Naming Conventions and Terminology}
\subsection{Glossary of All Terms, Including Acronyms, Used by Stakeholders
involved in the Project}
\lips

\section{Relevant Facts And Assumptions}
\subsection{Relevant Facts}
\lips
\subsection{Business Rules}
\lips
\subsection{Assumptions}
\lips

\section{The Scope of the Work}
\subsection{The Current Situation}
\lips
\subsection{The Context of the Work}
\lips
\subsection{Work Partitioning}
\lips
\subsection{Specifying a Business Use Case (BUC)}
\lips

\section{Business Data Model and Data Dictionary}
\subsection{Business Data Model}
\lips
\subsection{Data Dictionary}
\lips

\section{The Scope of the Product}
\subsection{Product Boundary}
The diagram below shows the components within the system and how they connect. The components that this project will aim on building include a user interface, an alternative text generation ML model, a session manager. Furthermore, these components will utilize or communicate with a screen reader software, McMaster Authentication system and external AI/ML Frameworks. 
 \begin{figure}[H]
    \centering
    \includegraphics[width=1.0\textwidth]{Product_Boundary_Diagram.jpg}
    \caption{Product Boundary Diagram}
    \label{img:usecase}
\end{figure}

\subsection{Product Use Case Table}
The table below discusses the different use cases. 
    \begin{table}[H]
      \centering
      \caption{Labels and Their Usage}
      \begin{tabular}{|p{1.3cm}|p{3.5cm}|p{3.5cm}|p{4cm}|p{2.8cm}|}
      \hline
      \textbf{PUC\#} & \textbf{PUC Name} & \textbf{Actor(s)} & \textbf{Input \& Output(s)} & \textbf{Requirement} \\ 
      \hline 
      PUC 1 & Login Using McMaster Credentials &  McMaster Student and Faculty , McMaster Authentication System & User Credentials (input), Authentication Results (output) & FR 7 \\
      \hline 
      PUC 2 & Upload Image & McMaster Student and Faculty & JPEG or PNG (input), Image upload status (output) & FR 1, UHR-EUR 3 \\
      \hline 
      PUC 3 & Generate Alternative Text &  McMaster Student and Faculty, Alternative Text Generation Model & Uploaded image (input), Generated alternative text  & FR 2, FR 3\\
      \hline
      PUC 4 & Copy or Download Text & McMaster Student and Faculty & User decision to copy or download (input), text copied to clipboard or downloaded & UHR-PIR 1\\ 
      \hline
      PUC 5 & View History of Inputted Images and their Alternative Text & McMaster Student and Faculty & User request to view history (input), Display of previously inputted images and their generated text within a session & FR 5\\
      \hline
\end{tabular}
\end{table}

\subsection{Individual Product Use Cases (PUC's)}
\textbf{PUC 1: Login Using McMaster Credentials }\\
\textbf{Trigger:} User selects "Login" and is directed to McMaster sign in page.\\
\textbf{Preconditions:}
\begin{itemize}
  \item The user is registered person in McMaster system and has valid credentials.
\end{itemize}
\textbf{Actors:} McMaster Student or Faculty, McMaster Authentication System.\\
\textbf{Outcome:} McMaster validates user's credentials are validated and they are given access to system.\\
\textbf{Input:} McMaster System username and password. \\
\textbf{Output:} User enters system or an error message is displayed.\\
\\
\textbf{PUC 2: Upload Image }\\
\textbf{Trigger:} User selects "Upload Image" and chooses a file\\
\textbf{Preconditions:}
\begin{itemize}
  \item User successfully logged into the system.
\end{itemize}
\textbf{Actors:} McMaster Student or Faculty \\
\textbf{Outcome:} The selected image is uploaded and stored for later text generation.\\
\textbf{Input:} Image file (JPEG or PNG) \\
\textbf{Output:} A confirmation message is displayed if the image was successfully uploaded, or an error message otherwise is displayed.\\
\\
\textbf{PUC 3: Generate Alternative Text}\\
\textbf{Trigger:} User selects \textit{Generate Alternative Text} for an uploaded image.\\
\textbf{Preconditions:}
\begin{itemize}
  \item A valid image has been successfully uploaded to the system. 
\end{itemize}
\textbf{Actors:} McMaster Student or Faculty.\\
\textbf{Outcome:} The system generates a descriptive alternative text for the uploaded image.\\
\textbf{Input:} User selection to generate alternative text\\
\textbf{Output:} Generated alternative text is displayed to the user.\\
\\
\textbf{PUC 4: Copy or Download Generated Alternative Text }\\
\textbf{Trigger:} User selects \textit{Copy} to clipboard or \textit{Download} .txt after generating alternative text.\\
\textbf{Preconditions:}
\begin{itemize}
  \item System has successfully generated alternative text.
  \item User is satisfied with generated alternative text and has made any desired changes. 
\end{itemize}
\textbf{Actors:} McMaster Student or Faculty. \\
\textbf{Outcome:} The user receives the alternative text through their preferred method. \\
\textbf{Input:} User decision to copy or download.\\
\textbf{Output:} Text is copied to clipboard or downloaded as .txt file on the users device.\\
\\
\textbf{PUC 5: View History of Uploaded Images and Generated Alternative Text}\\
\textbf{Trigger:} User selects \textit{View History} option.\\
\textbf{Preconditions:}
\begin{itemize}
  \item User is logged in with an active session. 
  \item User has previously uploaded at least one image and generated text within the session. 
\end{itemize}
\textbf{Actors:} McMaster Student or Faculty.\\
\textbf{Outcome:} The user views a list of their images and the corresponding generated alternative text within the current session. \\
\textbf{Input:} User request to view session history.\\
\textbf{Output:} Display of uploaded images and their corresponding generated alternative text.\\



\section{Functional Requirements}
\subsection{Functional Requirements}
\begin{enumerate}[label=FR \arabic*., wide=0pt, leftmargin=*]
  \item \emph{The system must accept technical diagrams in the format of JPEG and PNG.}\\[2mm]
    {\bf Rationale:} The system must process JPEG/PNG images in order to output alternative text. \\
    {\bf Fit Criterion:} The system successfully takes as accepts JPEG/PNG images and provides feedback to users when an invalid file type is inputted.  \\
    {\bf Priority:} High
  \item \emph{The system shall generate alternative text of uploaded gyms.}\\[2mm]
    {\bf Rationale:} The main purpose of the system is to make scientific diagrams more accessible by generating better alternative-text. \\
    {\bf Fit Criterion:} For a set of test diagrams, the alternative text generated must meet the pre-determined criteria.\\
    {\bf Priority:} High
  \item \emph{The system shall output alternative text in a format readable by screen readers.}\\[2mm]
    {\bf Rationale:} Students with disabilities utilize screen readers to access digital content; therefore, the alternative text must be displayed in away that enables screen readers to read it correctly. Furthermore, if the alternative text output format is not compatible with screen readers, then students cannot benefit from the application output.\\
    {\bf Fit Criterion:} The alternative text output must be readable by at least three commonly used screen readers.\\
    {\bf Priority:} High
  \item \emph{The system shall allow users to edit the outputted alternative texts.}\\[2mm] 
    {\bf Rationale:} Providing users with an option to edit the outputted text, enables them to adjust the output to better meet their needs if needed.\\
    {\bf Fit Criterion:} Users can add or delete text in any part of the outputted alternative text and save their changes.\\
    {\bf Priority:} High
  \item \emph{The system shall store and display all inputted images and their generated alternative texts within a session.}\\[2mm] 
    {\bf Rationale:} Storing previously inputted images and their generated alternative texts, allows users to easily review or reuse them without re-uploading. \\
    {\bf Fit Criterion:} Users can see view all previously inputted images with their generated alternative texts during the same session. \\
    {\bf Priority:} Medium
  \item \emph{The system must accept keyboard input for navigation.}\\[2mm] 
    {\bf Rationale:} Many users, including those with disabilities, use keyboard inputs to navigate through applications, the system must support this as a way to navigate.\\
    {\bf Fit Criterion:} Users can navigate to all the main functions and areas of the system using their keyboard. \\
    {\bf Priority:} High
 \item \emph{The system must validate users during login to confirm they are McMaster University students.}\\[2mm] 
    {\bf Rationale:} User verification will ensure that only McMaster University students have access to the system, ensuring that the system is used by the intended users.  \\
    {\bf Fit Criterion:} Users can only gain access to the system features after their McMaster University credentials are successfully validated.  \\
    {\bf Priority:} High
\end{enumerate}

\section{Look and Feel Requirements}
\subsection{Appearance Requirements}
\begin{enumerate}[label=LFR-AR \arabic*., wide=0pt, leftmargin=*]
  \item \emph{The system must allow all text on the interface to be resized up to 200 \%, without any loss of functionality or content. }\\[2mm] 
    {\bf Rationale:} Allowing text resizing will enable users with low vision to more easily utilize the system. This also ensures the system meets WCAG Success Criterion 1.4.4 Resize Text.
    User verification will ensure that only McMaster University students have access to the system, ensuring that the system is used by the intended users.  \\
    {\bf Fit Criterion:} All text, excluding any captions and images of text can be enlarged to 200 \% on a standard browser zoom (ex. Google Chrome) without any overlapping, hidden content, or broken features.  \\
    {\bf Priority:} High
  \item \emph{The system must not use color as the only method to provide information, indicate actions or prompt user input.}\\[2mm] 
    {\bf Rationale:} Users with color vision deficiencies or other visual impairments may not detect color differences accurately. This also ensures the system meets WCAG Success Criterion 1.4.1 Use of Color.\\
    {\bf Fit Criterion:} Any use of color communicates information to the user or requests information from he user must be appear with text.  \\
    {\bf Priority:} High
  \item \emph{The system must ensure sufficient contrasts of text and images of text.}\\[2mm] 
    {\bf Rationale:} Sufficient color contrast is important as it enables users with low vision or color vision deficiencies to easily read any system text. This also ensures the system meets WCAG Success Criterion 1.4.3 Contrast (Minimum)\\
    {\bf Fit Criterion:}  All text and images of text in the system interfaces has a contrast ratio of at least 4.5:1  \\
    {\bf Priority:} High
  \item \emph{The system must provide alternative text for all non-text content..}\\[2mm] 
    {\bf Rationale:} Users with visual impairment often use screen readers to navigate through software systems; therefore, it is essential that all images have sufficient alternative text, so that the purpose of the images can understood. This also ensures the system meets WCAG Success Criterion 1.1.1 Non-text Content  .\\
    {\bf Fit Criterion:} All images and non-text element are joined with descriptive alternative text that communicate their meaning. \\
    {\bf Priority:} High
\end{enumerate}

\subsection{Style Requirements}
\begin{enumerate}[label=LFR-SR \arabic*., wide=0pt, leftmargin=*]
\item \emph{The system interface must follow a simple and modern design style.}\\[2mm] 
    {\bf Rationale:} A simple interface will improve the systems usability as it better highlights the system's features, while also ensuring the system is visually appealing.\\
    {\bf Fit Criterion:} The system uses a clean layout with a maximum of three colors, consistent font styles and sizes, as well as only has key design elements that support usability. \\
    {\bf Priority:} High
\item \emph{The system interface must use McMaster University branding while maintaining accessibility standards and a modern style.}\\[2mm] 
    {\bf Rationale:} As the system is targeted towards McMaster University students, using the schools branding will build trust with users and ensure the system aligns with McMaster's identify. However, using McMaster branding must not interfere with usability and accessibility criteria..\\
    {\bf Fit Criterion:} The system interface includes McMaster University' official logo and meets the WCAG 2.1 contrast and non-text content success criteria. \\
    {\bf Priority:} High
\end{enumerate}


\section{Usability and Humanity Requirements}
\subsection{Ease of Use Requirements}
\begin{enumerate}[label=UHR-EUR \arabic*., wide=0pt, leftmargin=*]
\item \emph{The system interface must allow users to efficiently use the system features.}\\[2mm] 
    {\bf Rationale:} It is important the users can quickly access and use the system features, as they may be generating multiple alternative text outputs in a single session. \\
    {\bf Fit Criterion:} Users can upload images to the system and generate alternative text in 5 steps or fewer. \\
    {\bf Priority:} High
\item \emph{The system interface must be easy for users to remember how to use after not using it for some time.}\\[2mm] 
    {\bf Rationale:} Users should be able to quickly recall how to use the system without needing to relearn the features. An intuitive design will make it easier for returning users to find and use key features.  \\
    {\bf Fit Criterion:} Users who have not used the system in a month, can successfully login, upload an image and generate alternative text within 5 minutes, without needing any assistance. \\
    {\bf Priority:} Medium
\item \emph{The system interface must provide users with clear and immediate feedback for all actions.}\\[2mm] 
    {\bf Rationale:} Providing the users with feedback ensures they understand of the outcome of their actions and whether they are using the system correctly. This reduces confusion and makes users more confident while using the system. \\
    {\bf Fit Criterion:} The system provides textual feedback within 1 second after a user interaction, such as uploading an image.  \\
    {\bf Priority:} High 
\item \emph{The system interface must provide clear instructions, prevent common errors and allow users to easily correct them.}\\[2mm] 
    {\bf Rationale:} Providing easy to follow instructions will help ensure that users can easily use the system features and prevent errors. Additionally, if a user makes a mistake, they should easily be able to revert it.  \\
    {\bf Fit Criterion:} In user testing, at least 80\% of users can complete tasks without errors. When a user error occurs, the system explains the issue and how to recover within 2 seconds.\\
    {\bf Priority:} High 
\end{enumerate}


\subsection{Personalization and Internationalization Requirements}
\begin{enumerate}[label=UHR-PIR \arabic*., wide=0pt, leftmargin=*]
\item \emph{The system interface must allow users to choose how generated alternative text is stored or copied.}\\[2mm] 
    {\bf Rationale:} Providing users with the option to either copy generated text to their clipboard or download it as file, helps tailor the output to the users specific needs.  \\
    {\bf Fit Criterion:} After generating the alternative text users can choose to "Copy to Clipboard" or "Download as .txt" from the interface and system successfully completes the chosen option.\\
    {\bf Priority:} High
\end{enumerate}


\subsection{Learning Requirements}
\begin{enumerate}[label=UHR-LR \arabic*., wide=0pt, leftmargin=*]
\item \emph{The system must be easy for low-vision users to learn and operate with screen readers.}\\[2mm] 
    {\bf Rationale:} The system should be intuitive for users with low vision to use without prior training. Additionally, the system being highly compatible with screen readers, allows users to more easily navigate and use the system.  \\
    {\bf Fit Criterion:} In user testing, at least 90\% of first time users with low vision using a screen reader can upload an image and generate alternative text within 5 minutes without assistance. \\
    {\bf Priority:} High
\end{enumerate}


\subsection{Understandability and Politeness Requirements}
\begin{enumerate}[label=UHR-LR \arabic*., wide=0pt, leftmargin=*]
\item \emph{The system must be only display essential information and hide all technical details.}\\[2mm] 
    {\bf Rationale:} The system should only communicate the information needed to use the system. Displaying any technical details may cause the user to be confused and make the system less usable.   \\
    {\bf Fit Criterion:} In user testing, users do not encounter any technical terms, code outputs or information that is not relevant to them.  \\
    {\bf Priority:} High
\end{enumerate}


\subsection{Accessibility Requirements}
\begin{enumerate}[label=UHR-AR \arabic*., wide=0pt, leftmargin=*]
\item \emph{The system must meet the WCAG 2.1 Level AA accessibility standards.}\\[2mm] 
    {\bf Rationale:} The Accessibility for Ontarians with Disabilities Act (AODA) requires organizations to meet WCAG 2.0 Level AA for websites. Therefore, meeting WCAG 2.1 Level AA ensures the system meets AODA standards and is accessible for users with disabilities.  \\
    {\bf Fit Criterion:} The system will be evaluated using an accessibility testing tool such as Pope Tech and Wave Web Aim to ensure WCAG 2.1 criteria is met.\\
    {\bf Priority:} High
\end{enumerate}


\section{Performance Requirements}
\subsection{Speed and Latency Requirements}
\lips
\subsection{Safety-Critical Requirements}
\lips
\subsection{Precision or Accuracy Requirements}
\lips
\subsection{Robustness or Fault-Tolerance Requirements}
\lips
\subsection{Capacity Requirements}
\lips
\subsection{Scalability or Extensibility Requirements}
\lips
\subsection{Longevity Requirements}
\lips

\section{Operational and Environmental Requirements}
\subsection{Expected Physical Environment}
\lips
\subsection{Wider Environment Requirements}
\lips
\subsection{Requirements for Interfacing with Adjacent Systems}
\lips
\subsection{Productization Requirements}
\lips
\subsection{Release Requirements}
\lips

\section{Maintainability and Support Requirements}
\subsection{Maintenance Requirements}
\lips
\subsection{Supportability Requirements}
\lips
\subsection{Adaptability Requirements}
\lips

\section{Security Requirements}
\subsection{Access Requirements}
\lips
\subsection{Integrity Requirements}
\lips
\subsection{Privacy Requirements}
\lips
\subsection{Audit Requirements}
\lips
\subsection{Immunity Requirements}
\lips

\section{Cultural Requirements}
\subsection{Cultural Requirements}
\lips

\section{Compliance Requirements}
\subsection{Legal Requirements}
\lips
\subsection{Standards Compliance Requirements}
\lips

\section{Open Issues}
\lips

\section{Off-the-Shelf Solutions}
\subsection{Ready-Made Products}
\lips
\subsection{Reusable Components}
\lips
\subsection{Products That Can Be Copied}
\lips

\section{New Problems}
\subsection{Effects on the Current Environment}
\lips
\subsection{Effects on the Installed Systems}
\lips
\subsection{Potential User Problems}
\lips
\subsection{Limitations in the Anticipated Implementation Environment That May
Inhibit the New Product}
\lips
\subsection{Follow-Up Problems}
\lips

\section{Tasks}
\subsection{Project Planning}
\lips
\subsection{Planning of the Development Phases}
\lips

\section{Migration to the New Product}
\subsection{Requirements for Migration to the New Product}
\lips
\subsection{Data That Has to be Modified or Translated for the New System}
\lips

\section{Costs}
\lips
\section{User Documentation and Training}
\subsection{User Documentation Requirements}
\lips
\subsection{Training Requirements}
\lips

\section{Waiting Room}
\lips

\section{Ideas for Solution}
\lips

\newpage{}
\section*{Appendix --- Reflection}

\input{../Reflection.tex}

\begin{enumerate}
  \item What went well while writing this deliverable? 
  \item What pain points did you experience during this deliverable, and how did
  you resolve them?
  \item How many of your requirements were inspired by speaking to your
  client(s) or their proxies (e.g. your peers, stakeholders, potential users)?
  \item Which of the courses you have taken, or are currently taking, will help
  your team to be successful with your capstone project.
  \item What knowledge and skills will the team collectively need to acquire to
  successfully complete this capstone project?  Examples of possible knowledge
  to acquire include domain specific knowledge from the domain of your
  application, or software engineering knowledge, mechatronics knowledge or
  computer science knowledge.  Skills may be related to technology, or writing,
  or presentation, or team management, etc.  You should look to identify at
  least one item for each team member.
  \item For each of the knowledge areas and skills identified in the previous
  question, what are at least two approaches to acquiring the knowledge or
  mastering the skill?  Of the identified approaches, which will each team
  member pursue, and why did they make this choice?
\end{enumerate}

\textbf{Casey Francine Bulaclac - Reflection}
\begin{enumerate}
  \item What went well while writing this deliverable? \\[1ex]
  Having discussed the project thoroughly as a team and with our supervisor helped in writing this deliverable as the team
  was very knowledgeable about the needs for the project. 
  This deliverable went much smoother than the last due to stronger operational procedures, and better organization in how we structured and completed the SRS. 
  The team communicated well and were clear of the goals for this deliverable.
  \item What pain points did you experience during this deliverable, and how did
  you resolve them?\\[1ex]
  One pain point in writing the SRS was figuring out what each of the many sections entailed in the Volere's template. The template 
  is very thorough and needed many details, in which some sections seem to overlap which can be confusing. Another pain point was ensuring traceability
  between our goals in the project and the requirements. To resolve this, I made sure to ask the TA for feedback and clarification about specific sections.
  Additionally, communicating with each team member and ensuring our requirements aligned to the goals of the project was very helpful in aiding to ensure
  traceability.
  \item How many of your requirements were inspired by speaking to your
  client(s) or their proxies (e.g. your peers, stakeholders, potential users)? \\[1ex]
  Many, if not most, requirements were inspired through speaking with our supervisor, who had the most knowledge and experience with our project's 
  potential users and stakeholders. In this project, it is important to understand our target users as we are designing for accessibility, so it was critical 
  in making our requirements. 
  \item Which of the courses you have taken, or are currently taking, will help
  your team to be successful with your capstone project.\\[1ex]
  In this deliverable, the course that was most beneficial was Software Requirements and Security Considerations (SFWRENG 3RA3) as we learned how to create effected SRS documents.
  A course I've taken that will help thoroughly in ensuring our user interface is accessible is Human Computer Interfaces (SFWRENG 4HC3) as the course taught us principles of good design. Lastly,
  another course I took that contribute to the success of our project is Applications of Machine Learning (SFWRENG 4AL3) as this project heavily involves machine learning
  in generating alternative text.
  \item What knowledge and skills will the team collectively need to acquire to
  successfully complete this capstone project?  Examples of possible knowledge
  to acquire include domain specific knowledge from the domain of your
  application, or software engineering knowledge, mechatronics knowledge or
  computer science knowledge.  Skills may be related to technology, or writing,
  or presentation, or team management, etc.  You should look to identify at
  least one item for each team member. \\[1ex]
  
  \item For each of the knowledge areas and skills identified in the previous
  question, what are at least two approaches to acquiring the knowledge or
  mastering the skill?  Of the identified approaches, which will each team
  member pursue, and why did they make this choice?
\end{enumerate}

\end{document}