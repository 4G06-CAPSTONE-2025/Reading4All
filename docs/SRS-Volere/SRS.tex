% THIS DOCUMENT IS FOLLOWS THE VOLERE TEMPLATE BY Suzanne Robertson
% and James Robertson
% ONLY THE SECTION HEADINGS ARE PROVIDED
%
% Initial draft from https://github.com/Dieblich/volere
%
% Risks are removed because they are covered by the Hazard Analysis
\documentclass[12pt]{article}

\usepackage{booktabs}
\usepackage{tabularx}
\usepackage{hyperref}
\hypersetup{
  bookmarks=true,         % show bookmarks bar?
  colorlinks=true,      % false: boxed links; true: colored links
  linkcolor=red,          % color of internal links (change box color
  % with linkbordercolor)
  citecolor=green,        % color of links to bibliography
  filecolor=magenta,      % color of file links
  urlcolor=cyan           % color of external links
}

\newcommand{\lips}{\textit{Insert your content here.}}

%% Comments

\usepackage{color}

\newif\ifcomments\commentstrue %displays comments
%\newif\ifcomments\commentsfalse %so that comments do not display

\ifcomments
\newcommand{\authornote}[3]{\textcolor{#1}{[#3 ---#2]}}
\newcommand{\todo}[1]{\textcolor{red}{[TODO: #1]}}
\else
\newcommand{\authornote}[3]{}
\newcommand{\todo}[1]{}
\fi

\newcommand{\wss}[1]{\authornote{magenta}{SS}{#1}} 
\newcommand{\plt}[1]{\authornote{cyan}{TPLT}{#1}} %For explanation of the template
\newcommand{\an}[1]{\authornote{cyan}{Author}{#1}}

%% Common Parts

\newcommand{\progname}{ProgName} % PUT YOUR PROGRAM NAME HERE
\newcommand{\authname}{Team \#, Team Name
\\ Student 1 name
\\ Student 2 name
\\ Student 3 name
\\ Student 4 name} % AUTHOR NAMES                  

\usepackage{hyperref}
    \hypersetup{colorlinks=true, linkcolor=blue, citecolor=blue, filecolor=blue,
                urlcolor=blue, unicode=false}
    \urlstyle{same}
                                


\begin{document}

\title{Software Requirements Specification for \progname: subtitle
describing software}
\author{\authname}
\date{\today}

\maketitle

~\newpage

\pagenumbering{roman}

\tableofcontents

~\newpage

\section*{Revision History}

\begin{tabularx}{\textwidth}{p{3cm}p{2cm}X}
  \toprule {\textbf{Date}} & {\textbf{Version}} & {\textbf{Notes}}\\
  \midrule
  Date 1 & 1.0 & Notes\\
  Date 2 & 1.1 & Notes\\
  \bottomrule
\end{tabularx}

~\\

~\newpage
\section{Purpose of the Project}
\subsection{User Business}
\lips
\subsection{Goals of the Project}
\lips
\section{Stakeholders}
The project stakeholders consist of people who have a need or
interest, whether direct or indirect, for alternative text generation
for visual, idle
content (such as images or diagrams). These stakeholders will
influence and be affected by the
project's development decisions and progress. To meet user
needs, it is vital to understand the stakeholder roles and expectations.
\par First, this section introduces the client, customer and other
stakeholders involved in this project. Then, the product users are
described, specifically the hands-on users of the project. Finally,
personas, priority levels and anticipated
participation levels are listed for each stakeholder.
\subsection{Client}
This project's client is Ms. Jingchuan Sui who works as a Media Lab
Specialist Supervisor at the Faculty of Engineering, McMaster
University. As this project's supervisor, her main role is to provide
guidance and voice any concerns during the development phase with her
technical and domain expertise. She will be the main source for
setting requirements while also being directly involved in the
development of this project, providing feedback and opinions on the
Human-Computer interface components.
\subsection{Customer}
The customers of this product are McMaster users, specifically,
McMaster students, staff and teaching instructors who are directly
involved in learning from course content or making them accessible as per the
Accessibility for Ontarians with Disabilities Act (AODA). In other
words, McMaster University stakeholders who benefit directly from
accessible course content. For
example, primary customers can include students who use a screen reader for
learning purposes or
a teaching assistant who is making a course's content AODA compliant.
Furthermore, under her
position at McMaster University, Ms. Sui can also be considered a
customer, as she aids in course content remediation and thus, is also
one of the intended end-users.
\par For development, Group 22 is tailoring the solution to the
McMaster demographic in line with Ms. Sui’s requirements. However,
the product has the potential to support any users who require
alternative text generation for visual content. Feedback from
McMaster stakeholders will be prioritized to maintain a clear and
manageable scope.
\subsection{Other Stakeholders}
This subsection discusses other groups that are indirectly impacted
or who contribute to the ecosystem of accessibility, content
creation, and AODA compliance.
\subsubsection {Faculty of Engineering Instructors}
This group consists of professors and lecturers responsible for
creating and maintaining
course content. They may benefit from automated alternative text generation
to ensure their teaching materials are accessible.
\subsubsection {Teaching Assistants (TAs)}
As part of their work, TAs are often responsible for preparing,
modifying, and uploading course
content. They are stakeholders as they could use the system to
simplify accessibility compliance.
\subsubsection {Accessibility Services Office at McMaster}
This group includes staff members who oversee accessibility
compliance and provide
accommodations for students. They have a strong interest in ensuring
tools meet AODA standards.
\subsubsection {McMaster IT Services / Media Production Services}
These teams may be involved in system integration, technical support,
and maintenance of the product within the university’s digital infrastructure.
\subsubsection {Students with Accessibility Needs}
These students are those who may not be primary testers but are
indirectly impacted
by improved accessibility of course materials.
\subsection{Hands-On Users of the Project}
\subsubsection{Students with Accessibility Needs}
In some cases, students who use screen readers may provide feedback
loops to improve generated alternative text. While they are customers, they
may also be “hands-on” users if they test or adjust alt text themselves.
\subsubsection{Teaching Staff}
This group consists of TAs and instructors. As mentioned above, TAs
Frequently upload, adapt, and remediate course content. They would be
interacting directly with the tool to generate and refine alt text.
On the other hand, some instructors (especially those who prepare
their own slides, diagrams, or assignments) would use the system to
add or edit alternative text.
\subsection{Personas}
\subsubsection*{Persona: Alice Bayes}
\textbf{Age:} 27 \\
\textbf{Job Title:} Teaching Assistant at McMaster University\\
\textbf{Education:} Bachelor's in History\\
\textbf{Work Environment:} Alice works under several teaching
instructors to help deliver course content to students. She is in
charge of marking assignments and has recently been tasked with
auditing then remediating any inaccessible learning content. \\
\textbf{Professional Background:} Alice graduated two years ago, and
as part of her undergraduate career, she has experience in working
with students with disabilities. She is trained on making content
accessible and AODA compliant.\\[2mm]
\textbf{Need:} With so many courses to grade student work for, Alice
needs a tool that can easily and quickly generate content for her to
use as alternative text while she can ensure that her boss' teaching
content meets AODA compliance.\\
\textbf{Challenges:} Balancing her work and life has been difficult
as there are multiple images per document, and several documents per
course. She is overwhelmed with the amount of grading she has to do
on top of manually writing alternative text for over 50 images.

\subsubsection*{Persona: Chetan Dakshesh}
\textbf{Age:} 20 \\
\textbf{Job Title:} Chetan is a student at McMaster University.\\
\textbf{Education:} He is currently pursuing a Bachelor's in
Electrical Engineering\\
\textbf{Work Environment:} Chetan has a super busy course load with
six courses and volleyball club!\\
\textbf{Professional Background:} \\[2mm]
\textbf{Need:} With so many courses and volleyball practice to keep
up with, Chetan is finding it hard to keep track of course content.
Furthermore, through his screen reader, he has picked up that there
is no alternative text generated for several diagrams in a course he
is taking. These diagrams are vital to his learning experience but he
has little clue on what they indicate. \\
\textbf{Challenges:} Using large language models (LLMs) such as
Chat-GPT doesn't work for him as the text generated is too generic
and lacks substance. Chetan needs a tool that can effectively
describe the diagram to him while staying relevant to the course material.

\subsubsection*{Persona: Eyad Fahim}
\textbf{Age:} 40 \\
\textbf{Job Title:} Professor at McMaster University\\
\textbf{Education:} Doctor of Philosophy (PhD) in Engineering\\
\textbf{Work Environment:} Eyad works on a fast paced work
environment, connecting with over 100 students.
\textbf{Professional Background:} With over 20 years of experience both in
the workforce and academic, Eyad loves to teach the next generation
of leaders about various engineering techniques.\\[2mm]
\textbf{Need:} With the goal to
celebrate students of experiences, Eyad is looking for help to make
his teaching content accessible for all.\\
\textbf{Challenges:} Eyad needs a fast tool that can help gap the
accessibility knowledge he lacks. He wants to ensure all students can
learn from his materials with little to no barriers, including
alternative text but he has no idea how to get started.
\subsection{Priorities Assigned to Users}
\textbf{Primary users:}
\begin{itemize}
  \item Students with accessibility needs
  \item Ms. Jingchuan Sui
  \item Teaching Staff
\end{itemize}
\textbf{Secondary users:}
\begin{itemize}
  \item Accessibility Services Office at McMaster
  \item McMaster IT/Media Production Services
\end{itemize}
\subsection{User Participation}
During the development process, the requirements will be gathered
mainly from Ms. Sui. During testing phase, Group 22 will conduct
usability testing to ensure AODA compliance and to further refine the product.
\subsection{Maintenance Users and Service Technicians}
For this project, maintenance activities may involve updating
alternative text generation models, fixing bugs, or upgrading dependencies.
\subsubsection*{Expected Maintenance Users and Roles}
\begin{itemize}
  \item \textbf{McMaster IT Services / Media Production Services} \\
    These teams may oversee deployment, integration with
    institutional systems, and technical support. They require access
    to configuration tools, diagnostic information, and documentation
    for updates or troubleshooting.
  \item \textbf{Accessibility Services Office Staff} \\
    Although initially secondary stakeholders, some staff may
    contribute to iterative refinement of alt-text generation
    accuracy or compliance updates. Their participation may prompt
    system adjustments or patches.
  \item \textbf{Development Team (Group 22) or Future Maintainership Team} \\
    During initial deployment and handover, the development team or a
    designated successor group may perform updates to improve
    usability, resolve technical issues, or adapt to new
    accessibility standards.
\end{itemize}

\section{Mandated Constraints}
\subsection{Solution Constraints}
\lips
\subsection{Implementation Environment of the Current System}
\lips
\subsection{Partner or Collaborative Applications}
\lips
\subsection{Off-the-Shelf Software}
\lips
\subsection{Anticipated Workplace Environment}
\lips
\subsection{Schedule Constraints}
\lips
\subsection{Budget Constraints}
\lips
\subsection{Enterprise Constraints}
\lips

\section{Naming Conventions and Terminology}
\subsection{Glossary of All Terms, Including Acronyms, Used by Stakeholders
involved in the Project}
\lips

\section{Relevant Facts And Assumptions}
\subsection{Relevant Facts}
\lips
\subsection{Business Rules}
\lips
\subsection{Assumptions}
\lips

\section{The Scope of the Work}
\subsection{The Current Situation}
\lips
\subsection{The Context of the Work}
\lips
\subsection{Work Partitioning}
\lips
\subsection{Specifying a Business Use Case (BUC)}
\lips

\section{Business Data Model and Data Dictionary}
\subsection{Business Data Model}
\lips
\subsection{Data Dictionary}
\lips

\section{The Scope of the Product}
\subsection{Product Boundary}
\lips
\subsection{Product Use Case Table}
\lips
\subsection{Individual Product Use Cases (PUC's)}
\lips

\section{Functional Requirements}
\subsection{Functional Requirements}
\lips

\section{Look and Feel Requirements}
\subsection{Appearance Requirements}
\lips
\subsection{Style Requirements}
\lips

\section{Usability and Humanity Requirements}
\subsection{Ease of Use Requirements}
\lips
\subsection{Personalization and Internationalization Requirements}
\lips
\subsection{Learning Requirements}
\lips
\subsection{Understandability and Politeness Requirements}
\lips
\subsection{Accessibility Requirements}
\lips

\section{Performance Requirements}
\subsection{Speed and Latency Requirements}
\lips
\subsection{Safety-Critical Requirements}
\lips
\subsection{Precision or Accuracy Requirements}
\lips
\subsection{Robustness or Fault-Tolerance Requirements}
\lips
\subsection{Capacity Requirements}
\lips
\subsection{Scalability or Extensibility Requirements}
\lips
\subsection{Longevity Requirements}
\lips

\section{Operational and Environmental Requirements}
\subsection{Expected Physical Environment}
\lips
\subsection{Wider Environment Requirements}
\lips
\subsection{Requirements for Interfacing with Adjacent Systems}
\lips
\subsection{Productization Requirements}
\lips
\subsection{Release Requirements}
\lips

\section{Maintainability and Support Requirements}
\subsection{Maintenance Requirements}
\lips
\subsection{Supportability Requirements}
\lips
\subsection{Adaptability Requirements}
\lips

\section{Security Requirements}
\subsection{Access Requirements}
\lips
\subsection{Integrity Requirements}
\lips
\subsection{Privacy Requirements}
\lips
\subsection{Audit Requirements}
\lips
\subsection{Immunity Requirements}
\lips

\section{Cultural Requirements}
The following list conists of cultural requirements the system shall follow:
\begin{itemize}
  \item[\textbf{CR 1.}] \textit{The system shall generate alternative
      text using neutral and inclusive language appropriate for
    academic environments.}\\
    \textbf{Rationale:} Ensures that generated content is respectful
    to diverse cultural and educational backgrounds.\\
    \textbf{Fit Criterion:} Generated alt text contains no culturally
    biased, exclusionary, or inappropriate terminology.\\
    \textbf{Priority:} High

  \item[\textbf{CR 2.}] \textit{The system shall avoid using
      culturally specific references unless the visual content
    explicitly requires it.}\\
    \textbf{Rationale:} Prevents misinterpretation and maintains
    accessibility for a wide audience.\\
    \textbf{Fit Criterion:} Alt text focuses on visual description
    and context without unnecessary cultural assumptions.\\
    \textbf{Priority:} Medium

  \item[\textbf{CR 3.}] \textit{The system shall use professional and
    educationally appropriate tone in all generated content.}\\
    \textbf{Rationale:} Maintains usability across academic
    departments and contexts.\\
    \textbf{Fit Criterion:} Outputs remain formal, non-colloquial,
    and context-relevant.\\
    \textbf{Priority:} Medium
\end{itemize}

\section{Compliance Requirements}
\subsection{Legal Requirements}
\begin{itemize}
  \item[\textbf{CR-LR 1.}] \textit{The system shall comply with AODA
    standards for alternative text generation.}\\
    \textbf{Rationale:} Ensures the tool supports institutional
    accessibility requirements and legal obligations.\\
    \textbf{Fit Criterion:} All generated alt text meets WCAG 2.1
    Level AA criteria for accuracy, clarity, and relevance.\\
    \textbf{Priority:} High
\end{itemize}
\subsection{Standards Compliance Requirements}
\begin{itemize}
  \item[\textbf{CR-SCR 1.}] \textit{The system shall follow
      institutional privacy and data-handling guidelines for uploaded
    teaching materials.}\\
    \textbf{Rationale:} Prevents unauthorized distribution or
    mishandling of academic content.\\
    \textbf{Fit Criterion:} No files are stored beyond active use
    unless explicitly authorized; logs exclude proprietary content.\\
    \textbf{Priority:} High

  \item[\textbf{CR-SCR 2.}] \textit{The system shall provide verifiable
    documentation or statements of compliance upon request.}\\
    \textbf{Rationale:} Facilitates audits, approvals, and
    integration into university workflows.\\
    \textbf{Fit Criterion:} A compliance overview document or help
    section is available to stakeholders.\\
    \textbf{Priority:} Medium
\end{itemize}

\section{Open Issues}
\lips

\section{Off-the-Shelf Solutions}
\subsection{Ready-Made Products}
\lips
\subsection{Reusable Components}
\lips
\subsection{Products That Can Be Copied}
\lips

\section{New Problems}
\subsection{Effects on the Current Environment}
\lips
\subsection{Effects on the Installed Systems}
\lips
\subsection{Potential User Problems}
\lips
\subsection{Limitations in the Anticipated Implementation Environment That May
Inhibit the New Product}
\lips
\subsection{Follow-Up Problems}
\lips

\section{Tasks}
\subsection{Project Planning}
\lips
\subsection{Planning of the Development Phases}
\lips

\section{Migration to the New Product}
\subsection{Requirements for Migration to the New Product}
\begin{itemize}
  \item[\textbf{MNP-RMNP 1.}] \textit{The system shall support a
      phased implementation to allow gradual adoption while minimizing
    disruptions.}\\
    \textbf{Rationale:} Reduces organizational risk and allows
    controlled testing during rollout.\\
    \textbf{Fit Criterion:} Each phase is deployed and validated
    independently before progressing to the next.\\
    \textbf{Priority:} High

  \item[\textbf{MNP-RMNP 2.}] \textit{The organization shall operate
      the new system in parallel with the old product for a defined
    transition period.}\\
    \textbf{Rationale:} Ensures continuity and confirms correct
    operation before full cutover.\\
    \textbf{Fit Criterion:} Parallel operation lasts until all
    critical functions pass acceptance testing.\\
    \textbf{Priority:} High

  \item[\textbf{MNP-RMNP 3.}] \textit{The system shall provide
    procedures and tools for manual backup during transition.}\\
    \textbf{Rationale:} Maintains operational stability during migration.\\
    \textbf{Fit Criterion:} Backup processes are documented, tested,
    and accessible to staff.\\
    \textbf{Priority:} Medium

  \item[\textbf{MNP-RMNP 4.}] \textit{The transition plan shall
    identify and schedule major components and release phases.}\\
    \textbf{Rationale:} Guides project planning and resource allocation.\\
    \textbf{Fit Criterion:} A migration timeline with milestones and
    dependencies is documented.\\
    \textbf{Priority:} Medium
\end{itemize}

\subsection{Data That Has to be Modified or Translated for the New System}
This section does not apply to this project as there is no current system to replace, thus, no data at all.

\section{Costs}
\lips
\section{User Documentation and Training}
\subsection{User Documentation Requirements}
\lips
\subsection{Training Requirements}
\lips

\section{Waiting Room}
\lips

\section{Ideas for Solution}
\lips

\newpage{}
\section*{Appendix --- Reflection}

The purpose of reflection questions is to give you a chance to assess your own
learning and that of your group as a whole, and to find ways to improve in the
future. Reflection is an important part of the learning process.  Reflection is
also an essential component of a successful software development process.  

Reflections are most interesting and useful when they're honest, even if the
stories they tell are imperfect. You will be marked based on your depth of
thought and analysis, and not based on the content of the reflections
themselves. Thus, for full marks we encourage you to answer openly and honestly
and to avoid simply writing ``what you think the evaluator wants to hear.''

Please answer the following questions.  Some questions can be answered on the
team level, but where appropriate, each team member should write their own
response:

\textbf{Group Reflection  - Reflection}
\begin{enumerate}
  \item What went well while writing this deliverable? 
  
  Throughout the process, our team maintained organization and had good communication.  It was simpler to preserve flow and prevent redundancy because we had a clear framework and consistent formatting from earlier deliverables.  We were also able to maintain our efficiency and keep the material consistent with our previous work.  In order review sections, define expectations, and make sure the deliverable satisfied all requirements, we also had meetings with our TA and our team.
  \item What pain points did you experience during this deliverable, and how did
  you resolve them?

  One difficulty was avoiding redundancy by keeping formatting constant and clearly distinguishing design components from implementation specifics. In order to guarantee accuracy and uniformity throughout all sections, we addressed this by going over the course templates and conducting team editing sessions to make sure the formatting followed a consistent style. We also helped each other out by sharing tips and tricks to improve clarity and presentation.

  \item How many of your requirements were inspired by speaking to your
  client(s) or their proxies (e.g. your peers, stakeholders, potential users)?
  
  Key design and functionality components that were influenced by internal meetings, conversations with our supervisor, pertinent compliance guidelines, and my internship's prior experience working with AI included the accessible interface, the quality of the alt text that was generated, and adherence to AODA and WCAG 2.1 standards.
  \item Which of the courses you have taken, or are currently taking, will help
  your team to be successful with your capstone project.

  The Software Requirements course from third year was extremely useful in creating this deliverable.  It created a solid foundation for organizing requirement documents, developing precise specifications, and grasping the foundations of software documentation and traceability, all of which significantly aided our work on this project.

  \item What knowledge and skills will the team collectively need to acquire to
  successfully complete this capstone project?  Examples of possible knowledge
  to acquire include domain specific knowledge from the domain of your
  application, or software engineering knowledge, mechatronics knowledge or
  computer science knowledge.  Skills may be related to technology, or writing,
  or presentation, or team management, etc.  You should look to identify at
  least one item for each team member.}\newline
  
  \item \textbf{For each of the knowledge areas and skills identified in the previous
  question, what are at least two approaches to acquiring the knowledge or
  mastering the skill?  Of the identified approaches, which will each team
  member pursue, and why did they make this choice?}\newline

\end{enumerate}


\textbf{Moly Mikhail  - Reflection}
\begin{enumerate}
  \item \textbf{What went well while writing this deliverable?} \newline
  I believe writing this deliverable many things went well. I really enjoyed getting to think about the different
  non-functional requirements. I found that have different sections of non-functional requirements encouraged me to think about different 
  aspects of the system and things we will have to keep in mind during development. For example, prior to writing 
  this deliverable, we hadn’t considered personalization and internationalization requirements; however, having to
  complete that section led us to add important functionality of allowing the user to decide which way
  to store the alternative text. 
  
\item \textbf{What pain points did you experience during this
    deliverable, and how did you resolve them?} \newline
  One pain point I experience writing this deliverable dealt with completing the product boundary. 
  Initially, I was confused on The Scope of the Product section and what was expected. 
  To resolve this, I researched the Volere Requirements Specification Template and looked into the section. 
  However, I was still confused and what was expected of the section. Finally, during our meeting with
  our TA I was able to clarify the expectations for this section and I was able to complete the section. 
 
   \item \textbf{How many of your requirements were inspired by
      speaking to your client(s) or their proxies (e.g., your peers,
    stakeholders, potential users)?} \newline
  I believe many of our non-functional requirements, specifically look and feel 
  requirements, as well as usability and humanity requirements. Through our
  conversations with our supervisor Jing, we learned a lot of the accessibility
  requirements for website applications. For example, one specific requirement 
  that was derived from our conversations was that the system cannot use color alone to convey any messages
  or information. I believe without having this conversation, this is a requirement that would not have been discovered. 
 
 \item \textbf{Which of the courses you have taken, or are currently
      taking, will help your team be successful with your capstone
    project?} \newline
  I believe many courses that I have taken, and some that I’m currently taking will 
  contribute to the success of our capstone project. I completed
  SFWRENG 4HC3 - Human Computer Interfaces, which has taught me many 
  important design principles, such as Normans Design Principles. Furthermore, 
  completing COMPSCI 4AL3 - Applications of Machine Learning, also will be a lot of help when 
  completing our capstone. This course introduced me to developing machine learning models and 
  will be directly applicable. Finally, taking COMPSCI 3RA3 - Software Requirements and Security 
  Considerations will also help our team be successful. 
  
\end{enumerate}

\textbf{Casey Francine Bulaclac - Reflection}
\begin{enumerate}
  \item \textbf{What went well while writing this deliverable?} \newline
  Having discussed the project thoroughly as a team and with our supervisor helped in writing this deliverable as the team
  was very knowledgeable about the needs for the project. 
  This deliverable went much smoother than the last due to stronger operational procedures, and better organization in how we structured and completed the SRS. 
  The team communicated well and were clear of the goals for this deliverable.
  \item \textbf{What pain points did you experience during this
  deliverable, and how did you resolve them?} \newline
  One pain point in writing the SRS was figuring out what each of the many sections entailed in the Volere's template. The template 
  is very thorough and needed many details, in which some sections seem to overlap which can be confusing. Another pain point was ensuring traceability
  between our goals in the project and the requirements. To resolve this, I made sure to ask the TA for feedback and clarification about specific sections.
  Additionally, communicating with each team member and ensuring our requirements aligned to the goals of the project was very helpful in aiding to ensure
  traceability.
\item \textbf{How many of your requirements were inspired by
      speaking to your client(s) or their proxies (e.g., your peers,
    stakeholders, potential users)?} \newline
  Many, if not most, requirements were inspired through speaking with our supervisor, who had the most knowledge and experience with our project's 
  potential users and stakeholders. In this project, it is important to understand our target users as we are designing for accessibility, so it was critical 
  in making our requirements. 
 \item \textbf{Which of the courses you have taken, or are currently
      taking, will help your team be successful with your capstone
    project?} \newline
  In this deliverable, the course that was most beneficial was Software Requirements and Security Considerations (SFWRENG 3RA3) as we learned how to create effected SRS documents.
  A course I've taken that will help thoroughly in ensuring our user interface is accessible is Human Computer Interfaces (SFWRENG 4HC3) as the course taught us principles of good design. Lastly,
  another course I took that contribute to the success of our project is Applications of Machine Learning (SFWRENG 4AL3) as this project heavily involves machine learning
  in generating alternative text. 
\end{enumerate}

\end{document}
