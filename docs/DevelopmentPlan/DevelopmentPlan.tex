\documentclass{article}

\usepackage{booktabs}
\usepackage{tabularx}

\title{Development Plan\\\progname}

\author{\authname}

\date{}

\input{../Comments}
%% Common Parts

\newcommand{\progname}{ProgName} % PUT YOUR PROGRAM NAME HERE
\newcommand{\authname}{Team 22, READING4ALL
\\ Fiza Sehar
\\ Nawaal Fatima
\\ Dhruv Sardana
\\ Moly Mikhail
\\ Casey Francine Bulaclac } % AUTHOR NAMES                  

\usepackage{hyperref}
    \hypersetup{colorlinks=true, linkcolor=blue, citecolor=blue, filecolor=blue,
                urlcolor=blue, unicode=false}
    \urlstyle{same}
                                


\begin{document}

\maketitle

\begin{table}[hp]
\caption{Revision History} \label{TblRevisionHistory}
\begin{tabularx}{\textwidth}{llX}
\toprule
\textbf{Date} & \textbf{Developer(s)} & \textbf{Change}\\
\midrule
Date1 & Name(s) & Description of changes\\
Date2 & Name(s) & Description of changes\\
... & ... & ...\\
\bottomrule
\end{tabularx}
\end{table}

\newpage{}

\wss{Put your introductory blurb here.  Often the blurb is a brief roadmap of
what is contained in the report.}

\wss{Additional information on the development plan can be found in the
\href{https://gitlab.cas.mcmaster.ca/courses/capstone/-/blob/main/Lectures/L02b_POCAndDevPlan/POCAndDevPlan.pdf?ref_type=heads}
{lecture slides}.}

\section{Confidential Information}

This project doesn't contain any confidential information.


\section{IP to Protect}
This project doesn't contain any IP to protect. 
\section{Copyright License}

Our team is adopting the MIT License.

\section{Team Meeting Plan}

The team will meet weekly on Tuesdays from 3:00pm to 4:00pm virtually on Discord or in person on campus if needed. 
The team will meet with the industry advisor biweekly on Thursdays from 2:30pm to 3:30pm. These meetings with the industry advisor will 
be conducted either online on Microsoft Teams or in person on campus. \\[1ex]
The meetings will be structured as follows: 
\begin{enumerate}
	\item An agenda prepared by the meeting chair (who rotates among team members each week) 
  will be made to use as a guide for the meeting.
	\item The team will go over any announcements or completed To-Dos from the previous week's 
  meeting if needed.
  \item Each member will present what they have worked on so far and ask the remaining group 
  members for feedback or any questions if needed.
  \item The team will discuss and document any decisions needed about the deliverables 
  or the project. 
  \item Any concerns/questions will be documented for the next team meeting or for the next 
  industry advisor meeting.
\end{enumerate}

\section{Team Communication Plan}

\begin{itemize}
    \item \textbf{Discord}: Our main method of communication between group members. It will be used to discuss detailed deliverable and any code questions. Additionally, group meetings will be hosted on discord. 
    \item \textbf{Instagram}: Our secondary method of communication between group members. It will be used to discuss less technical details and for any urgent messages that require a quicker response. 
    \item \textbf{Teams}: Our main method of communication with our Supervisor. We will utilize our group chat with our supervisor for any quick questions or updates. Online meetings with our supervisor will be hosted on teams.
    \item \textbf{GitHub}: The issues feature will be utilized to communicate any bugs observed and meeting attendance. Additionally, as a week to see what other team members are working on.
\end{itemize}


\wss{Issues on GitHub should be part of your communication plan.}

\section{Team Member Roles}

The team will work collaboratively to develop and refine this project. To ensure a clear 
division of tasks, each team member have been assigned roles that 
align with their areas of expertise and contribute to achieving the goals 
of this project. These roles will rotate throughout the year to prevent overspecialization
and to ensure that all members can gain experience and knowledge in 
every aspect of the project. \\[1ex]
The defined roles and responsibilities per team member is as follows: 
\begin{itemize}
	\item \textbf{Fiza Sehar}: \textit{Developer, Documentation, Model Training Specialist} \\
	Fiza will be responsible for developing features and maintaining documentation for the 
  project. She will also be leading the model training for the project to ensure 
  efficient and accurate performance.
	\item \textbf{Dhruv Sardana}: \textit{Developer, Documentation, Full-Stack Specialist} \\
	Dhruv will work across both frontend and backend development, and in ensuring a 
  seamless integration and functionality. He will also support in writing documentation
  for this project. 
	\item \textbf{Nawaal Fatima}: \textit{Developer, Documentation, Data Specialist} \\
	Nawaal will also work on developing features with a 
  focus on data management, pre-processing, and analysis in this project. She 
  will also contribute in the documentation of the project.
	\item \textbf{Moly Mikhail}: \textit{Developer, Documentation, Backend Specialist} \\
	Moly will be handling the APIs, database management, and system logic focusing on 
  the backend of the project. She will also support in the documentation of the project.
	\item \textbf{Casey Francine Bulaclac}: \textit{Developer, Documentation, Frontend Specialist} \\
	Francine will be responsible for the design and implementation of the user interface, 
  ensuring correct usability and accessibility while assisting in the project 
  documentation.
\end{itemize}

\section{Workflow Plan}

\begin{itemize}
	\item\textbf{How will you be using git, including branches, pull request, etc.?}
	 We will be using git, branches and pull requests in order to divide work between group members and complete tasks concurrently. 
   Furthermore, we will follow a feature-branch based approach, our process will follow these steps:
   \begin{itemize}
    \item \textbf{Branch Creation} Group member will create a branch from the develop branch for each feature they work on and follow this naming convention: \newline
    \textit{feature\_[contributorName]/D\#\_[featureDescription]} \newline
    Similarly, for bugs the following naming convention will be followed: \newline
   \textit{bug\_[contributorName]/D\#\_[bugDescription]} \newline
    Beyond feature branches, we will also have \textit{develop} and \textit{main} branches. The develop branch will be used to integrate different features and once
    tested, as well as reviewed by the majority of group members can be merged into main. 
    \end{itemize}
    \begin{itemize}
      \item \textbf{Pull Request} Once a group member is done working on their feature, they will open a pull request to merge into the develop branch. Group members will use the Github comment feature to connect their feature to any applicable issue numbers.
    \end{itemize}
	\item\textbf{Continuous Integration\ Continuous Development}
	
  For continuous integration (CI), we will be using Github actions to automatically trigger a project build and run unit tests 
  whenever a commit occurs or pull requests are opened. This ensures errors are discovered as early as possible and that the project remains in a working state during development. 

  \item \textbf{Managing issues}\newline
  We will utilize the capTemplate template issues for tracking attendance, peer review, supervisor, TA, and team meetings. Furthermore, issues that fall outside these categories will utilize the labels mentioned below to ensure they are easily understood and managed. 
  \item \textbf{Use of Labels:} \newline
  The team will utilize issue tags to help us organize ongoing work as well as work that needs to be completed. The tags will enable us to see the status of ongoing work and their respective priority. 
    The issue tags we will have available are:
     \begin{itemize} 
        \item Bug, feature, question, in-progress, review needed, changes needed, ready for merge, done, high priority, low priority, meeting, frontend, backend, feedback 
    \end{itemize}
    \item \textbf{Use of Checklists:} \newline
  The group will utilize the deliverable checklists to ensure our work meets the expectations. 
  We will also have checklists in the Pull Request templates to ensure that all required items are completed prior to merging. The checklist will include the following items: 
  \begin{itemize}
    \item Unit Tests Passed
    \item Coding Standard is followed 
    \item Code Compiles without errors
    \item Reviewed by 2 team members
  \end{itemize}
  \end{itemize}

\section{Project Decomposition and Scheduling}

\textbf{Project Schedule}:
\begin{center}
  \begin{tabularx}{\textwidth}{Xc}
    \toprule
    \textbf{Deliverables} & \textbf{Due Date} \\
    \midrule
    Problem Statement, Proof of Concept, and Development Plan & Week  04 \\
    Software Requirements Specificaiton and Hazards Analysis (Revision 0) & Week 06 \\
    Verification \& Validation Plan (Revision 0) & Week 08 \\
    Design Document (Rev-1) & Week 10 \\
    Proof of Concept Demonstration & Week 11 + 12 \\
    Design Document (Revision 0) & Week 16 \\
    Project Demonstration (Revision 0) & Week 18 + 19 \\
    Verification \& Validation Report (Revision 0) & Week 22 \\
    Final Demonstration (Revision 1) & Week 24 \\
    Final Documentation & Week 26 \\
    Capstone EXPO & Week 26 \\
    \bottomrule
  \end{tabularx}
\end{center}

\begin{itemize}
  \item How will you be using GitHub projects?
  \item Include a link to your GitHub project
\end{itemize}

\wss{How will the project be scheduled?  This is the big picture schedule, not
details. You will need to reproduce information that is in the course outline
for deadlines.}

\section{Proof of Concept Demonstration Plan}

What is the main risk, or risks, for the success of your project?  What will you
demonstrate during your proof of concept demonstration to convince yourself that
you will be able to overcome this risk?

\section{Expected Technology}

\wss{What programming language or languages do you expect to use?  What external
libraries?  What frameworks?  What technologies.  Are there major components of
the implementation that you expect you will implement, despite the existence of
libraries that provide the required functionality.  For projects with machine
learning, will you use pre-trained models, or be training your own model?  }

\wss{The implementation decisions can, and likely will, change over the course
of the project.  The initial documentation should be written in an abstract way;
it should be agnostic of the implementation choices, unless the implementation
choices are project constraints.  However, recording our initial thoughts on
implementation helps understand the challenge level and feasibility of a
project.  It may also help with early identification of areas where project
members will need to augment their training.}

Topics to discuss include the following:

\begin{itemize}
\item Specific programming language
\item Specific libraries
\item Pre-trained models
\item Specific linter tool (if appropriate)
\item Specific unit testing framework
\item Investigation of code coverage measuring tools
\item Specific plans for Continuous Integration (CI), or an explanation that CI
  is not being done
\item Specific performance measuring tools (like Valgrind), if
  appropriate
\item Tools you will likely be using?
\end{itemize}

\wss{git, GitHub and GitHub projects should be part of your technology.}

\section{Coding Standard}

\wss{What coding standard will you adopt?}

\newpage{}

\section*{Appendix --- Reflection}

\wss{Not required for CAS 741}

\input{../Reflection.tex}

\begin{enumerate}
    \item Why is it important to create a development plan prior to starting the
    project?
    \item In your opinion, what are the advantages and disadvantages of using
    CI/CD?
    \item What disagreements did your group have in this deliverable, if any,
    and how did you resolve them?
\end{enumerate}

\newpage{}

\section*{Appendix --- Team Charter}

\wss{borrows from
\href{https://engineering.up.edu/industry_partnerships/files/team-charter.pdf}
{University of Portland Team Charter}}

\subsection*{External Goals}

\wss{What are your team's external goals for this project? These are not the
goals related to the functionality or quality fo the project.  These are the
goals on what the team wishes to achieve with the project.  Potential goals are
to win a prize at the Capstone EXPO, or to have something to talk about in
interviews, or to get an A+, etc.}

\subsection*{Attendance}

\subsubsection*{Expectations}

\wss{What are your team's expectations regarding meeting attendance (being on
time, leaving early, missing meetings, etc.)?}

\subsubsection*{Acceptable Excuse}

\wss{What constitutes an acceptable excuse for missing a meeting or a deadline?
What types of excuses will not be considered acceptable?}

\subsubsection*{In Case of Emergency}

\wss{What process will team members follow if they have an emergency and cannot
attend a team meeting or complete their individual work promised for a team
deliverable?}

\subsection*{Accountability and Teamwork}

\subsubsection*{Quality} 

\wss{What are your team's expectations regarding the quality
of team members' preparation for team meetings and the quality of the
deliverables that members bring to the team?}

\subsubsection*{Attitude}

\wss{What are your team's expectations regarding team members' ideas,
interactions with the team, cooperation, attitudes, and anything else regarding
team member contributions?  Do you want to introduce a code of conduct?  Do you
want a conflict resolution plan?  Can adopt existing codes of conduct.}

\subsubsection*{Stay on Track}

\wss{What methods will be used to keep the team on track? How will your team
ensure that members contribute as expected to the team and that the team
performs as expected? How will your team reward members who do well and manage
members whose performance is below expectations?  What are the consequences for
someone not contributing their fair share?}

\wss{You may wish to use the project management metrics collected for the TA and
instructor for this.}

\wss{You can set target metrics for attendance, commits, etc.  What are the
consequences if someone doesn't hit their targets?  Do they need to bring the
coffee to the next team meeting?  Does the team need to make an appointment with
their TA, or the instructor?  Are there incentives for reaching targets early?}

\subsubsection*{Team Building}

\wss{How will you build team cohesion (fun time, group rituals, etc.)? }

\subsubsection*{Decision Making} 

\wss{How will you make decisions in your group? Consensus?  Vote? How will you
handle disagreements? }

\end{document}