\documentclass{article}

\usepackage{booktabs}
\usepackage{tabularx}
\usepackage[hidelinks]{hyperref} 
\usepackage{placeins}
\usepackage{float}

\title{Development Plan\\\progname}

\author{\authname}

\date{}

%% Comments

\usepackage{color}

\newif\ifcomments\commentstrue %displays comments
%\newif\ifcomments\commentsfalse %so that comments do not display

\ifcomments
\newcommand{\authornote}[3]{\textcolor{#1}{[#3 ---#2]}}
\newcommand{\todo}[1]{\textcolor{red}{[TODO: #1]}}
\else
\newcommand{\authornote}[3]{}
\newcommand{\todo}[1]{}
\fi

\newcommand{\wss}[1]{\authornote{magenta}{SS}{#1}} 
\newcommand{\plt}[1]{\authornote{cyan}{TPLT}{#1}} %For explanation of the template
\newcommand{\an}[1]{\authornote{cyan}{Author}{#1}}

%% Common Parts

\newcommand{\progname}{ProgName} % PUT YOUR PROGRAM NAME HERE
\newcommand{\authname}{Team \#, Team Name
\\ Student 1 name
\\ Student 2 name
\\ Student 3 name
\\ Student 4 name} % AUTHOR NAMES                  

\usepackage{hyperref}
    \hypersetup{colorlinks=true, linkcolor=blue, citecolor=blue, filecolor=blue,
                urlcolor=blue, unicode=false}
    \urlstyle{same}
                                


\begin{document}

\maketitle

\begin{table}[hp]
\caption{Revision History} \label{TblRevisionHistory}
\begin{tabularx}{\textwidth}{lXl}
\toprule
\textbf{Date} & \textbf{Developer(s)} & \textbf{Change}\\
\midrule
September 22, 2025 & Dhruv Sardana, Moly Mikhail, Nawaal Fatima, Casey Francine Bulaclac & Initial version of document\\
\bottomrule
\end{tabularx}
\end{table}

\newpage{}

This document outlines the development plan for Team 22's Capstone Project ``Reading4All''. First, it highlights the confidential information, intellectual property to protect, and copyright 
license. It also explains team detail information such as team meeting plans, communication,
and roles. The workflow plan for the project is also established, along with the project 
decomposition and scheduling. A proof of concept demonstration plan is outlined in the document to 
identify risks and risk mitigation strategies. The expected technologies and coding standards are 
also highlighted in the demonstration plan followed by the appendix which contains the 
team charter and team reflection. Overall, this document provides the foundational structure 
and serves as the blueprint for the team's collaboration and technical direction for the project.

\section{Confidential Information}

This project does not contain any confidential information.


\section{IP to Protect}
This project does not contain any IP to protect. 
\section{Copyright License}

Our team is adopting the MIT License, which can be found \href{https://github.com/4G06-CAPSTONE-2025/Reading4All/blob/main/LICENSE} {here}

\section{Team Meeting Plan}

The team will meet weekly on Tuesdays from 3:00pm to 4:00pm virtually on Discord or in person on campus if needed. 
The team will meet with the industry advisor biweekly on Thursdays from 2:30pm to 3:30pm. These meetings with the industry advisor will 
be conducted either online on Microsoft Teams or in person on campus. \\[1ex]
The meetings will be structured as follows: 
\begin{enumerate}
	\item An agenda prepared by the meeting chair (who rotates among team members each week) 
  will be made to use as a guide for the meeting.
	\item The team will go over any announcements or completed To-Dos from the previous week's 
  meeting if needed.
  \item Each member will present what they have worked on so far and ask the remaining group 
  members for feedback or any questions if needed.
  \item The team will discuss and document any decisions needed about the deliverables 
  or the project. 
  \item Any concerns/questions will be documented for the next team meeting or for the next 
  industry advisor meeting.
\end{enumerate}

\section{Team Communication Plan}

\begin{itemize}
    \item \textbf{Discord}: Our main method of communication between group members. It will be used to discuss detailed deliverable and any code related questions. Additionally, online group meetings will be hosted on discord. 
    \item \textbf{Instagram}: Our secondary method of communication between group members. It will be used to discuss less technical details and for any urgent messages that require a quicker response. 
    \item \textbf{Microsoft Teams}: Our main method of communication with our Supervisor. We will utilize our group chat with our supervisor for any quick questions or updates. Online meetings with our supervisor will be hosted on teams.
    \item \textbf{GitHub}: The issues feature will be utilized to communicate any bugs observed and meeting attendances. Additionally, as a way to see what feature each team member is working on and their progress. 
\end{itemize}



\section{Team Member Roles}

The team will work collaboratively to develop and refine this project. To ensure a clear 
division of tasks, team members have been assigned roles that 
align with their areas of expertise and contribute to achieving the goals 
of this project. These roles will rotate throughout the year to prevent overspecialization
and to ensure that all members can gain experience and knowledge in 
every aspect of the project. The roles and responsibilities will rotate every two deliverables 
providing each member an opportunity to be assigned each role.\\[1ex]
The defined roles and current responsibilities per team member is as follows: 
\begin{itemize}
	\item \textbf{Fiza Sehar}: \textit{Developer, Documentation, Model Training Specialist} \\
	Fiza will be responsible for developing features and maintaining documentation for the 
  project. She will also be leading the model training for the project to ensure 
  efficient and accurate performance.
	\item \textbf{Dhruv Sardana}: \textit{Developer, Documentation, Full-Stack Specialist} \\
	Dhruv will work across both frontend and backend development, and in ensuring a 
  seamless integration and functionality. He will also support in writing documentation
  for this project. 
	\item \textbf{Nawaal Fatima}: \textit{Developer, Documentation, Data Specialist} \\
	Nawaal will also work on developing features with a 
  focus on data management, pre-processing, and analysis in this project. She 
  will also contribute in the documentation of the project.
	\item \textbf{Moly Mikhail}: \textit{Developer, Documentation, Backend Specialist} \\
	Moly will lead in the handling of the Application Programming Interfaces (API), database management, and system logic, focusing on 
  the backend of the project. She will also support in the documentation of the project.
	\item \textbf{Casey Francine Bulaclac}: \textit{Developer, Documentation, Frontend Specialist} \\
	Francine will be responsible for the design and implementation of the user interface, 
  ensuring correct usability and accessibility while assisting in the project 
  documentation.
\end{itemize}

\section{Workflow Plan}

\begin{itemize}
	\item\textbf{Git Workflow \newline}
	 We will be using git, branches and pull requests (PR) in order to divide work between group members and complete tasks concurrently. 
   Furthermore, we will follow a feature-branch based approach, our process will follow these steps:
   \begin{itemize}
    \item \textbf{Step 1: Permanent Branches}: The project repository will have two permanent branches. 
   \begin{itemize}
      \item \textbf{Develop}: This branch will be used integrate different features and ensure that they work successfully together. Code can only be merged into the \texttt{dev} branch after being reviewed by one other team member who wasn't working on the feature. 
      \item \textbf{Main}: This branch will be used to maintain the most stable version of the application. Code in the develop branch can only merged into the \texttt{main} branch after its been extensively tested and been reviewed and approved by a majority of team members. 
    \end{itemize}

    \item \textbf{Step 2: Feature/Bug Branches} Group member will create a branch from the \texttt{dev} branch (after pulling the most recent changes) for each feature they work on. \newline
    \textbf{Naming Conventions:}
    \begin{itemize}
      \item \textbf{Features: }\texttt{feature\_[contributorName]/D\#\_[featureDescription]} 
      \item \textbf{Bugs: }  \texttt{bug\_[contributorName]/D\#\_[bugDescription]}
    \end{itemize}

  \item \textbf{Step 3: Pull Request} Once a group member is done working on their feature, they will open a pull request to merge into the \texttt{dev} branch. Group members will use the Github comment feature to connect their feature to any applicable issue numbers. They will also assign a group member to review their code.
    \end{itemize}

	\item\textbf{Continuous Integration (CI) and Continuous Deployment (CD)}
	
  For CI and CD, we will be using Github Actions to automatically trigger a project build and run unit tests 
  whenever a commit occurs or pull requests are opened. This ensures errors are discovered as early as possible 
  and that the project remains in a working state during development. 

  \item \textbf{Use of Labels/Tags} \newline
  The team will utilize labels to help us organize ongoing work as well as work that needs to be completed. The tags will enable us to see the status of ongoing work and their respective priority. 
    The additional issue tags we will have available are:

    \begin{table}[!htbp]
      \centering
      \caption{Labels and Their Usage}
      \begin{tabular}{|p{4cm}|p{4cm}|}
      \hline
      \textbf{Labels} & \textbf{Usage} \\
      \hline
      High-Priority & This will be used on issues that need to be urgently completed.\\
      \hline
      Low-Priority & This will be used on issues that are not urgent and can be addressed at a later time.\\
      \hline
      In-Progress & This will be used to indicate that an issue is being worked on.\\
      \hline 
      Review Needed & This will be used to indicate that the PR needs a review.\\
      \hline
      Feedback & This will be used to indicate that a feature is based on feedback \\ 
      \hline
      Testing Needed & This indicates that testing is needed prior to completing the issue or feature\\
      \hline
      \end{tabular}
    \end{table}


   \item \textbf{Managing Issues}\newline
  We will utilize the ``capTemplate" template issues for tracking attendance, peer review, supervisor, teaching assistant (TA), and team meetings. Furthermore, issues that fall outside these categories will utilize the blank issue and the labels mentioned above. \newline
  The issue will have a clear title, description and an owner to be responsible for the issue. This will ensure issues are easily identified and managed. Once an issue has been resolved, it will be linked to the PR containing the code, allowing the team to keep track of the issue's status.  
  \item \textbf{Use of Milestones}\newline
  Milestones will be used to represent our deliverables. Each milestone will group the related issues that compose the deliverable and as issues are closed, we can track our progress and the additional effort needed to submit the deliverable.
  \item \textbf{Use of Checklists} \newline
  The group will utilize the deliverable checklists to ensure our work meets the expectations. 
  We will also have checklists in the Pull Request templates to ensure that all required items are completed prior to merging. The checklist will include the following items: 
  \begin{itemize}
    \item Unit Tests Passed
    \item Coding Standard is followed 
    \item Code Compiles without errors
    \item Reviewed by one team member (\texttt{dev} branch) or majority of team members (\texttt{main} branch)
  \end{itemize}
  \item \textbf{Quality Standards for Contribution}\newline
  The following criteria must be met in order to merge any changes into the ``Reading4All'' Repository
  \begin{itemize}
    \item All unit tests pass (CI pipeline is successful)
    \item Test coverage is \textbf{95}\%
    \item Linting passes with zero warnings or errors
    \item At least one team member has reviewed and approved the PR
    \item User interface (UI) changes meet the Accessibility for Ontarians with Disabilities Act (AODA) and Web Content Accessibility Guidelines 2.1 (WCAG)
    \item \texttt{README.md} is updated with any necessary changes
  \end{itemize}
  
  \end{itemize}

\section{Project Decomposition and Scheduling}

The project is hosted under the 4G06-CAPSTONE-2025 organization. The repository can be 
accessed here: \href{https://github.com/4G06-CAPSTONE-2025/Reading4All}{Reading4All Repository}. \\[1ex]
The team will use a Github Project to track and manage the project deliverables and tasks, as well 
as to ensure accountability for each team member's assigned tasks. The GitHub 
project is named ``Reading4All Project Planner'' which can be accessed here: 
\href{https://github.com/orgs/4G06-CAPSTONE-2025/projects/4}{Reading4All GitHub Project}. \\[1ex]
The project planner will have items to track and manage including:
\begin{itemize}
  \item Issues for project deliverables
  \item Supervisor and team meeting logs 
  \item Pull requests 
\end{itemize}
\subsection{Project Schedule}
\begin{center}
  \begin{tabularx}{\textwidth}{Xc}
    \toprule
    \textbf{Deliverables} & \textbf{Due Date} \\
    \midrule
    Problem Statement, Proof of Concept, and Development Plan & Week  04 \\
    Software Requirements Specifications and Hazards Analysis (Revision 0) & Week 06 \\
    Verification \& Validation Plan (Revision 0) & Week 08 \\
    Design Document (Rev-1) & Week 10 \\
    Proof of Concept Demonstration & Week 11 + 12 \\
    Design Document (Revision 0) & Week 16 \\
    Project Demonstration (Revision 0) & Week 18 + 19 \\
    Verification \& Validation Report (Revision 0) & Week 22 \\
    Final Demonstration (Revision 1) & Week 24 \\
    Final Documentation & Week 26 \\
    Capstone EXPO & Week 26 \\
    \bottomrule
  \end{tabularx}
\end{center}

\section{Proof of Concept Demonstration Plan}

Our Proof Of Concept(POC) will consist of roughly the following steps:
\begin{enumerate}
    \item Identify the types of technical diagrams to target for alt text generation.
    \begin{itemize}
        \item Such as but not limited to flowcharts, circuit diagrams, graphs, UML diagrams, and mathematical figures.
    \end{itemize}

    \item Determine the accessibility requirements and standards for alt text.
    \begin{itemize}
        \item These requirements will be based on AODA and Web Content Accessibility Guidelines (WCAG) 2.1, focusing on clarity, conciseness, and compatibility with screen readers.
    \end{itemize}

    \item Develop and evaluate the model’s ability to generate alt text for a selected set of diagrams.
    \begin{itemize}
        \item We will test outputs using a curated dataset of diagrams.
        \item Evaluation will include both automated metrics (e.g., text length, keyword coverage) and qualitative user testing with screen reader compatibility checks.
    \end{itemize}

    \item Incorporate user feedback to refine the generated descriptions.
    \begin{itemize}
        \item Feedback will be collected from users using metrics defined in the STEM Alt Text User Testing Project conducted by Ms. Jing in January 2025.
        \item A human-in-the-loop process will validate and correct outputs to increase trustworthiness and usability.
    \end{itemize}

    \item Compare generated alt text against baseline approaches.
    \begin{itemize}
        \item We will benchmark our tool’s outputs against existing alt text practices (manual descriptions, generic auto-generation) and results from the STEM Alt Text User Testing Project.
        \item The goal is to show measurable improvements in accessibility and user comprehension.
    \end{itemize}

    \item Demonstrate proof of integration with assistive technologies.
    \begin{itemize}
        \item Final validation will involve testing with screen readers (e.g., NVDA, JAWS) to ensure proper reading order and interpretability.
    \end{itemize}
\end{enumerate}

The following is a brief list of primary risks to our success and how the POC results can mitigate them: 
\begin{itemize}
    \item \textbf{Risk:} The model may not be able to accurately generate alternative text that is considered accessible. \\
    \textbf{Mitigation:} The POC will include testing for the model on a diverse set of diagrams to evaluate its performance and regulate according to  Accessibility for Ontarians with Disabilities Act (AODA) standards.
      
    \item \textbf{Risk:} The tool may not integrate smoothly with assistive technologies (e.g., screen readers).  \\
    \textbf{Mitigation:} The POC will explicitly test generated alt text with standard screen readers to confirm compatibility and usability.  

    \item \textbf{Risk:} End-users may perceive the generated alt text as inaccurate or misleading, reducing trust in the tool.  \\
    \textbf{Mitigation:} The POC will incorporate a human-in-the-loop feedback mechanism, enabling iterative validation and correction of generated text.  
    
    \item \textbf{Risk:} The project may face time constraints that could impact the depth of development. \\
    \textbf{Mitigation:} The POC will prioritize essential features and functionalities, allowing for a focused approach that can be expanded upon in future iterations.
\end{itemize}

\section{Expected Technology}

\begin{table}[!htbp]
  \centering
  \caption{Expected Technologies}
  \begin{tabular}{|p{4cm}|p{4cm}|p{5cm}|}
  \hline
  \textbf{Technology} & \textbf{Choice} & \textbf{Reasoning} \\ 
  \hline
  Backend Language & Python & The team is most familiar with this language and it provides many free and easy to use machine learning (ML) libraries.\\
  \hline 
  Database and Server Management & McMaster Database and Servers & The technology used for user authentication and verification is currently unknown and will be further investigated upon discussing with McMaster University\\
  \hline 
  Frontend Language and Framework & JavaScript with React & The team is also familiar with using JavaScript and React. The combination allows you to make interactive and maintainable interfaces through components. \\
  \hline
  UI Design & Figma & Figma will allow us to design and visualize our UI prior to building it. Figma also allows collaboration.\\ 
  \hline
  Python Libraries & Pandas, Numpy, Tensorflow or Pytorch & These Python libraries will be utilized to build our own ML model. \\
  \hline
  Pre-trained models & None & We will be making our own model and not used a pre-trained model.\\
  \hline 
  Linter & Flake8 (Python) and ESLint (JavaScript) &  We will use these linters as they are free tools that will allow us to detect potential errors early on and follows a consistent coding style. \\
  \hline
  Unit Testing Framework & Pytest (Python) and Jest (Javascript) & These unit testing frameworks will allow us to easily write and run unit tests. \newline They also support parameterized unit tests, allowing us to run the same test with different data. \\
  \hline
  Coverage Tools & Coverage.py (Python) and JSover (JavaScript) & We will use these these free tools to confirm that our unit tests reach all possible code paths.\\
  \hline
  Version Control & Git and GitHub & Git and GitHub will allow the team to easily collaborate and present our code. \\
  \hline 
  CI and CD & GitHub Actions & GitHub Actions will allow us to automate builds and run unit tests after a commit occurs. \\
  \hline 
  Project Management Tool & GitHub Projects & GitHub Projects will be used to plan and organize tasks as well as track progress in order to meet deadlines.\\
  \hline 
  

\end{tabular}
\end{table}

\FloatBarrier 

\section{Coding Standard}
The team will be adopting the Python Enhancement Proposal 8 (\href{https://peps.python.org/pep-0008/}{PEP8}) 
coding standard for the project. This coding standard is commonly used in the 
Python community and ensures that the team's code remains 
consistent and readable across the entire project.\\
To support this, the team will also use the Flake8 linter tool to automate the 
checking of the adherance of the team's code to the PEP8 standard. The tool will
also assist in highlighting any style and quality issues.

\newpage{}

\section*{Appendix --- Reflection}

\textbf{Fiza Sehar Reflection}
\begin{enumerate}
    \item Why is it important to create a development plan prior to starting the
    project?
    Creating a development plan before beginning a project is critical because it provides a clear path for the team, specifies the roles and responsibilities, and establishes realistic deadlines and milestones.  It enables the early identification of possible risks and dependencies, ensuring the implementation of mitigation methods.  The plan helps in the coordination of team efforts, effective progress tracking, and optimal resource use by establishing communication techniques and workflow processes.  In conclusion, it provides structure, decreases uncertainty, and increases the confidence of stakeholders that the project will go as planned.
    \item In your opinion, what are the advantages and disadvantages of using CI/CD?
    CI/CD has many advantages, including early problem detection through automated builds and tests, which improves overall code quality.  It accelerates development by allowing teams to produce smaller, more regular updates and respond quickly to customer feedback.  CI/CD also encourages collaboration by continuously merging code changes, minimizing merge conflicts, and facilitating reviews.  However, there are some drawbacks: version-control complexities (for example, frequent rebasing and merge strategies across branches) can cause friction; and complex debugging and reporting can slow teams, when a build or deployment fails, progress may halt, and pinpointing the root cause across pipeline stages, environments, and logs is often difficult.    
    \item What disagreements did your group have in this deliverable, if any,
    and how did you resolve them?
    We had no disagreements during this deliverable.  We agreed on scope and dates from the start, assigned tasks based on each member's strengths, and tracked progress through weekly check-ins.  This collaborative approach kept things going smoothly and ensured the deliverables are completed on time.
\end{enumerate}
The purpose of reflection questions is to give you a chance to assess your own
learning and that of your group as a whole, and to find ways to improve in the
future. Reflection is an important part of the learning process.  Reflection is
also an essential component of a successful software development process.  

Reflections are most interesting and useful when they're honest, even if the
stories they tell are imperfect. You will be marked based on your depth of
thought and analysis, and not based on the content of the reflections
themselves. Thus, for full marks we encourage you to answer openly and honestly
and to avoid simply writing ``what you think the evaluator wants to hear.''

Please answer the following questions.  Some questions can be answered on the
team level, but where appropriate, each team member should write their own
response:

\textbf{Nawaal Fatima Reflection}
\begin{enumerate}
    \item  \textbf{Why is it important to create a development plan prior to starting the
    project?} \newline
    Our project is very user-oriented and has more risks when it comes to reliability and functionality as Group 22 is designing for a demographic with which we have little/no working experience with.
    Knowing this, it is very important to have a blueprint of what we are bulding before we waste resources and cause our testers/end-users any unnecessary frustration. 
    When we have a plan, we can also make sure that we are all working towards the same goal. It also helps us to identify potential challenges and risks early on, allowing us to develop strategies to mitigate them.
    The development plan asked a couple questions we did't consider, which helped us to think more critically about our project and how we can make it successful.
    
    \item \textbf{In your opinion, what are the advantages and disadvantages of using
    CI/CD?} \newline
    I think there are more advantages than disadvantages when it comes to CI/CD.
    The main advantage is that it allows for faster and more frequent releases, which can lead to quicker feedback from users and a more responsive development process.
    It also helps to catch bugs and issues early in the development process, which can save time and resources in the long run.
    However, one disadvantage is that it can be difficult to set up and maintain, especially for smaller teams or projects with limited resources.
    It also requires a certain level of discipline and commitment from the development team to ensure that code is properly tested and reviewed before being merged into the main branch.
    Overall, I think the benefits of CI/CD outweigh the challenges, and it is a valuable practice for modern software development.

    \item \textbf{What disagreements did your group have in this deliverable, if any,
    and how did you resolve them?}\newline
    We're in agreement about most aspects of the development plan. Most `disagreements' we had were minor - such as naming conventiones to follow or what processes to esablish to ensure everything remains organized.
    We resolved these disagreements through open communication, making sure to listen to each other's perspectives and find solutions that worked for everyone.
    I imagine as we continue to work together, we may have more disagreements, but I am confident that we will be able to resolve them in a similar manner.

\end{enumerate}

\textbf{Moly Mikhail Reflection}
\begin{enumerate}
    \item  \textbf{Why is it important to create a development plan prior to starting the
    project?} \newline
    I believe it is important to create a development prior to starting a project as it lets you consider ahead of time the different 
    components. It allows you to plan out the work to be done and the different technologies that will be required. Furthermore, if a technology 
    is needed such as machine learning or a python library that some team members are not familiar with, planning prior to starting the project will give
    them a chance to schedule time to learn and practice with the technology. Another reason it is important to create a development plan is that it ensures that the project begins with a strong foundation.
    For example, discussing unit testing, code coverage and git practices prior to starting the project will ensure the project remains organized and stable throughout the development process. It also sets 
    expectations between group members of the code quality expected and the pull request review process to be completed.
    
    \item \textbf{In your opinion, what are the advantages and disadvantages of using
    CI/CD?} \newline
    I believe there are many advantages and disadvantages to using CI/CD. One advantage is that it ensures that the project remains stable and behaves as expected when new features are implemented. 
    For example, when a developer completes a feature, CI/CD checks that all previously functioning application features remain working as expected and no bugs are introduced. Another advantage
    is that when the application is in production and being used by customers, CI/CD will help ensure that releases only contain code that has passed all written tests.  This improves customer experiences
    as it reduces the chance of them encountering bugs or errors.  
    One disadvantage of CI/CD is that it can be very time-consuming and require many resources. For example, if a project is hosted on AWS, continuously redeploying the application
    can require many resources and increase cost. Another disadvantage of CI/CD is that when the CI/CD pipeline fails, it often requires a more 
    experienced individual or a specialist to investigate the reason for the failure.

    \item \textbf{What disagreements did your group have in this deliverable, if any,
    and how did you resolve them?}\newline
    We didn't face any disagreements when completing this deliverable. All group members were very open minded to other members ideas. Additionally, having meetings throughout the week and easily being able to communicate through discord ensured that everyone was aligned and on the same page. 

\end{enumerate}

\textbf{Dhruv Sardana Reflection}
\begin{enumerate}
    \item  \textbf{Why is it important to create a development plan prior to starting the project?} \newline

    Creating a development plan prior to starting the project is extremely important as it sets the foundation for the entire project. It sets a clear roadmap and outlines everyone's strength in different areas which could be useful to distribuite tasks.
    It allows the team to outline the goals, objectives, and deliverables of the project, ensuring that everyone is on the same page from the beginning. It also helps set the groundrules for effective and effcient communication
    A development plan also helps to identify potential risks and challenges that may arise during the project, allowing the team to develop strategies to mitigate them. 
    It also provides a roadmap for the project, outlining the timeline, milestones, and deadlines, which helps to keep the team on track and ensures that the project is completed on time.
    Additionally, a development plan helps to allocate resources effectively, ensuring that the team has the necessary tools, technologies, and personnel to complete the project successfully. 
    Overall, creating a development plan prior to starting the project is essential for ensuring that the project is well-organized, efficient, and successful.

    
    \item \textbf{In your opinion, what are the advantages and disadvantages of using CI/CD?} \newline

    In my opinion, the advantages of using CI/CD include faster development cycles, early detection of bugs, and improved collaboration among team members. It ensures that the codebase remains in a deployable state and reduces the risk of integration issues.
    We have two branches main and developer, this is done in order to ensure that main branch always remain stable and deployable. CI/CD ensures that maintainable code is merged into the main branch. It helps track changes and helps out in keeping the project organised throughout its course.
    However, the disadvantages include the initial setup complexity, the need for robust test coverage, and the potential for increased costs due to frequent builds and deployments.
  velopment.

    \item \textbf{What disagreements did your group have in this deliverable, if any, and how did you resolve them?}\newline
    Our group did not have any disagreements in this deliverable. We ensured proper and consistent communication from the start of this project. The team made sure to stay organized and on track in this deliverable
    by holding meetings to assign individual tasks, keeping each other up to date with the progress of these tasks, and providing support when needed. Each decision was taken with consensus and everyone's opinion was considered.
\end{enumerate}

\textbf{Casey Francine Bulaclac Reflection}

\begin{enumerate}
    \item Why is it important to create a development plan prior to starting the
    project? \\[1ex]
    It is important to create a development plan before starting the project to ensure that 
    the team is on the same page and working on the same objectives. The development helps to establish 
    the team's goals and outlines how to meet these goals through components such as the workflow plan. 
    It also establishes clear project timelines and individual accountability by 
    defining the team member roles and meeting plans, and project scheduling.
    Overall, creating a development plan provides a strong foundation for achieving the goals
    of the project which the team can refer to throughout the process of working on this project.
    \item In your opinion, what are the advantages and disadvantages of using CI/CD? \\[1ex]
    Continuous Integration (CI) allows multiple users to merge frequently in a shared repository by 
    implementing an automated pipeline consisting of build, unit tests, and more to 
    ensure that the new code does not break the system. Therefore, an advantage to CI is 
    that it benefits these users by making the integration of their code safer and 
    more efficient. A disadvantage, however, is that it demands time and complex automation such as 
    setting up the pipelines. Another disadvantage is that CI highly depends on having a 
    strong test coverage, therefore, having weak tests can decrease reliability. \\
    Continuous Deployment (CD) allows users to automatically release changes that have 
    passed automated tests and checks, delivering new features, bug fixes, and more into 
    production reliably and quickly. Therefore, an advantage to CD is that it speeds up 
    the delivery of new features or fixes but a disadvantage to this is that there are 
    high requirements for testing to ensure quality. Additionally, it requires a reliable 
    monitoring system to ensure that no defects or bugs reach production.

    \item What disagreements did your group have in this deliverable, if any,
    and how did you resolve them? \\[1ex]
    Our group had no disagreements in this deliverable as we ensured proper 
    and consistent communciation from the start of this project. The team made
    sure to stay organized and on track in this deliverable 
    by holding meetings to assign individual tasks, keeping each other up to date with 
    the progress of these tasks, and providing support when needed. 

\end{enumerate}

\textbf{Fiza Sehar}
\begin{enumerate}
   \item Why is it important to create a development plan prior to starting the
    project? \\[1ex]
	  Creating a development plan before beginning a project is critical because it provides a clear path for the team, specifies the roles and 		responsibilities, and establishes realistic deadlines and milestones.  It enables the early identification of possible risks and dependencies, ensuring the implementation of mitigation methods.  The plan helps in the coordination of team efforts, effective progress tracking, and optimal resource use by establishing communication techniques and workflow processes.  In conclusion, it provides structure, decreases uncertainty, and increases the confidence of stakeholders that the project will go as planned.
	  
    \item In your opinion, what are the advantages and disadvantages of using CI/CD?
    CI/CD has many advantages, including early problem detection through automated builds and tests, which improves overall code quality.  It  accelerates development by allowing teams to produce smaller, more regular updates and respond quickly to customer feedback.  CI/CD also encourages collaboration by continuously merging code changes, minimizing merge conflicts, and facilitating reviews.  However, there are some drawbacks: version-control complexities (for example, frequent rebasing and merge strategies across branches) can cause friction; and complex debugging and reporting can slow teams, when a build or deployment fails, progress may halt, and pinpointing the root cause across pipeline stages, environments, and logs is often difficult.   
	  
    \item What disagreements did your group have in this deliverable, if any,
    and how did you resolve them?
    We had no disagreements during this deliverable.  We agreed on scope and dates from the start, assigned tasks based on each member's strengths, and tracked progress through weekly check-ins.  This collaborative approach kept things going smoothly and ensured the deliverables are completed on time.
\end{enumerate}

\newpage{}

\section*{Appendix --- Team Charter}

\wss{borrows from
\href{https://engineering.up.edu/industry_partnerships/files/team-charter.pdf}
{University of Portland Team Charter}}

\subsection*{External Goals}

Our team’s primary external goal is to gain valuable, workforce-relevant experience by 
developing a project that enhances our technical and professional skills, particularly 
in machine learning, natural language processing, accessibility standards (AODA), and inclusive design. We aim
to create a portfolio-worthy project that could be showcased in interviews, 
highlighting our ability to address real-world accessibility challenges. Our goals also 
include achieving an A+ in the course, presenting our work at the Capstone EXPO for a chance 
to win a prize, and contributing to McMaster's commitment to accessibility by creating a tool to 
improve inclusion for students with disabilities.  

\subsection*{Attendance}
This section explains rules and expectations regarding team member attendance.

\subsubsection*{Expectations}

Our team expects all members to attend weekly meetings consistently and arrive 
on time to ensure productive collaboration and effective communication. Members 
are expected to stay for the entire duration of the meeting. If someone must 
leave early or miss a meeting, they should notify the team at least 24 hours 
in advance and take responsibility for catching up on any missed discussions 
or tasks by reviewing the meeting notes. We prioritize respect for each other’s time and aim to keep meetings efficient, 
focused, and adaptable to everyone’s schedule. If any team member misses two 
meetings in a row, they must treat the rest of the team to timbits. For every 
meeting missed after that, the member must bring a snack for each team member.
If a team member is consistently late or misses a meeting, they must provide an 
appropriate reason. Failure to do so will result in escalation to the TA and 
then the instructor. To stay organized, the team will use a discord event bot to 
schedule meetings and send reminders to ensure everyone is aware of upcoming 
sessions. 
\subsubsection*{Acceptable Excuse}

An acceptable excuse for missing a meeting or deadline includes unavoidable circumstances 
such as illness, family emergencies, technical issues, work events or unavoidable academic conflicts (midterms/exams) beyond one’s control,
that are communicated to the team in advance.
Excuses such as forgetting, poor time management, lack of preparation, or not informing the team
ahead of time are not acceptable. We anticipate clear and timely communication to alter duties as
appropriate to ensure the project stays on schedule. 

\subsubsection*{In Case of Emergency}

If a team member has an emergency and cannot attend a meeting or complete their assigned work, they 
are expected to notify the team as soon as possible through the group’s discord channel or teams channel. In the event, 
a team member cannot complete their assigned duties, they should inform the team atleast 72-96 hours prior to submission deadlines.
The member should clearly explain the situation, indicate whether they will need support or a change 
of tasks, and provide any available progress or notes so others can continue the work if necessary. In the case where tasks are redistributed,
the team member must update the progress board on Github to reflect the changes. 
The team will then adjust responsibilities collaboratively to ensure that deadlines and deliverables are still met.

\subsection*{Accountability and Teamwork}

\subsubsection*{Quality} 

Every member of the team is expected to produce high-quality work that meets the agreed-upon standards and contributes positively to the 
overall project. All members should prepare for weekly check-ins and meetings by doing the
following:

\begin{itemize}
  \item Before every meeting, the team members must review meeting notes from the last meeting and be up-to-date on the meeting agenda.
  \item The team members must review all the relevant materials and documents related to the project. Team members must create a list of questions to ask at the biweekly meeting with supervisor in order to gain more clarity towards the final goal.
  \item Assigned tasks should be completed in advance so that meetings can focus on collaboration and decision-making rather than catching up on unfinished work.
  \item Every team member must provide status updates on their assigned tasks during meetings, highlighting any challenges or roadblocks they are facing.
  \item All team members should actively participate in discussions, offering constructive feedback and suggestions to improve the project.
  \item In case any team members require any assistance or support, they should communicate this to the team promptly so that help can be provided.
\end{itemize}

\noindent In order to ensure high-quality work, the team will implement the following practices:

\begin{itemize}
  \item All code contributions must adhere to the team's coding standards and be reviewed by at least one other team member before being merged 
  into the dev branch and a majority review before being merged into main branch.
  \item Issues must be created on the Github progress board for all tasks, and team members should update the status of their tasks regularly.
  \item All tasks should be completed on the time agreed by the team and the quality of the work should adhere to the rubric of the given deliverable.
  \item Each team member is responsible for getting their work reviewed and approved by the team before submission.
  \item All the team members must actively participate in revieweing as well as sincerly and honestly considering feedback provided by other team members.
  \item If challenges arise that may affect quality, members should communicate early so that the team can provide support or make adjustments.
  \item The standard of the work should be well documented, well comunicated, well tested and reviewed by the team to ensure it meets the coding standards set by the team.
\end{itemize}

\subsubsection*{Attitude}

Our team expects all members to approach the project professionally, respectfully, and open to new ideas. Ideas should be openly shared, and all contributions
will be carefully considered to foster an environment creativity and innovation. Team interactions should be collaborative and helpful, with members working together to 
achieve a common goal rather than focusing on individual accomplishments. A positive and accountable attitude is required, in which each individual accepts
responsibility for their task while respecting the time and efforts of others. The team will follow the standard code of conduct to ensure respect, diversity and a no discrimination/harassement policy.
In case of disagreements/conflicts, the team will address them constructively and professionally, seeking to understand different perspectives and finding common ground internally in a team meeting.
If the issue rises, we will inform the TA and then escalate the issue to instructor defining the problem.

\subsubsection*{Stay on Track}

In order to stay track on the project and have effective teamwork, the team will implement the following strategies:


\noindent For each weekly meeting, attendance will be tracked and recorded by the meeting notes. The team will be tracking issues and tasks for each member using 
github projects and github kanban board. Each team member will be assigned weekly tasks and will be expected to complete them by the agreed-upon deadlines. 
The team will also track individual contributions through git commits and pull requests. The team will track participation in meetings and discussions, 
ensuring that all members are actively engaged and contributing to the team's progress.

\noindent Each week weekly status updates will be given by each team member. We will be evaluating member's contributions through various factors such as 
attendance, task completion, code contributions, helping team members, complexity of the tasks completed, ideation, code quality, research as well as the number of tickets closed. These
factors will be discussed before every deliverable and used as performance indicators. The team will be maintain a folder and using a weighted matrix with the above mentioned factors, a team champion will be declared every 2 weeks to recogonise the work
\noindent If a team member is not contributing their fair share, the team will first address the issue internally by discussing it with the member and understanding any challenges they may be facing.
The main focus is to maintain open communication and provide support to help the member get back on track. In this internal meeting, every member should be understanding and offer solutions to maintain a positive and fair environment. If the issue persists, the team will escalate the matter to the TA and then to the instructor if necessary. A performance improvement plan may be implemented to help the member improve their contributions. In the case if there is still no improvement,
the team may consider reassigning tasks or redistributing workload to ensure the project's success. The team will document any actions taken to address the issue and ensure that all members are aware of the expectations and consequences of not contributing their fair share.
In order to maintain equitable work and fairshare, the team member might be subject to disciplinary action such as removal from the group or grade adjustment.

\subsubsection*{Team Building}

We plan on building a harmonious team by organizing regular team-building activities, such as team lunches once every month and celebrating birthdays or special occassions. 
To recogonise good work, we will be giving a team medal to the best contributor every other week and celebrating small wins together. 

\subsubsection*{Decision Making} 

Decisions will be made collaboratively, with all team members having an equal say in the decision-making process. The team will strive to reach consensus on major decisions,
but if consensus cannot be reached, a majority vote will be used to make the final decision. In case of a tie, the team supervisor will have the deciding vote.

\end{document}
