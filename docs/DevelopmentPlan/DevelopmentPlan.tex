\documentclass{article}

\usepackage{booktabs}
\usepackage{tabularx}
\usepackage[hidelinks]{hyperref} 
\usepackage{placeins}
\usepackage{float}

\title{Development Plan\\\progname}

\author{\authname}

\date{}

%% Comments

\usepackage{color}

\newif\ifcomments\commentstrue %displays comments
%\newif\ifcomments\commentsfalse %so that comments do not display

\ifcomments
\newcommand{\authornote}[3]{\textcolor{#1}{[#3 ---#2]}}
\newcommand{\todo}[1]{\textcolor{red}{[TODO: #1]}}
\else
\newcommand{\authornote}[3]{}
\newcommand{\todo}[1]{}
\fi

\newcommand{\wss}[1]{\authornote{magenta}{SS}{#1}} 
\newcommand{\plt}[1]{\authornote{cyan}{TPLT}{#1}} %For explanation of the template
\newcommand{\an}[1]{\authornote{cyan}{Author}{#1}}

%% Common Parts

\newcommand{\progname}{ProgName} % PUT YOUR PROGRAM NAME HERE
\newcommand{\authname}{Team \#, Team Name
\\ Student 1 name
\\ Student 2 name
\\ Student 3 name
\\ Student 4 name} % AUTHOR NAMES                  

\usepackage{hyperref}
    \hypersetup{colorlinks=true, linkcolor=blue, citecolor=blue, filecolor=blue,
                urlcolor=blue, unicode=false}
    \urlstyle{same}
                                


\begin{document}

\maketitle

\begin{table}[hp]
\caption{Revision History} \label{TblRevisionHistory}
\begin{tabularx}{\textwidth}{llX}
\toprule
\textbf{Date} & \textbf{Developer(s)} & \textbf{Change}\\
\midrule
Date1 & Name(s) & Description of changes\\
Date2 & Name(s) & Description of changes\\
... & ... & ...\\
\bottomrule
\end{tabularx}
\end{table}

\newpage{}

\wss{Put your introductory blurb here.  Often the blurb is a brief roadmap of
what is contained in the report.}

\wss{Additional information on the development plan can be found in the
\href{https://gitlab.cas.mcmaster.ca/courses/capstone/-/blob/main/Lectures/L02b_POCAndDevPlan/POCAndDevPlan.pdf?ref_type=heads}
{lecture slides}.}

\section{Confidential Information}

This project does not contain any confidential information.


\section{IP to Protect}
This project does not contain any IP to protect. 
\section{Copyright License}

Our team is adopting the MIT License, which can be found \href{https://github.com/4G06-CAPSTONE-2025/Reading4All/blob/main/LICENSE} {here}

\section{Team Meeting Plan}

The team will meet weekly on Tuesdays from 3:00pm to 4:00pm virtually on Discord or in person on campus if needed. 
The team will meet with the industry advisor biweekly on Thursdays from 2:30pm to 3:30pm. These meetings with the industry advisor will 
be conducted either online on Microsoft Teams or in person on campus. \\[1ex]
The meetings will be structured as follows: 
\begin{enumerate}
	\item An agenda prepared by the meeting chair (who rotates among team members each week) 
  will be made to use as a guide for the meeting.
	\item The team will go over any announcements or completed To-Dos from the previous week's 
  meeting if needed.
  \item Each member will present what they have worked on so far and ask the remaining group 
  members for feedback or any questions if needed.
  \item The team will discuss and document any decisions needed about the deliverables 
  or the project. 
  \item Any concerns/questions will be documented for the next team meeting or for the next 
  industry advisor meeting.
\end{enumerate}

\section{Team Communication Plan}

\begin{itemize}
    \item \textbf{Discord}: Our main method of communication between group members. It will be used to discuss detailed deliverable and any code questions. Additionally, group meetings will be hosted on discord. 
    \item \textbf{Instagram}: Our secondary method of communication between group members. It will be used to discuss less technical details and for any urgent messages that require a quicker response. 
    \item \textbf{Teams}: Our main method of communication with our Supervisor. We will utilize our group chat with our supervisor for any quick questions or updates. Online meetings with our supervisor will be hosted on teams.
    \item \textbf{GitHub}: The issues feature will be utilized to communicate any bugs observed and meeting attendances. Additionally, as a way to see what feature each team member is working on and their progress. 
\end{itemize}



\section{Team Member Roles}

The team will work collaboratively to develop and refine this project. To ensure a clear 
division of tasks, each team member have been assigned roles that 
align with their areas of expertise and contribute to achieving the goals 
of this project. These roles will rotate throughout the year to prevent overspecialization
and to ensure that all members can gain experience and knowledge in 
every aspect of the project. \\[1ex]
The defined roles and responsibilities per team member is as follows: 
\begin{itemize}
	\item \textbf{Fiza Sehar}: \textit{Developer, Documentation, Model Training Specialist} \\
	Fiza will be responsible for developing features and maintaining documentation for the 
  project. She will also be leading the model training for the project to ensure 
  efficient and accurate performance.
	\item \textbf{Dhruv Sardana}: \textit{Developer, Documentation, Full-Stack Specialist} \\
	Dhruv will work across both frontend and backend development, and in ensuring a 
  seamless integration and functionality. He will also support in writing documentation
  for this project. 
	\item \textbf{Nawaal Fatima}: \textit{Developer, Documentation, Data Specialist} \\
	Nawaal will also work on developing features with a 
  focus on data management, pre-processing, and analysis in this project. She 
  will also contribute in the documentation of the project.
	\item \textbf{Moly Mikhail}: \textit{Developer, Documentation, Backend Specialist} \\
	Moly will be handling the APIs, database management, and system logic focusing on 
  the backend of the project. She will also support in the documentation of the project.
	\item \textbf{Casey Francine Bulaclac}: \textit{Developer, Documentation, Frontend Specialist} \\
	Francine will be responsible for the design and implementation of the user interface, 
  ensuring correct usability and accessibility while assisting in the project 
  documentation.
\end{itemize}

\section{Workflow Plan}

\begin{itemize}
	\item\textbf{Git Workflow \newline}
	 We will be using git, branches and pull requests in order to divide work between group members and complete tasks concurrently. 
   Furthermore, we will follow a feature-branch based approach, our process will follow these steps:
   \begin{itemize}
    \item \textbf{Step 1: Permanent Branches}: The project repository will have two permanent branches. 
   \begin{itemize}
      \item \textbf{Develop}: This branch will be used integrate different features and ensure that they work successfully together. Code can only be merged into the develop branch after being reviewed by one other team member who wasn't working on the feature. 
      \item \textbf{Main}: This branch will be used to maintain the most stable version of the application. Code in the develop branch can only merged into the main branch after its been extensively tested and been reviewed and approved by a majority of team members. 
    \end{itemize}

    \item \textbf{Step 2: Feature/Bug Branches} Group member will create a branch from the \texttt{develop} branch (after pulling the most recent changes) for each feature they work on. \newline
    \textbf{Naming Conventions:}
    \begin{itemize}
      \item \textbf{Features: }\texttt{feature\_[contributorName]/D\#\_[featureDescription]} 
      \item \textbf{Bugs: }  \texttt{bug\_[contributorName]/D\#\_[bugDescription]}
    \end{itemize}

  \item \textbf{Step 3: Pull Request} Once a group member is done working on their feature, they will open a pull request to merge into the develop branch. Group members will use the Github comment feature to connect their feature to any applicable issue numbers. They will also assign a group member to review their code.
    \end{itemize}

	\item\textbf{Continuous Integration\ Continuous Development}
	
  For continuous integration (CI), we will be using Github actions to automatically trigger a project build and run unit tests 
  whenever a commit occurs or pull requests are opened. This ensures errors are discovered as early as possible and that the project remains in a working state during development. 

  \item \textbf{Use of Labels} \newline
  The team will utilize labels to help us organize ongoing work as well as work that needs to be completed. The tags will enable us to see the status of ongoing work and their respective priority. 
    The issue tags we will have available are:
     \begin{itemize} 
        \item Bug, feature, question, in-progress, review needed, changes needed, ready for merge, done, high priority, low priority, meeting, frontend, backend, feedback 
    \end{itemize}
   \item \textbf{Managing issues}\newline
  We will utilize the capTemplate template issues for tracking attendance, peer review, supervisor, TA, and team meetings. Furthermore, issues that fall outside these categories will utilize the blank issue and the labels mentioned above. \newline
  The issue will have a clear title, description and an owner to be responsible for the issue. This will ensure issues are easily identified and managed. Once an issue has been resolved, it will be linked to the PR containing the code, allowing the team to keep track of the issues status.
  \item \textbf{Use of Checklists} \newline
  The group will utilize the deliverable checklists to ensure our work meets the expectations. 
  We will also have checklists in the Pull Request templates to ensure that all required items are completed prior to merging. The checklist will include the following items: 
  \begin{itemize}
    \item Unit Tests Passed
    \item Coding Standard is followed 
    \item Code Compiles without errors
    \item Reviewed by 2 team members
  \end{itemize}
  \end{itemize}

\section{Project Decomposition and Scheduling}

The project is hosted under the 4G06-CAPSTONE-2025 organization. The repository can be 
accessed here: \href{https://github.com/4G06-CAPSTONE-2025/Reading4All}{Reading4All Repository}. \\[1ex]
The team will use a Github Project to track and manage the project deliverables and tasks, as well 
as to ensure accountability for each team member's assigned tasks. The GitHub 
project is named ``Reading4All Project Planner'' which can be accessed here: 
\href{https://github.com/orgs/4G06-CAPSTONE-2025/projects/4}{Reading4All GitHub Project}. \\[1ex]
The project planner will have items to track and manage including:
\begin{itemize}
  \item Issues for project deliverables
  \item Supervisor and team meeting logs 
  \item Pull requests 
  \item Draft items as placeholders for future issues
\end{itemize}
\subsection{Project Schedule}
\begin{center}
  \begin{tabularx}{\textwidth}{Xc}
    \toprule
    \textbf{Deliverables} & \textbf{Due Date} \\
    \midrule
    Problem Statement, Proof of Concept, and Development Plan & Week  04 \\
    Software Requirements Specificaiton and Hazards Analysis (Revision 0) & Week 06 \\
    Verification \& Validation Plan (Revision 0) & Week 08 \\
    Design Document (Rev-1) & Week 10 \\
    Proof of Concept Demonstration & Week 11 + 12 \\
    Design Document (Revision 0) & Week 16 \\
    Project Demonstration (Revision 0) & Week 18 + 19 \\
    Verification \& Validation Report (Revision 0) & Week 22 \\
    Final Demonstration (Revision 1) & Week 24 \\
    Final Documentation & Week 26 \\
    Capstone EXPO & Week 26 \\
    \bottomrule
  \end{tabularx}
\end{center}

\section{Proof of Concept Demonstration Plan}

What is the main risk, or risks, for the success of your project?  What will you
demonstrate during your proof of concept demonstration to convince yourself that
you will be able to overcome this risk?

\section{Expected Technology}

\begin{table}[!htbp]
  \centering
  \caption{Expected Technologies}
  \begin{tabular}{|p{4cm}|p{4cm}|p{5cm}|}
  \hline
  \textbf{Technology} & \textbf{Choice} & \textbf{Reasoning} \\ 
  \hline
  Backend Language & Python & The team is most familiar with this language and it provides many free and easy to use machine learning (ML) libraries.\\
  \hline 
  Frontend Language and Framework & JavaScript with React & The team is also familiar with using JavaScript and React. The combination allows you to make interactive and maintainable interfaces through components. \\
  \hline
  UI Design & Figma & Figma will allow us to design and visualize our UI prior to building it. Figma also allows collaboration.\\ 
  \hline
  Libraries & Pandas, Tensorflow or Pytorch & These Python libraries will be utilized to build our own ML model. \\
  \hline
  Pre-trained models & None & We will be making our own model and not used a pre-trained model.\\
  \hline 
  Linter & Pylint(Python) and ESLint (JavaScript) &  We will use these linters as they are free tools that will allow us to detect potential errors early on and follows a consistent coding style. \\
  \hline
  Unit Testing Framework & Pytest(Python) and Jest (Javascript) & These unit testing frameworks will allow us to easily write and run unit tests. \newline They also support parameterized unit tests, allowing us to run the same test with different data. \\
  \hline
  Coverage Tools & Coverage.py (Python) and JSover(JavaScript) & We will use these these free tools to confirm that our unit tests reach all possible code paths.\\
  \hline
  Version Control & Git and GitHub & Git and GitHub will allow the team to easily collaborate and present our code. \\
  \hline 
  Continuous Integration & GitHub Actions & GitHub Actions will allow us to automates builds and unit tests on commits. \\
  \hline 
  Project Management Tool & GitHub Projects & GitHub Projects will be used to plan and organize tasks as well as track progress in order to meet deadlines.\\
  \hline 

\end{tabular}
\end{table}

\FloatBarrier 

\section{Coding Standard}
The team will be adopting the \href{https://peps.python.org/pep-0008/}{PEP8} (Python Enhancement Proposal 8) 
coding standard for the project. This coding standard is commonly used in the 
Python community and ensures that the team's code remains 
consistent and readable across the entire project.\\
To support this, the team will also use the Flake8 linter tool to automate the 
checking of the adherance of the team's code to the PEP8 standard. The tool will
also assist in highlighting any style and quality issues.

\newpage{}

\section*{Appendix --- Reflection}

\wss{Not required for CAS 741}

The purpose of reflection questions is to give you a chance to assess your own
learning and that of your group as a whole, and to find ways to improve in the
future. Reflection is an important part of the learning process.  Reflection is
also an essential component of a successful software development process.  

Reflections are most interesting and useful when they're honest, even if the
stories they tell are imperfect. You will be marked based on your depth of
thought and analysis, and not based on the content of the reflections
themselves. Thus, for full marks we encourage you to answer openly and honestly
and to avoid simply writing ``what you think the evaluator wants to hear.''

Please answer the following questions.  Some questions can be answered on the
team level, but where appropriate, each team member should write their own
response:


\textbf{Moly Mikhail Reflection}
\begin{enumerate}
    \item  \textbf{Why is it important to create a development plan prior to starting the
    project?} \newline
    I believe it is important to create a development prior to starting a project as it lets you consider ahead of time the different 
    components. It allows you to plan out the work to be done and the different technologies that will be required. Furthermore, if a technology 
    is needed such as machine learning or a python library that some team members are not familiar with, planning prior to starting the project will give
    them a chance to schedule time to learn and practice with the technology. Another reason it is important to create a development plan is that it ensures that the project begins with a strong foundation.
    For example, discussing unit testing, code coverage and git practices prior to starting the project will ensure the project remains organized and stable throughout the development process. It also sets 
    expectations between group members of the code quality expected and the pull request review process to be completed.
    
    \item \textbf{In your opinion, what are the advantages and disadvantages of using
    CI/CD?} \newline
    I believe there are many advantages and disadvantages to using CI/CD. One advantage is that it ensures that the project remains stable and behaves as expected when new features are implemented. 
    For example, when a developer completes a feature, CI/CD checks that all previously functioning application features remain working as expected and no bugs are introduced. Another advantage
    is that when the application is in production and being used by customers, CI/CD will help ensure that releases only contain code that has passed all written tests.  This improves customer experiences
    as it reduces the chance of them encountering bugs or errors.  
    One disadvantage of CI/CD is that it can be very time-consuming and require many resources. For example, if a project is hosted on AWS, continuously redeploying the application
    can require many resources and increase cost. Another disadvantage of CI/CD is that when the CI/CD pipeline fails, it often requires a more 
    experienced individual or a specialist to investigate the reason for the failure.

    \item \textbf{What disagreements did your group have in this deliverable, if any,
    and how did you resolve them?}
    We didn't face any disagreements when completing this deliverable. 

\end{enumerate}


\textbf{Casey Francine Bulaclac Reflection}

\begin{enumerate}
    \item Why is it important to create a development plan prior to starting the
    project? \\[1ex]
    It is important to create a development plan before starting the project to ensure that 
    the team is on the same page and working on the same objectives. The development helps to establish 
    the team's goals and outlines how to meet these goals through components such as the workflow plan. 
    It also establishes clear project timelines and individual accountability by 
    defining the team member roles and meeting plans, and project scheduling.
    Overall, creating a development plan provides a strong foundation for achieving the goals
    of the project which the team can refer to throughout the process of working on this project.
    \item In your opinion, what are the advantages and disadvantages of using CI/CD? \\[1ex]
    Continuous Integration (CI) allows multiple users to merge frequently in a shared repository by 
    implementing an automated pipeline consisting of build, unit tests, and more to 
    ensure that the new code does not break the system. Therefore, an advantage to CI is 
    that it benefits these users by making the integration of their code safer and 
    more efficient. A disadvantage, however, is that it demands time and complex automation such as 
    setting up the pipelines. Another disadvantage is that CI highly depends on having a 
    strong test coverage, therefore, having weak tests can decrease reliability. \\
    Continuous Deployment (CD) allows users to automatically release changes that have 
    passed automated tests and checks, delivering new features, bug fixes, and more into 
    production reliably and quickly. Therefore, an advantage to CD is that it speeds up 
    the delivery of new features or fixes but a disadvantage to this is that there are 
    high requirements for testing to ensure quality. Additionally, it requires a reliable 
    monitoring system to ensure that no defects or bugs reach production.

    \item What disagreements did your group have in this deliverable, if any,
    and how did you resolve them? \\[1ex]
    Our group had no disagreements in this deliverable as we ensured proper 
    and consistent communciation from the start of this project. The team made
    sure to stay organized and on track in this deliverable 
    by holding meetings to assign individual tasks, keeping each other up to date with 
    the progress of these tasks, and providing support when needed. 

\end{enumerate}

\newpage{}

\section*{Appendix --- Team Charter}

\wss{borrows from
\href{https://engineering.up.edu/industry_partnerships/files/team-charter.pdf}
{University of Portland Team Charter}}

\subsection*{External Goals}

\wss{What are your team's external goals for this project? These are not the
goals related to the functionality or quality fo the project.  These are the
goals on what the team wishes to achieve with the project.  Potential goals are
to win a prize at the Capstone EXPO, or to have something to talk about in
interviews, or to get an A+, etc.}

\subsection*{Attendance}

\subsubsection*{Expectations}

\wss{What are your team's expectations regarding meeting attendance (being on
time, leaving early, missing meetings, etc.)?}

\subsubsection*{Acceptable Excuse}

\wss{What constitutes an acceptable excuse for missing a meeting or a deadline?
What types of excuses will not be considered acceptable?}

\subsubsection*{In Case of Emergency}

\wss{What process will team members follow if they have an emergency and cannot
attend a team meeting or complete their individual work promised for a team
deliverable?}

\subsection*{Accountability and Teamwork}

\subsubsection*{Quality} 

\wss{What are your team's expectations regarding the quality
of team members' preparation for team meetings and the quality of the
deliverables that members bring to the team?}

\subsubsection*{Attitude}

\wss{What are your team's expectations regarding team members' ideas,
interactions with the team, cooperation, attitudes, and anything else regarding
team member contributions?  Do you want to introduce a code of conduct?  Do you
want a conflict resolution plan?  Can adopt existing codes of conduct.}

\subsubsection*{Stay on Track}

\wss{What methods will be used to keep the team on track? How will your team
ensure that members contribute as expected to the team and that the team
performs as expected? How will your team reward members who do well and manage
members whose performance is below expectations?  What are the consequences for
someone not contributing their fair share?}

\wss{You may wish to use the project management metrics collected for the TA and
instructor for this.}

\wss{You can set target metrics for attendance, commits, etc.  What are the
consequences if someone doesn't hit their targets?  Do they need to bring the
coffee to the next team meeting?  Does the team need to make an appointment with
their TA, or the instructor?  Are there incentives for reaching targets early?}

\subsubsection*{Team Building}

\wss{How will you build team cohesion (fun time, group rituals, etc.)? }

\subsubsection*{Decision Making} 

\wss{How will you make decisions in your group? Consensus?  Vote? How will you
handle disagreements? }

\end{document}