\documentclass{article}

\usepackage{booktabs}
\usepackage{tabularx}

\title{Development Plan\\\progname}

\author{\authname}

\date{}

%% Comments

\usepackage{color}

\newif\ifcomments\commentstrue %displays comments
%\newif\ifcomments\commentsfalse %so that comments do not display

\ifcomments
\newcommand{\authornote}[3]{\textcolor{#1}{[#3 ---#2]}}
\newcommand{\todo}[1]{\textcolor{red}{[TODO: #1]}}
\else
\newcommand{\authornote}[3]{}
\newcommand{\todo}[1]{}
\fi

\newcommand{\wss}[1]{\authornote{magenta}{SS}{#1}} 
\newcommand{\plt}[1]{\authornote{cyan}{TPLT}{#1}} %For explanation of the template
\newcommand{\an}[1]{\authornote{cyan}{Author}{#1}}

%% Common Parts

\newcommand{\progname}{ProgName} % PUT YOUR PROGRAM NAME HERE
\newcommand{\authname}{Team \#, Team Name
\\ Student 1 name
\\ Student 2 name
\\ Student 3 name
\\ Student 4 name} % AUTHOR NAMES                  

\usepackage{hyperref}
    \hypersetup{colorlinks=true, linkcolor=blue, citecolor=blue, filecolor=blue,
                urlcolor=blue, unicode=false}
    \urlstyle{same}
                                


\begin{document}

\maketitle

\begin{table}[hp]
\caption{Revision History} \label{TblRevisionHistory}
\begin{tabularx}{\textwidth}{llX}
\toprule
\textbf{Date} & \textbf{Developer(s)} & \textbf{Change}\\
\midrule
Date1 & Name(s) & Description of changes\\
Date2 & Name(s) & Description of changes\\
... & ... & ...\\
\bottomrule
\end{tabularx}
\end{table}

\newpage{}

\wss{Put your introductory blurb here.  Often the blurb is a brief roadmap of
what is contained in the report.}

\wss{Additional information on the development plan can be found in the
\href{https://gitlab.cas.mcmaster.ca/courses/capstone/-/blob/main/Lectures/L02b_POCAndDevPlan/POCAndDevPlan.pdf?ref_type=heads}
{lecture slides}.}

\section{Confidential Information?}

\wss{State whether your project has confidential information from industry, or
not.  If there is confidential information, point to the agreement you have in
place.}

\wss{For most teams this section will just state that there is no confidential
information to protect.}
\section{IP to Protect}

\wss{State whether there is IP to protect.  If there is, point to the agreement.
All students who are working on a project that requires an IP agreement are also
required to sign the ``Intellectual Property Guide Acknowledgement.''}

\section{Copyright License}

\wss{What copyright license is your team adopting.  Point to the license in your
repo.}

\section{Team Meeting Plan}

\wss{How often will you meet? where?}

\wss{If the meeting is a physical location (not virtual), out of an abundance of
caution for safety reasons you shouldn't put the location online}

\wss{How often will you meet with your industry advisor?  when?  where?}

\wss{Will meetings be virtual?  At least some meetings should likely be
in-person.}

\wss{How will the meetings be structured?  There should be a chair for all meetings.  There should be an agenda for all meetings.}

\section{Team Communication Plan}

\wss{Issues on GitHub should be part of your communication plan.}

\section{Team Member Roles}

\wss{You should identify the types of roles you anticipate, like notetaker,
leader, meeting chair, reviewer.  Assigning specific people to those roles is
not necessary at this stage.  In a student team the role of the individuals will
likely change throughout the year.}

\section{Workflow Plan}

\begin{itemize}
	\item How will you be using git, including branches, pull request, etc.?
	\item How will you be managing issues, including template issues, issue
	classification, etc.?
  \item Use of CI/CD
\end{itemize}

\section{Project Decomposition and Scheduling}

\begin{itemize}
  \item How will you be using GitHub projects?
  \item Include a link to your GitHub project
\end{itemize}

\wss{How will the project be scheduled?  This is the big picture schedule, not
details. You will need to reproduce information that is in the course outline
for deadlines.}

\section{Proof of Concept Demonstration Plan}

What is the main risk, or risks, for the success of your project?  What will you
demonstrate during your proof of concept demonstration to convince yourself that
you will be able to overcome this risk?

\section{Expected Technology}

\wss{What programming language or languages do you expect to use?  What external
libraries?  What frameworks?  What technologies.  Are there major components of
the implementation that you expect you will implement, despite the existence of
libraries that provide the required functionality.  For projects with machine
learning, will you use pre-trained models, or be training your own model?  }

\wss{The implementation decisions can, and likely will, change over the course
of the project.  The initial documentation should be written in an abstract way;
it should be agnostic of the implementation choices, unless the implementation
choices are project constraints.  However, recording our initial thoughts on
implementation helps understand the challenge level and feasibility of a
project.  It may also help with early identification of areas where project
members will need to augment their training.}

Topics to discuss include the following:

\begin{itemize}
\item Specific programming language
\item Specific libraries
\item Pre-trained models
\item Specific linter tool (if appropriate)
\item Specific unit testing framework
\item Investigation of code coverage measuring tools
\item Specific plans for Continuous Integration (CI), or an explanation that CI
  is not being done
\item Specific performance measuring tools (like Valgrind), if
  appropriate
\item Tools you will likely be using?
\end{itemize}

\wss{git, GitHub and GitHub projects should be part of your technology.}

\section{Coding Standard}

\wss{What coding standard will you adopt?}

\newpage{}

\section*{Appendix --- Reflection}

\wss{Not required for CAS 741}

The purpose of reflection questions is to give you a chance to assess your own
learning and that of your group as a whole, and to find ways to improve in the
future. Reflection is an important part of the learning process.  Reflection is
also an essential component of a successful software development process.  

Reflections are most interesting and useful when they're honest, even if the
stories they tell are imperfect. You will be marked based on your depth of
thought and analysis, and not based on the content of the reflections
themselves. Thus, for full marks we encourage you to answer openly and honestly
and to avoid simply writing ``what you think the evaluator wants to hear.''

Please answer the following questions.  Some questions can be answered on the
team level, but where appropriate, each team member should write their own
response:

\begin{enumerate}
    \item Why is it important to create a development plan prior to starting the
    project?
    \item In your opinion, what are the advantages and disadvantages of using
    CI/CD?
    \item What disagreements did your group have in this deliverable, if any,
    and how did you resolve them?
\end{enumerate}

\newpage{}

\section*{Appendix --- Team Charter}

\wss{borrows from
\href{https://engineering.up.edu/industry_partnerships/files/team-charter.pdf}
{University of Portland Team Charter}}

\subsection*{External Goals}

Our team’s primary external goal is to gain valuable, workforce-relevant experience by 
developing a project that enhances our technical and professional skills, particularly 
in machine learning, natural language processing, accessibility standards (AODA), and inclusive design. We aim
to create a portfolio-worthy project that could be showcased in interviews, 
highlighting our ability to address real-world accessibility challenges. Our goals also 
include achieving an A+ in the course, presenting our work at the Capstone EXPO for a chance 
to win a prize, and contributing to McMaster's commitment to accessibility by creating a tool to 
improve inclusion for students with disabilities.  

\subsection*{Attendance}
This section explains rules and expectations regarding team member attendance.

\subsubsection*{Expectations}

Our team expects all members to attend weekly meetings consistently and arrive 
on time to ensure productive collaboration and effective communication. Members 
are expected to stay for the entire duration of the meeting. If someone must 
leave early or miss a meeting, they should notify the team at least 24 hours 
in advance and take responsibility for catching up on any missed discussions 
or tasks by reviewing the meeting notes. We prioritize respect for each other’s time and aim to keep meetings efficient, 
focused, and adaptable to everyone’s schedule. If any team member misses two 
meetings in a row, they must treat the rest of the team to timbits. For every 
meeting missed after that, the member must bring a snack for each team member.
If a team member is consistently late or misses a meeting, they must provide an 
appropriate reason. Failure to do so will result in escalation to the TA and 
then the instructor. To stay organized, the team will use a discord event bot to 
schedule meetings and send reminders to ensure everyone is aware of upcoming 
sessions. 
\subsubsection*{Acceptable Excuse}

An acceptable excuse for missing a meeting or deadline includes unavoidable circumstances 
such as illness, family emergencies, technical issues, work events or unavoidable academic conflicts (midterms/exams) beyond one’s control,
that are communicated to the team in advance.
Excuses such as forgetting, poor time management, lack of preparation, or not informing the team
ahead of time are not acceptable. We anticipate clear and timely communication to alter duties as
appropriate to ensure the project stays on schedule. 

\subsubsection*{In Case of Emergency}

If a team member has an emergency and cannot attend a meeting or complete their assigned work, they 
are expected to notify the team as soon as possible through the group’s discord channel or teams channel. In the event, 
a team member cannot complete their assigned duties, they should inform the team atleast 72-96 hours prior to submission deadlines.
The member should clearly explain the situation, indicate whether they will need support or a change 
of tasks, and provide any available progress or notes so others can continue the work if necessary. In the case where tasks are redistributed,
the team member must update the progress board on Github to reflect the changes. 
The team will then adjust responsibilities collaboratively to ensure that deadlines and deliverables are still met.

\subsection*{Accountability and Teamwork}

\subsubsection*{Quality} 

Every member of the team is expected to produce high-quality work that meets the agreed-upon standards and contributes positively to the 
overall project. All members should prepare for weekly check-ins and meetings by doing the
following:

\begin{itemize}
  \item Before every meeting, the team members must review meeting notes from the last meeting and be up-to-date on the meeting agenda.
  \item The team members must review all the relevant materials and documents related to the project. Team members must create a list of questions to ask at the biweekly meeting with supervisor in order to gain more clarity towards the final goal.
  \item Assigned tasks should be completed in advance so that meetings can focus on collaboration and decision-making rather than catching up on unfinished work.
  \item Every team member must provide status updates on their assigned tasks during meetings, highlighting any challenges or roadblocks they are facing.
  \item All team members should actively participate in discussions, offering constructive feedback and suggestions to improve the project.
  \item In case any team members require any assistance or support, they should communicate this to the team promptly so that help can be provided.
\end{itemize}

\noindent In order to ensure high-quality work, the team will implement the following practices:

\begin{itemize}
  \item All code contributions must adhere to the team's coding standards and be reviewed by at least one other team member before being merged 
  into the main branch.
  \item Issues must be created on the Github progress board for all tasks, and team members should update the status of their tasks regularly.
  \item All tasks should be completed on the time agreed by the team and the quality of the work should adhere to the rubric of the given deliverable.
  \item Each team member is responsible for getting their work reviewed and approved by the team before submission.
  \item All the team members must actively participate in revieweing as well as sincerly and honestly considering feedback provided by other team members.
  \item If challenges arise that may affect quality, members should communicate early so that the team can provide support or make adjustments.
  \item The standard of the work should be well documented, well comunicated, well tested and reviewed by the team to ensure it meets the coding standards set by the team.
\end{itemize}

\subsubsection*{Attitude}

Our team expects all members to approach the project professionally, respectfully, and open to new ideas. Ideas should be openly shared, and all contributions
will be carefully considered to foster an environment creativity and innovation. Team interactions should be collaborative and helpful, with members working together to 
achieve a common goal rather than focusing on individual accomplishments. A positive and accountable attitude is required, in which each individual accepts
responsibility for their task while respecting the time and efforts of others. The team will follow the standard code of conduct to ensure respect, diversity and a no discrimination/harassement policy.
In case of disagreements/conflicts, the team will address them constructively and professionally, seeking to understand different perspectives and finding common ground internally in a team meeting.
If the issue rises, we will inform the TA and then escalate the issue to instructor defining the problem.

\subsubsection*{Stay on Track}

In order to stay track on the project and have effective teamwork, the team will implement the following strategies:


\noindent For each weekly meeting, attendance will be tracked and recorded by the meeting notes. The team will be tracking issues and tasks for each member using 
github projects and github kanban board. Each team member will be assigned weekly tasks and will be expected to complete them by the agreed-upon deadlines. 
The team will also track individual contributions through git commits and pull requests. The team will track participation in meetings and discussions, 
ensuring that all members are actively engaged and contributing to the team's progress.

\noindent Each week weekly status updates will be given by each team member. We will be evaluating member's contributions through various factors such as 
attendance, task completion, code contributions, helping team members, complexity of the tasks completed, ideation, code quality, research as well as the number of tickets closed. These
factors will be discussed before every deliverable and used as performance indicators. The team will be maintain a folder and using a weighted matrix with the above mentioned factors, a team champion will be declared every 2 weeks to recogonise the work
\noindent If a team member is not contributing their fair share, the team will first address the issue internally by discussing it with the member and understanding any challenges they may be facing.
The main focus is to maintain open communication and provide support to help the member get back on track. In this internal meeting, every member should be understanding and offer solutions to maintain a positive and fair environment.
. If the issue persists, the team will escalate the matter to the TA and then to the instructor if necessary. A performance improvement plan may be implemented to help the member improve their contributions. In the case if there is still no improvement,
the team may consider reassigning tasks or redistributing workload to ensure the project's success. The team will document any actions taken to address the issue and ensure that all members are aware of the expectations and consequences of not contributing their fair share.
In order to maintain equitable work and fairshare, the team member might be subject to disciplinary action such as removal from the group or grade adjustment.

\subsubsection*{Team Building}

We plan on building a harmonious team by organizing regular team-building activities, such as team lunches once every month and celebrating birthdays or special occassions. 
To recogonise good work, we will be giving a team medal to the best contributor every other week and celebrating small wins together. 

\subsubsection*{Decision Making} 

Decisions will be made collaboratively, with all team members having an equal say in the decision-making process. The team will strive to reach consensus on major decisions,
but if consensus cannot be reached, a majority vote will be used to make the final decision. In case of a tie, the team supervisor will have the deciding vote.

\end{document}