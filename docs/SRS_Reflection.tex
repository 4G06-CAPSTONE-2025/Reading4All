\textbf{Group Reflection  - Reflection}
\begin{enumerate}
  \item What went well while writing this deliverable? 
  
  Throughout the process, our team maintained organization and had good communication.  It was simpler to preserve flow and prevent redundancy because we had a clear framework and consistent formatting from earlier deliverables.  We were also able to maintain our efficiency and keep the material consistent with our previous work.  In order review sections, define expectations, and make sure the deliverable satisfied all requirements, we also had meetings with our TA and our team.
  \item What pain points did you experience during this deliverable, and how did
  you resolve them?

  One difficulty was avoiding redundancy by keeping formatting constant and clearly distinguishing design components from implementation specifics. In order to guarantee accuracy and uniformity throughout all sections, we addressed this by going over the course templates and conducting team editing sessions to make sure the formatting followed a consistent style. We also helped each other out by sharing tips and tricks to improve clarity and presentation.

  \item How many of your requirements were inspired by speaking to your
  client(s) or their proxies (e.g. your peers, stakeholders, potential users)?
  
  Key design and functionality components that were influenced by internal meetings, conversations with our supervisor, pertinent compliance guidelines, and my internship's prior experience working with AI included the accessible interface, the quality of the alt text that was generated, and adherence to AODA and WCAG 2.1 standards.
  \item Which of the courses you have taken, or are currently taking, will help
  your team to be successful with your capstone project.

  The Software Requirements course from third year was extremely useful in creating this deliverable.  It created a solid foundation for organizing requirement documents, developing precise specifications, and grasping the foundations of software documentation and traceability, all of which significantly aided our work on this project.

  \item What knowledge and skills will the team collectively need to acquire to
  successfully complete this capstone project?  Examples of possible knowledge
  to acquire include domain specific knowledge from the domain of your
  application, or software engineering knowledge, mechatronics knowledge or
  computer science knowledge.  Skills may be related to technology, or writing,
  or presentation, or team management, etc.  You should look to identify at
  least one item for each team member. \newline
  
  \item \textbf{For each of the knowledge areas and skills identified in the previous
  question, what are at least two approaches to acquiring the knowledge or
  mastering the skill?  Of the identified approaches, which will each team
  member pursue, and why did they make this choice?}\newline

\end{enumerate}


\textbf{Moly Mikhail  - Reflection}
\begin{enumerate}
  \item \textbf{What went well while writing this deliverable?} \newline
  I believe writing this deliverable many things went well. I really enjoyed getting to think about the different
  non-functional requirements. I found that have different sections of non-functional requirements encouraged me to think about different 
  aspects of the system and things we will have to keep in mind during development. For example, prior to writing 
  this deliverable, we hadn’t considered personalization and internationalization requirements; however, having to
  complete that section led us to add important functionality of allowing the user to decide which way
  to store the alternative text. 
  
\item \textbf{What pain points did you experience during this
    deliverable, and how did you resolve them?} \newline
  One pain point I experience writing this deliverable dealt with completing the product boundary. 
  Initially, I was confused on The Scope of the Product section and what was expected. 
  To resolve this, I researched the Volere Requirements Specification Template and looked into the section. 
  However, I was still confused and what was expected of the section. Finally, during our meeting with
  our TA I was able to clarify the expectations for this section and I was able to complete the section. 
 
   \item \textbf{How many of your requirements were inspired by
      speaking to your client(s) or their proxies (e.g., your peers,
    stakeholders, potential users)?} \newline
  I believe many of our non-functional requirements, specifically look and feel 
  requirements, as well as usability and humanity requirements. Through our
  conversations with our supervisor Jing, we learned a lot of the accessibility
  requirements for website applications. For example, one specific requirement 
  that was derived from our conversations was that the system cannot use color alone to convey any messages
  or information. I believe without having this conversation, this is a requirement that would not have been discovered. 
 
 \item \textbf{Which of the courses you have taken, or are currently
      taking, will help your team be successful with your capstone
    project?} \newline
  I believe many courses that I have taken, and some that I’m currently taking will 
  contribute to the success of our capstone project. I completed
  SFWRENG 4HC3 - Human Computer Interfaces, which has taught me many 
  important design principles, such as Normans Design Principles. Furthermore, 
  completing COMPSCI 4AL3 - Applications of Machine Learning, also will be a lot of help when 
  completing our capstone. This course introduced me to developing machine learning models and 
  will be directly applicable. Finally, taking COMPSCI 3RA3 - Software Requirements and Security 
  Considerations will also help our team be successful. 
  
\end{enumerate}

\textbf{Casey Francine Bulaclac - Reflection}
\begin{enumerate}
  \item \textbf{What went well while writing this deliverable?} \newline
  Having discussed the project thoroughly as a team and with our supervisor helped in writing this deliverable as the team
  was very knowledgeable about the needs for the project. 
  This deliverable went much smoother than the last due to stronger operational procedures, and better organization in how we structured and completed the SRS. 
  The team communicated well and were clear of the goals for this deliverable.
  \item \textbf{What pain points did you experience during this
  deliverable, and how did you resolve them?} \newline
  One pain point in writing the SRS was figuring out what each of the many sections entailed in the Volere's template. The template 
  is very thorough and needed many details, in which some sections seem to overlap which can be confusing. Another pain point was ensuring traceability
  between our goals in the project and the requirements. To resolve this, I made sure to ask the TA for feedback and clarification about specific sections.
  Additionally, communicating with each team member and ensuring our requirements aligned to the goals of the project was very helpful in aiding to ensure
  traceability.
\item \textbf{How many of your requirements were inspired by
      speaking to your client(s) or their proxies (e.g., your peers,
    stakeholders, potential users)?} \newline
  Many, if not most, requirements were inspired through speaking with our supervisor, who had the most knowledge and experience with our project's 
  potential users and stakeholders. In this project, it is important to understand our target users as we are designing for accessibility, so it was critical 
  in making our requirements. 
 \item \textbf{Which of the courses you have taken, or are currently
      taking, will help your team be successful with your capstone
    project?} \newline
  In this deliverable, the course that was most beneficial was Software Requirements and Security Considerations (SFWRENG 3RA3) as we learned how to create effected SRS documents.
  A course I've taken that will help thoroughly in ensuring our user interface is accessible is Human Computer Interfaces (SFWRENG 4HC3) as the course taught us principles of good design. Lastly,
  another course I took that contribute to the success of our project is Applications of Machine Learning (SFWRENG 4AL3) as this project heavily involves machine learning
  in generating alternative text. 
\end{enumerate}