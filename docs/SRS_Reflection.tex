\begin{enumerate}
  \item What went well while writing this deliverable? 
  
  Throughout the process, our team maintained organization and had good communication.  It was simpler to preserve flow and prevent redundancy because we had a clear framework and consistent formatting from earlier deliverables.  We were also able to maintain our efficiency and keep the material consistent with our previous work.  In order review sections, define expectations, and make sure the deliverable satisfied all requirements, we also had meetings with our TA and our team.
  \item What pain points did you experience during this deliverable, and how did
  you resolve them?

  One difficulty was avoiding redundancy by keeping formatting constant and clearly distinguishing design components from implementation specifics. In order to guarantee accuracy and uniformity throughout all sections, we addressed this by going over the course templates and conducting team editing sessions to make sure the formatting followed a consistent style. We also helped each other out by sharing tips and tricks to improve clarity and presentation.

  \item How many of your requirements were inspired by speaking to your
  client(s) or their proxies (e.g. your peers, stakeholders, potential users)?
  
  Key design and functionality components that were influenced by internal meetings, conversations with our supervisor, pertinent compliance guidelines, and my internship's prior experience working with AI included the accessible interface, the quality of the alt text that was generated, and adherence to AODA and WCAG 2.1 standards.
  \item Which of the courses you have taken, or are currently taking, will help
  your team to be successful with your capstone project.

  The Software Requirements course from third year was extremely useful in creating this deliverable.  It created a solid foundation for organizing requirement documents, developing precise specifications, and grasping the foundations of software documentation and traceability, all of which significantly aided our work on this project.

  \item What knowledge and skills will the team collectively need to acquire to
  successfully complete this capstone project?  Examples of possible knowledge
  to acquire include domain specific knowledge from the domain of your
  application, or software engineering knowledge, mechatronics knowledge or
  computer science knowledge.  Skills may be related to technology, or writing,
  or presentation, or team management, etc.  You should look to identify at
  least one item for each team member.
  \item For each of the knowledge areas and skills identified in the previous
  question, what are at least two approaches to acquiring the knowledge or
  mastering the skill?  Of the identified approaches, which will each team
  member pursue, and why did they make this choice?
\end{enumerate}
