\textbf{Moly Mikhail  - Reflection}
\begin{enumerate}
  \item What went well while writing this deliverable? \\[1ex]
  I believe writing this deliverable many things went well. I really enjoyed getting to think about the different
  non-functional requirements. I found that have different sections of non-functional requirements encouraged me to think about different 
  aspects of the system and things we will have to keep in mind during development. For example, prior to writing 
  this deliverable, we hadn’t considered personalization and internationalization requirements; however, having to
  complete that section led us to add important functionality of allowing the user to decide which way
  to store the alternative text. 
  
  \item What pain points did you experience during this deliverable, and how did
  you resolve them?\\[1ex]
  One pain point I experience writing this deliverable dealt with completing the product boundary. 
  Initially, I was confused on The Scope of the Product section and what was expected. 
  To resolve this, I researched the Volere Requirements Specification Template and looked into the section. 
  However, I was still confused and what was expected of the section. Finally, during our meeting with
  our TA I was able to clarify the expectations for this section and I was able to complete the section. 
 
  \item How many of your requirements were inspired by speaking to your
  client(s) or their proxies (e.g. your peers, stakeholders, potential users)?\\[1ex]
  I believe many of our non-functional requirements, specifically look and feel 
  requirements, as well as usability and humanity requirements. Through our
  conversations with our supervisor Jing, we learned a lot of the accessibility
  requirements for website applications. For example, one specific requirement 
  that was derived from our conversations was that the system cannot use color alone to convey any messages
  or information. I believe without having this conversation, this is a requirement that would not have been discovered. 
 
  \item Which of the courses you have taken, or are currently taking, will help
  your team to be successful with your capstone project.\\[1ex]
  I believe many courses that I have taken, and some that I’m currently taking will 
  contribute to the success of our capstone project. I completed
  SFWRENG 4HC3 - Human Computer Interfaces, which has taught me many 
  important design principles, such as Normans Design Principles. Furthermore, 
  completing COMPSCI 4AL3 - Applications of Machine Learning, also will be a lot of help when 
  completing our capstone. This course introduced me to developing machine learning models and 
  will be directly applicable. Finally, taking COMPSCI 3RA3 - Software Requirements and Security 
  Considerations will also help our team be successful. 
  
  \item What knowledge and skills will the team collectively need to acquire to
  successfully complete this capstone project?  Examples of possible knowledge
  to acquire include domain specific knowledge from the domain of your
  application, or software engineering knowledge, mechatronics knowledge or
  computer science knowledge.  Skills may be related to technology, or writing,
  or presentation, or team management, etc.  You should look to identify at
  least one item for each team member.
  
  \item For each of the knowledge areas and skills identified in the previous
  question, what are at least two approaches to acquiring the knowledge or
  mastering the skill?  Of the identified approaches, which will each team
  member pursue, and why did they make this choice?
\end{enumerate}
